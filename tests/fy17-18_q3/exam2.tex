\documentclass[addpoints]{exam}

\usepackage{epic,array,ecltree,url,calrsfs}
\usepackage[nointegrals]{wasysym}

\usepackage{array,epsfig}
\usepackage{amsmath}
\usepackage{amsfonts}
\usepackage{amssymb}
\usepackage{amsxtra}
\usepackage{amsthm}
\usepackage{mlextra} % must be below ams packages
\usepackage{mathrsfs}
\usepackage{color}
\usepackage{array}
\usepackage{graphicx}
\graphicspath{ {../../art/} }
\usepackage{bm}
\usepackage{tikz}
\usepackage{multicol}

%Pagination stuff.
\setlength{\topmargin}{-.3 in}
\setlength{\oddsidemargin}{0in}
\setlength{\evensidemargin}{0in}
\setlength{\textheight}{9.in}
\setlength{\textwidth}{6.5in}

\newcommand{\tf}[1][{}]{%
\fillin[#1][0.25in]%
}

\noprintanswers
\unframedsolutions
\SolutionEmphasis{\itshape\small}
\SolutionEmphasis{\color{NavyBlue}}
\checkboxchar{$\Box$}
\checkedchar{$\blacksquare$}


\pointsinmargin
\begin{document}


\noindent
\begin{tabular*}{\textwidth}{l @{\extracolsep{\fill}} r @{\extracolsep{6pt}} l}
{\large CS3920: Foundations of Computer Science} &  \makebox[3in]{\large Name:\enspace\hrulefill}\\
{\large May 31, 2018} & \\
{\large Exam 2} & 
\end{tabular*}\\

\fbox{\fbox{\parbox{6in}{\textbf{Instructions}: Please answer the questions 
  below to the best of your ability. Be sure to show your work where appropriate. 
   This exam is closed book, closed notes, closed computer. There are \numpoints\ 
   points in total and \numbonuspoints\ possible extra credit points. }}}\\
\begin{questions}

\question[4] Let $A$ be $(p \land q) \imp (r \land s)$. Write $A$ in
\emph{\underline{clausal}} form.
\vspace{30mm}

\question[3] Transform the following formula to prenex normal form:
$\forall x(p(x) \imp \forall y q(y))$
\vspace{30mm}



\question[2] Let $S, S'$ be propositional logic formulas in clausal form, 
 and $C, C' \in S$ clauses of $S$. 

\begin{parts}
\part \tf[T] (T/F) If $S$ is satisfiable and $S' \subseteq S$, then $S'$ 
must be satisfiable.
\part \tf[T] (T/F) If $C$ is unsatisfiable and $C' \subseteq C$, then $C'$
must be unsatisfiable.
\end{parts}
\vspace{5mm}

\question[3] Let $S = \{sr,ps\bar{r},\bar{u}\bar{t},t\}$.

\begin{parts}
\part \tf[T] (T/F) If $S$ is satisfiable then $t = T$.
\part \tf[F] (T/F) If $S$ is satisfiable then $u = T$.
\part \tf[T] (T/F) If $S$ is satisfiable then $ps = \top$.
\end{parts}
\vspace{5mm}


%For final, need to add interpretations
%\question[4] Which of the following statements in first order logic correctly
%             expresses the anti-symmetric property (choose all that apply):
%\begin{checkboxes}
%\CorrectChoice $\forall x \forall y(p(x,y) \land p(y,x) \imp q(x,y)$
%\choice        $\forall x \forall y (\ngg q(x,y) \imp (p(x,y) \imp \ngg %p(y,x)))
%\end{checkboxes}

\question[4] Consider the following interpretation $\mathcal{I} = (P,\{L,H\},\{\})$,
where $P$ is a set of people, $L$ is a unary relation such that $L(x)$ if and
only if $x$ is a librarian, and $H$ is a unary relation such that $H(x)$ if
and only if $x$ is happy. Write a statement in first order logic to express
the following:
\begin{parts}
\part All librarians are happy if only librarians are happy.
\vspace{20mm}
\part If it is the case that if someone is a librarian, no one is happy, then
no librarian is happy.
\vspace{20mm}
\end{parts}


\question[6] Let $S$ be a propositional logic formula in clausal form, such that
$S = \{s\bar{t}, p\bar{r}\bar{s}, \bar{p}\bar{r}, r\bar{s}, \bar{p}rs, st, pt,
  rt, t\}$. Use DPLL to prove $S$ is a contradiction. Draw the execution tree.

\vspace{60mm}


\question The following is proposed as an alternative one-bit adder design:

\begin{center}
\begin{picture}(280,120)
\put(-10,  0){\makebox(20,20)[l]{$b_2$}}
\put(-10,100){\makebox(20,20)[l]{$b_1$}}
\put(260, 20){\makebox(25,20)[l]{Sum}}
\put(260, 80){\makebox(25,20)[l]{Carry}}
\put(20, 10){\line(1,0){50}}
\put(20,110){\line(1,0){60}}
\put(30, 10){\line(1, 2){40}}
\put(30,110){\line(1,-2){40}}
\put(70, 90){\line(1,0){10}}
\put(70,80){ %OR gate
  \put(20,20){\oval(40,40)[r]}
  \put(0,20){\oval(20,40)[r]}
  \put(0,0){\line(1,0){20}}
  \put(0,40){\line(1,0){20}}
}
\put(70,0){ %AND gate
  \put(20,20){\oval(40,40)[r]}
  \put(0,0){\line(0,1){40}}
  \put(0,0){\line(1,0){20}}
  \put(0,40){\line(1,0){20}}
}
\put(110,100){\line(1,0){60}}
\put(110, 20){\line(1,0){60}}
\put(210, 30){\line(1,0){40}}
\put(210, 90){\line(1,0){40}}
\put(170,70){ %AND gate
  \put(20,20){\oval(40,40)[r]}
  \put(0,0){\line(0,1){40}}
  \put(0,0){\line(1,0){20}}
  \put(0,40){\line(1,0){20}}
}\put(170,10){ %AND gate
  \put(20,20){\oval(40,40)[r]}
  \put(0,0){\line(0,1){40}}
  \put(0,0){\line(1,0){20}}
  \put(0,40){\line(1,0){20}}
}
\put(120,80){  % NOT gate
  \put( 0, 0){\line(1,0){20}}
  \put( 0, 0){\line(1,-2){10}}
  \put(20, 0){\line(-1,-2){10}}
  \put(10,-23){\circle{6}}
}
\put(130,80){\line(0,1){20}}
\put(130,54){\line(0,-1){14}}
\put(130,40){\line(1,0){40}}
\put(150,80){\line(1,0){20}}
\put(150,20){\line(0,1){60}}
\end{picture}
\end{center}

\begin{parts}
\part[4] Write a propositional logic statement using $b_1$ and $b_2$ that is
true if and only if:
\begin{subparts}
\subpart the carry bit is set
\vspace{15mm}
\subpart the sum bit is set
\vspace{15mm}
\end{subparts}
\part[4] Use your statements to evaluate the proposed design:
\begin{subparts}
\subpart Does the carry-bit give the correct output? If yes, explain how you
know this. If no, provide an example. 
\vspace{20mm}
\subpart For which values of $b_1$ and $b_2$, if any, does the circuit as a
whole give the correct output?
\vspace{20mm}
\end{subparts}
\end{parts}




\clearpage
\question[4] Fill in the blanks to produce an interpretation that
\emph{\underline{satisfies}} the
formula: 

\begin{parts}
\part $A = \forall n p(n,f(a,n))$, and $\mathcal{I}_A = \{\Z^+, \{=\}, \{\times\}, \{\fillin[1][1cm]\} \}$
\part $A = \forall n p(n,f(a,n))$, and $\mathcal{I}_A = \{\N_0, \{=\}, \{\fillin[+][1cm]\}, \{0\}\}$
\part $A = \exists x p(x,a) \imp \forall x p(x,a)$, and $\mathcal{I}_A = \{\fillin[\{1\}][1cm], \{\geq\}, \{\fillin[1][1cm]\} \}$
\end{parts}
\vspace{5mm}

\question[4] Fill in the blanks to produce an interpretation that
\emph{\underline{falsifies}} the
formula: 

\begin{parts}
\part $A = \forall x \forall y (p(x,y) \imp
    p(f(x,a),f(y,a)))$, $\mathcal{I}_A = \{\Z, \{< \}, \{\fillin[][1cm]\}, \fillin[][1cm] \}$
\part $A = (\exists x p(x) \land \exists x q(x) ) \imp \exists x (p(x) \land
    q(x))$, $\mathcal{I}_A = \{\N_0, \{\{1\},\fillin[][1cm] \}, \{\} \}$ 
\part $A = \exists x \exists y (p(x) \land \ngg p(y))$, $\mathcal{I}_A =
\{\fillin[][1cm], \{\{1\} \}, \{\} \}$
\end{parts}
\vspace{5mm}



\question[4] Let $A$ and $B$ be problems such that $A$ reduces to $B$, and
evaluate each of the following claims:
\begin{parts}
\part \tf[T] (T/F) If $B$ is decidable, $A$ is decidable.
\part \tf[F] (T/F) If $B$ is NP-Hard, $A$ is decidable.
\part \tf[T] (T/F) If FOL-VAL reduces to $A$, $B$ is undecidable.
\part \tf[F] (T/F) If $B$ reduces to FOL-SAT, $B$ is undecidable.
\end{parts}

\vspace{5mm}

\question Let $\mathcal{F}$ be the set of all formulas in propositional logic.
Define $A \subseteq \mathcal{F}$ to be the set of formulas that your instructor
finds interesting, and let $B$ be the set of all other formulas.
\begin{parts}

\part[4] Assume we show that a non-deterministic Turing machine can determine
whether a formula is in $A$ in polynomial time. Which of the following can you 
conclude (choose all that apply)?
\begin{checkboxes}
\CorrectChoice $A$ is in NP. 
\choice        $A$ is NP-Hard.
\CorrectChoice $A$ is decidable. 
\CorrectChoice $A$ is semi-decidable. 
\end{checkboxes}

\vspace{5mm}
\part[4] Now assume PL-SAT reduces to $A$ in polynomial time. Which of the following can you 
conclude (choose all that apply)?
\begin{checkboxes}
\CorrectChoice $A$ is NP-Complete.
\CorrectChoice $A$ is NP-Hard.
\CorrectChoice $B$ is in CoNP.
\CorrectChoice $B$ is CoNP-Complete.
\end{checkboxes}
\end{parts}

\vspace{5mm}
%\question Let $FOL$ be the set of all formulas in first-order logic, and $A$-FOL  



%\part[1] $A = (\forall x p(x) \imp \forall x q(x)) \imp \forall x (p(x) \imp q(x))$ and $\mathcal{I}_A = \{\N_0, \{=\}, \{\times\}, \fillin \}$

\bonusquestion[2] Let $D$ be the relation such that $(a,b) \in D$ iff $a$ divides
$b$, for integers $a$ and $b$. 

\begin{parts}
\part Write an expression in first order logic and give an 
appropriate interpretation to define $D$ in terms of equality and
multiplication. 
\vspace{10mm}
\part Express the claim that $D$ is antisymmetric. 
\end{parts}

\end{questions}
\end{document}


