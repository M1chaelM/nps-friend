\documentclass[addpoints]{exam}

%These tell TeX which packages to use.
\usepackage{array,epsfig}
\usepackage{amsmath}
\usepackage{amsfonts}
\usepackage{amssymb}
\usepackage{amsxtra}
\usepackage{amsthm}
\usepackage{mlextra} % must be below ams packages
\usepackage{mathrsfs}
\usepackage{color}
\usepackage{array}
\usepackage{graphicx}
\graphicspath{ {../../art/} }
\usepackage{bm}
\usepackage{tikz}
\usepackage{multicol}

%Pagination stuff.
\setlength{\topmargin}{-.3 in}
\setlength{\oddsidemargin}{0in}
\setlength{\evensidemargin}{0in}
\setlength{\textheight}{9.in}
\setlength{\textwidth}{6.5in}



%\pagestyle{empty} 
%\footer{}{\thepage}{}


\begin{document}


\noindent
\begin{tabular*}{\textwidth}{l @{\extracolsep{\fill}} r @{\extracolsep{6pt}} l}
{\large CS3920: Foundations of Computer Science} &  \makebox[3in]{\large Name:\enspace\hrulefill}\\
%{Points: /\numpoints\. } \\
%{\large June 8, 2018} & \\
{\large Quiz 1} & \\
{\large Points: \hspace{1cm}/\numpoints} & 
\end{tabular*}\\

\fbox{\fbox{\parbox{6in}{\textbf{Instructions}: Please answer the questions 
  below to the best of your ability. \textbf{Show your work} where appropriate to receive credit. 
   This quiz is closed book, closed notes, closed computer. }}}\\
\begin{questions}

\question[2]
Consider a set $Z=\{-2, -1, 0, 1, 2\}$. \textbf{How many relations} are there on $Z$? (Hint: $|Z|=?$)
\vspace{10mm}	
	
	


%\question[4]
%Consider the function $f$: $\mathbb{R} \to \mathbb{R}$ 
%such that $fx) = x^4$.
%
%Is $f$ \textbf{injective}? Give a proof of your answer. 
%\vspace{18mm}
%
%Is $f$ \textbf{surjective}? Give a proof of your answer.
%\vspace{18mm}


\question[6]
Prove the division relation $a|b$ is a \textbf{partial order} on the set of positive integers.
\vspace{70mm}


\question[6]
Show that the relation $R= \{ (x,y) | a \equiv   b \enspace (mod \  7)\}$ is an \textbf{equivalence relation} on the set of integers.
\vspace{70mm}




\question[2]
List the \textbf{equivalence classes} of 0 and 1 for congruence modulo 3.
\vspace{30mm}



%\newpage

\end{questions}
\end{document}


