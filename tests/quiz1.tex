\documentclass[addpoints]{exam}

%These tell TeX which packages to use.
\usepackage{array,epsfig}
\usepackage{amsmath}
\usepackage{amsfonts}
\usepackage{amssymb}
\usepackage{amsxtra}
\usepackage{amsthm}
\usepackage{mathrsfs}
\usepackage{color}
\usepackage{array}
\usepackage{graphicx}
\usepackage{bm}
\usepackage{tikz}
\usepackage{multicol}

\renewcommand\qedsymbol{$\blacksquare$}

%Here I define some theorem styles and shortcut commands for symbols I use often
\theoremstyle{definition}
\newtheorem{defn}{Definition}
\newtheorem{thm}{Theorem}
\newtheorem{cor}{Corollary}
\newtheorem*{rmk}{Remark}
\newtheorem{lem}{Lemma}
\newtheorem*{joke}{Joke}
\newtheorem{ex}{Example}
\newtheorem*{soln}{Solution}
\newtheorem{prop}{Proposition}

\newcommand{\lra}{\longrightarrow}
\newcommand{\ra}{\rightarrow}
\newcommand{\surj}{\twoheadrightarrow}
\newcommand{\graph}{\mathrm{graph}}
\newcommand{\bb}[1]{\mathbb{#1}}
\newcommand{\Ell}{\mathscr{L}}
\newcommand{\Z}{\bb{Z}}
\newcommand{\Q}{\bb{Q}}
\newcommand{\R}{\bb{R}}
\newcommand{\C}{\bb{C}}
\newcommand{\N}{\bb{N}}
\newcommand{\M}{\mathbf{M}}
\newcommand{\m}{\mathbf{m}}
\newcommand{\MM}{\mathscr{M}}
\newcommand{\HH}{\mathscr{H}}
\newcommand{\Om}{\Omega}
\newcommand{\Ho}{\in\HH(\Om)}
\newcommand{\bd}{\partial}
\newcommand{\del}{\partial}
\newcommand{\bardel}{\overline\partial}
\newcommand{\textdf}[1]{\textbf{\textsf{#1}}\index{#1}}
\newcommand{\img}{\mathrm{img}}
\newcommand{\ip}[2]{\left\langle{#1},{#2}\right\rangle}
\newcommand{\inter}[1]{\mathrm{int}{#1}}
\newcommand{\exter}[1]{\mathrm{ext}{#1}}
\newcommand{\cl}[1]{\mathrm{cl}{#1}}
\newcommand{\ds}{\displaystyle}
\newcommand{\vol}{\mathrm{vol}}
\newcommand{\cnt}{\mathrm{ct}}
\newcommand{\osc}{\mathrm{osc}}
\newcommand{\LL}{\mathbf{L}}
\newcommand{\UU}{\mathbf{U}}
\newcommand{\support}{\mathrm{support}}
\newcommand{\AND}{\;\wedge\;}
\newcommand{\OR}{\;\vee\;}
\newcommand{\Oset}{\varnothing}
\newcommand{\st}{\ni}
\newcommand{\wh}{\widehat}

%Pagination stuff.
\setlength{\topmargin}{-.3 in}
\setlength{\oddsidemargin}{0in}
\setlength{\evensidemargin}{0in}
\setlength{\textheight}{9.in}
\setlength{\textwidth}{6.5in}



%\pagestyle{empty} 
%\footer{}{\thepage}{}


\begin{document}


\noindent
\begin{tabular*}{\textwidth}{l @{\extracolsep{\fill}} r @{\extracolsep{6pt}} l}
{\large CS3920: Foundations of Computer Science} &  \makebox[3in]{\large Name:\enspace\hrulefill}\\
{\large June 8, 2018} & \\
{\large Quiz 1 Take 4} & 
\end{tabular*}\\

\fbox{\fbox{\parbox{6in}{\textbf{Instructions}: Please answer the questions 
  below to the best of your ability. Be sure to show your work where appropriate. 
   This quiz is closed book, closed notes, closed computer. There are \numpoints\ 
   points in total. }}}\\
\begin{questions}
\question[3]
Prove $A \cap (B \cup C) = (A \cap B) \cup (A \cap C)$. (Reminder: proof by
    membership table is disallowed for this exercise.)


\vspace{30mm}

\question[4]
Define the function \emph{trunc}: $\Sigma^+ \to \Sigma$ for some
alphabet $\Sigma$ such that $trunc(x) = c$
where $x$ is any string in $\Sigma^+$ and $c$ is the first character in $x$.

Is $trunc$ injective? Give a proof of your answer. 
\vspace{18mm}

Is $trunc$ surjective? Give a proof of your answer.
\vspace{18mm}


\question[3]
Let $\Sigma = \{a,b\}$
Consider the relation $P$ on $\Sigma^*$ where $xPy$ for $x,y \in \Sigma^*$
if and only if there exists a string $w$ in $\Sigma^*$ such that $y = xw$. 
For example, if $x = aba$ and $y = ababb$, then $(x,y) \in P$ because $w = bb$
and $bb \in \Sigma^*$.

Show that $P$ is a partial order.
\vspace{30mm}


\question[4]
Let $\Sigma = \{a,b,c,d\}$.
Consider the relation $T$ on $\Sigma^+$ where for any two strings $x,y \in
\Sigma^+$, $xTy$ if and only if the first character in $x$ is equal to the
first character in $y$. Prove $T$ is an equivalence relation.
\vspace{30mm}

How many equivalences classes are there with respect to $T$ on $\Sigma^+$?
\vspace{10mm}



%\newpage

\end{questions}
\end{document}


