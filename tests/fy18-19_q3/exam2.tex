\documentclass[addpoints]{exam}

\usepackage{epic,array,ecltree,url,calrsfs}
\usepackage[nointegrals]{wasysym}

\usepackage{array,epsfig}
\usepackage{amsmath}
\usepackage{amsfonts}
\usepackage{amssymb}
\usepackage{amsxtra}
\usepackage{amsthm}
\usepackage{../mlextra} % must be below ams packages
\usepackage{mathrsfs}
\usepackage{color}
\usepackage{array}
\usepackage{graphicx}
\graphicspath{ {../../art/} }
\usepackage{bm}
\usepackage{tikz}
\usepackage{multicol}

%Pagination stuff.
\setlength{\topmargin}{-.3 in}
\setlength{\oddsidemargin}{0in}
\setlength{\evensidemargin}{0in}
\setlength{\textheight}{9.in}
\setlength{\textwidth}{6.5in}

\newcommand{\tf}[1][{}]{%
\fillin[#1][0.25in]%
}

\noprintanswers
\unframedsolutions
\SolutionEmphasis{\itshape\small}
\SolutionEmphasis{\color{NavyBlue}}
\checkboxchar{$\Box$}
\checkedchar{$\blacksquare$}


\pointsinmargin
\begin{document}


\noindent
\begin{tabular*}{\textwidth}{l @{\extracolsep{\fill}} r @{\extracolsep{6pt}} l}
{\large CS3001: Foundations of Computer Science} &  \makebox[3in]{\large Name:\enspace\hrulefill}\\
{\large May 28, 2019} & \\
{\large Exam 2} & 
\end{tabular*}\\

\fbox{\fbox{\parbox{6in}{\textbf{Instructions}: Please answer the questions 
  below to the best of your ability. Be sure to show your work where appropriate. 
   This exam is closed book, closed notes, closed computer. There are \numpoints\ 
   points and \numbonuspoints\ extra credit points in total.}}}\\
\begin{questions}

\question[4] Write the following $\id{PL}$ formula in \emph{\underline{clausal}} form:\\[2mm]
$  (\ngg r \imp (p \imp q)) \land (s \imp t)$
\vspace{30mm}


\question[5] Transform the following $\id{FOL}$ formula to prenex normal form:\\[2mm]
$\forall x \exists y r(x,y) \land (\exists x p(x) \imp \forall x q(x))$
\vspace{40mm}



\question[4] Fill in the blanks to produce an interpretation that \emph{\underline{satisfies}} the formula: 
\vspace{2mm}
\begin{parts}
\part $A = \forall x p(a,f(x,a))$, and $\mathcal{I}_A = \{\Z, \{=\}, \{\times\}, \{\fillin[0][1cm]\} \}$
\vspace{2mm}
\part $A = \forall n p(n,f(a,n))$, and $\mathcal{I}_A = \{\N, \{\ < \}, \{\fillin[\times][1cm]\}, \{1\}\}$
\end{parts}
\vspace{5mm}

\question[4] Fill in the blanks to produce an interpretation that
\emph{\underline{falsifies}} the formula: 

\vspace{2mm}
\begin{parts}
\part $A = \forall x \forall y (p(x,y) \imp p(f(x,a),f(y,a)))$, $\mathcal{I}_A = \{\Z, \{ > \}, \{\fillin[][1cm]\}, \fillin[][1cm] \}$
\vspace{2mm}
\part $A = (\exists x p(x,a) \land \exists x q(x,a) ) \imp \exists x (p(x,a) \land
    q(x,a))$, $\mathcal{I}_A = \{\N_0, \{\fillin[<][1cm] ,\fillin[>][1cm] \}, \{5\} \}$ 
\vspace{2mm}
\end{parts}
\vspace{5mm}

\clearpage
\question[4] Let $S, S'$ be propositional logic formulas in clausal form, 
 and let $C, C'$ be individual clauses. 

\begin{parts}
\part \tf[T] (T/F) If $S$ is satisfiable and $S' \subseteq S$, then $S'$ 
must be satisfiable.
\part \tf[T] (T/F) If $S$ is satisfiable and $C$ is trivial, then $S \cup {C}$ 
must be satisfiable.
\part \tf[F] (T/F) If $S$ is satisfiable and $C$ is satisfiable, $S \cup {C}$
must be satisfiable.
\part \tf[F] (T/F) If $C$ is unsatisfiable and $C' \subseteq C$, then $C'$
must be unsatisfiable.
\end{parts}
\vspace{5mm}

%For final, need to add interpretations
%\question[4] Which of the following statements in first order logic correctly
%             expresses the anti-symmetric property (choose all that apply):
%\begin{checkboxes}
%\CorrectChoice $\forall x \forall y(p(x,y) \land p(y,x) \imp q(x,y)$
%\choice        $\forall x \forall y (\ngg q(x,y) \imp (p(x,y) \imp \ngg %p(y,x)))
%\end{checkboxes}
\question Consider the following interpretation $\mathcal{I} = (P,\{L,H\},\{\})$,
where 
\begin{itemize}
\item $P$ is a set of people, 
\item $G$ is a unary relation such that $G(x)$ iff $x$ is a grad student, and
\item $R$ is a unary relation such that $R(x)$ iff and only if $x$ is rich. 
\end{itemize}
Write a statement in first order logic to express the following:
\begin{parts}
\part[3] If all grad students are rich then everyone is rich.
\vspace{20mm}
\part[3] If someone is rich and not a grad student, then no grad student is rich.
\vspace{20mm}
\end{parts}


\question[6] Let $S$ be a propositional logic formula in clausal form, given
below.  Use DPLL to prove $S$ is a contradiction. Draw the execution tree.
\[S = \{\bar{r}\bar{s}, s\bar{p}, r\bar{s}\bar{p}, pt, rt, \bar{r}sp, st,
  p\bar{t}, t\}\]. 

\vspace{60mm}

\clearpage
\question The following is proposed as an alternative one-bit adder design:

\begin{center}
\begin{picture}(380,120)
\put(-10,  0){\makebox(20,20)[l]{$b_2$}}
\put(-10,100){\makebox(20,20)[l]{$b_1$}}
\put(300, 20){\makebox(25,20)[l]{s}}
\put(300, 80){\makebox(25,20)[l]{c}}
\put(20, 10){\line(1,0){58}}
\put(20,110){\line(1,0){60}}
\put(30, 10){\line(1, 2){40}}
\put(30,110){\line(1,-2){38}}
\put(70, 90){\line(1,0){10}}
\put(68, 33){\line(1,0){10}}
\put(104, 33){\line(1,0){10}}
\put(104, 10){\line(1,0){10}}

\put(78,23){  % NOT gate
  \put( 0, 0){\line(0,1){20}}
  \put( 0, 0){\line(2,1){20}}
  \put(0, 20){\line(2,-1){20}}
  \put(23,10){\circle{6}}
}
\put(78,0){  % NOT gate
  \put( 0, 0){\line(0,1){20}}
  \put( 0, 0){\line(2,1){20}}
  \put(0, 20){\line(2,-1){20}}
  \put(23,10){\circle{6}}
}
\put(70,80){ %OR gate
  \put(20,20){\oval(40,40)[r]}
  \put(0,20){\oval(20,40)[r]}
  \put(0,0){\line(1,0){20}}
  \put(0,40){\line(1,0){20}}
}
\put(110,100){\line(1,0){130}}
\put(150,40){\line(0,1){60}}
\put(150,40){\line(1,0){20}}

\put(105,0){ %OR gate
  \put(20,20){\oval(40,40)[r]}
  \put(0,20){\oval(20,40)[r]}
  \put(0,0){\line(1,0){20}}
  \put(0,40){\line(1,0){20}}
}
\put(145, 20){\line(1,0){25}}

\put(170,10){ %AND gate
  \put(20,20){\oval(40,40)[r]}
  \put(0,0){\line(0,1){40}}
  \put(0,0){\line(1,0){20}}
  \put(0,40){\line(1,0){20}}
}
\put(210, 30){\line(1,0){85}}

\put(220,30){\line(0,1){14}}
\put(210,45){  % NOT gate
  \put( 0, 0){\line(1,0){20}}
  \put( 0, 0){\line(1,2){10}}
  \put(20, 0){\line(-1,2){10}}
  \put(10,23){\circle{6}}
}
\put(220,71){\line(0,1){14}}
\put(220, 85){\line(1,0){20}}

\put(240,70){ %AND gate
  \put(20,20){\oval(40,40)[r]}
  \put(0,0){\line(0,1){40}}
  \put(0,0){\line(1,0){20}}
  \put(0,40){\line(1,0){20}}
}
\put(280, 90){\line(1,0){15}}
\end{picture}
\end{center}


Let $S$ be a set of formulas in propositional logic that describe the
properties of the proposed one-bit adder and let $P_S$ be the set of atoms in
$S$.
\begin{parts}
\part[2] Give an explicit definition of $P_S$ (write the elements).
\vspace{5mm}

\part[3] Let $A \in S$ be a formula that describes the relationship between $b_1$,
  $b_2$ and $s$ in the proposed adder. Fill in the blank below to complete
  $A$:\\  

$A \equiv \fillin[(b_1 \lor b_2) \land (\ngg b_1 \lor b_2)][6cm] \eqv s $

\vspace{2mm}
\part[3] Write a formula $B \in S$ that describes the relationship between $b_1$,
  $b_2$ and $c$.\\

  $B \equiv$

\vspace{2mm}
\part Evaluate $A$ and $B$ above under the following interpretation (show
    your work):\\
    $\mathcal{I}_S(b_1) = 0, \mathcal{I}_S(b_0) = 1, \mathcal{I}_S(s) = 1, \mathcal{I}_S(c) = 1$
\begin{subparts}
\vspace{2mm}
\subpart[2]\label{addervala} $v_{\mathcal{I}_S}(A) = $ 
\vspace{10mm}
\subpart[2]\label{addervalb} $v_{\mathcal{I}_S}(B) = $ 
\vspace{10mm}
\subpart[3] Looking just at the results in parts \ref{addervala} and
\ref{addervalb} above, is the adder functioning correctly? Briefly justify your
answer.

\vspace{10mm}
\end{subparts}

%\part[2] Is $S$ satisfiable?
%
%\part[2] Is $S$ valid?


%\part[4] Recall that the \emph{desired} relationship between $b_1$, $b_2$ and $c$ can
%be expressed by the statement: 
%\[c \eqv b_1 \land b_2\] 
%
%Does the proposed circuit create an equivalent relationship? If yes, show that it is 
%equivalent. If no, provide a counterexample.

\vspace{20mm}
\end{parts}




\clearpage
\question[6] 
\begin{itemize}
\item Let $\mathcal{F}$ be the set of all formulas in propositional logic.
\item Define $A \subseteq \mathcal{F}$ to be the set of formulas that your instructor
finds pleasing, and
\item let $B$ be the set of all other formulas.
\end{itemize}
Assume we show that a non-deterministic Turing machine can determine
whether a formula is in $A$ in polynomial time. Which of the following can you 
conclude (choose all that apply)?
\begin{checkboxes}
\choice $A$ is in P. 
\CorrectChoice $A$ is in NP. 
\choice $A$ is in CoNP. 
\CorrectChoice $A$ is decidable. 
\CorrectChoice $B$ is in CoNP. 
\CorrectChoice $B$ is decidable. 
\CorrectChoice $A$ is semi-decidable. 
\end{checkboxes}
\vspace{5mm}

\question[4] Let $A$ and $B$ be problems and assume the following facts have
been established:
\begin{itemize}
\item $A$ reduces to $B$
\item $\id{FOL-VAL}$ reduces to $A$
\item $A$ reduces to $\id{FOL-UNSAT}$.
\end{itemize}

Given the information above, which of the following can you conclude (choose all that apply)?
\begin{checkboxes}
\CorrectChoice Neither $A$ nor $B$ are decidable.
\choice  Both $A$ and $B$ are semi-decidable.
\CorrectChoice The complement of $A$ is neither decidable nor semi-decidable. 
\choice The complement of $B$ is neither decidable nor semi-decidable. 
\end{checkboxes}


%part[2] Now assume you \textbf{\emph{also}} know that $\id{FOL-VAL}$ reduces to $A$
%nd $A$ reduces to $\id{FOL-UNSAT}$.

\vspace{5mm}
%\part[4] Now assume PL-SAT reduces to $A$ in polynomial time. Which of the following can you 
%conclude (choose all that apply)?
%\begin{checkboxes}
%\CorrectChoice $A$ is NP-Complete.
%\CorrectChoice $A$ is NP-Hard.
%\CorrectChoice $B$ is in CoNP.
%\CorrectChoice $B$ is CoNP-Complete.
%\end{checkboxes}

\vspace{5mm}
%\question Let $FOL$ be the set of all formulas in first-order logic, and $A$-FOL  
\bonusquestion[2] Fill in the blanks to produce an interpretation that
\emph{\underline{satisfies}} the formula:\\[2mm]
$A = (\exists x p(x,a) \imp \forall x p(x,a)) \land (\exists y p(a,y) \imp \forall y p(a,y))$, and $\mathcal{I}_A = \{\fillin[\{1\}][1cm], \{\ \geq \}, \{\fillin[1][1cm]\} \}$

\bonusquestion[2] Fill in the blanks to produce an interpretation that
\emph{\underline{falsifies}} the formula: 

$A = \exists x \exists y (p(x,a) \land \ngg p(y,a))$, $\mathcal{I}_A = \{\fillin[][1cm], \{ = \}, \{2\} \}$


%\part[1] $A = (\forall x p(x) \imp \forall x q(x)) \imp \forall x (p(x) \imp q(x))$ and $\mathcal{I}_A = \{\N_0, \{=\}, \{\times\}, \fillin \}$
\end{questions}
\end{document}


