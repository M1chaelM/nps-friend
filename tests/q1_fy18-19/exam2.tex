\documentclass[addpoints]{exam}

\usepackage{epic,array,ecltree,url}
\usepackage[nointegrals]{wasysym}

\usepackage{array,epsfig}
\usepackage{amsmath}
\usepackage{amsfonts}
\usepackage{amssymb}
\usepackage{amsxtra}
\usepackage{amsthm}
\usepackage{mlextra} % must be below ams packages
\usepackage{mathrsfs}
\usepackage{color}
\usepackage{array}
\usepackage{graphicx}
\graphicspath{ {../art/} }
\usepackage{bm}
\usepackage{tikz}
\usepackage{multicol}
\usepackage{enumitem}
\usepackage{pifont}

%Pagination stuff.
\setlength{\topmargin}{-.3 in}
\setlength{\oddsidemargin}{0in}
\setlength{\evensidemargin}{0in}
\setlength{\textheight}{9.in}
\setlength{\textwidth}{6.5in}

\newcommand{\tf}[1][{}]{%
\fillin[#1][0.25in]%
}


\pointsinmargin
\begin{document}


\noindent
\begin{tabular*}{\textwidth}{l @{\extracolsep{\fill}} r @{\extracolsep{6pt}} l}
	{\large CS3920: Foundations of Computer Science} &  \makebox[3in]{\large Name:\enspace\hrulefill}\\
	%{Points: /\numpoints\. } \\
	%{\large June 8, 2018} & \\
	{\large Exam 2} & \\
	{\large Points: \hspace{1cm}/\numpoints} & 
\end{tabular*}\\

\fbox{\fbox{\parbox{6in}{\textbf{Instructions}: Please answer the questions 
  below to the best of your ability. Be sure to show your work where appropriate. 
   This exam is closed book, closed notes, closed computer. There are \numpoints\ 
   points in total with points denoted in parentheses.
%    and \numbonuspoints\ possible extra credit points. 
    }}}\\
\begin{questions}
\vspace{1cm}

\question Consider the following formula: $A=( p \rightarrow q) \lor p$.
\begin{parts}
	
	
	\part[1] Which of the following conclusions can be made? (Select all that apply.)
	
	$\square$ Valid \hspace{1cm} $\square$ Falsifiable \hspace{1cm} $\square$ Satisfiable \hspace{1cm} $\square$ Unsatisfiable
	
	\vspace{.5cm}
	
	\part[1] Which of the following correctly characterize(s) $\lnot A$? (Select all that apply.)
	
	$\square$ Valid \hspace{1cm} $\square$ Falsifiable \hspace{1cm} $\square$ Satisfiable \hspace{1cm} $\square$ Unsatisfiable
\end{parts}

\vspace{1cm}



%\question[6]
%Let $A$ be a problem relati such that $A$ is in class . Which of the following conclusions can be made? (Select all that apply.)

\question[2] What does it mean for a problem $A$ to be decidable in time $t$?

\vspace{3cm}

\question[3]
Let $A= p \leftrightarrow (q \land r)$.
%Let $A= (p\land t) \rightarrow (q \land s)$.
Write $A$ in clausal form. Circle your final answer.

\vspace{8cm}

\question[2] Let $B$ is equivalent to some formula $(\lnot(p \land q) \rightarrow r) \rightarrow (s \land \lnot s)$, and let $A$ be a logical consequent of $U$.

What can we say about $A$ with respect to $U \cup \{B\}$?

\vspace{3cm}

\newpage


\question[4] Use DPLL to show that $A=\{p\bar{q}r, q \bar{r}, \bar{r}p, \bar{q}\bar{p}\}$ is satisfiable or unsatisfiable. Draw the full execution tree. 

If $A$ is satisfiable, give an interpretation that satisfies $A$, else explain why it is unsatisfiable.
\vspace{7cm}





\question Let $p$ be a binary relation on the set S. 

\begin{parts}
\part[2] Write a formula in first order logic to express the claim that $p$ is reflexive. (Do \textbf{ not} write an interpretation.)

\vspace{1.5cm}

\part[2] Write a formula in first order logic to express the claim that $p$ is symmetric. (Do \textbf{ not} write an interpretation.)
\vspace{1.5cm}


\part[2] Write a formula in first order logic to express the claim that $p$ is transitive. (Do \textbf{ not} write an interpretation.)


\vspace{1.5cm}

\part[2] Write a formula in first order logic to express the claim that $p$ is an equivalence relation. (Do \textbf{ not} write an interpretation.)
	
\end{parts}



\vspace{2cm}

	
	\question[4]
	Let $A$ and $B$ be two problems such that $A$ is reducible to $B$. Which of the following conclusions can be made? (Select all that apply.)
	
	\begin{itemize}
		\renewcommand{\labelitemi}{$\square$}
		\renewcommand{\labelitemii}{$\square$}
		\renewcommand{\labelitemiii}{$\square$}
		\renewcommand{\labelitemiv}{$\square$}
		\item If $A$ is P, $B$ is NP.
		\item If $B$ is P, $A$ is NP.
		\item If $A$ is NP-hard, $B$ is NP-hard.
		\item If $B$ is NP-hard, $A$ is NP-hard.
		%\item If $A$ is semi-decidable, $B$ is semi-decidable.
		%\item If $B$ is semi-decidable, $A$ is semi-decidable. 
	\end{itemize}
	
	\vspace{5mm}


\question[4]

Let $A,A'$ be propositional logic formulas in clausal form, and $ C, C' \in A$ be clauses of A.

\begin{parts}
\part \underline{\hspace{1cm}} (T/F) If $A$ is satisfiable and $A \subseteq A'$ then $A'$ must be satisfiable.
\part \underline{\hspace{1cm}} (T/F) If $C$ is unsatisfiable and $C \subseteq C'$ then $C'$ must be unsatisfiable. 
\end{parts}

\vspace{.5cm}



\question
Consider the following interpretation $\mathscr{I} = (P, \{H, T, Q\}, \{\})$ where  $P$ is a set of people and
\begin{itemize}
\item $H$ is a unary
relation such that $H(x)$ if and only if $x$ has pie on Thanksgiving,
\item $T$ is a unary
relation such that $T(x)$ if and only if $x$ has turkey on Thanksgiving,
\item and $Q$ is a unary relation such that $Q(x)$ if and
only if $x$ is celebrating.
\end{itemize}  Write formulas in first order logic to express each of the following statements:

\begin{parts}
	\part[2] Everyone who has pie on Thanksgiving is celebrating.
	
	\vspace{1.5cm}
	
	\part[2] If everyone is celebrating, then someone has either pie or turkey on Thanksgiving. 
	
%	someone does not have either pie or turkey on Thanksgiving, then they are not celebrating.
%	\vspace{1cm}
	
\end{parts}

\vspace{2cm}
	
	
	\question
	Consider the following formula: 
	
	$\exists x \forall y p(x,y) \land p (a, x)$. 
	\begin{parts}
		
		
		
		\part[2] Write a sentence in English for the meaning of the interpretation $(\mathbb{N}, \{ \leq \}, \{0\})$ under this formula. 
		
		
		\vspace{1.5cm}
		
		\part[1] Fill in the blank to produce an interpretation that satisfies the formula: \\
		
		$\mathscr{S}=(\mathbb{R}, \{\underline{\hspace{3cm}}\}, \{1\} )$.
		
		
	\end{parts}
	
	
	\vspace{.5cm}
	
	
	
	
%\question[4]
%Let $\mathscr{F}$ be the set of all formulas in propositional logic. Define $A \subseteq \mathscr{F}$ to be the set of formulas that your instructor finds interesting, and let $B$ be the set of all other formulas. 	
%
%Assume that we show that a non-deterministic Turing machine can determine whether or not a formula is in $A$ in polynomial time $t$. Which of the following can you conclude (check all that apply)?
%
%\begin{itemize}
%	\renewcommand{\labelitemi}{$\square$}
%	\renewcommand{\labelitemii}{$\square$}
%	\renewcommand{\labelitemiii}{$\square$}
%	\renewcommand{\labelitemiv}{$\square$}
%	\item $A$ is NP.
%	\item $A$ is NP-hard.
%	\item $A$ is decidable.
%	\item $A$ is semi-decidable. 
%	
%	%\item If $A$ is semi-decidable, $B$ is semi-decidable.
%	%\item If $B$ is semi-decidable, $A$ is semi-decidable. 
%\end{itemize}
%%	\part[2] If someone does not 


\question[4] Use resolution to show that $S=\{p\bar{q}r, q \bar{r}, \bar{r}p, \bar{q}\bar{p}, r \}$ is satisfiable or unsatisfiable.

 State what the new full set of clauses $S$ was before the last step of the resolution algorithm.


\vspace{7cm}






\question[4] Prove $\models \exists x (A(x)\rightarrow B(x)) \leftrightarrow (\forall x A(x) \rightarrow \exists x B(x))$.

\vspace{8cm}

\question[6]
Consider the reduction $A \leq_{T} B$.  

Claim: Turing reduction, $\leq_{T}$, is \textbf{not} a partial order on the set of all problems. 

Which properties of partial ordering hold and which do not? Justify your answers.

\vspace{8cm}

% DPLL \done
%Resoltion \done
% theory
% FOL proof
% FOL interpretation
% FOL satisfy / falsify
% draw dpll tree \done
% draw resolution tree  \done


\end{questions}
\end{document}


