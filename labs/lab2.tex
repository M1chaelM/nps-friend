\documentclass[]{exam}

%These tell TeX which packages to use.
\usepackage{array,epsfig}
\usepackage{amsmath}
\usepackage{amsfonts}
\usepackage{amssymb}
\usepackage{amsxtra}
\usepackage{amsthm}
\usepackage{mathrsfs}
\usepackage{color}
\usepackage{array}
\usepackage{graphicx}
\graphicspath{ {../art/} }
\usepackage{bm}
\usepackage{tikz}
\usepackage{multicol}
\usepackage{enumitem}

\renewcommand\qedsymbol{$\blacksquare$}

%Here I define some theorem styles and shortcut commands for symbols I use often
\theoremstyle{definition}
\newtheorem{defn}{Definition}
\newtheorem{thm}{Theorem}
\newtheorem{cor}{Corollary}
\newtheorem*{rmk}{Remark}
\newtheorem{lem}{Lemma}
\newtheorem*{joke}{Joke}
\newtheorem{ex}{Example}
\newtheorem*{soln}{Solution}
\newtheorem{prop}{Proposition}

\newcommand{\lra}{\longrightarrow}
\newcommand{\ra}{\rightarrow}
\newcommand{\surj}{\twoheadrightarrow}
\newcommand{\graph}{\mathrm{graph}}
\newcommand{\bb}[1]{\mathbb{#1}}
\newcommand{\Ell}{\mathscr{L}}
\newcommand{\Z}{\bb{Z}}
\newcommand{\Q}{\bb{Q}}
\newcommand{\R}{\bb{R}}
\newcommand{\C}{\bb{C}}
\newcommand{\N}{\bb{N}}
\newcommand{\M}{\mathbf{M}}
\newcommand{\m}{\mathbf{m}}
\newcommand{\MM}{\mathscr{M}}
\newcommand{\HH}{\mathscr{H}}
\newcommand{\Om}{\Omega}
\newcommand{\Ho}{\in\HH(\Om)}
\newcommand{\bd}{\partial}
\newcommand{\del}{\partial}
\newcommand{\bardel}{\overline\partial}
\newcommand{\textdf}[1]{\textbf{\textsf{#1}}\index{#1}}
\newcommand{\img}{\mathrm{img}}
\newcommand{\ip}[2]{\left\langle{#1},{#2}\right\rangle}
\newcommand{\inter}[1]{\mathrm{int}{#1}}
\newcommand{\exter}[1]{\mathrm{ext}{#1}}
\newcommand{\cl}[1]{\mathrm{cl}{#1}}
\newcommand{\ds}{\displaystyle}
\newcommand{\vol}{\mathrm{vol}}
\newcommand{\cnt}{\mathrm{ct}}
\newcommand{\osc}{\mathrm{osc}}
\newcommand{\LL}{\mathbf{L}}
\newcommand{\UU}{\mathbf{U}}
\newcommand{\support}{\mathrm{support}}
\newcommand{\AND}{\;\wedge\;}
\newcommand{\OR}{\;\vee\;}
\newcommand{\Oset}{\varnothing}
\newcommand{\st}{\ni}
\newcommand{\wh}{\widehat}

%Pagination stuff.
\setlength{\topmargin}{-.3 in}
\setlength{\oddsidemargin}{0in}
\setlength{\evensidemargin}{0in}
\setlength{\textheight}{9.in}
\setlength{\textwidth}{6.5in}


\title{Lab 2: Recusive Definitions and Inductive Proofs}
\author{Foundations of Computer Science}
\date{\today}
%\pagestyle{empty} 
%\footer{}{\thepage}{}

\begin{document}

\maketitle

\begin{questions}

\section*{Recursive Definitions}
\question Write a Post system that defines the odd natural numbers.


\question Write a Post system that defines the integers.

\question Write a Post system that defines the set of binary strings of odd
length that have a ``0'' as the middle character.

\question Write a Post system that defines the set of palindromes with odd
length.


\section*{Inductive Proofs}
For the following proofs, begin by setting up the correct basis, hypothesis
and inductive step in terms of $S(n)$.


\question Prove for all $n \geq 1$, $\sum_{i=1}^{n} (2n - 1) = n^2$. 
\question Consider the following Post System:
\begin{tabbing}
{\bf R2}XX \=  \kill
{\bf B} \>
        \(\begin{array}[t]{l}
        2\in S
        \end{array}\) \\[2ex]
{\bf R} \>
        \(\begin{array}[t]{l}
        x \in S \;\;\;y \in S \\
        \hline
        x + y \in S
        \end{array}\)
\end{tabbing}

\begin{enumerate}[label=\alph*)]
\item Prove $x \in S$ only if $x$ is a positive even integer (soundness). 
\item Prove if $x$ is a positive even integer, $x \in S$ (completeness).
\end{enumerate}

\question Consider the following Post System:
\begin{tabbing}
{\bf R2}XX \=  \kill
{\bf B} \>
        \(\begin{array}[t]{l}
        a \in D
        \end{array}\) \\[2ex]
{\bf R} \>
        \(\begin{array}[t]{l}
        x \in D \\
        \hline
        xx \in D
        \end{array}\) 
\end{tabbing}

Prove for all $x \in D$, $|x| = 2^n$, where $n$ is the derivation height
of $x$. 
%\newpage

\question Consider the following Post System:
\begin{tabbing}
{\bf R2}XX \=  \kill
{\bf B1} \>
        \(\begin{array}[t]{l}
        aa \in E
        \end{array}\) \\[2ex]
{\bf B2} \>
        \(\begin{array}[t]{l}
        bb \in E
        \end{array}\) \\[2ex]
{\bf R} \>
        \(\begin{array}[t]{l}
        x \in E \;\;\;y \in E \\
        \hline
        x(yx) \in E
        \end{array}\) 
\end{tabbing}
Prove for all $m \in E$, $\frac{|m|}{2}$ is a positive odd integer.  

\end{questions}
\end{document}


