\documentclass[]{exam}
\usepackage{epic,array,ecltree,url,calrsfs}
\usepackage[nointegrals]{wasysym}

%These tell TeX which packages to use.
\usepackage{array,epsfig}
\usepackage{amsmath}
\usepackage{amsfonts}
\usepackage{amssymb}
\usepackage{amsxtra}
\usepackage{amsthm}
\usepackage{mlextra} % must come after ams packages
\usepackage{mathrsfs}
\usepackage[dvipsnames]{xcolor}
\usepackage{array}
\usepackage{graphicx}
\graphicspath{ {../art/} }
\usepackage{bm}
\usepackage{tikz}
\usepackage{multicol}
\usepackage{enumitem}


\title{Lab 2: Recursive Definitions and Inductive Proofs}
\author{Foundations of Computer Science}
\date{\today}
%\pagestyle{empty} 
%\footer{}{\thepage}{}

\unframedsolutions
\SolutionEmphasis{\itshape\small}
\SolutionEmphasis{\color{NavyBlue}}


\begin{document}

\maketitle

\begin{questions}

\section*{Recursive Definitions}
\question Write a Post system that defines the odd natural numbers.
\begin{solution}
\begin{tabbing}
{\bf R2}XX \=  \kill
{\bf B} \>
        \(\begin{array}[t]{l}
        1\in O
        \end{array}\) \\[2ex]
{\bf R} \>
        \(\begin{array}[t]{l}
        x \in O \\
        \hline
        x + 2 \in O
        \end{array}\)
\end{tabbing}
\end{solution}

\question Write a Post system that defines the integers.
\begin{solution}
\begin{tabbing}
{\bf R2}XX \=  \kill
{\bf B} \>
        \(\begin{array}[t]{l}
        1\in Z
        \end{array}\) \\[2ex]
{\bf R1} \>
        \(\begin{array}[t]{l}
        x \in Z \\
        \hline
        -x \in Z
        \end{array}\) \\[2ex]
{\bf R2} \>
        \(\begin{array}[t]{l}
        x \in Z \;\;\;y \in Z \\
        \hline
        x + y \in Z
        \end{array}\)
\end{tabbing}
\end{solution}


\question Write a Post system that defines the set of binary strings of odd
length that have a ``0'' as the middle character.
\begin{solution}
\begin{tabbing}
{\bf R2}XX \=  \kill
{\bf B} \>
        \(\begin{array}[t]{l}
        0\in B
        \end{array}\) \\[2ex]
{\bf R1} \>
        \(\begin{array}[t]{l}
        x \in B \\
        \hline
        0x0 \in B
        \end{array}\) \\[2ex]
{\bf R2} \>
        \(\begin{array}[t]{l}
        x \in B \\
        \hline
        1x1 \in B
        \end{array}\) \\[2ex]
{\bf R3} \>
        \(\begin{array}[t]{l}
        x \in B \\
        \hline
        0x1 \in B
        \end{array}\) \\[2ex]
{\bf R4} \>
        \(\begin{array}[t]{l}
        x \in B \\
        \hline
        1x0 \in B
        \end{array}\)
\end{tabbing}

\end{solution}


\question Write a Post system that defines the set of palindromes with odd
length.
\begin{solution}
\begin{tabbing}
{\bf R2}XX \=  \kill
{\bf B1} \>
        \(\begin{array}[t]{l}
        a \in OPal
        \end{array}\) \\[2ex]
{\bf B2} \>
        \(\begin{array}[t]{l}
        b \in OPal 
        \end{array}\) \\[2ex]
{\bf R} \>
        \(\begin{array}[t]{l}
        x \in Opal \;\;\;y \in Opal \\
        \hline
        y(xy) \in Opal
        \end{array}\)
\end{tabbing}
\end{solution}

\section*{Inductive Proofs}
For the following proofs, begin by setting up the correct basis, hypothesis
and inductive step in terms of $S(n)$.


\question Prove for all $n \geq 1$, $\sum_{i=1}^{n} (2n - 1) = n^2$. 
\begin{solution}
\begin{proof}
First, set up the basis, hypothesis and induction step in terms of $S(n)$:
~\\
\textbf{Basis:} Show $S(1): \sum_{i=1}^1 (2i - 1) = 1^2$. We have $ \sum_{i=1}^1
(2i - 1) = 2(1) - 1 = 1^2 = 1$. Therefore, this is true.
~\\
\textbf{Inductive Hypothesis:} Suppose $S(n): \sum_{i=1}^{n}(2i - 1) = n^2$, 
  where $n \geq 1$.
~\\
\textbf{Inductive Step:} Show $S(n+1): \sum_{i=1}^{n+1} (2i - 1) = (n + 1)^2$. 
First, rewrite the sum $\sum_{i=1}^{n+1} (2i - 1)$ as $\sum_{i=1}^{n}
(2i - 1) + 2(n+1) - 1$. Then, by the inductive hypothesis, we have $\sum_{i=1}^{n}
(2i - 1) + 2n + 1 = n^2 + 2n + 1$. Factoring, we get $n^2 + 2n + 1 = (n + 1)^2$,
which proves the result.
\end{proof}

\end{solution}

\question Consider the following Post System:
\begin{tabbing}
{\bf R2}XX \=  \kill
{\bf B} \>
        \(\begin{array}[t]{l}
        2\in S
        \end{array}\) \\[2ex]
{\bf R} \>
        \(\begin{array}[t]{l}
        x \in S \;\;\;y \in S \\
        \hline
        x + y \in S
        \end{array}\)
\end{tabbing}

\begin{enumerate}[label=\alph*)]
\item Prove $m \in S$ only if $m$ is a positive even integer (soundness). 
\begin{solution}
\begin{proof}
First, set up the basis, hypothesis and induction step in terms of $S(n)$ and
derivation height:
~\\
\textbf{Basis:} Show $S(0)$: if the derivation height of $m \in S$ is $0$, $m$ is a
positive even number. This is the case because if the derivation height is $0$,
the derivation ends with the basis step, so $m = 2$, which is a positive even
number.
~\\
\textbf{Inductive Hypothesis:} Suppose for all $k$, where $0 \leq k \leq n$ and
$n \geq 0$, $S(k):$ $m$ is a positive even integer if $m \in S$ has a derivation
height $k$.
~\\
\textbf{Inductive Step:} Show $S(n+1)$: if $m' \in S$ and $m'$ has derivation
height $n + 1$ for $n \geq 0$, $m'$ is a positive even integer. Since $n \geq 0$,
we have $n + 1 \geq 1$, so the derivation of $m'$ must end with an application of
rule $R$. Therefore, $m' = x + y$, where $x,y \in S$, and the derivation heights
of $x$ and $y$ are less than the derivation height of $m'$. Therefore, by the
inductive hypothesis, $x$ and $y$ are positive even integers. It follows that
their sum, $m'$, is also a positive even integer.
\end{proof}

\end{solution}
\item Prove if $m$ is a positive even integer, $m \in S$ (completeness).
\begin{solution}
\begin{proof}
First, set up the basis, hypothesis and induction step in terms of $S(k)$,
where $k$ is a positive integer such that $m = 2k$, and we want to show
$2k \in S$ for all $k \geq 1$:
~\\
\textbf{Basis:} Show $S(1)$: $2(1) = 2 \in S$. This is the case by axiom $B$.
~\\
\textbf{Inductive Hypothesis:} Suppose $S(k):$ $2k \in S$ for $k \geq 1$.
~\\
\textbf{Inductive Step:} Show $S(k+1)$: $2(k+1) \in S$, where $k \geq 1$.
Observe that $2(k + 1) = 2k + 2$. By induction $2k \in S$, and by axiom
$B$, $2 \in S$, so rule $R$ gives us $2k + 2 \in S$.
\end{proof}

\end{solution}
\end{enumerate}

\question Consider the following Post System:
\begin{tabbing}
{\bf R2}XX \=  \kill
{\bf B} \>
        \(\begin{array}[t]{l}
        a \in D
        \end{array}\) \\[2ex]
{\bf R} \>
        \(\begin{array}[t]{l}
        x \in D \\
        \hline
        xx \in D
        \end{array}\) 
\end{tabbing}

Prove for all $x \in D$, $|x| = 2^n$, where $n$ is the derivation height
of $x$. 
%\newpage
\begin{solution}
\begin{proof}
First, set up the basis, hypothesis and induction step in terms of $S(n)$ and
derivation height:
~\\
\textbf{Basis:} Show $S(0):$ if $x \in D$ and $x$ has a derivation height of
$0$, then $x$ must end with rule $B$. Therefore $x$ is $a$ and $|x| = 1 = 2^0$,
which proves the formula for $n = 0$.
~\\
\textbf{Inductive Hypothesis:} Suppose $S(n)$: Assume $|x| = 2^n$ when $x$ has 
a derivation height of $n$.
~\\
\textbf{Inductive Step:} Show $S(n+1)$: Show if $y \in D$ and $y$ has a
derivation height of $n + 1$, then $|y| = 2^{n + 1}$. Since $n \geq 0$,
we have that the derivation height of $y$ is greater than or equal to
$1$, and therefore must have been produced using rule $R$. It follows
that $y$ can be expressed as $xx$ where $x \in D$. By the inductive hypothesis,
$|x| = 2^n$, and therefore $|xx| = 2(2^n) = 2^{n+1}$.
\end{proof}

\end{solution}

\question Consider the following Post System:
\begin{tabbing}
{\bf R2}XX \=  \kill
{\bf B1} \>
        \(\begin{array}[t]{l}
        aa \in E
        \end{array}\) \\[2ex]
{\bf B2} \>
        \(\begin{array}[t]{l}
        bb \in E
        \end{array}\) \\[2ex]
{\bf R} \>
        \(\begin{array}[t]{l}
        x \in E \;\;\;y \in E \\
        \hline
        x(yx) \in E
        \end{array}\) 
\end{tabbing}
Prove for all $m \in E$, $\frac{|m|}{2}$ is a positive odd integer.  
\begin{solution}
\begin{proof}
First, set up the basis, hypothesis and induction step in terms of $S(n)$ and
derivation height:
~\\
\textbf{Basis:} Show $S(0):$ if $m \in E$ and $m$ has a derivation height of
$0$, then $m$ must be produced by rule $B1$ or $B2$. Therefore $m$ is $aa$ or $bb$ 
and $|m| = 2$. Since $\frac{2}{2} = 1$ is a positive odd integer, the base
case holds.
~\\
\textbf{Inductive Hypothesis:} Suppose $S(n)$: Assume $\frac{|m|}{2}$ is odd
for $m \in E$ when $m$ has a derivation height of $n$.
~\\
\textbf{Inductive Step:} Show $S(n+1)$: Show if $m' \in E$ and $m'$ has a
derivation height of $n + 1$, then $\frac{|m'|}{2}$ is odd.  Since $n \geq 0$,
we have that the derivation height of $m'$ is greater than or equal to
$1$, and therefore must have been produced using rule $R$. It follows
that $m'$ can be expressed as $x(yx)$ where $x,y \in D$. Therefore $|m'| = |x| +
|y| + |x| = 2|x| + |y|$. We must show that $\frac{|m'|}{2} = \frac{2|x| + |y|}{2} = |x| +
\frac{|y|}{2}$ is odd.  Since $x$ and $y$ have a derivation height of $n$,
we know by the inductive hypothesis, $\frac{|x|}{2}$ 
and $\frac{|y|}{2}$ are odd, so $|x|$ is even. The sum of an even and
an odd is odd, so $|m'| = |x| + \frac{|y|}{2}$ is odd.
\end{proof}
\end{solution}
\end{questions}
\end{document}


