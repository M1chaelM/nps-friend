\documentclass[]{exam}
\usepackage{epic,array,ecltree,url,calrsfs}
\usepackage[nointegrals]{wasysym}

%These tell TeX which packages to use.
\usepackage{array,epsfig}
\usepackage{amsmath}
\usepackage{amsfonts}
\usepackage{amssymb}
\usepackage{amsxtra}
\usepackage{amsthm}
\usepackage{mlextra} % must come after ams packages
\usepackage{mathrsfs}
\usepackage[dvipsnames]{xcolor}
\usepackage{array}
\usepackage{graphicx}
\graphicspath{ {../art/} }
\usepackage{bm}
\usepackage{tikz}
\usepackage{multicol}
\usepackage{enumitem}


\title{Lab 1.5: Functions}
\author{Foundations of Computer Science}
\date{\today}
%\pagestyle{empty} 
%\footer{}{\thepage}{}
\printanswers
\unframedsolutions
\SolutionEmphasis{\itshape\small}
\SolutionEmphasis{\color{NavyBlue}}

\begin{document}

\maketitle

Recall a function $f: A \to B$ is injective if and only if 
$f(x) = f(y) \implies x = y$ for all $x,y \in A$.

A function $f: A \to B$ is surjective if and only if for all $y \in B$ there
exists $x \in A$ such that $f(x) = y$. 


\begin{questions}

\question Consider $f: \N \to \N$, where $f(x) = x^2$.

\begin{parts}
\part Is $f$ injective? Give a proof of your answer.
\begin{solution}
Yes.
\begin{proof}
$f(x) = f(y)$ implies $x^2 = y^2$. Since $x,y \in \N$, they
are positive, so it must be the case that $x = y$.
\end{proof}
\end{solution}

\part Is $f$ surjective? Give a proof of your answer.
\begin{solution}
No.
\begin{proof}
$2$ is a natural number, but there is no natural number $k$ such that
$k^2 = 2$.
\end{proof}
\end{solution}

\end{parts}


\question Consider $f: \N \to \Z$, where $f(x) = \frac{x}{2}$ if $x$ is even
and $f(x) = -\left (\frac{x+1}{2} \right) $ if $x$ is odd.
\begin{parts}
\part Is $f$ injective? Give a proof of your answer.
\begin{solution}
Yes.
\begin{proof}
We want to show $f(x) = f(y)$ implies $x = y$. There are three cases:
$x$ and $y$ are even, $x$ and $y$ are odd and $x$ is even and $y$ is odd.
\\~\\
If $x$ and $y$ are even, $f(x) = f(y) \implies \frac{x}{2} = \frac{y}{2}
\implies x = y$.
\\~\\
If $x$ and $y$ are odd, 
$f(x) = f(y) \implies - \left (\frac{x + 1}{2} \right ) = - \left (\frac{y + 1}{2} \right
    ) \implies x + 1 = y + 1 \implies x = y$.
\\~\\
If $x$ is even and $y$ is odd, then $f(x) \neq f(y)$, since $f(x)$ will be
positive and $f(y)$ will be negative. Therefore, it is vacously true that
$f(x) = f(y) \implies x = y$.
\end{proof}


\end{solution}
\part Is $f$ surjective? Give a proof of your answer.
\begin{solution}
Yes.
\begin{proof}
We must show that for any integer $z$ there is a natural number $n$ such that
$f(n) = z$. If $z$ is positive or zero, choose $n = 2z$. Clearly, $n$ is an 
even natural number, so $f(n) = \frac{2z}{2} = z$. If $z$ is negative, 
choose $n = -2z - 1$. Since $-2z$ is even and positive, $n$ must be odd.
The smallest even positive number is $2$, so $n \geq 1$. Therefore, $n$
is a natural number and $f(n) = -\frac{-2z -1 + 1}{2} = z$.
\end{proof}
\end{solution}
\end{parts}

\question
Define the function \emph{floor}: $\R^+ \to \N$ as $floor(x) = n$ if and only if
$n \leq x < n+1$.(Note that $\R^+$ denotes the positive real numbers.)

\begin{parts}
\part Is $floor$ injective? Give a proof of your answer. 
\begin{solution}
No.
\begin{proof}
Choose $x = 1.5$ and $y = 1.75$. $f(x) = f(y)$, but $x \neq y$.
\end{proof}
\end{solution}

\part Is $floor$ surjective? Give a proof of your answer.
\begin{solution}
Yes.
\begin{proof}
Any natural number is also a positive real number, so for any $n$ we have $f(n) = n$. 
\end{proof}
\end{solution}

\end{parts}
%

\end{questions}
\end{document}


