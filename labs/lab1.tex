\documentclass[]{exam}

%These tell TeX which packages to use.
\usepackage{array,epsfig}
\usepackage{amsmath}
\usepackage{amsfonts}
\usepackage{amssymb}
\usepackage{amsxtra}
\usepackage{amsthm}
\usepackage{mathrsfs}
\usepackage{color}
\usepackage{array}
\usepackage{graphicx}
\usepackage{bm}
\usepackage{tikz}
\usepackage{multicol}
\usepackage{enumitem}

\renewcommand\qedsymbol{$\blacksquare$}

%Here I define some theorem styles and shortcut commands for symbols I use often
\theoremstyle{definition}
\newtheorem{defn}{Definition}
\newtheorem{thm}{Theorem}
\newtheorem{cor}{Corollary}
\newtheorem*{rmk}{Remark}
\newtheorem{lem}{Lemma}
\newtheorem*{joke}{Joke}
\newtheorem{ex}{Example}
\newtheorem*{soln}{Solution}
\newtheorem{prop}{Proposition}

\newcommand{\lra}{\longrightarrow}
\newcommand{\ra}{\rightarrow}
\newcommand{\surj}{\twoheadrightarrow}
\newcommand{\graph}{\mathrm{graph}}
\newcommand{\bb}[1]{\mathbb{#1}}
\newcommand{\Ell}{\mathscr{L}}
\newcommand{\Z}{\bb{Z}}
\newcommand{\Q}{\bb{Q}}
\newcommand{\R}{\bb{R}}
\newcommand{\C}{\bb{C}}
\newcommand{\N}{\bb{N}}
\newcommand{\M}{\mathbf{M}}
\newcommand{\m}{\mathbf{m}}
\newcommand{\MM}{\mathscr{M}}
\newcommand{\HH}{\mathscr{H}}
\newcommand{\Om}{\Omega}
\newcommand{\Ho}{\in\HH(\Om)}
\newcommand{\bd}{\partial}
\newcommand{\del}{\partial}
\newcommand{\bardel}{\overline\partial}
\newcommand{\textdf}[1]{\textbf{\textsf{#1}}\index{#1}}
\newcommand{\img}{\mathrm{img}}
\newcommand{\ip}[2]{\left\langle{#1},{#2}\right\rangle}
\newcommand{\inter}[1]{\mathrm{int}{#1}}
\newcommand{\exter}[1]{\mathrm{ext}{#1}}
\newcommand{\cl}[1]{\mathrm{cl}{#1}}
\newcommand{\ds}{\displaystyle}
\newcommand{\vol}{\mathrm{vol}}
\newcommand{\cnt}{\mathrm{ct}}
\newcommand{\osc}{\mathrm{osc}}
\newcommand{\LL}{\mathbf{L}}
\newcommand{\UU}{\mathbf{U}}
\newcommand{\support}{\mathrm{support}}
\newcommand{\AND}{\;\wedge\;}
\newcommand{\OR}{\;\vee\;}
\newcommand{\Oset}{\varnothing}
\newcommand{\st}{\ni}
\newcommand{\wh}{\widehat}

%Pagination stuff.
\setlength{\topmargin}{-.3 in}
\setlength{\oddsidemargin}{0in}
\setlength{\evensidemargin}{0in}
\setlength{\textheight}{9.in}
\setlength{\textwidth}{6.5in}


\title{Lab 1: Sets, Functions, Relations}
\author{Foundations of Computer Science}
\date{\today}
%\pagestyle{empty} 
%\footer{}{\thepage}{}

\begin{document}

\maketitle

\begin{questions}

\section*{Warm up}
\question
Sam says: ``I'll call you if I'm interested.'' A few days later, he calls you.

Is Sam interested? Assume he's not a liar.

\question
Rouben says: ``If I call you, I'm interested.'' A few days later, he calls you.

Is Rouben interested? Assume he's not a liar.

\question If it's raining, Steve brings an umbrella.\\
    Today, Steve brought an umbrella.\\
    Is it raining today?

\question Joe breathes when he eats.\\
    Joe is breathing.\\
    Is Joe eating?

\question Let $p$ be the statement ``Beau likes fish.''\\
    Let $q$ be the statement ``1 = 0.''\\
    Let $s$ be the statement $p \implies q$.\\
    Assume $s$ is True.\\
    Does Beau like fish?

\section*{Review of Sets and Functions}
\subsection*{Sets}
Recall $a|b$ for $a,b \in \Z$ if and only if there exists some $k \in \Z$ such that $b = ka$.
\question Let $c, d$ be natural numbers such that $c$ divides $d$.
   Let $A = \{a \in \N| a \text{ divides } c \}$ and let\\ 
   $B = \{b \in \N| b\text{ divides }d \}$. Prove $A \subseteq B$.

\question
Prove $A \cup (B \cap C) = (A \cup B) \cap (A \cup C)$.



\subsection*{Functions}
Recall a function $f: A \to B$ is injective if and only if $f(x) = f(y) \implies
x = y$ for all $x,y \in A$.

A function $f: A \to B$ is surjective if and only if for all $y \in B$ there
exists $x \in A$ such that $f(x) = y$. 

\question Consider $f: \N \to \N$, where $f(x) = x^2$.

\begin{enumerate}[label=\alph*)]
\item Is $f$ injective? Give a proof of your answer.

\item Is $f$ surjective? Give a proof of your answer.
\end{enumerate}


\question Consider $f: \N \to \Z$, where $f(x) = \frac{x}{2}$ if $x$ is even
and $f(x) = -\left (\frac{x+1}{2} \right) $ if $x$ is odd.
\begin{enumerate}[label=\alph*)]
\item Is $f$ injective? Give a proof of your answer.
\item Is $f$ surjective? Give a proof of your answer.
\end{enumerate}

\question
Define the function \emph{floor}: $\R^+ \to \N$ as $floor(x) = n$ if and only if
$n \leq x < n+1$.(Note that $\R^+$ denotes the positive real numbers.)

\begin{enumerate}[label=\alph*)]
\item Is $floor$ injective? Give a proof of your answer. 

\item Is $floor$ surjective? Give a proof of your answer.

\end{enumerate}

\section*{Relations}

\question Recall that two integers have the same parity if they are both
          even or both odd. Let $P$ be a relation on $\Z$ where $aPb$
          if and only if $a$ and $b$ have the same parity. Prove that
          $P$ is an equivalence relation.


\question
Let $\Sigma = \{a,b\}$
Consider the relation $P$ on $\Sigma^*$ where $xPy$ for $x,y \in \Sigma^*$
if and only if there exists a string $w$ in $\Sigma^*$ such that $y = wx$. 
\begin{enumerate}[label=\alph*)]
\item Assume $w = aba$. Find $3$ pairs $(x_1,y_1),(x_2,y_2),(x_3,y_3)$ that belong
to the relation $P$.

\item Now assume $w$ can be any string in $\Sigma^*$. Is the pair $(x,y) \in P$,
where $x = bb$ and $y = abbabbabb$? If so, what is $w$? If not, why
not?

\item Consider $x = abababa$ and $y = aba$. Is the pair $(x,y) \in P$? If so,
what is $w$? If not, why not?

\item Characterize the relation in informal English. What does $xPy$ mean about $x$
and $y$?\\

\end{enumerate}
\question Prove the division relation, $a | b$, is a partial order on the set
of positive integers. 

\question Prove the subset relation, $A \subseteq B$, is a partial order on
sets. 

\question 
Let $\Sigma = \{0,1,2\}$.  Let $T = \{t \in \Sigma^* | |t| \geq 3\}$.
Now define $Z$ to be a relation on $T$ such that $xZy$ for $x,y \in
T$ if and only if there exist strings $r,s \in \Sigma^*$
and string $m$ in $\Sigma^3$ such that $x = rm$ and $y = sm$. 

\begin{enumerate}[label=\alph*)]
\item Prove $Z$ is an equivalence relation.

\item How many equivalences classes are there with respect to $Z$ on $T$?
\end{enumerate}

%\newpage

\end{questions}
\end{document}


