\documentclass[]{exam}

%These tell TeX which packages to use.
\usepackage{array,epsfig}
\usepackage{amsmath}
\usepackage{amsfonts}
\usepackage{amssymb}
\usepackage{amsxtra}
\usepackage{amsthm}
\usepackage{mathrsfs}
\usepackage{color}
\usepackage{array}
\usepackage{graphicx}
\usepackage{bm}
\usepackage{tikz}
\usepackage{multicol}
\usepackage{enumitem}

\renewcommand\qedsymbol{$\blacksquare$}

%Here I define some theorem styles and shortcut commands for symbols I use often
\theoremstyle{definition}
\newtheorem{defn}{Definition}
\newtheorem{thm}{Theorem}
\newtheorem{cor}{Corollary}
\newtheorem*{rmk}{Remark}
\newtheorem{lem}{Lemma}
\newtheorem*{joke}{Joke}
\newtheorem{ex}{Example}
\newtheorem*{soln}{Solution}
\newtheorem{prop}{Proposition}

\newcommand{\lra}{\longrightarrow}
\newcommand{\ra}{\rightarrow}
\newcommand{\surj}{\twoheadrightarrow}
\newcommand{\graph}{\mathrm{graph}}
\newcommand{\bb}[1]{\mathbb{#1}}
\newcommand{\Ell}{\mathscr{L}}
\newcommand{\Z}{\bb{Z}}
\newcommand{\Q}{\bb{Q}}
\newcommand{\R}{\bb{R}}
\newcommand{\C}{\bb{C}}
\newcommand{\N}{\bb{N}}
\newcommand{\M}{\mathbf{M}}
\newcommand{\m}{\mathbf{m}}
\newcommand{\MM}{\mathscr{M}}
\newcommand{\HH}{\mathscr{H}}
\newcommand{\Om}{\Omega}
\newcommand{\Ho}{\in\HH(\Om)}
\newcommand{\bd}{\partial}
\newcommand{\del}{\partial}
\newcommand{\bardel}{\overline\partial}
\newcommand{\textdf}[1]{\textbf{\textsf{#1}}\index{#1}}
\newcommand{\img}{\mathrm{img}}
\newcommand{\ip}[2]{\left\langle{#1},{#2}\right\rangle}
\newcommand{\inter}[1]{\mathrm{int}{#1}}
\newcommand{\exter}[1]{\mathrm{ext}{#1}}
\newcommand{\cl}[1]{\mathrm{cl}{#1}}
\newcommand{\ds}{\displaystyle}
\newcommand{\vol}{\mathrm{vol}}
\newcommand{\cnt}{\mathrm{ct}}
\newcommand{\osc}{\mathrm{osc}}
\newcommand{\LL}{\mathbf{L}}
\newcommand{\UU}{\mathbf{U}}
\newcommand{\support}{\mathrm{support}}
\newcommand{\AND}{\;\wedge\;}
\newcommand{\OR}{\;\vee\;}
\newcommand{\Oset}{\varnothing}
\newcommand{\st}{\ni}
\newcommand{\wh}{\widehat}

%Pagination stuff.
\setlength{\topmargin}{-.3 in}
\setlength{\oddsidemargin}{0in}
\setlength{\evensidemargin}{0in}
\setlength{\textheight}{9.in}
\setlength{\textwidth}{6.5in}


\title{Lab 1: Sets, Functions, Relations}
\author{Foundations of Computer Science}
\date{\today}
%\pagestyle{empty} 
%\footer{}{\thepage}{}

\begin{document}

\maketitle

\begin{questions}

\section*{Warm up}
\question
Sam says: ``I'll call you if I'm interested.'' A few days later, he calls you.

Is Sam interested? Assume he's not a liar.
\\~\\
Ans: Uknown.


\question
Rouben says: ``If I call you, I'm interested.'' A few days later, he calls you.

Is Rouben interested? Assume he's not a liar.
\\~\\
Ans: Yes.

\question If it's raining, Steve brings an umbrella.\\
    Today, Steve brought an umbrella.\\
    Is it raining today?
\\~\\
Ans: Uknown.


\question Joe breathes when he eats.\\
    Joe is breathing.\\
    Is Joe eating?
\\~\\
Ans: Uknown.


\question Let $p$ be the statement ``Beau likes fish.''\\
    Let $q$ be the statement ``1 = 0.''\\
    Let $s$ be the statement $p \implies q$.\\
    Assume $s$ is True.\\
    Does Beau like fish?
\\~\\
Ans: No.


\section*{Review of Sets and Functions}
\subsection*{Sets}
Recall $a|b$ for $a,b \in \Z$ if and only if there exists some $k \in \Z$ such that $b = ka$.
\question Let $c, d$ be natural numbers such that $c$ divides $d$.
   Let $A = \{a \in \N| a \text{ divides } c \}$ and let\\ 
   $B = \{b \in \N| b\text{ divides }d \}$. Prove $A \subseteq B$.

\begin{proof}
To show $A \subseteq B$ we will show that $x \in A$ implies $x \in B$.
By definition of division, $x \in A$ implies that there is an integer
$k$ such that $c = kx$. Moreover, we know that $c$ divides $d$ so
there is some integer $m$ such that $d = mc$. Subsituting for
$c$ yields $c = (mk)x$. Since $mk \in \Z$, $x$ divides $d$, so $x \in B$.


\end{proof}

\question
Prove $A \cup (B \cap C) = (A \cup B) \cap (A \cup C)$.

\begin{proof}
To show equality between sets, we show first $A \cup (B \cap C) \subseteq (A
    \cup B) \cap (A \cup C)$, then 
$(A \cup B) \cap (A \cup C) \subseteq A \cup (B \cap C)$:

Beginning with the former, let $x$ be an arbitrary element of 
$A \cup (B \cap C)$. We will show it must also be in $(A \cup B) \cap (A \cup C)$. 
There are two cases: $x \in A$ and $x \notin A$. If $x \in A$, it is in both
 $(A \cup B)$ and $(A \cup C)$, by definition of union. It must also therefore
 be in $(A \cup B) \cap (A \cup C)$, by definition of intersection. If $x \notin A$, 
 it is in $B \cap C$. Therefore $x \in B$ and $x \in C$, so $x \in (A \cup B)$ and 
 $x \in (A \cup C)$, and also their intersection $(A \cup B) \cap (A \cup C)$.
This shows $A \cup (B \cap C) \subseteq (A \cup B) \cap (A \cup C)$.
\\~\\
Next, we show $x \in (A \cup B) \cap (A \cup C)$ implies 
$x \in A \cup (B \cap C)$. Again consider the cases $x \in A$ and $x \notin A$.
If $x \in A$, it is immediate that $x \in A \cup (B \cap C)$, by definition
of union. If $x \notin A$, note that by definition of intersection we have that
$x \in (A \cup B) \cap (A \cup C)$ implies $x \in (A \cup B)$ and $x \in (A \cup C)$,
so it must follow that $x \in C$ and $x \in B$. Therefore, it is in the
intersection of $B$ and $C$ and, consequently, $x \in A \cup (B \cap C)$.

\end{proof}

\subsection*{Functions}
Recall a function $f: A \to B$ is injective if and only if $f(x) = f(y) \implies
x = y$ for all $x,y \in A$.

A function $f: A \to B$ is surjective if and only if for all $y \in B$ there
exists $x \in A$ such that $f(x) = y$. 

\question Consider $f: \N \to \N$, where $f(x) = x^2$.

\begin{enumerate}[label=\alph*)]
\item Is $f$ injective? Give a proof of your answer.
\\
Yes.
\begin{proof}
$f(x) = f(y)$ implies $x^2 = y^2$. Since $x,y \in \N$, they
are positive, so it must be the case that $x = y$.
\end{proof}

\item Is $f$ surjective? Give a proof of your answer.
\\
No.
\begin{proof}
$2$ is a natural number, but there is no natural number $k$ such that
$k^2 = 2$.
\end{proof}

\end{enumerate}


\question Consider $f: \N \to \Z$, where $f(x) = \frac{x}{2}$ if $x$ is even
and $f(x) = -\left (\frac{x+1}{2} \right) $ if $x$ is odd.
\begin{enumerate}[label=\alph*)]
\item Is $f$ injective? Give a proof of your answer.
\\
Yes.
\begin{proof}
We want to show $f(x) = f(y)$ implies $x = y$. There are three cases:
$x$ and $y$ are even, $x$ and $y$ are odd and $x$ is even and $y$ is odd.
\\~\\
If $x$ and $y$ are even, $f(x) = f(y) \implies \frac{x}{2} = \frac{y}{2}
\implies x = y$.
\\~\\
If $x$ and $y$ are odd, 
$f(x) = f(y) \implies - \left (\frac{x + 1}{2} \right ) = - \left (\frac{y + 1}{2} \right
    ) \implies x + 1 = y + 1 \implies x = y$.
\\~\\
If $x$ is even and $y$ is odd, then $f(x) \neq f(y)$, since $f(x)$ will be
positive and $f(y)$ will be negative. Therefore, it is vacously true that
$f(x) = f(y) \implies x = y$.

\end{proof}

\item Is $f$ surjective? Give a proof of your answer.
\\
Yes.
\begin{proof}
We must show that for any integer $z$ there is a natural number $n$ such that
$f(n) = z$. If $z$ is positive or zero, choose $n = 2z$. Clearly, $n$ is an 
even natural number, so $f(n) = \frac{2z}{2} = z$. If $z$ is negative, 
choose $n = -2z - 1$. Since $-2z$ is even and positive, $n$ must be odd.
The smallest even positive number is $2$, so $n \geq 1$. Therefore, $n$
is a natural number and $f(n) = -\frac{-2z -1 + 1}{2} = z$.
\end{proof}

\end{enumerate}

\question
Define the function \emph{floor}: $\R^+ \to \N$ as $floor(x) = n$ if and only if
$n \leq x < n+1$. (Note that $\R^+$ denotes the positive real numbers.)

\begin{enumerate}[label=\alph*)]
\item Is $floor$ injective? Give a proof of your answer. 
\\
No.
\begin{proof}
Choose $x = 1.5$ and $y = 1.75$. $f(x) = f(y)$, but $x \neq y$.
\end{proof}

\item Is $floor$ surjective? Give a proof of your answer.
\\
Yes.
\begin{proof}
Any natural number is also a positive real number, so for any $n$ we have $f(n) = n$. 
\end{proof}

\end{enumerate}

\section*{Relations}

\question Recall that two integers have the same parity if they are both
          even or both odd. Let $P$ be a relation on $\Z$ where $aPb$
          if and only if $a$ and $b$ have the same parity. Prove that
          $P$ is an equivalence relation.

\textbf{Note:} This is a trivial proof; the intent is just to show the form of the
  argument.
\begin{proof}
To show $P$ is an equivalence relation, we must show it is reflexive, symmetric
and transitive.
\\~\\
REFLEXIVE: Any integer $a \in \Z$ has the same parity as itself, so $(a,a) \in
P$.
\\~\\
SYMMETRIC: If $aPb$, $a$ and $b$ have the same parity, so clearly it is also the
case that $bPa$.
\\~\\
TRANSITIVE: If $aPb$ and $bPc$, then $a$ has the same parity as $b$ and $b$ has
the same parity as $c$ so $a$ has the same parity of $c$, and therefore $(a,c)
\in P$. 
\end{proof}

\question
Let $\Sigma = \{a,b\}$
Consider the relation $P$ on $\Sigma^*$ where $xPy$ for $x,y \in \Sigma^*$
if and only if there exists a string $w$ in $\Sigma^*$ such that $y = wx$. 
\begin{enumerate}[label=\alph*)]
\item Assume $w = aba$. Find $3$ pairs $(x_1,y_1),(x_2,y_2),(x_3,y_3)$ that belong
to the relation $P$.\\
$y = aba$, $x = \epsilon$\\
$y = abaa$, $x = a$\\
$y = abab$, $x = b$\\
\item Now assume $w$ can be any string in $\Sigma^*$. Is the pair $(x,y) \in P$,
where $x = bb$ and $y = abbabbabb$? If so, what is $w$? If not, why
not?\\
Yes. $w = abbabba$.

\item Consider $x = abababa$ and $y = aba$. Is the pair $(x,y) \in P$? If so,
what is $w$? If not, why not?\\
No. $|x| > |y|$ so there is no $w$ such that $y = wx$.

\item Characterize the relation in informal English. What does $xPy$ mean about $x$
and $y$?\\
$x$ is a suffix of $y$.

\end{enumerate}
\question Prove the division relation, $a | b$, is a partial order on the set
of positive integers.\\ 
See slides on partial orders for the proof.
%\begin{proof}
%To show the division relation is a partial order on integers, we must show it is 
%reflexive, antisymmetric and transitive.
%\\~\\
%REFLEXIVE: For all $x \in \Z$, $x|x$, so the division relation is reflexive.
%\\~\\
%ANTISYMMETRIC: For all $x, y \in \Z$, $x | y$ if and only if there is some $k
%\in \Z$ such that $y = kx$. Assume it is also the case that $y | x$. Then
%there is some integer $m \in \Z$ such that $x = my$. Substituting, we have
%$y = kmy$, which is only true if $km = 1$. This implies $k = z = 1$.
%Therefore, $x | y$ and $y | x$ only in the case where $x = y$.
%\\~\\
%TRANSITIVE: See proof of question 6. 
%
%\end{proof}
%

\question Prove the subset relation, $A \subseteq B$, is a partial order on
sets. 
\begin{proof}
To show the subset relation is a partial order, we must show it is reflexive, 
antisymmetric and transitive.
\\~\\
REFLEXIVE: For any set $A$, $A \subseteq A$, so the subset relation is
reflexive.
\\~\\
ANTISYMMETRIC: $A \subseteq B$ and $B \subseteq A$ if and only if any element
of $A$ is in $B$ and any element of $B$ is in $A$. That is, if and only if 
$A = B$. 
\\~\\
TRANSITIVE: $A \subseteq B$ and $B \subseteq C$ implies that all elements of $A$
are in $B$ and all elements of $B$ are in $C$. Therefore all elements of $A$ are
in $C$ so $A \subseteq C$.
\end{proof}

\question 
Let $\Sigma = \{0,1,2\}$.  Let $T = \{t \in \Sigma^* | |t| \geq 3\}$.
Now define $Z$ to be a relation on $T$ such that $xZy$ for $x,y \in
T$ if and only if there exist strings $r,s \in \Sigma^*$
and string $m$ in $\Sigma^3$ such that $x = rm$ and $y = sm$. 

\begin{enumerate}[label=\alph*)]
\item Prove $Z$ is an equivalence relation.

\begin{proof}
To show $Z$ is an equivalence relation, we must show it is reflexive, symmetric
and transitive.
\\~\\
REFLEXIVE: $(x,x) \in Z$ if and only if there exist strings $r,s \in \Sigma^*$ and
string $m \in \Sigma^3$ such that $x = rm$ and $x = sm$. This is the case
for any $m$ and any $r$ if we choose $s = r$. Note that this applies to
all $x$ in $T$ because $x \in T$ implies $x$ has at least $3$ characters. (If
$x$ had fewer than $3$ characters there would be no choice of $s,m$ that would
satisfy $x = sm$.)
\\~\\
SYMMETRIC: $(x,y) \in Z$ if and only if there exist strings $r,s \in \Sigma^*$ and
string $m \in \Sigma^3$ such that $x = rm$ and $y = sm$. However, this is
the same criteria required to show $(y,x) \in Z$ (since the choice of variable
names $r,s$ is arbitrary), so $bZa$.
\\~\\
TRANSITIVE: We want to show that $xZy$ and $yZz$ implies $xZz$. $(x,y) \in Z$
if and only if there exists $r,s \in \Sigma^*$ and $m \in \Sigma^3$ such 
that $x = rm$ and $y = sm$.  Also, $(y,z) \in Z$ if and only if there 
exists $r',s' \in \Sigma^*$ 
and $m' \in \Sigma^3$ such that $y = r'm'$ and $z = s'm'$. If $y = rm$
and $y = s'm'$ where $m$ and $m'$ are strings of length three, then
it must be the case that $m = m'$. Therefore $r, s'$ are strings in
$\Sigma^*$ such that $x = rm$ and $z = s'm$, which proves $(x,z) \in Z$.
\end{proof}


\item How many equivalences classes are there with respect to $Z$ on $T$?
\\
27
\\~\\
($Z$ is the equivalence class of strings of length $3$ or more that end in
the same last three characters. There are $3$ characters in the alphabet,
so the total number of distinct strings of length $3$ is $3^3 = 27$.
\end{enumerate}

%\newpage

\end{questions}
\end{document}


