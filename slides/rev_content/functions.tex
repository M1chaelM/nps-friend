\begin{slide}[bm=,toc=]{Definition}
\begin{defn}{A.23}[Ben Ari]
~\\
Let $\mathcal{F}$ be a relation on $S_1 \times \cdots \times S_n$.
\\~\\
\pause
$\mathcal{F}$ is a \emph{function} if and only if for every \emph{(n -- 1)-tuple}
$(x_1,...,x_{n-1}) \in S_1 \times \cdots \times S_{n-1}$, there is at most
one $x_n \in S_n$, such that $\mathcal{F}(x_1,...,x_n)$.
\begin{itemize}
\item<3-> For a function $\mathcal{F}$ we write $x_n = \mathcal{F}(x_1,...,x_{n-1})$.
\item<4-> The \emph{domain} of $\mathcal{F}$ is the set of all
$(x_1,...,x_{n-1})\in S_1 \times \cdots \times S_{n-1}$ for which (exactly one)
$x_n = \mathcal{F}(x_1,...,x_{n-1})$ exists.
\item<5->The \emph{range} of $\mathcal{F}$ is the set of all $x_n \in S_n$ such
that $x_n = \mathcal{F}(x_1,...,x_{n-1})$ for at least one $(x_1,...,x_{n-1})$. 
\end{itemize}
\end{defn}
\end{slide}

\begin{slide}[bm=,toc=]{Examples of Functions}
\textbf{As a Special Relation:}
\begin{itemize}
\item<2-> With explicit set notation: \[f_1 = \{(1,3),(2,1),(3,2)\}\]
\vspace{-5mm}
\item<3-> With set comprehension: \[f_2 = \{(n_1,n_2)|n_2 = n_1^2\}\]
\end{itemize}
\pause[3]
\textbf{Functional Notation:}
\vspace{-3mm}
\[
  f_3(x) = x^2 + 1
  \]
\vspace{-5mm}
\pause
\textbf{Arrow Notation:}
\vspace{-3mm}
\begin{align*}
f_4:\N &\longrightarrow \R \\
    x &\mapsto \sqrt{x}
\end{align*}

\end{slide}

\begin{slide}[bm=,toc=]{Total and Partial Functions}
\begin{defn}{A.23}[Ben Ari]~\\
\pause
\vspace{2mm}
$\mathcal{F}$ is \emph{total} if the domain of $\mathcal{F}$ is (all of)
  $S_1 \times \cdots \times S_{n-1}$;\\
\pause
\vspace{2mm}
Otherwise, $\mathcal{F}$ is \emph{partial}.
\end{defn}

\begin{figure}[b]
\subcaptionbox*{Total Function}[.4\linewidth]{
\pause
\begin{tikzpicture}[scale=.75, every node/.style={scale=.75}]
\tikzset{vertex/.style = {shape=circle,draw,minimum size=1.5em}}
\tikzset{edge/.style = {->,> = latex'}}

\draw (0, 2) ellipse (0.75cm and 2.7cm);

\node[vertex](a5) at (0,0) {\tiny $3,1$};
\node[vertex](a4) at (0,1) {\tiny $2,2$};
\node[vertex](a3) at (0,2) {\tiny $2,1$};
\node[vertex](a2) at (0,3) {\tiny $1,2$};
\node[vertex](a1) at (0,4) {\tiny $1,1$};

\draw (3, 2) ellipse (0.75cm and 2.7cm);

\node[vertex](b6) at (3,0)   {\tiny  $6$};
\node[vertex](b5) at (3,0.8) {\tiny  $5$};
\node[vertex](b4) at (3,1.6) {\tiny  $4$};
\node[vertex](b3) at (3,2.4) {\tiny  $3$};
\node[vertex](b2) at (3,3.2) {\tiny  $2$};
\node[vertex](b1) at (3,4) {\tiny    $1$};

\draw[edge] (a1) to (b2);
\draw[edge] (a2) to (b3);
\draw[edge] (a3) to (b3);
\draw[edge] (a4) to (b4);
\draw[edge] (a5) to (b4);

\end{tikzpicture}
}
\qquad
\subcaptionbox*{Partial Function}[.4\linewidth]{
\pause
\begin{tikzpicture}[scale=.75, every node/.style={scale=.75}]
\tikzset{vertex/.style = {shape=circle,draw,minimum size=1.5em}}
\tikzset{edge/.style = {->,> = latex'}}
\draw (0, 2) ellipse (0.75cm and 2.7cm);

\node[vertex](a5) at (0,0) {\tiny $3,1$};
\node[vertex](a4) at (0,1) {\tiny $2,2$};
\node[vertex](a3) at (0,2) {\tiny $2,1$};
\node[vertex](a2) at (0,3) {\tiny $1,2$};
\node[vertex](a1) at (0,4) {\tiny $1,1$};

\draw (3, 2) ellipse (0.75cm and 2.7cm);

\node[vertex](b3) at (3,0)   {\tiny  $3$};
\node[vertex](b2) at (3,2) {\tiny  $2$};
\node[vertex](b1) at (3,4) {\tiny  $1$};

\draw[edge] (a2) to (b1);
\draw[edge] (a3) to (b2);
\draw[edge] (a4) to (b2);
\draw[edge] (a5) to (b3);

\end{tikzpicture}
}

\end{figure}

\end{slide}

\begin{slide}[bm=,toc=]{Injective / one-to-one Functions}
\begin{defn}{A.23}[Ben Ari]~\\
\vspace{2mm}
$\mathcal{F}$ is \emph{injective} or \emph{one-to-one} if and only if:
\begin{itemize}
\item<2-> $(x_1,...,x_{n-1}) \neq (y_1,...,y_{n-1})$
\end{itemize}
\pause[2]
implies
\begin{itemize}
\item<4-> $\mathcal{F}(x_1,...,x_{n-1}) \neq \mathcal{F}(y_1,...,y_{n-1})$
\end{itemize}
\end{defn}

\vspace{-6mm}
\begin{figure}[b]
\subcaptionbox*{Injective}[.4\linewidth]{
\pause[2]
\begin{tikzpicture}[scale=.75, every node/.style={scale=.75}]
\tikzset{vertex/.style = {shape=circle,draw,minimum size=1.5em}}
\tikzset{edge/.style = {->,> = latex'}}

\draw (0, 2) ellipse (0.75cm and 2.7cm);

\node[vertex](a5) at (0,0) {\tiny $3,1$};
\node[vertex](a4) at (0,1) {\tiny $2,2$};
\node[vertex](a3) at (0,2) {\tiny $2,1$};
\node[vertex](a2) at (0,3) {\tiny $1,2$};
\node[vertex](a1) at (0,4) {\tiny $1,1$};

\draw (3, 2) ellipse (0.75cm and 2.7cm);

\node[vertex](b6) at (3,0)   {\tiny  $6$};
\node[vertex](b5) at (3,0.8) {\tiny  $5$};
\node[vertex](b4) at (3,1.6) {\tiny  $4$};
\node[vertex](b3) at (3,2.4) {\tiny  $3$};
\node[vertex](b2) at (3,3.2) {\tiny  $2$};
\node[vertex](b1) at (3,4) {\tiny    $1$};

\draw[edge] (a1) to (b1);
\draw[edge] (a2) to (b3);
\draw[edge] (a3) to (b5);
\draw[edge] (a5) to (b2);
\end{tikzpicture}
}
\qquad
\subcaptionbox*{Not Injective}[.4\linewidth]{
\pause
\begin{tikzpicture}[scale=.75, every node/.style={scale=.75}]
\tikzset{vertex/.style = {shape=circle,draw,minimum size=1.5em}}
\tikzset{edge/.style = {->,> = latex'}}
\draw (0, 2) ellipse (0.75cm and 2.7cm);

\node[vertex](a5) at (0,0) {\tiny $3,1$};
\node[vertex](a4) at (0,1) {\tiny $2,2$};
\node[vertex](a3) at (0,2) {\tiny $2,1$};
\node[vertex](a2) at (0,3) {\tiny $1,2$};
\node[vertex](a1) at (0,4) {\tiny $1,1$};

\draw (3, 2) ellipse (0.75cm and 2.7cm);

\node[vertex](b3) at (3,0)   {\tiny  $3$};
\node[vertex](b2) at (3,2) {\tiny  $2$};
\node[vertex](b1) at (3,4) {\tiny  $1$};

\draw[edge] (a1) to (b1);
\draw[edge] (a2) to (b1);
\draw[edge] (a3) to (b2);
\draw[edge] (a4) to (b2);
\draw[edge] (a5) to (b3);

\end{tikzpicture}
}

\end{figure}


\end{slide}

\begin{slide}[bm=,toc=]{Surjective / onto Functions}
\begin{defn}{A.23}[Ben Ari]~\\
\vspace{2mm}
$\mathcal{F}$ is \emph{surjective} or \emph{onto} if and only if its range is all of $S_n$.\\
\vspace{2mm}
\pause
Equivalently, $\mathcal{F}$ from $S_1$ to $S_2$ is \emph{surjective} if 
and only if for all $y \in S_2$ there is an $x \in S_1$ such that
$\mathcal{F}(x) = y$. 
\end{defn}
\vspace{-5mm}
\begin{figure}[b]
\subcaptionbox*{Surjective}[.4\linewidth]{
\pause
\begin{tikzpicture}[scale=.75, every node/.style={scale=.75}]
\tikzset{vertex/.style = {shape=circle,draw,minimum size=1.5em}}
\tikzset{edge/.style = {->,> = latex'}}
\draw (0, 2) ellipse (0.75cm and 2.7cm);

\node[vertex](a5) at (0,0) {\tiny $3,1$};
\node[vertex](a4) at (0,1) {\tiny $2,2$};
\node[vertex](a3) at (0,2) {\tiny $2,1$};
\node[vertex](a2) at (0,3) {\tiny $1,2$};
\node[vertex](a1) at (0,4) {\tiny $1,1$};

\draw (3, 2) ellipse (0.75cm and 2.7cm);

\node[vertex](b3) at (3,0)   {\tiny  $3$};
\node[vertex](b2) at (3,2) {\tiny  $2$};
\node[vertex](b1) at (3,4) {\tiny  $1$};

\draw[edge] (a1) to (b1);
\draw[edge] (a2) to (b1);
\draw[edge] (a3) to (b2);
\draw[edge] (a4) to (b2);
\draw[edge] (a5) to (b3);

\end{tikzpicture}
}
\qquad
\subcaptionbox*{Not Surjective}[.4\linewidth]{
\pause
\begin{tikzpicture}[scale=.75, every node/.style={scale=.75}]
\tikzset{vertex/.style = {shape=circle,draw,minimum size=1.5em}}
\tikzset{edge/.style = {->,> = latex'}}

\draw (0, 2) ellipse (0.75cm and 2.7cm);

\node[vertex](a5) at (0,0) {\tiny $3,1$};
\node[vertex](a4) at (0,1) {\tiny $2,2$};
\node[vertex](a3) at (0,2) {\tiny $2,1$};
\node[vertex](a2) at (0,3) {\tiny $1,2$};
\node[vertex](a1) at (0,4) {\tiny $1,1$};

\draw (3, 2) ellipse (0.75cm and 2.7cm);

\node[vertex](b6) at (3,0)   {\tiny  $6$};
\node[vertex](b5) at (3,0.8) {\tiny  $5$};
\node[vertex](b4) at (3,1.6) {\tiny  $4$};
\node[vertex](b3) at (3,2.4) {\tiny  $3$};
\node[vertex](b2) at (3,3.2) {\tiny  $2$};
\node[vertex](b1) at (3,4) {\tiny    $1$};

\draw[edge] (a1) to (b1);
\draw[edge] (a2) to (b2);
\draw[edge] (a3) to (b4);
\draw[edge] (a4) to (b5);
\draw[edge] (a5) to (b6);

\end{tikzpicture}
}

\end{figure}

\end{slide}

\begin{slide}[bm=,toc=]{Bijective Functions}
\begin{defn}{A.23}[Ben Ari]~\\
$\mathcal{F}$ is \emph{bijective} or \emph{one-to-one and onto} if and only if
it is injective and surjective.
\end{defn}

\begin{figure}[b]
\pause
\begin{tikzpicture}[scale=.75, every node/.style={scale=.75}]
\tikzset{vertex/.style = {shape=circle,draw,minimum size=1.5em}}
\tikzset{edge/.style = {->,> = latex'}}

\draw (0, 2) ellipse (0.75cm and 2.7cm);

\node[vertex](a5) at (0,0) {\tiny $3,1$};
\node[vertex](a4) at (0,1) {\tiny $2,2$};
\node[vertex](a3) at (0,2) {\tiny $2,1$};
\node[vertex](a2) at (0,3) {\tiny $1,2$};
\node[vertex](a1) at (0,4) {\tiny $1,1$};

\draw (3, 2) ellipse (0.75cm and 2.7cm);

\node[vertex](b5) at (3,0) {\tiny  $5$};
\node[vertex](b4) at (3,1) {\tiny  $4$};
\node[vertex](b3) at (3,2) {\tiny  $3$};
\node[vertex](b2) at (3,3) {\tiny  $2$};
\node[vertex](b1) at (3,4) {\tiny  $1$};

\draw[edge] (a1) to (b2);
\draw[edge] (a2) to (b3);
\draw[edge] (a3) to (b4);
\draw[edge] (a4) to (b5);
\draw[edge] (a5) to (b1);

\end{tikzpicture}
\caption*{Bijective Function}
\end{figure}

\end{slide}

\begin{slide}[bm=,toc=]{Composition of Functions}
As with relations, we can combine functions using the \emph{composition} operation, 
denoted $f \circ g$.
\begin{itemize}
\item<2-> Assume $g$ is a function from $S_1$ to $S_2$ and $f$ is a function from 
$S_3$ to $S_4$ and let $S = S_2 \cap S_3$.
\item<3->The composition forms a new function on $S_1 \times S_4$, such that:
\[
  f \circ g(x) = f(g(x))
\]
\vspace{-5mm}
\item<4-> \textbf{Note:} since functions are relations, the above follows immediately from
the definition of composition of relations.
\end{itemize}
\vspace{-5mm}
\begin{figure}[b]
\pause[4]
\begin{subfigure}{.3\linewidth}
\begin{tikzpicture}[scale=.5, every node/.style={scale=.5}]
\tikzset{vertex/.style = {shape=circle,draw,minimum size=1.5em}}
\tikzset{edge/.style = {->,> = latex'}}

\draw (0, 1.2) ellipse (.75cm and 2cm);
\node[vertex](c) at (0,0.4) {\tiny $1,1$};
\node[vertex](b) at (0,1.2) {\tiny $1,2$};
\node[vertex](a) at (0,2)   {\tiny $2,1$};
\node (s1) at (0, 2.9) {$S_1$};

\node (g) at (1, -0.2) {$g$};

\draw (2, 1.2) ellipse (.75cm and 2cm);
\node[vertex](0) at (2,0) {$0$};
\node[vertex](1) at (2,0.8) {$1$};
\node[vertex](2) at (2,1.6) {$2$};
\node[vertex](3) at (2,2.4) {$3$};
\node (s) at (2, 3) {$S$};

\node (f) at (3, -0.2) {$f$};

\draw (4, 1.2) ellipse (.75cm and 2cm);
\node[vertex](b1) at (4,0.4) {$1$};
\node[vertex](b2) at (4,1.2) {$2$};
\node[vertex](b3) at (4,2)   {$3$};
\node (s4) at (4, 2.9) {$S_4$};

\draw[edge] (a) to (3);
\draw[edge] (b) to (3);
\draw[edge] (c) to (2);

\draw[edge] (0) to (b1);
\draw[edge] (1) to (b2);
\draw[edge] (2) to (b3);
\draw[edge] (3) to (b1);

\end{tikzpicture}

\end{subfigure}
\pause
\begin{subfigure}{.3\linewidth}
\begin{align*}
  f(x)       &= x \pmod{3} + 1 \\
  g(x_1,x_2) &= x_1 + x_2 \\
  f \circ g(x_1,x_2) &= f(g(x_1,x_2)) \\
                     &= (x_1 + x_2) \pmod{3} + 1
\end{align*}
\end{subfigure}
\end{figure}
\end{slide}
