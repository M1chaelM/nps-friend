\section[slide=true,tocsection=false]{Sets}

\begin{slide}[bm=,toc=]{Finite and infinite Sets}

\emph{\textbf{Definition: A set is a collection of elements.}}
\pause
\begin{itemize}
   \item<2-> $a \in S$ means $a$ is an element of set $S$ 
   \item<3-> $a \notin S$ means $a$ is not an element of set $S$
   \item<4-> The \emph{cardinality} of a set $S$ is denoted $|S|$. 
   \begin{itemize}
       \item<5-> Cardinality indicates the number of elements in the set.
   \end{itemize}
   \item<6-> $\emptyset$ represents the set with no elements (``empty set'').
   \begin{itemize}
       \item<7-> $\emptyset$ has cardinality 0. 
   \end{itemize}
\end{itemize} 
\end{slide}

\begin{slide}[bm=,toc=]{Specifying Sets}
\textbf{Three ways of defining a set:}
\begin{itemize}
   \item<2-> Explict (write out the elements).
   \begin{itemize}
      \item<3-> $S = \{red, yellow, green\}$ 
      \item<4-> $R = \{1,2,3\}$
      \item<5-> This does not work for infinite sets.
   \end{itemize} 
   \item<6-> Through \emph{set comprehension}. 
   \begin{itemize}
      \item<7-> $\N = \{n| n \in \Z, n \geq 0\}$ 
      \item<8-> $E = \{n| n \in \N, n \mod{2} = 0 \}$
   \end{itemize}
   \item<9-> Through operations on sets that have already be defined. 
   \begin{itemize}
      \item<10-> $K = S \cup R = \{red, yellow, green, 1, 2, 3\}$ 
      \item<11-> See ``Operations on Sets and Strings'' for examples. 
   \end{itemize}
\end{itemize} 

\end{slide}


\begin{slide}[bm=,toc=]{Applications of sets}
Sets are the basis for:
\begin{itemize}
   \item<2-> Relations
   \item<3-> Functions
   \item<4-> Equivalence classes
   \item<5-> Partial orders
\end{itemize}
\pause[5]Important infinite sets: 
\begin{itemize}
   \item<7-> $\N$: Natural Numbers
   \item<8-> $\Z$: Integers
   \item<9-> $\Q$: Rational Numbers
   \item<10-> $\R$: Real Numbers
   \item<11-> $\C$: Complex Numbers
\end{itemize}
\end{slide}

\section[slide=true,tocsection=false]{Sequences}

\begin{slide}[bm=,toc=]{Finite and Infinite Sequences}
\pause
\begin{defn}{A.12}[Ben Ari]~\\\pause
Let $\mathcal{S}$ be a set.
\begin{itemize}
    \item<4-> A \emph{finite sequence} $f$ is a function from $\{0,...,n-1\}$ to $\mathcal{S}$.
    \begin{itemize}
        \item<5-> The length of the sequence is $n$.
    \end{itemize}
    \item<6-> An \emph{infinite sequence} $f$ on $\mathcal{S}$ is a function from $\N$ to $\mathcal{S}$.
    \end{itemize}
\end{defn}
\pause[4]
\begin{defn}{A.14}[Ben Ari]~\\\pause
Let $f$ be a sequence on $\mathcal{S}$. The sequence is denoted:
\pause
\[
  (s_0,s_1,s_2,...)
  \]
\pause
where $s_i = f(i)$.
\end{defn}
~\\
\pause
\begin{defn}{A.15}[Ben Ari]
~\\
\pause
A finite sequence of length $n$ is an \emph{n-tuple}.
\end{defn}

\end{slide}


\section[slide=true,tocsection=false]{Strings}
\begin{slide}[bm=,toc=]{Strings: Definitions}
\pause
\begin{defn}{--- Alphabet}~\\
\pause Any finite set. 
\end{defn}
\vspace{-5mm}
\begin{itemize}
\item<4-> For example $\{0,1,2,...,9\}$ and $\{0,1\}$ are alphabets. 
\item<5-> The set $\{a,b\}$ is an alphabet. 
\item<6->  $\Sigma$ denotes an arbitrary alphabet. 
\item<7->  We call the elements of $\Sigma$ \emph{letters} or \emph{symbols}.
\end{itemize}
\pause[5]
\begin{defn}{--- String}~\\
\pause
A \emph{string} over $\Sigma$ is any finite-length sequence of elements of
$\Sigma$. 
\end{defn}
\vspace{-5mm}

\begin{itemize}
\item<10-> For example, if $\Sigma = \{a,b\}$ then $aabab$ is a string of length $5$ over $\Sigma$.
\item<11-> The length of a string $x$ is denoted $|x|$. 
\item<12-> There is a unique string of length zero over any alphabet $\Sigma$ called
      the \emph{empty string} and it is denoted $\epsilon$.
\end{itemize}
\end{slide}

\begin{slide}[bm=,toc=]{$\emptyset \neq \{\epsilon\} \neq \epsilon$}
    Note that $\emptyset$, $\{\epsilon\}$ and $\epsilon$ are three different things.
    \begin{itemize}
    \item<2-> $\emptyset$ is a set with no elements.
    \item<3-> $\{\epsilon\}$ is a set with one element, namely the empty string.
    \item<4-> $\epsilon$ is a string, not a set.
    \end{itemize}
\end{slide}

