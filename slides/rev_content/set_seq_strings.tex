\section[slide=true,tocsection=false]{Sets}

\begin{slide}[bm=,toc=]{Finite and infinite Sets}

\emph{\textbf{Definition: A set is a collection of elements.}}
\pause
\begin{itemize}
   \item<2-> $a \in S$ means $a$ is an element of set $S$ 
   \item<3-> $a \notin S$ means $a$ is not an element of set $S$
   \item<4-> The \emph{cardinality} of set $A$ is denoted $|A|$. 
   \begin{itemize}
       \item<5-> Cardinality indicates the number of elements in the set.
   \end{itemize}
   \item<6-> $\emptyset$ represents the set with no elements (``empty set'').
   \begin{itemize}
       \item<7-> $\emptyset$ has cardinality 0. 
   \end{itemize}
\end{itemize} 
\end{slide}

\begin{slide}[bm=,toc=]{Specifying Sets}
\textbf{Three ways of defining a set:}
\begin{itemize}
   \item<2-> Explict (write out the elements).
   \begin{itemize}
      \item<3-> $S = \{red, yellow, green\}$ 
      \item<4-> $R = \{1,2,3\}$
      \item<5-> This does not work for infinite sets.
   \end{itemize} 
   \item<6-> Through \emph{set comprehension}. 
   \begin{itemize}
      \item<7-> $\N = \{n| n \in \Z, n \geq 0\}$ 
      \item<8-> $E = \{n| n \in \N, n \mod{2} = 0 \}$
   \end{itemize}
   \item<9-> Through operations on sets that have already be defined. 
   \begin{itemize}
      \item<10-> $K = S \cup R = \{red, yellow, green, 1, 2, 3\}$ 
      \item<11-> See ``Operations on Sets'' for examples. 
   \end{itemize}
\end{itemize} 

\end{slide}


\begin{slide}[bm=,toc=]{Applications of sets}
Sets are the basis for:
\begin{itemize}
   \item<2-> Relations
   \item<3-> Functions
   \item<4-> Equivalence classes
   \item<5-> Partial orders
\end{itemize}
\pause[5]Important infinite sets: 
\begin{itemize}
   \item<7-> $\N$: Natural Numbers
   \item<8-> $\Z$: Integers
   \item<9-> $\Q$: Rational Numbers
   \item<10-> $\R$: Real Numbers
   \item<11-> $\C$: Complex Numbers
\end{itemize}
\end{slide}

\section[slide=true,tocsection=false]{Sequences}

\begin{slide}[bm=,toc=]{Finite and Infinite Sequences}
\pause
\begin{defn}{A.12}[Ben Ari]~\\\pause
Let $\mathcal{S}$ be a set.
\begin{itemize}
    \item<4-> A \emph{finite sequence} $f$ is a function from $\{0,...,n-1\}$ to $\mathcal{S}$.
    \begin{itemize}
        \item<5-> The length of the sequence is $n$.
    \end{itemize}
    \item<6-> An \emph{infinite sequence} $f$ on $\mathcal{S}$ is a function from $\N$ to $\mathcal{S}$.
    \end{itemize}
\end{defn}
\pause[4]
\begin{defn}{A.14}[Ben Ari]~\\\pause
Let $f$ be a sequence on $\mathcal{S}$. The sequence is denoted:
\pause
\[
  (s_0,s_1,s_2,...)
  \]
\pause
where $s_i = f(i)$.
\end{defn}
~\\
\pause
\begin{defn}{A.15}[Ben Ari]
~\\
\pause
A finite sequence of length $n$ is an \emph{n-tuple}.
\end{defn}

\end{slide}


\section[slide=true,tocsection=false]{Strings}
\begin{slide}[bm=,toc=]{Strings: Definitions}
\pause
\begin{defn}{--- Alphabet}~\\
\pause Any finite set. 
\end{defn}
\vspace{-5mm}
\begin{itemize}
\item<4-> For example $\{0,1,2,...,9\}$ and $\{0,1\}$ are alphabets. 
\item<5->  $\Sigma$ denotes an arbitrary alphabet. 
\item<6->  We call the elements of $\Sigma$ \emph{letters} or \emph{symbols}.
\end{itemize}
\pause[4]
\begin{defn}{--- String}~\\
\pause
A \emph{string} over $\Sigma$ is any finite-length sequence of elements of
$\Sigma$. 
\end{defn}
\vspace{-5mm}

\begin{itemize}
\item<9-> For example, if $\Sigma = \{a,b\}$ then $aabab$ is a string of length $5$ over $\Sigma$.
\item<10-> The length of a string $x$ is denoted $|x|$. 
\item<11-> There is a unique string of length zero over any alphabet $\Sigma$ called
      the \emph{empty string} and it is denoted $\epsilon$.
\end{itemize}
\end{slide}

\begin{slide}[bm=,toc=]{$\emptyset \neq \{\epsilon\} \neq \epsilon$}
    Note that $\emptyset$, $\{\epsilon\}$ and $\epsilon$ are three different things.
    \begin{itemize}
    \item<2-> $\emptyset$ is a set with no elements.
    \item<3-> $\{\epsilon\}$ is a set with one element, namely the empty string.
    \item<4-> $\epsilon$ is a string, not a set.
    \end{itemize}
\end{slide}

\section[slide=true]{Operations on Sets and Strings}

\begin{slide}[bm=,toc=]{Set Inclusion}
\begin{defn}{A.3}[Ben Ari]
Let $S$ and $T$ be sets.
\begin{itemize}
\item $S$ is a \emph{subset} of $T$ if and only if every element of $S$ is an
      element of $T$.
\item Equivalently, $S \subseteq T$ if and only if $x \in S$ implies $x \in T$.
\item $S$ is a \emph{proper subset} of $T$, (i.e.\ $S \subset T$) if and only
      if $S \subseteq T$ and $S \neq T$.
\end{itemize}
\end{defn}
\begin{thm}{A.5}[Ben Ari]
$\emptyset \subseteq T$
\end{thm}
\begin{thm}{A.6}[Ben Ari]
\emph{The subset property is transitive.}
\end{thm}
\begin{thm}{A.7}[Ben Ari]
$S = T$ if and only if $S \subseteq T$ and $T \subseteq S$.
\end{thm}



\end{slide}

\begin{slide}[bm=,toc=]{Basic operations on sets}
\begin{itemize}
   \item \emph{Set union}: 
   \[
     A \cup B = \{x|x \in A \text{ or } x \in B\}
   \]

   \item \emph{Set intersection}: 
   \[
     A \cap B = \{x|x \in A \text{ and } x \in B\}
   \]

   \item \emph{Set difference}: 
   \[
     A - B = \{x|x \in A \text{ and } x \notin B\}
   \]

\end{itemize}
\end{slide}

\begin{wideslide}[bm=,toc=]{Illustration of Set Operations}
\vspace*{15mm}

\unitlength=1.0pt
\begin{center}
\begin{picture}(260,110)
\put(80,40){\oval(160,80)}
\put(180,40){\oval(160,80)}
\put(  0,0){\makebox(100,80){$S-T$}}
\put(100,0){\makebox( 60,80){$S\cap T$}}
\put(160,0){\makebox(100,80){$T-S$}}
\put(0,60){\makebox(20,20)[r]{S}}
\put(240,60){\makebox(20,20)[l]{T}}
\put(20,85){\makebox(220,10){$\overbrace{\hspace*{220pt}}$}}
\put(100,100){\makebox( 60,10){$S\cup T$}}
\end{picture}
\end{center}
\end{wideslide}


\begin{slide}[bm=,toc=]{Cartesian Products}
\begin{defn}{A.17}[Ben Ari]
~\\
\emph{Cartesian Product of 2 Sets:}
\begin{itemize}
\item Let $S$ and $T$ be sets.
\item We define their \emph{Cartesian Product}, $S \times T$, as the set of all
pairs $(s,t)$ such that $s \in S$ and $t \in T$.
\end{itemize}
\emph{n-ary Cartesian Product:}
\begin{itemize}
\item This is the general case.
\item Let $S_1,...,S_n$ be sets.
\item We define their \emph{Cartesian Product}, $S_1 \times \cdots \times S_n$, as the set of all
$\emph{n-tuples}$ $(s_1,...,s_n)$ such that $s_i \in S_i$.
\item If all sets $S_i$ are the same set, the notation $S^n$ is used for $S
\times \cdots \times S$.
\end{itemize}
\end{defn}
\end{slide}

\begin{slide}[bm=,toc=]{String Concatenation}

\begin{itemize}
    \item If $x$ and $y$ are strings then string $xy$ is called the
    \emph{concatenation} of $x$ and $y$.
    \item Concatenation is associative: $(xy)z = x(yz)$. 
    \item Concatenation is not commutative. $xy$ and $yx$ are different strings
    in general.
    \item $\epsilon$ is an identity for concatenation: ${\epsilon}x = x\epsilon = x$. 
    \item $|xy| = |x| + |y|$. 
    \item We write $a^n$ for a string of $a$'s of length $n$.
    \begin{itemize}
        \item For example, $a^5 = aaaaa$ and $a^0 = \epsilon$. 
        \item And $a^ma^n = a^{m+n}$ for all $m,n \geq 0.$
    \end{itemize} 
\end{itemize} 
\end{slide}

\begin{slide}[bm=,toc=]{Set Concatenation}
\emph{\textbf{Definition:}}
\begin{itemize}
   \item  $\mli{AB} = \{xy | x \in A \text{ and } y \in B\}$. 
   \item When forming a set concatenation, you form \emph{all} strings that can be obtained in this way.
   \item Example:
   \[\{\textcolor{red}{a},\textcolor{brown}{ab}\}\{\textcolor{green}{b},\textcolor{blue}{ba}\} = \
                      \{\textcolor{red}{a}\textcolor{green}{b}, \
                        \textcolor{red}{a}\textcolor{blue}{ba}, \
                        \textcolor{brown}{ab}\textcolor{green}{b}, \
                        \textcolor{brown}{ab}\textcolor{blue}{ba}\}
   \]
\end{itemize} 
   Note that $\mli{AB}$ and $\mli{BA}$ are different sets in general.
\end{slide}


\begin{slide}[bm=,toc=]{Powers of sets of strings}
\begin{itemize}
\item \emph{Powers} of a set $A$ (i.e.\ $A^n$) are defined inductively:
\[
  A^0 = \{\epsilon\}
\]
\[
  A^{n+1} = AA^n
\]

\item $A^n$ is formed by concatenating $n$ copies of $A$ together. 
\item Taking $A^0 = \{\epsilon\}$ by definition makes $A^{m+n} = A^mA^n$ true, 
      even when one of $m$ or $n$ is zero. 
\item As a special case, if $A$ is an \emph{alphabet} then $A^n$ is the set of all strings
over $A$ of length $n$.
\item \textbf{Note:} 
\begin{itemize}
    \item The notation for the $n^{th}$ power of a set of strings is identical to
    the notation for the \emph{n-ary} Cartesian power.
    \item Need to determine the meaning from context.
\end{itemize}
\end{itemize}
\end{slide}

\begin{slide}[bm=,toc=]{Powers of sets}
Let $A = \{ab,aab\}$. Then
\[
\begin{array}{lll}
A^0 &= \{\epsilon\} & \\[2ex]

A^1 &= AA^0         &= \{ab,aab\}\{\epsilon\} \\
    &               &= \{ab,aab\} \\[2ex]

A^2 &= AA^1         &= \{ab,aab\}\{ab,aab\}   \\
    &               &= \{abab,abaab,aabab,aabaab\} \\[2ex]

A^3 &= AA^2         &= \{ab,aab\}\{abab,abaab,aabab,aabaab\}  \\
    &               &= \{ababab,ababaab,abaabab,abaabaab,  \\
    &               &\;\;\;\;\;\;aababab,aababaab,aabaabab,aabaabaab \}  \\
\end{array}
\]
%\\~\\
\end{slide}


\begin{slide}[bm=,toc=]{Kleene closure}
\emph{Kleene closure} of a set $A$, denoted $A^*$, is the infinite union of all
finite powers of $A$:
\[
A^* = \bigcup_{n \geq 0} A^n = A^0 \cup A^1 \cup A^2 \cup A^3 \cup \cdots
\]
As a special case, if $A$ is an alphabet then $A^*$ is the set of all strings
over $A$ of any length, including zero length.
\end{slide}

\begin{slide}[bm=,toc=]{$\bar{A}$ and $A^+$}
\begin{itemize}
   \item The \emph{complement} of $A$ with respect to $\Sigma^*$ is defined 
         $\bar{A} = \{x \in \Sigma^* | x \notin A\}$. The complement of $A$
         depends on $\Sigma^*$, hence $\bar{A}$ is sometimes denoted 
         $\Sigma^* - A$ to emphasize this dependence.
   \item $A^+$ is the infinite union of all nonzero powers of $A$:
         \[
           A^+ = AA^* = \bigcup_{n>0} A^n
           \]
\end{itemize}
\end{slide}




\begin{slide}[bm=,toc=]{Properties of Set Operations}
\begin{itemize}
   \item Set union, set intersection and set concatenation are associative: 
   \[
     \begin{split}
     (A\cup B) \cup C = A \cup (B \cup C) \\
     (A\cap B) \cap C = A \cap (B \cap C) \\
     (\mli{AB})C = A(\mli{BC}) \\
     \end{split}
     \]
   \item Set union and set intersection are commutative
   \[
     \begin{split}
     A\cup B = B \cup A \\
     A\cap B = B \cap A \\
     \end{split}
    \]
    \item Set concatenation is not commutative.
\end{itemize}
\end{slide}

\begin{slide}[bm=,toc=]{Identity for Union and set Concatenation}
\begin{itemize}
   \item The empty set $\emptyset$ is the identity for $\bigcup$:
   \[
     A\cup \emptyset = \emptyset \cup A = A
     \]
   \item The set $\{\epsilon\}$ is an identity for set concatenation:
       \[
         \{\epsilon\}A = A\{\epsilon\} = A
       \]
   \item $A\emptyset = {\emptyset}A = \emptyset$
\end{itemize}
\end{slide}

\begin{slide}[bm=,toc=]{Distributive Properties}
\begin{itemize}
   \item  Set union and intersection distribute over each other:
   $A \cup (B \cap C) = (A\cup B)\cap(A \cup C)$\\
   $A \cap (B \cup C) = (A\cap B)\cup(A \cap C)$

   \item Set concatenation distributes over union. 
    $A(B\cup C) = \mli{AB} \cup \mli{AC}$\\
    $(A\cup B)C = \mli{AC} \cup \mli{BC}$
    \item Set concatenation does \emph{not} distribute over intersection. For
    example, let $A = \{a,ab\}$, $B = \{b\}$, $C = \{\epsilon\}$. Then
       $A(B\cap C) = A\emptyset = \emptyset$ \\
       $\mli{AB} \cap \mli{AC} = \{ab,abb\} \cap \{a,ab\} = \{ab\}$
\end{itemize}
\end{slide}

\begin{slide}[bm=,toc=]{Other useful identities}
De Morgan Laws:
\[
\overline{A \cup B} = \bar{A} \cap \bar{B}  
\]
\[
\overline{A \cap B} = \bar{A} \cup \bar{B}  
\]

Properties of Kleene closure:
\begin{itemize}
\item $A^*A^* = A^*$
\item $(A^*)^* = A^*$
\item $A^* = \{\epsilon\} \cup AA^* = \{\epsilon\}\cup A^*A $
\item $\emptyset^* = \{\epsilon\}$ 
\item $\{\epsilon\}^* = \{\epsilon\}$ 
\item $AA^* = A^*A$ 
\end{itemize}
\end{slide}
