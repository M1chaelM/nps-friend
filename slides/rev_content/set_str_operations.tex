\begin{slide}[bm=,toc=]{Set Inclusion}
\begin{defn}{A.3}[Ben Ari]~\\
\pause 
Let $S$ and $T$ be sets.
\begin{itemize}
\item<3-> $S$ is a \emph{subset} of $T$ if and only if every element of $S$ is an
      element of $T$.
\item<4-> Equivalently, $S \subseteq T$ if and only if $x \in S$ implies $x \in T$.
\item<5-> $S$ is a \emph{proper subset} of $T$, (i.e.\ $S \subset T$) if and only
      if $S \subseteq T$ and $S \neq T$.
\end{itemize}
\end{defn}
\pause[4]
\begin{thm}{A.5}[Ben Ari]~\\
$\emptyset \subseteq T$
\end{thm}
\vspace{1.5mm}
\pause
\begin{thm}{A.6}[Ben Ari]~\\
\emph{The subset property is transitive.}
\end{thm}
\vspace{1.5mm}
\pause
\begin{thm}{A.7}[Ben Ari]~\\
$S = T$ if and only if $S \subseteq T$ and $T \subseteq S$.
\end{thm}



\end{slide}

\begin{slide}[bm=,toc=]{Basic operations on sets}
\begin{itemize}
   \item<2-> \emph{Set union}: 
   \pause[2]
   \[
     A \cup B = \{x|x \in A \text{ or } x \in B\}
   \]

   \item<4-> \emph{Set intersection}: 
   \pause[2]
   \[
     A \cap B = \{x|x \in A \text{ and } x \in B\}
   \]

   \item<6-> \emph{Set difference}: 
   \pause[2]
   \[
     A - B = \{x|x \in A \text{ and } x \notin B\}
   \]

\end{itemize}
\end{slide}

\begin{wideslide}[bm=,toc=]{Illustration of Set Operations}
\vspace*{15mm}

\unitlength=1.0pt
\begin{center}
\begin{picture}(260,110)
\put(80,40){\oval(160,80)}
\put(180,40){\oval(160,80)}
\put(  0,0){\makebox(100,80){$S-T$}}
\put(100,0){\makebox( 60,80){$S\cap T$}}
\put(160,0){\makebox(100,80){$T-S$}}
\put(0,60){\makebox(20,20)[r]{S}}
\put(240,60){\makebox(20,20)[l]{T}}
\put(20,85){\makebox(220,10){$\overbrace{\hspace*{220pt}}$}}
\put(100,100){\makebox( 60,10){$S\cup T$}}
\end{picture}
\end{center}
\end{wideslide}


\begin{slide}[bm=,toc=]{Cartesian Products}
\begin{defn}{A.17}[Ben Ari]
~\\
\emph{Cartesian Product of 2 Sets:}
\begin{itemize}
\item<2-> Let $S$ and $T$ be sets.
\item<3-> We define their \emph{Cartesian Product}, $S \times T$, as the set of all
pairs $(s,t)$ such that $s \in S$ and $t \in T$.
\end{itemize}
\pause[3]
\emph{n-ary Cartesian Product:}
\begin{itemize}
\item<5-> This is the general case.
\item<6-> Let $S_1,...,S_n$ be sets.
\item<7-> We define their \emph{Cartesian Product}, $S_1 \times \cdots \times S_n$, as the set of all
$\emph{n-tuples}$ $(s_1,...,s_n)$ such that $s_i \in S_i$.
\item<8-> If all sets $S_i$ are the same set, the notation $S^n$ is used for $S
\times \cdots \times S$.
\end{itemize}
\end{defn}
\end{slide}

\begin{slide}[bm=,toc=]{String Concatenation}
Informally, concatentation is a basic operation that joins strings together.

\begin{itemize}
    \item<2-> Formally, if $x$ and $y$ are strings then string $xy$ is called the
          \emph{concatenation} of $x$ and $y$.
    \item<3-> Concatenation is associative: $(xy)z = x(yz)$. 
    \item<4-> Concatenation is not commutative. $xy$ and $yx$ are different strings
    in general.
    \item<5-> $\epsilon$ is an identity for concatenation: ${\epsilon}x = x\epsilon = x$. 
    \item<6-> $|xy| = |x| + |y|$. 
    \item<7-> We write $a^n$ for a string of $a$'s of length $n$.
    \begin{itemize}
        \item<8-> For example, $a^5 = aaaaa$ and $a^0 = \epsilon$. 
        \item<9-> And $a^ma^n = a^{m+n}$ for all $m,n \geq 0.$
    \end{itemize} 
\end{itemize} 
\end{slide}

\begin{slide}[bm=,toc=]{Set Concatenation}
\begin{defn}{--- Concatenation of Sets of Strings}~\\
\pause If $A$ and $B$ are sets and all elements of $A$ and $B$ are strings, we define their concatenation, $AB$, as follows: 
\pause
\[
\mli{AB} = \{xy | x \in A \text{ and } y \in B\}
\] 
\end{defn}
\pause
\textbf{Notes:}
\begin{itemize}
   \item<4-> When forming a set concatenation, you form \emph{all} strings that can be obtained in this way.
   \item<5-> Example:
   \[\{\textcolor{red}{a},\textcolor{brown}{ab}\}\{\textcolor{green}{b},\textcolor{blue}{ba}\} = \
                      \{\textcolor{red}{a}\textcolor{green}{b}, \
                        \textcolor{red}{a}\textcolor{blue}{ba}, \
                        \textcolor{brown}{ab}\textcolor{green}{b}, \
                        \textcolor{brown}{ab}\textcolor{blue}{ba}\}
   \]
    \vspace{-5mm}
    \item<6-> Set concatenation is not commutative. That is, $\mli{AB}$ and $\mli{BA}$ are different sets in general.
\end{itemize} 
\end{slide}


\begin{slide}[bm=,toc=]{Powers of sets of strings}
\begin{defn}{--- Powers of Sets of Strings}~\\
\pause
\emph{Powers} of a set $A$ (i.e.\ $A^n$) are defined inductively:
\pause
\[
  A^0 = \{\epsilon\}
\]
\pause
\[
  A^{n+1} = AA^n
\]
\end{defn}
\vspace{-5mm}
\pause
\textbf{Notes:}
\begin{itemize}
\item<5-> $A^n$ is formed by concatenating $n$ copies of $A$ together. 
\item<6-> Taking $A^0 = \{\epsilon\}$ by definition makes $A^{m+n} = A^mA^n$ true, 
      even when one of $m$ or $n$ is zero. 
\item<7-> As a special case, if $A$ is an \emph{alphabet} then $A^n$ is the set of all strings
over $A$ of length $n$.
\item<8-> \textbf{Notation:} The notation for the $n^{th}$ power of a set of strings is identical to
    the notation for the \emph{n-ary} Cartesian power.
\begin{itemize}
\item<9-> Need to determine the meaning from context.
\end{itemize}
\end{itemize}
\end{slide}

\begin{slide}[bm=,toc=]{Powers of sets}
Let $A = \{ab,aab\}$. Then
\pause
\[
\begin{array}{lll}
A^0 &= \{\epsilon\} & \\[2ex]
\pause

A^1 &= AA^0         &= \{ab,aab\}\{\epsilon\} \\
    &               &= \{ab,aab\} \\[2ex]

\pause
A^2 &= AA^1         &= \{ab,aab\}\{ab,aab\}   \\
    &               &= \{abab,abaab,aabab,aabaab\} \\[2ex]

\pause
A^3 &= AA^2         &= \{ab,aab\}\{abab,abaab,aabab,aabaab\}  \\
    &               &= \{ababab,ababaab,abaabab,abaabaab,  \\
    &               &\;\;\;\;\;\;aababab,aababaab,aabaabab,aabaabaab \}  \\
\end{array}
\]
%\\~\\
\end{slide}

\begin{slide}[bm=,toc=]{Kleene closure}
\begin{defn}{--- Kleene Closure}~\\
\pause
The \emph{Kleene closure} of a set $A$, denoted $A^*$, is the infinite union of all
finite powers of $A$:
\pause
\[
A^* = \bigcup_{n \geq 0} A^n = A^0 \cup A^1 \cup A^2 \cup A^3 \cup \cdots
\]
\end{defn}
\pause
As a special case, if $A$ is an alphabet then $A^*$ is the set of all strings
over $A$ of any length, including zero length.
\end{slide}

\begin{slide}[bm=,toc=]{$\bar{A}$ and $A^+$}
\begin{defn}{--- The complement $\bar{A}$}~\\
\pause
Let $A$ be a set of strings formed over the alphabet $\Sigma$. \\~\\
\pause
We define the \emph{complement} of $A$ with respect to $\Sigma^*$, $\bar{A}$, as 
\pause
\[
\bar{A} = \{x \in \Sigma^* | x \notin A\}
\]
\end{defn}
\pause
\begin{itemize}
\item The complement of $A$ depends on $\Sigma^*$, hence $\bar{A}$ is sometimes denoted $\Sigma^* - A$ to emphasize this dependence.
\end{itemize}
\pause
\begin{defn}{--- Kleene plus: $A^+$}~\\
\pause
$A^+$ is the infinite union of all nonzero powers of $A$:
\pause
\[
A^+ = AA^* = \bigcup_{n>0} A^n
\]
\end{defn}

\end{slide}

\begin{slide}[bm=,toc=]{Properties of Set Operations}
Set union, set intersection and set concatenation are associative: 
\pause
\begin{align*}
     (A\cup B) \cup C &= A \cup (B \cup C) \\
     (A\cap B) \cap C &= A \cap (B \cap C) \\
          (\mli{AB})C &= A(\mli{BC})
\end{align*}
\pause
Set union and set intersection are commutative
\pause
   \[
     \begin{split}
     A\cup B = B \cup A \\
     A\cap B = B \cap A \\
     \end{split}
    \]
\pause
Set concatenation is not commutative.
\end{slide}

\begin{slide}[bm=,toc=]{Identity for Union and set Concatenation}
The empty set $\emptyset$ is the identity for $\bigcup$:
\pause
\[
A\cup \emptyset = \emptyset \cup A = A
\]
\pause
The set $\{\epsilon\}$ is an identity for set concatenation:
\pause
\[
\{\epsilon\}A = A\{\epsilon\} = A
\]
\pause
\textbf{Note:} \\
The concatenation of the empty set with any set is the empty set:
\pause
\[
A\emptyset = {\emptyset}A = \emptyset
\]
\end{slide}

\begin{slide}[bm=,toc=]{Distributive Properties}
Set union and intersection distribute over each other:
\pause
\begin{align*}
   A \cup (B \cap C) &= (A\cup B)\cap(A \cup C)\\
   A \cap (B \cup C) &= (A\cap B)\cup(A \cap C)
\end{align*}

\pause
Set concatenation distributes over union. 
\pause
\begin{align*}
    A(B\cup C) &= \mli{AB} \cup \mli{AC}\\
    (A\cup B)C &= \mli{AC} \cup \mli{BC}
\end{align*}
\pause
Set concatenation does \emph{not} distribute over intersection. \\

\pause
\textbf{Example:}~\\
Let $A = \{a,ab\}$, $B = \{b\}$, $C = \{\epsilon\}$. Then
\vspace{-3mm}
\pause
\begin{align*}
       A(B\cap C)             &= A\emptyset               = \emptyset \\
       \mli{AB} \cap \mli{AC} &= \{ab,abb\} \cap \{a,ab\} = \{ab\}
\end{align*}
\end{slide}

\begin{slide}[bm=,toc=]{Other useful identities}
De Morgan Laws:
\pause
\[
\overline{A \cup B} = \bar{A} \cap \bar{B}  
\]
\[
\overline{A \cap B} = \bar{A} \cup \bar{B}  
\]

\pause
Properties of Kleene closure:
\begin{itemize}
\item<4-> $A^*A^* = A^*$
\item<5-> $(A^*)^* = A^*$
\item<6-> $A^* = \{\epsilon\} \cup AA^* = \{\epsilon\}\cup A^*A $
\item<7-> $\emptyset^* = \{\epsilon\}$ 
\item<8-> $\{\epsilon\}^* = \{\epsilon\}$ 
\item<9-> $AA^* = A^*A$ 
\end{itemize}
\end{slide}
