
\documentclass[style=sailor,size=12pt]{powerdot}
\usepackage{epic,array,ecltree,url}
\usepackage[nointegrals]{wasysym}
\usepackage{mathtools}

\newcommand{\id}[1]{\mbox{\it #1\/}}
\newcommand{\rid}[1]{\mbox{\rm #1}}
\newcommand{\sid}[1]{\mbox{\sf #1}}
\newcommand{\bid}[1]{\mbox{\bf #1}}
\newcommand{\tinysz}[1]{\mbox{\tiny $#1$}}

\newcommand{\lra}{\longrightarrow}
\newcommand{\ra}{\rightarrow}
\newcommand{\surj}{\twoheadrightarrow}
\newcommand{\graph}{\mathrm{graph}}
\newcommand{\bb}[1]{\mathbb{#1}}
\newcommand{\Ell}{\mathscr{L}}
\newcommand{\Z}{\bb{Z}}
\newcommand{\Q}{\bb{Q}}
\newcommand{\R}{\bb{R}}
\newcommand{\C}{\bb{C}}
\newcommand{\N}{\bb{N}}
\newcommand{\M}{\mathbf{M}}
\newcommand{\m}{\mathbf{m}}
\newcommand{\MM}{\mathscr{M}}
\newcommand{\HH}{\mathscr{H}}
\newcommand{\Om}{\Omega}
\newcommand{\Ho}{\in\HH(\Om)}
\newcommand{\bd}{\partial}
\newcommand{\del}{\partial}
\newcommand{\bardel}{\overline\partial}
\newcommand{\textdf}[1]{\textbf{\textsf{#1}}\index{#1}}
\newcommand{\img}{\mathrm{img}}
\newcommand{\ip}[2]{\left\langle{#1},{#2}\right\rangle}
\newcommand{\inter}[1]{\mathrm{int}{#1}}
\newcommand{\exter}[1]{\mathrm{ext}{#1}}
\newcommand{\cl}[1]{\mathrm{cl}{#1}}
\newcommand{\ds}{\displaystyle}
\newcommand{\vol}{\mathrm{vol}}
\newcommand{\cnt}{\mathrm{ct}}
\newcommand{\osc}{\mathrm{osc}}
\newcommand{\LL}{\mathbf{L}}
\newcommand{\UU}{\mathbf{U}}
\newcommand{\support}{\mathrm{support}}
\newcommand{\AND}{\;\wedge\;}
\newcommand{\OR}{\;\vee\;}
\newcommand{\Oset}{\varnothing}
\newcommand{\st}{\ni}
\newcommand{\wh}{\widehat}
\newcommand{\mli}[1]{\mathit{#1}}
\newcommand{\ndiv}{\hspace{-3pt}\not|\hspace{2pt}}

\pdsetup{method=normal,
list={labelsep=1em,leftmargin=1cm,itemsep=0pt,topsep=5pt,parsep=0pt}
}
% import from truth and PTL.
\title{Discrete Math Review}
\author{Foundations of Computer Science}
\date{\today}


\begin{document}
\maketitle
\section[slide=false]{Overview}
\begin{slide}[bm=,toc=]{This course builds on concepts from MA2025.}
In particular, we assume familiarity with:
\begin{enumerate}
\item Finite and infinite sets
\item Set operators
\item Sequences
\item Relations
\item Functions
\item Cardinality
\item Basic strategies for proving properties of sets
\end{enumerate} 
A working knowledge of these topics is useful for formulating precise descriptions
of the systems and problems that interest us.
\\~\\
(This review is based on the Appendix provided in the Ben Ari.)
\end{slide}

\section[slide=false]{Finite and infinite sets}
\begin{slide}[bm=,toc=]{Finite and infinite Sets}

\emph{\textbf{Definition: A set is a collection of elements.}}
\begin{itemize}
\item $a \in S$ means $a$ is an element of set $S$ 
\item $a \notin S$ means $a$ is not an element of set $S$
\item $\emptyset$ represents the set with no elements (``empty set'').
\end{itemize} 

\textbf{Two ways of defining a set:}
\begin{itemize}
\item Explict (write out the elements).
   \begin{itemize}
   \item $S = \{red, yellow, green\}$ 
   \item $R = \{1,2,3\}$
   \item This does not work for infinite sets.
   \end{itemize} 
\item Through \emph{set comprehension}. 
   \begin{itemize}
   \item $\N = \{n| n \in \Z, n \geq 0\}$ 
   \item $E = \{n| n \in \N, n \mod{2} = 0 \}$
   \end{itemize}
\end{itemize} 

Important infinite sets: $\N$, $\Z$, $\Q$, $\R$.

\end{slide}


\begin{wideslide}[bm=,toc=]{Applications of sets}
Sets are the basis for:
\begin{itemize}
\item Relations
\item Functions
\item Equivalence classes
\item Partial orders
\end{itemize}
We will describe these in more detail.

In addition, sets of signs form alphabets, languages.



\begin{itemize}
\item Different computational models have been developed independently:
%\item Machine independent: recursive function theory.
%\item Machine dependent:
\begin{enumerate}
\item Turing machines, Alan Turing (covered in CS3101)
\item Lambda calculus, Alonzo Church
\item Post systems, Emil Post
\item and others.
\end{enumerate}
\item Post systems are of interest to us.
\item They can be used to define something (data structure, program).
\item Definitions may be recursive 

\hspace{2em}$A \equiv \ldots A\ldots$

or mutually recursive

\hspace{2em}$A \equiv \ldots B\ldots$

\hspace{2em}$B \equiv \ldots A\ldots$

\item Post systems admit formal proofs---is a definition correct?
\end{itemize}
\end{wideslide}

\begin{wideslide}[bm=,toc=]{Recursive definitions}
\begin{itemize}
\item A recursive definition can be represented in a {\em Post system\/}.
\item Named after mathematician Emil Leon Post.
\item A Post system comprises a finite set of {\em variables\/}, {\em signs\/} and {\em productions\/} 
(inference rules).
\item An inference rule has the form
\begin{displaymath}
\begin{array}{c}
t_1\;t_2\;\cdots\;t_n\\
\hline
t
\end{array}
\end{displaymath}
where $t,t_1,\ldots ,t_n$ ($n\geq 0$) are terms (strings of signs and variables).
\item $t_1,\ldots , t_n$ are the {\em antecedents\/} and $t$ the {\em consequent\/} or conclusion.
\item When $n=0$, an inference rule is called an {\em axiom\/}.
\end{itemize}
\end{wideslide}

\begin{wideslide}[bm=,toc=]{Recursive definitions}
\begin{itemize}
\item A Post system of two productions:
\vspace{-1em}
\begin{tabbing}
{\bf R2}XX \=  \kill
{\bf B} \>
        \(\begin{array}[t]{l}
        3\in S
        \end{array}\) \\[2ex]
{\bf R} \>
        \(\begin{array}[t]{l}
        x \in S \;\;\;y \in S \\
        \hline
        x + y \in S
        \end{array}\)
\end{tabbing}
\item The set of signs is $\{3,\in,+\}$ and variables $\{x,y\}$.
\item Claim: system defines the set of all positive multiples of 3.
\item Must prove $n\in S$ iff $n$ is a positive multiple of 3.
\item ``$n\in S$ only if $n$ is a positive multiple of 3'' (system {\em soundness\/}).
\item ``$n\in S$ if $n$ is a positive multiple of 3'' (system {\em completeness\/}).
\end{itemize}
\end{wideslide}

\begin{wideslide}[bm=,toc=]{Recursive definitions}
\begin{itemize}
\item A Post system recursively defining the set {\em P\/} of all well-formed formulae in propositional logic:
\begin{displaymath}
\begin{array}{lll}
        \begin{array}[t]{l}
        \bid{T}\in P
        \end{array}
&
        \begin{array}[t]{l}
        \bid{F}\in P
        \end{array}
&
	\begin{array}[t]{l}
        x \in P \\
        \hline
        \neg x \in P
        \end{array} \\[6ex]

	\begin{array}[t]{l}
	x \in P \;\;y \in P \\
	\hline
	x \wedge y \in P
	\end{array}
&
	\begin{array}[t]{l}
	x \in P \;\;y \in P \\
	\hline
	x \vee y \in P
	\end{array}
&
	\begin{array}[t]{l}
	x \in P \;\;y \in P \\
	\hline
	x \Rightarrow y \in P
	\end{array} \\[6ex]

	\begin{array}[t]{l}
        x \in \id{Var} \\
        \hline
        x \in P
        \end{array}
\end{array}
\end{displaymath}
%\item Consequents have larger terms than antecedents except in last rule.
\end{itemize}
\end{wideslide}

\begin{wideslide}[bm=,toc=]{Recursive definitions}
\begin{itemize}
\item A mutually-recursive definition of {\em rooted trees\/} {\em RT\/}:
\vspace{-1em}
\begin{tabbing}
{\bf R2}XX \=  \kill
{\bf B} \>
        \(\begin{array}[t]{l}
        \id{nil}\in\id{RTL}
        \end{array}\) \\[2ex]
{\bf R1} \>
        \(\begin{array}[t]{l}
        x\in\id{RT}\;\;\;y\in\id{RTL} \\
        \hline
        \id{cons}(x,y)\in\id{RTL}
        \end{array}\) \\[2ex]
{\bf R2} \>
        \(\begin{array}[t]{l}
        x\in\id{RTL} \\
        \hline
        \id{node}(x)\in\id{RT}
        \end{array}\)
\end{tabbing}
\item Compare with sloppy definition in Rosen 7th Ed., pg.\ 351.
\item It ignores mutual recursion and hence mutual induction.
\item It is not induction friendly.
\end{itemize}
\end{wideslide}

\begin{wideslide}[bm=,toc=]{Beyond recursive definitions}
\begin{itemize}
\item Post systems are as powerful as any programming language.
\item Productions can define any computation.
\item In practice they are used as {\em deductive systems\/}.
\item Gentzen and Hilbert systems are systems for deducing validity in propositional logic.
\item For instance, here's a rule of inference for {\em modus ponens\/}:
\vspace{-1em}
\begin{tabbing}
{\bf MP}XX \=  \kill
{\bf MP} \>
        \(\begin{array}[t]{l}
        x\Rightarrow y \in\id{Thm}\;\;\;x\in\id{Thm} \\
        \hline
        y\in\id{Thm}
        \end{array}\)
\end{tabbing}
\item Notice term ``$x\Rightarrow y$'' is larger than ``$y$'' in the consequent.
\item So structural induction is not possible here.
\end{itemize}
\end{wideslide}

\end{document}
