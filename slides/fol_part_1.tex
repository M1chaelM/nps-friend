\documentclass[style=sailor,size=12pt]{powerdot}
\usepackage{epic,array,ecltree,url,calrsfs}
\usepackage[nointegrals]{wasysym}
\usepackage{listings}
\usepackage{epsfig}
\usepackage{amsmath}
\usepackage{amsfonts}
\usepackage{amssymb}
\usepackage{amsxtra}
\usepackage{amsthm}
\usepackage{mlextra} % Must be below ams packages
\usepackage{mathrsfs}
\usepackage{color}
\usepackage{array}
\usepackage{graphicx}
\graphicspath{ {../art/} }
\usepackage{bm}
\usepackage{tikz}
\usepackage{multicol}
\usepackage{enumitem}

\pdsetup{method=normal}

\title{First Order Logic: Part I}
\author{Foundations of Computer Science}
\date{\today}

\begin{document}
\maketitle
\begin{slide}[toc=,bm=]{Overview}
This slide set covers the following topics in first order logic:

\vspace{5mm}
\tableofcontents[content=sections]
\end{slide}

\section[slide=true]{Motivations and Applications}
\begin{wideslide}[bm=,toc=]{Limitations of Propositional Logic}
Previously, we used propositional logic to simplify the conditional statement:
\begin{program}
if (a < b || (a >= b \&\& c == d)) ...
\end{program}
\pause 
to the equivalent statement
\vspace{-1em}
\begin{program}
if (a < b || c == d) ...
\end{program}
\pause
However, there are other equivalences that propositional logic cannot justify.
\end{wideslide}

\begin{wideslide}[bm=,toc=]{First Order Logic Equivalence Example}
Consider the expression:
\begin{program}
if ((a < b) and (a < c) and (b < c))...
\end{program}
\pause
which can be simplified to
\vspace{-1em}
\begin{program}
if ((a < b) and (b < c))...
\end{program}
\vspace{-1em}
\pause
In propositional logic, this equivalence would be written as:\\~\\
$p \land q \land r \equiv p \land r$\\~\\
\pause
which is not true for the interpretation $v_{\mathcal{I}}(p) = T$,
   $v_{\mathcal{I}}(q) = F$, $v_{\mathcal{I}}(r) = T$
\end{wideslide}

\begin{wideslide}[bm=,toc=]{First Order Logic and Predicates}
First-order logic allows propositions to contain ``\emph{arguments}.''
\begin{itemize}
\item<2-> Called \emph{predicates}
\item<3->Interpretations are no longer restricted to assignments of $T$ or $F$.
\item<4->Predicates can express set-theoretic relations.
\end{itemize}
\pause[4]
\begin{ex}{}[Ullman]
Define the relation $\id{lt}$ such that $\id{lt}(x,y)$ iff $x < y$ and 
let $p = \id{lt}(a,b)$, $q = \id{lt}(a,c)$, and $r = \id{lt}(b,c)$.
\\~\\
\pause
We can now rewrite the equivalent statements as:\\
$(\id{lt}(a,b) \land \id{lt}(a,c) \land \id{lt}(b,c)) \equiv (\id{lt}(a,b) \land \id{lt}(b,c))$
\\~\\
\pause
First order logic allows us to express that $\id{lt}$ is transitive, and further that the above
equivalence holds for \emph{all} predicates that express transitive relations.
\end{ex}
\end{wideslide}

\begin{wideslide}[bm=,toc=]{FOL Application to Access Control Policies}
First order logic can also be used to express access control policies.
\\~\\
\pause
\textbf{Motivation:} 
\begin{itemize}
\item Informal policy statements often invite ambiguity.\footnote{ \footnotesize
J. Halpern and V. Weissman, 
Using First-Order Logic to Reason about Policies, {\em ACM Trans on Information and System Security\/},
(11)4, July 2008.}
\item Difficult to ensure consistency automatically. \pause
\begin{itemize}
\item (How difficult? How do we measure the difficulty?) 
\end{itemize}
\end{itemize}
\pause
~\\
\textbf{Example:} 
\begin{itemize}
\item ``Only librarians may edit the online catalog.''
\begin{itemize}
\item<5-> Forbids anyone who is not a librarian from editing the catalog.
\item<6-> But can a librarian edit the catalog?
\end{itemize}
\end{itemize}
\end{wideslide}

\begin{wideslide}[bm=,toc=]{Policies in first-order logic}
\textbf{Approaches:}
\begin{itemize}
\item<2-> Structured policy languages
\begin{itemize}
\item<3-> Extensible rights Markup Language, Open Digital Rights Language
\item<4-> Permit more formal statements but their semantics are usually in English.
\end{itemize}
\item<2-> First-order logic 
\begin{itemize}
\item<5-> Has a precise semantics.
\item<6-> Encourages precise policy specification. 
\end{itemize}
\end{itemize}
\pause[6]
\textbf{Consider:}
\vspace{-2mm}
\begin{displaymath}
\begin{array}{l}
\forall x(\neg\bid{Librarian}(x))\imp \neg\bid{Permitted}(x,\rid{edit catalog}) \\[1ex]
\forall x(\bid{Librarian}(x))\imp \bid{Permitted}(x,\rid{edit catalog})
\end{array}
\end{displaymath}
\vspace{3mm}
\pause
First formula alone may convey the meaning of the English statement.
\end{wideslide}

\begin{wideslide}[bm=,toc=]{Halpern and Weissman's Access Control Proposal}
\textbf{General form of a policy:}
\begin{displaymath}
\begin{array}{l}
\forall x_1\ldots x_m(f \imp (\neg)\bid{Permitted}(t,t'))
\end{array}
\end{displaymath}
\vspace{-15mm}
\begin{itemize}
\item[] \phantom{empty bullet}
\begin{itemize}
\item $f$ is a first-order formula
\item $t$ denotes a subject and 
\item $t'$ an action performed by that subject.
\end{itemize}
\end{itemize}
\vspace{3mm}
\pause
\textbf{Example:}\\
\vspace{2mm}
``An article may be downloaded if a fee has been paid within the past 6 weeks.''
\vspace{-2mm}
\begin{displaymath}
\begin{array}{l}
\forall i\, \forall t\, \forall a\, ((\bid{PaidFee}(i,t)\wedge (\bid{now} - 6 < t < \bid{now})\wedge \\
\bid{Customer}(i,\bid{now})\wedge\bid{Article}(a) \Rightarrow \bid{Permitted}(i,\rid{download}(a)))
\end{array}
\end{displaymath}
$\bid{now}$ is a constant denoting current time (provided by a global clock).
\end{wideslide}

\begin{wideslide}[bm=,toc=]{Halpern and Weissman's Access Control Proposal}
\textbf{Expressing Permission:}
\begin{itemize}
\item<2-> Let $E$ be an {\em environment\/} of statements (e.g.\ $\bid{Librarian}(\id{Alice})$).
\item<3-> Let $p_1,\ldots ,p_n$ be policies governing subjects.
\item<4-> Subject $t$ can perform action $t'$ iff the following is {\em valid\/}:
\pause[4]
\begin{displaymath}
E\wedge p_1 \wedge \cdots \wedge p_n \imp \bid{Permitted}(t,t')
\end{displaymath}
\vspace{-5mm}
\item<6->  that is, $E\wedge p_1 \wedge \cdots \wedge p_n \wedge \neg\bid{Permitted}(t,t')$ 
is {\em unsatisfiable\/}.
\end{itemize}
\end{wideslide}

\begin{wideslide}[bm=,toc=]{Halpern and Weissman's Access Control Proposal}
\textbf{Checking Consistency:}
\begin{itemize}
\item Policies $p_1,\dots p_n$ and environment $E$ are {\em consistent\/} iff
the following is {\em satisfiable\/}:
\begin{displaymath}
E\wedge p_1 \wedge \cdots \wedge p_n 
\end{displaymath}
\end{itemize}
\pause
\textbf{Difficulty of Checking Consistency:}
\begin{itemize}
\item<3-> As we shall see, neither validity or satisfiability is decidable in first-order logic!
\item<4-> Halpern and Weissman give restrictions on $E$ and $p_1,\ldots ,p_n$ under which
these problems are:
\begin{itemize}
\item<5->decidable,
\item<5->tractable
\item<5->sufficiently expressive to remain useful. 
\end{itemize}
\end{itemize}
\end{wideslide}

\section[slide=true]{Syntax}
\begin{slide}[bm=,toc=]{First Order Logic: Syntax}
\begin{defn}{7.6}[Ben Ari]
Let $\mathcal{P}$, $\mathcal{A}$ and $\mathcal{V}$ be countable sets of
\emph{predicate symbols}, \emph{constant symbols} and \emph{variables}.
\pause
\begin{itemize}
\item Each predicate symbol $p^n \in \mathcal{P}$ is associated with an \emph{arity}.
\begin{itemize}
\item Refers to the number $n \geq 1$ of \emph{arguments} that it takes. 
\end{itemize}
\item $p^n$ is called an $n$-ary predicate. 
\item For $n = 1,2$, the terms \emph{unary} and \emph{binary},
respectively, are also used.
\end{itemize}
\end{defn}
\pause
\begin{defn}{7.7}[Ben Ari]
~\\
$\forall$ is the \emph{universal quantifier} and is read \emph{for all}.\\
$\exists$ is the \emph{existential quantifier} and is read \emph{there exists}.
\end{defn}

\end{slide}

\begin{slide}[bm=,toc=]{First Order Logic: Notational Conventions}
\textbf{Writing Symbols:}
\begin{itemize}
\item<2-> By convention, we associate the following lower-case letters with each
set:
\begin{itemize}
\item $\mathcal{P} = \{p,q,r\}$, for predicate symbols
\item $\mathcal{A} = \{a,b,c\}$, for constant symbols
\item $\mathcal{V} = \{x,y,z\}$, for variables
\end{itemize}
\item<3-> The superscript denoting the arity of the predicate is omitted when clear (can
    be inferred from the number of arguments).
\begin{itemize}
\item<4-> \textbf{Example:} \pause[4] $p(x,y,z)$ instead of $p^3(x,y,z)$
\end{itemize}
\end{itemize}
\end{slide}

\begin{wideslide}[bm=,toc=]{Atomic Formulas in FOL}
\begin{defn}{7.8}[Ben Ari]
~\\~\\
\textbf{ An \emph{atomic formula}} is
\begin{itemize}
\item<2-> an $n$-ary predicate 
\item<3-> followed by a list of $n$ arguments in parentheses:, 
\begin{itemize}
\item<4-> $p(t_1,t_2,...,t_n)$
\end{itemize}
\item<5-> where each argument $t_i$ is either a variable or a constant. 
\end{itemize}
\pause[5]
\textbf{Examples:}
\begin{itemize}
\item $lt(3,4)$
\item $p(user, edit)$
\end{itemize}

\end{defn}
\end{wideslide}


\begin{wideslide}[bm=,toc=]{Formulas in FOL}
\begin{defn}{7.8}[Ben Ari] (Continued)\\~\\
\textbf{ A \emph{formula}} in first-order logic is
a tree defined recursively as follows:
\end{defn}
\vspace{-3ex}
\begin{itemize}
\item<2-> A formula is a leaf labeled by an atomic formula.
\item<3-> A formula is a node labeled by $\ngg$ with a single child that is a
formula.
\item<4-> A formula is a node labeled by $\forall x$ or $\exists x$ (for some
 variable $x$) with a single child that is a formula.
\item<5-> A formula is a node labeled by a binary Boolean operator with two
children both of which are formulas.
\end{itemize}
\pause[5]
{\bf Universal vs Existential}:\\
A formula of the form $\forall x A$ is a \emph{universal formula}.\\
A formula of the form $\exists x A$ is an \emph{existential formula}.\\
\end{wideslide}

\begin{wideslide}[bm=,toc=]{Figure 7.1: Tree for
$\forall x(\ngg\exists yp(x,y)\vee\ngg\exists yp(y,x))$}
\begin{multicols}{2}
\begin{itemize}
\item Formula tree structure follows PL.
\item Similar concept of induction on structure.
\item In the string representation, $\exists$ and $\forall$ have the same precedence as $\ngg$.
\end{itemize}
\begin{center}
\setlength{\GapWidth}{8mm}
\setlength{\GapDepth}{8mm}
\begin{bundle}{$\forall x$\rule[-1mm]{0mm}{1mm}}
\chunk{
  \begin{bundle}{$\vee$\rule[-1mm]{0mm}{1mm}}
  \chunk{
    \begin{bundle}{$\ngg$\rule[-1mm]{0mm}{1mm}}
    \chunk{
      \begin{bundle}{$\exists y$\rule[-1mm]{0mm}{1mm}}
      \chunk{$p(x,y)$}
      \end{bundle}
    }
    \end{bundle}
  }
  \chunk{
    \begin{bundle}{$\ngg$\rule[-1mm]{0mm}{1mm}}
    \chunk{
      \begin{bundle}{$\exists y$\rule[-1mm]{0mm}{1mm}}
      \chunk{$p(y,x)$}
      \end{bundle}
    }
    \end{bundle}
  }
  \end{bundle}
}
\end{bundle}
\end{center}
\end{multicols}
\end{wideslide}

\begin{wideslide}[bm=,toc=]{Scope of Variables}
\begin{defn}{7.11}[Ben Ari]~\\
\textbf{Quantification and Scope:}
\begin{itemize}
\item<2-> A universal or existential formula $\forall x A$ or $\exists x A$
is a \emph{quantified formula}. 
\item<3-> $x$ is the \emph{quantified variable}
and its \emph{scope} is the formula $A$. 
\item<4-> It is not required that $x$ actually appear in the scope of its 
quantification.
\end{itemize}
\end{defn}
\pause[4]
\textbf{Comments:}
\begin{itemize}
\item<6-> Conceptually similar to scope of variables in programming languages.
\item<7-> Technically, local declaration hides global declarations.
\item<8-> Best practice is to avoid name conflicts.
\end{itemize}
\end{wideslide}

\begin{wideslide}[bm=,toc=]{Figure 7.2: Global and local variables}
\begin{multicols}{2}
\textbf{Scope Example:}
\begin{itemize}
\item $x$ is declared globally and locally within $p$.
\item Local declaration takes precedence in $p$.
\item This code is error prone and confusing.
\item Using another variable name improves readability.
\end{itemize}
\vspace*{-2ex}
\begin{program}
class MyClass \{\\
\>int x;\\
\>void p() \{\\
\>\>int x;\\
\>\>x = 1;\\
\>\>// Print the value of x\\
\>\}\\
\>void q() \{\\
\>\>// Print the value of x\\
\>\}\\
\>... void main(...) \{\\
\>x = 5;\\
\>p;\\
\>q;\\
\}
\end{program}
\end{multicols}
\end{wideslide}

\begin{wideslide}[bm=,toc=]{Free, Bound, Closed}
\begin{defn}{7.12}[Ben Ari] Let $A$ be a formula.\\~\\
\pause
\textbf{Free and Bound Variables:}
\begin{itemize}
\item An occurrence of a variable $x$ in $A$ is a \emph{free variable of A}
iff $x$ is not within the scope of a quantified variable $x$.
\item A variable which is not free is \emph{bound}.
\end{itemize}
\end{defn}
\pause
\textbf{Closed Formulas:}
\begin{itemize}
\item<4-> A formula with no free variable is \emph{closed}.
\item<5-> If $\{x_1,...,x_n\}$ are all free variables of $A$...
  \begin{itemize}
      \item<6->$\forall x_1 \cdots \forall x_n A$ is the \emph{universal closure}.
      \item<7->$\exists x_1 \cdots \exists x_n A$ is the \emph{existential closure}.
  \end{itemize}
\end{itemize}
\vspace*{-2ex}
\end{wideslide}

\begin{wideslide}[bm=,toc=]{Examples}
\begin{ex}{7.13}
$p(x,y)$
\end{ex}
\vspace*{-2ex}
\begin{itemize}
\item<2-> Has two free variables ($x$ and $y$).
\item<3-> $\exists y\; p(x,y)$ has one free variable \pause[3] ($x$).
\item<5-> $\forall x \exists y \;p(x,y)$ is closed.
\item<6-> Universal closure: \pause[3] $\forall x \forall y \;p(x,y)$.
\item<8-> Existential closure: \pause[2] $\exists x \exists y \;p(x,y)$.
\end{itemize}
\pause
\begin{ex}{7.14}
$\forall x \; p(x) \land q(x)$
\end{ex}
\vspace*{-2ex}
\begin{itemize}
\item<11-> Bound: \pause[2] $x$ in $p(x)$
\item<13-> Free: \pause[2] $x$ in $q(x)$
\item<15-> Universal Closure: \pause[2] $\forall x(\forall x \; p(x) \land q(x))$
\item<17-> How to write this better?
\item<18-> $\forall y(\forall x \; p(x) \land q(y))$
\end{itemize}
\end{wideslide}

\section[slide=true]{Semantics}
\begin{slide}[bm=,toc=]{Interpretations: FOL vs PL}
{\bf Interpretation in PL:}
\begin{itemize}
\item Map from atomic \emph{propositions} to truth values.
\end{itemize}
\pause
{\bf Interpretation in FOL:} 
\begin{itemize}
\item Map from atomic \emph{formulas} to truth values.
\end{itemize}
\vspace{1ex}
\pause
{\bf \emph{Mapping atomic formulas is more complex}}
\begin{itemize}
\item Atomic formulas contain variables and constants.
\item First these must be assigned elements from some domain.
\item Then predicates are interpreted as relations on that domain. 
\end{itemize}
\end{slide}

\begin{wideslide}[bm=,toc=]{Interpretations in FOL}
\begin{defn}{7.16}
Let $A$ be a formula where 
\begin{itemize}
\item<2-> $\{p_1,...,p_m\}$ are all the predicates appearing in $A$ and 
\item<3-> $\{a_1,...,a_k\}$ are all the constants appearing in $A$.
\end{itemize}
\pause[3]
An \emph{interpretation} $\mathcal{I}_A$ for $A$ is a triple:
\pause
\[(D,\{R_1,...R_m\}, \{d_1,...d_k\},)\]
\pause
where 
\begin{itemize}
\item<6-> $D$ is a \emph{non-empty} set called the \emph{domain}, 
\item<7-> $R_i$ is an $n_i$-ary relation on $D$ that is assigned to the $n_i$-ary predicate
$p_i$ and 
\item<8-> $d_i \in D$ is assigned to the constant $a_i$.
\end{itemize}
\end{defn}
\end{wideslide}
\begin{wideslide}[bm=,toc=]{Examples of Interpretations in FOL}
\begin{ex}{7.17}
Four interpretations for the formula $\forall x \; p(a,x)$:
\end{ex}
\vspace*{-2ex}
\begin{enumerate}
\item<2-> $\mathcal{I}_1 = (\N_0,\{\leq\},\{0\})$
\begin{itemize}
\item<3-> $\N_0$ is assigned to $D$.
\item<3-> The \emph{less-than or equal} relation is assigned to $p$.
\item<3-> $0$ is assigned to $a$.
\end{itemize}
\item<4-> $\mathcal{I}_2 = (\N_0,\{\leq\},\{1\})$
\begin{itemize}
\item<5-> Same as above but $1$ is assigned to $a$. 
\end{itemize}
\item<6-> $\mathcal{I}_3 = (\Z,\{\leq\},\{0\})$
\begin{itemize}
\item<7-> Same as first example but $\Z$ is assigned to $D$. 
\end{itemize}
\item<8-> $\mathcal{I}_4 = (\mathcal{S},\{substr\},\{ \epsilon \})$
\begin{itemize}
\item The domain, $\mathcal{S}$, is a set of strings; $\epsilon$ is the empty string.
\item The $substr$ relation is a binary relation such that $(s_1,s_2) \in substr$
iff $s_1$ is a substring of $s_2$.
\end{itemize}

\end{enumerate}
\end{wideslide}


\begin{wideslide}[bm=,toc=]{Assignment}
\begin{defn}{7.18}[Ben Ari]~\\
\begin{itemize}
\item<2-> Let $\mathcal{I}_A$ be an interpretation for a formula $A$. 
\item<3-> An \emph{assignment} $\sigma_{\mathcal{I}_A}: \mathcal{V} \mapsto D$ is 
\begin{itemize}
\item<4-> a function, which 
\item<5-> maps every variable $v \in \mathcal{V}$ 
\item<6-> to an element $d \in D$, the domain of $\mathcal{I}_A$.
\end{itemize}
\item<7-> $\sigma_{\mathcal{I}_A}[x_i \leftarrow d_i]$ is an assignment that is
the same as $\sigma_{\mathcal{I}_A}$ except that $x_i$ is mapped to
$d_i$.
\end{itemize}
\end{defn}
\pause[7]
\textbf{Intuition:}
\begin{itemize}
\item $\sigma_{\mathcal{I}_A}[x_i \leftarrow d_i]$ can be
understood as a \emph{modification} of $\sigma_{\mathcal{I}_A}$ in which 
\item the value of $x_i$ is assigned to $d_i$ instead.
\end{itemize}
\end{wideslide}


\begin{wideslide}[bm=,toc=]{Truth in First Order Logic}
\begin{defn}{7.19}[Ben Ari] Let $A$ be a formula, $\mathcal{I}_A$ an interpretation and
$\sigma_{\mathcal{I}_A}$ an assignment. 
\begin{itemize}
\item<2-> The \emph{truth value} of $A$ \emph{under} $\mathcal{I}_A$ \emph{and}
$v_{\sigma_{\mathcal{I}_A}}(A)$ is written: $\sigma_{\mathcal{I}_A}$
\item<3-> (for simplicity we write $v_{\sigma}$ for $v_{\sigma_{\mathcal{I}_A}}$)
\item<4-> Defined by induction on the structure of $A$ :
\begin{itemize}
\item<5->  Let $A = p_k(c_1,...,c_n)$ be an atomic formula where each $c_i$ is
either a variable $x_i$ or a constant $a_i$. 
\item<6->  $v_{\sigma}(A) = T$ iff $(d_1,...,d_n)
\in R_k$ where $R_k$ is the relation assigned by $\mathcal{I}_A$ to $p_k$, and
$d_i$ is the domain element assigned to $c_i$, 
\item<7->  either by $\mathcal{I}_A$ if $c_i$ is a constant or by $\sigma_{\mathcal{I}_A}$ 
      if $c_i$ is a variable.
\item<8-> $v_{\sigma}(\ngg A) = T$ iff $v_{\sigma}(A) = F$.
\item<9-> $v_{\sigma}(A_1 \lor A_2) = T$ iff $v_{\sigma}(A_1) = T$ or
$v_{\sigma}(A_2) = T$ (sim. for $\land$, $\imp$, etc. ) 
\item<10->  $v_{\sigma}(\forall x A_1) = T$ iff $\sigma_{[x \leftarrow d]}(A_1) = T$
for \emph{all} $d \in D$.
\item<11->  $v_{\sigma}(\exists x A_1) = T$ iff $\sigma_{[x \leftarrow d]}(A_1) = T$
for \emph{some} $d \in D$.
\end{itemize}
\end{itemize}
\end{defn}
\end{wideslide}

\begin{wideslide}[bm=,toc=]{Truth Values of Interpretations in Example 7.17}
\begin{ex}{7.21}
Recall the four interpretations given for the formula $\forall x \; p(a,x)$:
\end{ex}
\vspace*{-2ex}
\begin{enumerate}
\item<2-> $\mathcal{I}_1 = (\N_0,\{\leq\},\{0\})$
\begin{itemize}
\item<3-> $T$: all natural numbers are greater than or equal to $0$.
\end{itemize}
\item<4-> $\mathcal{I}_2 = (\N_0,\{\leq\},\{1\})$
\begin{itemize}
\item<5-> $F$: $0 \in \N_0$ but $1 \not \leq 0$. 
\end{itemize}
\item<6-> $\mathcal{I}_3 = (\Z,\{\leq\},\{0\})$
\begin{itemize}
\item<7-> $F$: Not all integers are greater than or equal to $0$.
\end{itemize}
\item<8-> $\mathcal{I}_4 = (\mathcal{S},\{substr\},\{ \epsilon \})$
\begin{itemize}
\item<9-> $T$: $\epsilon$ is a substring of all strings, therefore also all
strings in $\mathcal{S}$. 
\end{itemize}
\end{enumerate}
\end{wideslide}

\begin{wideslide}[bm=,toc=]{Truth for Some / All Assignments}
\begin{thm}{7.22}[Ben Ari]
Let $A' = A(x_1,...,x_n)$ be a (non-closed) formula with free variables
$x_1,...,x_n$, and let $\mathcal{I}$ be an interpretation. 
\end{thm}
\pause
Then:
\begin{itemize}
\item<3-> $v_{\sigma_{\mathcal{I}_A}}(A') = T$ for \emph{some} assignment $\sigma_{\mathcal{I}_A}$ iff $v_{\mathcal{I}}(\exists x_1 ... \exists x_n A') = T$.
\item<4-> $v_{\sigma_{\mathcal{I}_A}}(A') = T$ for \emph{all} assignments
$\sigma_{\mathcal{I}_A}$ iff $v_{\mathcal{I}}(\forall x_1 ... \forall x_n A') = T$.
\end{itemize}
\pause[3]
Informally:
\begin{itemize}
\item<6-> The value of $A'$ under interpretation $\mathcal{I}$ and assignment
$\sigma$ is $T$ for some assignment iff the existential closure of $A'$ is true 
under $\mathcal{I}$. 
\item<7-> Similarly, the value of $A'$ under interpretation $\mathcal{I}$ and assignment
$\sigma$ is $T$ for all assignments iff the universal closure of $A'$ is true 
under $\mathcal{I}$. 
\end{itemize}
\end{wideslide}

\section[slide=true]{Validity and Satisfiability}

\begin{wideslide}[bm=,toc=]{Validity and Satisfiability}
\begin{defn}{7.23}[Ben Ari]
Let $A$ be a closed formula of first-order logic.
\end{defn}
\vspace{-2ex}
\begin{itemize}
\item<2-> $A$ is \emph{true} in $\mathcal{I}$ or $\mathcal{I}$ is a \emph{model} for
$A$ iff $v_{\mathcal{I}}(A) = T$. 
\begin{itemize}
\item<3-> Notation: $\mathcal{I} \models A$.
\end{itemize}
\item<4-> $A$ is \emph{valid} if for \emph{all} interpretations $\mathcal{I}$,
$\mathcal{I} \models A$. 
\begin{itemize}
\item<5-> Notation: $\models A$ 
\end{itemize}
\item<6-> $A$ is \emph{satisfiable} if for \emph{some} interpretation $\mathcal{I}$,
$\mathcal{I} \models A$.
\item<7-> $A$ is \emph{unsatisfiable} if it is not satisfiable. 
\item<8-> $A$ is \emph{falsifiable} if it is not valid. 
\end{itemize}

\end{wideslide}

\begin{wideslide}[bm=,toc=]{Satisfiability and Validity in $\forall x \; p(a,x)$}
Let $A$ be the formula $\forall x \; p(a,x)$ from examples $7.17$ and $7.21$:
\begin{enumerate}
\item<2-> $\mathcal{I}_1 = (\N_0,\{\leq\},\{0\})$
\begin{itemize}
\item<3-> $T$: $A$ is satisfiable. 
\end{itemize}
\item<4-> $\mathcal{I}_2 = (\N_0,\{\leq\},\{1\})$
\begin{itemize}
\item<5-> $F$: $A$ is not valid. 
\end{itemize}
\item<6-> $\mathcal{I}_3 = (\Z,\{\leq\},\{0\})$
\begin{itemize}
\item<7-> $F$: Also shows $A$ is not valid. 
\end{itemize}
\item<8-> $\mathcal{I}_4 = (\mathcal{S},\{substr\},\{ \epsilon \})$
\begin{itemize}
\item<9-> $T$: Also shows $A$ is satisfiable. 
\end{itemize}
\end{enumerate}
\end{wideslide}


\begin{wideslide}[bm=,toc=]{Satisfiability and Validity Examples}
\begin{defn}{7.25}{Ben Ari}
\end{defn}
\begin{enumerate}
\item<2-> $\forall x \forall y (p(x,y) \imp p(y,x))$
\begin{itemize}
\item<3-> Satisfiable? \pause[3] Yes, in interpretations where $p$ is a
symmetric relation.
\item<3-> Valid? \pause No. Falisified in interpretations where $p$ is
not symmetric. 
\end{itemize}
\item<6-> $\forall x \exists y \; p(x,y)$
\begin{itemize}
\item<7-> Satisfiable? \pause[3] Satisfiable in interpretations where 
$p$ is a total function. For example, $(x,y) \in R$ iff $y = x + 1$
for $x,y \in \Z$.
\item<7-> Valid? \pause No. Falsified if the domain is changed to $\Z^-$.  
\end{itemize}
\item<10-> $\exists x \exists y(p(x) \land \ngg p (y))$
\begin{itemize}
\item<11-> Satisfiable? \pause[3] Yes. Assume for example that $D$ is $\N$ and
$p$ is the unary predicate corresponding to the property of being an
odd number.
\item<11-> Valid? \pause No. Consider a domain with only one element.
\end{itemize}
\end{enumerate}
\end{wideslide}

\begin{wideslide}[bm=,toc=]{Satisfiability and Validity Examples (continued)}
\begin{defn}{7.25}{Ben Ari}
(continued from previous slide)
\end{defn}
\begin{enumerate}
\setcounter{enumi}{3}
\item<2-> $\forall x \; p(a,x)$
\begin{itemize}
\item<3-> Expresses the existence of an element with special properties.
\item<3-> Satisfiable? \pause[3] Yes, when $D$ is $\N$ and $p$ is $\leq$.
\item<3-> Valid? \pause No. Falisified if we change $D$ to $\Z$ 
\end{itemize}
\item<6-> $\forall x (p(x) \land q(x)) \eqv (\forall x p(x) \land \forall x
    q(x))$
\begin{itemize}
\item<7-> Satisfiable? \pause[3] Yes. Ex: $D$ is $\N$, $p$ is the property of
not being negative and $q$ is the property of being divisible by $1$. 
\item<7-> Valid? \pause Yes. Proof uses theorem 7.22. 
\end{itemize}
\item<10-> $\forall x (p(x) \imp q(x)) \imp (\forall x p(x) \imp \forall x q(x))$
\begin{itemize}
\item<11-> Valid? \pause Yes. 
\item<11-> What about the converse: $(\forall x p(x) \imp \forall x q(x)) \imp \forall x (p(x) \imp q(x))$?
\end{itemize}
\end{enumerate}
\end{wideslide}

\begin{wideslide}[bm=,toc=]{Extending to Sets of Formulas}
As in PL, we can extend concepts of interpretation, satisfiability and other
properties to sets of formulas.
\pause
\begin{defn}{7.26}[Ben Ari]
Let $U = \{A_1,...\}$ be a set of formulas where 
\begin{itemize}
\item<3-> $\{p_1,...,p_m\}$ are all the predicates appearing in all $A_i \in U$ and
\item<4-> $\{a_1,...,a_k\}$ are all the constants appearing in all $A_i \in U$. 
\end{itemize}
\pause[3]
An \emph{interpretation} $\mathcal{I}_U$ for $U$ is a triple:
\[(D, \{R_1,...,R_m\},\{d_1,...,d_k\})\]
\pause
where 
\begin{itemize}
\item<6-> $D$ is a \emph{non-empty} set called the \emph{domain}, 
\item<7-> $R_i$ is an $n_i$-ary relation on $D$ that is assigned to the $n_i$-ary predicate $p_i$ and 
\item<8-> $d_i \in D$ is an element of $D$ that is assigned to the constant $a_i$.
\end{itemize}
\end{defn}

\end{wideslide}

\begin{wideslide}[bm=,toc=]{Consistency}
\begin{defn}{7.27}[Ben Ari]~\\~\\
A set of closed formulas $U = \{A_1,...\}$ is \emph{(simultaneously)
  satisfiable} iff 
\begin{itemize}
\item<2-> there exists an interpretation $\mathcal{I}_U$ such that $v_{\mathcal{I}_U}(A_i) = T$ for all $i$. 
\item<3-> The satisfying interpretation is a \emph{model} of $U$.
\item<4-> $U$ is \emph{valid} iff for every interpretation
  $\mathcal{I}_U$, $v_{\mathcal{I}_U}(A_i) = T$ for all $i$.
\end{itemize}
\end{defn}
\end{wideslide}

\section[slide=true]{Logical Equivalence}

\begin{wideslide}[bm=,toc=]{Definition of Logical Equivalence}
\begin{defn}{7.28}[Ben Ari]
\end{defn}
\vspace{-2ex}
\begin{itemize}
\item<2-> Let $U = \{A_1,A_2\}$ be a pair of closed formulas. 
\item<3-> $A_1$ is \emph{logically equivalent} to $A_2$ iff \pause[3] $v_{{\mathcal{I}_U}}(A_1) =
v_{{\mathcal{I}_U}}(A_2)$ for \emph{all} $\mathcal{I}_U$.
\begin{itemize}
\item<5-> Notation: $A_1 \equiv A_2$.
\end{itemize}
\item<6-> Let $A$ be a closed formula and $U$ a set of closed formulas.
\item<7-> $A$ is a \emph{logical consequence} of $U$ iff \pause[4] for all interpretations
$\mathcal{I}_{U \cup \{A\}}$, $v_{\mathcal{I}_{U \cup \{A\}}}(A_i) = T$ for
all $A_i \in U$ implies $v_{\mathcal{I}_{U \cup \{A\}}}(A) = T$. 
\begin{itemize}
\item<9-> Notation: $U \models A$.
\end{itemize}
\end{itemize}
\vspace{-5mm}
\pause[2]
\begin{thm}{7.29}
Let $A,B$ be closed formulas and $U = \{A_1,...,A_n\}$ be a set of
closed formulas. Then:
\end{thm}
\vspace{-5ex}
\begin{tabbing}
closed formulas. Then X \= \kill
\> $A \equiv B$ iff $\models A \eqv B$\\
\> $U \models A$ iff $\models (A_1 \land \cdots \land A_n) \imp B$
\end{tabbing}

\end{wideslide}

\begin{wideslide}[bm=,toc=]{Duality}
The two quantifiers are duals:
\pause
\begin{tabbing}
The two quantifiers are dua \= \kill
\> $\models \forall x A(x) \eqv \ngg \exists x \ngg A(x)$.\\
\> ~\\ 
\pause
\> $\models \exists x A(x) \eqv \ngg \forall x \ngg A(x)$.
\end{tabbing}
\pause
\begin{itemize}
\item It is only necessary to define one of the two.
\item If $\forall$ is defined, $\exists$ can be considered an abbreviation of
      $\ngg \forall \ngg$.
\end{itemize}

\end{wideslide}

\begin{wideslide}[bm=,toc=]{Commutativity}
Quantifiers of the same type commute:
\pause
\begin{tabbing}
Quantifiers of the same \= \kill
\> $\models \forall x \forall y A(x,y) \eqv \forall y \forall x A(x,y)$.\\
\pause
\> ~\\ 
\> $\models \exists x \exists y A(x,y) \eqv \exists y \exists x A(x,y)$.\\
\end{tabbing}

\pause
but $\forall$ and $\exists$ commute only in one direction:
\pause
\begin{tabbing}
Quantifiers of the same \= \kill
\> $\models \exists x \forall y A(x,y) \imp \forall y \exists x A(x,y)$.\\
\end{tabbing}

\end{wideslide}

\begin{wideslide}[bm=,toc=]{Distributivity}
\begin{tabbing}
Universal \= \kill
Universal quantifiers distribute over conjunction:  \> \\[2mm]
\pause
\>$\models \forall x(A(x) \land B(x)) \eqv \forall x A(x) \land \forall x B(x)$.\\
~\\
\pause
Existential quantifiers distribute over disjunction:\\[2mm]
\pause
\>$\models \exists x(A(x) \lor B(x)) \eqv \exists x A(x) \lor \exists x B(x)$.\\
~\\
\pause
Distributing universal over disjunction goes only one direction:\\[2mm]
\pause
\>$\models \forall x A(x) \lor \forall x B(x) \imp \forall x(A(x) \lor B(x))$.\\
~\\
\pause
Distributing existential over conjunction goes the other direction:\\[2mm]
\pause
\>$\models \exists x(A(x) \land B(x)) \imp \exists x A(x) \land \exists x B(x)$.\\
\end{tabbing}

\end{wideslide}

\begin{wideslide}[bm=,toc=]{Quantification Without the Free Variable in Its Scope}

Quantifiers over disjunction or conjunction are always distributive when
one subformula does not contain the quantified variable:

\vspace*{-5mm}
\begin{displaymath}
\def\arraystretch{1.5}
\begin{array}{ll}
 \models \exists xA(x) \vee B \eqv \exists x(A(x) \vee  B), &
 \models \forall xA(x) \vee B \eqv \forall x(A(x) \vee  B), \\
 \models B \vee \exists xA(x) \eqv \exists x(B \vee A(x)), &
 \models B \vee \forall xA(x) \eqv \forall x(B \vee A(x)), \\
 \models \exists xA(x) \wedge B \eqv \exists x(A(x) \wedge B), &
 \models \forall xA(x) \wedge B \eqv \forall x(A(x) \wedge B), \\
 \models B \wedge \exists xA(x) \eqv \exists x(B \wedge A(x)), &
 \models B \wedge \forall xA(x) \eqv \forall x(B \wedge A(x)).
\end{array}
\end{displaymath}
\pause
For implication, the following equivalences hold (note they are not symmetric):
\begin{displaymath}
\begin{array}{l}
 \models \forall x(A\imp B(x)) \eqv (A \imp \forall xB(x)) \\ 
 \models \forall x(A(x)\imp B) \eqv (\exists xA(x) \imp B) \\
\end{array}
\end{displaymath}

\end{wideslide}
\begin{wideslide}[bm=,toc=]{Quantification Over Implication}
The following formulas hold for distributing quantifiers over implication:
\vspace*{-2mm}
\begin{displaymath}
\def\arraystretch{1.5}
\begin{array}{l}
\models \forall x(A(x)\imp B(x))\imp (\forall xA(x)\imp\forall xB(x)) \\
\models \forall x(A(x)\imp B(x))\imp (\exists xA(x)\imp\exists xB(x)) \\
\models \forall x(A(x)\imp B(x))\imp (\forall xA(x)\imp\exists xB(x)) \\
\models \exists x(A(x)\imp B(x))\eqv (\forall xA(x)\imp\exists xB(x))  \\
\models (\exists xA(x)\imp\forall xB(x))\imp \forall x(A(x)\imp B(x))  \\
\end{array}
\end{displaymath}
\pause
Example to verify intuition:
\begin{itemize}
\item Let $A(x)$ mean ``a person, $x$, is rich.''
\item Let $B(x)$ mean ``a person, $x$, is happy.''
\item Check the formulas under this interpretation.
\end{itemize}
\end{wideslide}

\begin{wideslide}[bm=,toc=]{Quantification Over Equivalence}
Distribution of universal quantification works in one direction:
\begin{displaymath}
\begin{array}{l}
 \models \forall x(A(x)\eqv B(x))\imp (\forall xA(x) \eqv \forall xB(x)) \\
\end{array}
\end{displaymath}
\pause
The following formula holds for existential quantification:
\begin{displaymath}
\begin{array}{l}
 \models \forall x(A(x)\eqv B(x))\imp (\exists xA(x) \eqv \exists xB(x)) \\
\end{array}
\end{displaymath}

\end{wideslide}

\begin{wideslide}[bm=,toc=]{Example Derivation for Distributing Quantifiers Over
Implications / Biconditionals}
To derive an equivalent formula, translate $\imp$ or $\eqv$ into disjuctions and
conjunctions, then apply relevance substitutions.
\begin{displaymath}
\def\arraystretch{1.5}
\begin{array}{ll}
\exists x (A(x) \imp B(x))  & \equiv \exists x(\ngg A(x) \lor B(x)) \\
\pause &\equiv \exists x\ngg A(x) \lor \exists x B(x) \\
\pause  &\equiv \ngg \exists x\ngg A(x) \imp \exists x B(x) \\
\pause &\equiv \forall x A(x) \imp \exists x B(x) \\

\end{array}
\end{displaymath}

\end{wideslide}

\section[slide=true]{Functions}
\begin{slide}[bm=,toc=]{The Case for Functions in FOL}
FOL can express many properties of relations. 
\\~\\
\pause
{\bf Example:}
\begin{itemize}
\item $A = \forall x \forall y \forall z (p(x,y) \land p (y,z) \imp p(x,z))$ with
\item $\mathcal{I}_A = (\Z,\{<\},\{\})$
\item expresses transitivity for $<$ in the domain of integers.
\end{itemize}
\vspace{2ex}
\pause
{\bf Compare to}:
\[
  for\; all\;  x,y,z: (x < y) \imp (x + z < y + z)
  \]

\vspace{-1ex}
\pause
\begin{itemize}
\item The difference is the $+$ function.
\item How do we express this in first order logic?
\end{itemize}

\end{slide}

\begin{wideslide}[bm=,toc=]{Functions and Terms}
\begin{defn}{9.1}[Ben Ari]
Let $\mathcal{F}$ be a countable set of \emph{function symbols}, where
each symbol has an \emph{arity} denoted by a superscript. 
\\~\\
\pause
\emph{Terms} are defined recursively as follows:
\end{defn}
\vspace{-2ex}
\begin{itemize}
\item<3-> A variable, constant or $0$-ary function symbol is a term.
\item<4-> If $f^n$ is an $n$-ary function symbol $(n > 0)$ and $\{t_1,t_2,...,t_n\}$
are terms, then $f^n(t_1,t_2,...,t_n)$ is a term.
\end{itemize}
\pause[3]
An \emph{atomic formula} is 
\begin{itemize}
\item<6-> an $n$-ary predicate 
\item<7-> followed by a list of $n$ \emph{arguments} 
\item<8-> where each argument $t_i$ is a term: $p(t_1,t_2,...,t_n)$.
\end{itemize}
\end{wideslide}

\begin{wideslide}[bm=,toc=]{Notation}
Conventions we adopt from Ben Ari:
\begin{itemize}
\item<2-> The word `function' refers only to syntactical symbols (replaces word `symbol') 
\item<3-> Functions are denoted by $\{f,g,h\}$ (and subscripts)
\item<4-> Superscript denoting arity is usually omitted.
\item<5-> Constant symbols are now unnecessary, but kept for convenience.
\begin{itemize}
\item<6-> A constant is equivalent to a $0$-ary function.
\end{itemize}
\end{itemize}
~\\
\pause[6]
{\bf Examples of terms:}
\[
  a,\;\; x,\;\; f(a,x),\;\; f(g(x),y),\;\; g(f(a,g(b)))
  \]
\pause
{\bf Examples of atomic formulas:}
\[
  p(a,b),\;\; p(x,f(a,x)),\;\; q(f(a,a), f(g(x),g(x)))
\]
\end{wideslide}

\begin{wideslide}[bm=,toc=]{Interpretations for FOL with Functions}
We must extend the definition of an interpretation to include function symbols:
\pause
\begin{defn}{9.3}[Ben Ari]
Let $U$ be a set of formulas such that 
\begin{itemize}
\item<3-> $\{p_1,...,p_k\}$ are all the predicate symbols, 
\item<4-> $\{f_1^{n_1},...,f_l^{n_l}\}$ are all the functions symbols and
\item<5-> $\{a_1,...,a_m\}$ are all the constant symbols appearing in $U$. 
\end{itemize}
\pause[4]
An \emph{interpretation} $\mathcal{I}$ is a $4$-tuple:
\vspace{-2mm}
\[
  \mathcal{I} = (D, \{R_1,...,R_k\}, \{F_1^{n_1},...,F_l^{n_l}\},\{d_1,...,d_m\})
  \]~\\
\vspace{-8mm}
\pause
\textbf{\emph{consisting of}}
\begin{itemize}
\item<8-> a \emph{non-empty} domain $D$,
\item<9-> $n_i$-ary relations $R_i$ on $D$ to be assigned to the $n_i$-ary predicate symbols $p_i$, 
\item<10-> $n_j$-ary functions $F_j^{n_j}$ on $D$ to be assigned to the function symbol $f_j^{n_j}$, and 
\item<11-> assignments of elements $d_n \in D$ to the constant symbols $a_n$ for $1 \leq n \leq m$.
\end{itemize}
\end{defn}
\end{wideslide}

\begin{wideslide}[bm=,toc=]{Meaning of Atomic Formulas with Functions}
Let $\mathcal{D}_{\mathcal{I}}$ be a map from terms to domain elements such that:
\[
 \mathcal{D}_{\mathcal{I}} (f_i(t_1,...,t_n)) = F_i(\mathcal{D}_{\mathcal{I}}(t_1),...,\mathcal{D}_{\mathcal{I}}(t_n)) 
  \]
\pause
{\bf Given an atomic formula} 
\begin{itemize}
\item $A = p_k(t_1,...,t_n)$
\end{itemize}
\pause
{\bf The truth value is given by}
\begin{itemize}
\item $v_{\sigma_{\mathcal{I}}}(A) = T$ \emph{if and only if} $(\mathcal{D}_{\mathcal{I}}(t_1),...,\mathcal{D}_{\mathcal{I}}(t_n)) \in R_k$
\end{itemize}
\end{wideslide}
\begin{wideslide}[bm=,toc=]{Examples of Interpretations with Functions}
\begin{ex}{9.4}
Consider the formula
\[
  A = \forall x \forall y(p(x,y) \imp p(f(x,a),f(y,a)))
\]
We claim it is true under the interpretation:
\[
  (\Z,\{\leq\}, \{+\}, \{1\}) 
\]
\vspace{-4ex}
\pause
\begin{proof}
~\\For arbitrary $m,n \in \Z$ assigned to $x,y$, $f(x,a)$ and $f(y,a)$ evaluate to:
\vspace{-1ex}
\pause
\[ \mathcal{D}_{\mathcal{I}}(f(x,a)) = +(\mathcal{D}_{\mathcal{I}}(x), \mathcal{D}_{\mathcal{I}}(a)) = +(m,1) = m+1 \]
\pause
\vspace{-4ex}
\[ \mathcal{D}_{\mathcal{I}}(f(y,a)) = +(\mathcal{D}_{\mathcal{I}}(y), \mathcal{D}_{\mathcal{I}}(a)) = +(n,1) = n+1 \]
\pause
Substituting $\leq$ for $p$ yields $m \leq n \imp (m+1) \leq n + 1$, which is
true for $m,n \in \Z$.
\end{proof}
\end{ex}
\end{wideslide}

\begin{wideslide}[bm=,toc=]{Examples of Interpretations with Functions (cont)}
\begin{ex}{9.4}[cont]
Consider the same formula
\[
  A = \forall x \forall y(p(x,y) \imp p(f(x,a),f(y,a)))
\]
Under the interpretation:
\[
  (\Sigma^*,\{\id{suffix}\}, \{\cdot\}, \{\id{tuv}\}) 
\]
\vspace{-4ex}
\end{ex}
\begin{itemize}
\item<2-> $\id{suffix}$ is the relation such that $(s_1,s_2) \in \id{suffix}$ iff $s_1$ is a suffix of $s_2$.
\item<3-> $\cdot$ is the function that concatenates its arguments
\item<4-> $\id{tuv}$ is a string
\item<5-> What is $v_{\mathcal{I}_A}$?
\item<6-> Is $A$ satisfiable?
\item<7-> Valid?
\begin{itemize}
\item<8-> No. Consider $(\Z,\{>\},\{\cdot\},\{-1\})$
\end{itemize}
\end{itemize}
\end{wideslide}


%\begin{wideslide}[bm=,toc=]{Logically equivalent formulas (2)}
%\begin{displaymath}
%\begin{array}{l}
% \forall x(A(x)\vee B(x)) \imp (\forall xA(x) \vee \exists xB(x)) \\
%
%\end{array}
%\end{displaymath}
%\end{wideslide}
%
\end{document}

