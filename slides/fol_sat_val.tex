\begin{wideslide}[bm=,toc=]{Validity and Satisfiability}
\begin{defn}{7.23}[Ben Ari]
Let $A$ be a closed formula of first-order logic.
\end{defn}
\vspace{-2ex}
\begin{itemize}
\item<2-> $A$ is \emph{true} in $\mathcal{I}$ or $\mathcal{I}$ is a \emph{model} for
$A$ iff $v_{\mathcal{I}}(A) = T$. 
\begin{itemize}
\item<3-> Notation: $\mathcal{I} \models A$.
\end{itemize}
\item<4-> $A$ is \emph{valid} if for \emph{all} interpretations $\mathcal{I}$,
$\mathcal{I} \models A$. 
\begin{itemize}
\item<5-> Notation: $\models A$ 
\end{itemize}
\item<6-> $A$ is \emph{satisfiable} if for \emph{some} interpretation $\mathcal{I}$,
$\mathcal{I} \models A$.
\item<7-> $A$ is \emph{unsatisfiable} if it is not satisfiable. 
\item<8-> $A$ is \emph{falsifiable} if it is not valid. 
\end{itemize}

\end{wideslide}

\begin{wideslide}[bm=,toc=]{Satisfiability and Validity in $\forall x \; p(a,x)$}
Let $A$ be the formula $\forall x \; p(a,x)$ from examples $7.17$ and $7.21$:
\begin{enumerate}
\item<2-> $\mathcal{I}_1 = (\N_0,\{\leq\},\{0\})$
\begin{itemize}
\item<3-> $T$: $A$ is satisfiable. 
\end{itemize}
\item<4-> $\mathcal{I}_2 = (\N_0,\{\leq\},\{1\})$
\begin{itemize}
\item<5-> $F$: $A$ is not valid. 
\end{itemize}
\item<6-> $\mathcal{I}_3 = (\Z,\{\leq\},\{0\})$
\begin{itemize}
\item<7-> $F$: Also shows $A$ is not valid. 
\end{itemize}
\item<8-> $\mathcal{I}_4 = (\mathcal{S},\{substr\},\{ \epsilon \})$
\begin{itemize}
\item<9-> $T$: Also shows $A$ is satisfiable. 
\end{itemize}
\end{enumerate}
\end{wideslide}


\begin{wideslide}[bm=,toc=]{Satisfiability and Validity Examples}
\begin{defn}{7.25}{Ben Ari}
\end{defn}
\begin{enumerate}
\item<2-> $\forall x \forall y (p(x,y) \imp p(y,x))$
\begin{itemize}
\item<3-> Satisfiable? \pause[3] Yes, in interpretations where $p$ is a
symmetric relation.
\item<3-> Valid? \pause No. Falisified in interpretations where $p$ is
not symmetric. 
\end{itemize}
\item<6-> $\forall x \exists y \; p(x,y)$
\begin{itemize}
\item<7-> Satisfiable? \pause[3] Satisfiable in interpretations where 
$p$ is a total function. For example, $(x,y) \in R$ iff $y = x + 1$
for $x,y \in \Z$.
\item<7-> Valid? \pause No. Falsified if the domain is changed to $\Z^-$.  
\end{itemize}
\item<10-> $\exists x \exists y(p(x) \land \ngg p (y))$
\begin{itemize}
\item<11-> Satisfiable? \pause[3] Yes. Assume for example that $D$ is $\N$ and
$p$ is the unary predicate corresponding to the property of being an
odd number.
\item<11-> Valid? \pause No. Consider a domain with only one element.
\end{itemize}
\end{enumerate}
\end{wideslide}

\begin{wideslide}[bm=,toc=]{Satisfiability and Validity Examples (continued)}
\begin{defn}{7.25}{Ben Ari}
(continued from previous slide)
\end{defn}
\begin{enumerate}
\setcounter{enumi}{3}
\item<2-> $\forall x \; p(a,x)$
\begin{itemize}
\item<3-> Expresses the existence of an element with special properties.
\item<3-> Satisfiable? \pause[3] Yes, when $D$ is $\N$ and $p$ is $\leq$.
\item<3-> Valid? \pause No. Falisified if we change $D$ to $\Z$ 
\end{itemize}
\item<6-> $\forall x (p(x) \land q(x)) \eqv (\forall x p(x) \land \forall x
    q(x))$
\begin{itemize}
\item<7-> Satisfiable? \pause[3] Yes. Ex: $D$ is $\N$, $p$ is the property of
not being negative and $q$ is the property of being divisible by $1$. 
\item<7-> Valid? \pause Yes. Proof uses theorem 7.22. 
\end{itemize}
\item<10-> $\forall x (p(x) \imp q(x)) \imp (\forall x p(x) \imp \forall x q(x))$
\begin{itemize}
\item<11-> Valid? \pause Yes. 
\item<11-> What about the converse: $(\forall x p(x) \imp \forall x q(x)) \imp \forall x (p(x) \imp q(x))$?
\end{itemize}
\end{enumerate}
\end{wideslide}

\begin{wideslide}[bm=,toc=]{Extending to Sets of Formulas}
As in PL, we can extend concepts of interpretation, satisfiability and other
properties to sets of formulas.
\pause
\begin{defn}{7.26}[Ben Ari]
Let $U = \{A_1,...\}$ be a set of formulas where 
\begin{itemize}
\item<3-> $\{p_1,...,p_m\}$ are all the predicates appearing in all $A_i \in U$ and
\item<4-> $\{a_1,...,a_k\}$ are all the constants appearing in all $A_i \in U$. 
\end{itemize}
\pause[3]
An \emph{interpretation} $\mathcal{I}_U$ for $U$ is a triple:
\[(D, \{R_1,...,R_m\},\{d_1,...,d_k\})\]
\pause
where 
\begin{itemize}
\item<6-> $D$ is a \emph{non-empty} set called the \emph{domain}, 
\item<7-> $R_i$ is an $n_i$-ary relation on $D$ that is assigned to the $n_i$-ary predicate $p_i$ and 
\item<8-> $d_i \in D$ is an element of $D$ that is assigned to the constant $a_i$.
\end{itemize}
\end{defn}

\end{wideslide}

\begin{wideslide}[bm=,toc=]{Consistency}
\begin{defn}{7.27}[Ben Ari]~\\~\\
A set of closed formulas $U = \{A_1,...\}$ is \emph{(simultaneously)
  satisfiable} iff 
\begin{itemize}
\item<2-> there exists an interpretation $\mathcal{I}_U$ such that $v_{\mathcal{I}_U}(A_i) = T$ for all $i$. 
\item<3-> The satisfying interpretation is a \emph{model} of $U$.
\item<4-> $U$ is \emph{valid} iff for every interpretation
  $\mathcal{I}_U$, $v_{\mathcal{I}_U}(A_i) = T$ for all $i$.
\end{itemize}
\end{defn}
\end{wideslide}


