\begin{slide}[bm=,toc=]{First Order Logic: Syntax}
\begin{defn}{7.6}[Ben Ari]
Let $\mathcal{P}$, $\mathcal{A}$ and $\mathcal{V}$ be countable sets of
\emph{predicate symbols}, \emph{constant symbols} and \emph{variables}.
\pause
\begin{itemize}
\item Each predicate symbol $p^n \in \mathcal{P}$ is associated with an \emph{arity}.
\begin{itemize}
\item Refers to the number $n \geq 1$ of \emph{arguments} that it takes. 
\end{itemize}
\item $p^n$ is called an $n$-ary predicate. 
\item For $n = 1,2$, the terms \emph{unary} and \emph{binary},
respectively, are also used.
\end{itemize}
\end{defn}
\pause
\begin{defn}{7.7}[Ben Ari]
~\\
$\forall$ is the \emph{universal quantifier} and is read \emph{for all}.\\
$\exists$ is the \emph{existential quantifier} and is read \emph{there exists}.
\end{defn}

\end{slide}

\begin{slide}[bm=,toc=]{First Order Logic: Notational Conventions}
\textbf{Writing Symbols:}
\begin{itemize}
\item<2-> By convention, we associate the following lower-case letters with each
set:
\begin{itemize}
\item $\mathcal{P} = \{p,q,r\}$, for predicate symbols
\item $\mathcal{A} = \{a,b,c\}$, for constant symbols
\item $\mathcal{V} = \{x,y,z\}$, for variables
\end{itemize}
\item<3-> The superscript denoting the arity of the predicate is omitted when clear (can
    be inferred from the number of arguments).
\begin{itemize}
\item<4-> \textbf{Example:} \pause[4] $p(x,y,z)$ instead of $p^3(x,y,z)$
\end{itemize}
\end{itemize}
\end{slide}

\begin{wideslide}[bm=,toc=]{Atomic Formulas in FOL}
\begin{defn}{7.8}[Ben Ari]
~\\~\\
\textbf{ An \emph{atomic formula}} is
\begin{itemize}
\item<2-> an $n$-ary predicate 
\item<3-> followed by a list of $n$ arguments in parentheses:, 
\begin{itemize}
\item<4-> $p(t_1,t_2,...,t_n)$
\end{itemize}
\item<5-> where each argument $t_i$ is either a variable or a constant. 
\end{itemize}
\pause[5]
\textbf{Examples:}
\begin{itemize}
\item $lt(3,4)$
\item $p(user, edit)$
\end{itemize}

\end{defn}
\end{wideslide}


\begin{wideslide}[bm=,toc=]{Formulas in FOL}
\begin{defn}{7.8}[Ben Ari] (Continued)\\~\\
\textbf{ A \emph{formula}} in first-order logic is
a tree defined recursively as follows:
\end{defn}
\vspace{-3ex}
\begin{itemize}
\item<2-> A formula is a leaf labeled by an atomic formula.
\item<3-> A formula is a node labeled by $\ngg$ with a single child that is a
formula.
\item<4-> A formula is a node labeled by $\forall x$ or $\exists x$ (for some
 variable $x$) with a single child that is a formula.
\item<5-> A formula is a node labeled by a binary Boolean operator with two
children both of which are formulas.
\end{itemize}
\pause[5]
{\bf Universal vs Existential}:\\
A formula of the form $\forall x A$ is a \emph{universal formula}.\\
A formula of the form $\exists x A$ is an \emph{existential formula}.\\
\end{wideslide}

\begin{wideslide}[bm=,toc=]{Figure 7.1: Tree for
$\forall x(\ngg\exists yp(x,y)\vee\ngg\exists yp(y,x))$}
\begin{multicols}{2}
\begin{itemize}
\item Formula tree structure follows PL.
\item Similar concept of induction on structure.
\item In the string representation, $\exists$ and $\forall$ have the same precedence as $\ngg$.
\end{itemize}
\begin{center}
\setlength{\GapWidth}{8mm}
\setlength{\GapDepth}{8mm}
\begin{bundle}{$\forall x$\rule[-1mm]{0mm}{1mm}}
\chunk{
  \begin{bundle}{$\vee$\rule[-1mm]{0mm}{1mm}}
  \chunk{
    \begin{bundle}{$\ngg$\rule[-1mm]{0mm}{1mm}}
    \chunk{
      \begin{bundle}{$\exists y$\rule[-1mm]{0mm}{1mm}}
      \chunk{$p(x,y)$}
      \end{bundle}
    }
    \end{bundle}
  }
  \chunk{
    \begin{bundle}{$\ngg$\rule[-1mm]{0mm}{1mm}}
    \chunk{
      \begin{bundle}{$\exists y$\rule[-1mm]{0mm}{1mm}}
      \chunk{$p(y,x)$}
      \end{bundle}
    }
    \end{bundle}
  }
  \end{bundle}
}
\end{bundle}
\end{center}
\end{multicols}
\end{wideslide}

\begin{wideslide}[bm=,toc=]{Scope of Variables}
\begin{defn}{7.11}[Ben Ari]~\\
\textbf{Quantification and Scope:}
\begin{itemize}
\item<2-> A universal or existential formula $\forall x A$ or $\exists x A$
is a \emph{quantified formula}. 
\item<3-> $x$ is the \emph{quantified variable}
and its \emph{scope} is the formula $A$. 
\item<4-> It is not required that $x$ actually appear in the scope of its 
quantification.
\end{itemize}
\end{defn}
\pause[4]
\textbf{Comments:}
\begin{itemize}
\item<6-> Conceptually similar to scope of variables in programming languages.
\item<7-> Technically, local declaration hides global declarations.
\item<8-> Best practice is to avoid name conflicts.
\end{itemize}
\end{wideslide}

\begin{wideslide}[bm=,toc=]{Figure 7.2: Global and local variables}
\begin{multicols}{2}
\textbf{Scope Example:}
\begin{itemize}
\item $x$ is declared globally and locally within $p$.
\item Local declaration takes precedence in $p$.
\item This code is error prone and confusing.
\item Using another variable name improves readability.
\end{itemize}
\vspace*{-2ex}
\begin{program}
class MyClass \{\\
\>int x;\\
\>void p() \{\\
\>\>int x;\\
\>\>x = 1;\\
\>\>// Print the value of x\\
\>\}\\
\>void q() \{\\
\>\>// Print the value of x\\
\>\}\\
\>... void main(...) \{\\
\>x = 5;\\
\>p;\\
\>q;\\
\}
\end{program}
\end{multicols}
\end{wideslide}

\begin{wideslide}[bm=,toc=]{Free, Bound, Closed}
\begin{defn}{7.12}[Ben Ari] Let $A$ be a formula.\\~\\
\pause
\textbf{Free and Bound Variables:}
\begin{itemize}
\item An occurrence of a variable $x$ in $A$ is a \emph{free variable of A}
iff $x$ is not within the scope of a quantified variable $x$.
\item A variable which is not free is \emph{bound}.
\end{itemize}
\end{defn}
\pause
\textbf{Closed Formulas:}
\begin{itemize}
\item<4-> A formula with no free variable is \emph{closed}.
\item<5-> If $\{x_1,...,x_n\}$ are all free variables of $A$...
  \begin{itemize}
      \item<6->$\forall x_1 \cdots \forall x_n A$ is the \emph{universal closure}.
      \item<7->$\exists x_1 \cdots \exists x_n A$ is the \emph{existential closure}.
  \end{itemize}
\end{itemize}
\vspace*{-2ex}
\end{wideslide}

\begin{wideslide}[bm=,toc=]{Examples}
\begin{ex}{7.13}
$p(x,y)$
\end{ex}
\vspace*{-2ex}
\begin{itemize}
\item<2-> Has two free variables ($x$ and $y$).
\item<3-> $\exists y\; p(x,y)$ has one free variable \pause[3] ($x$).
\item<5-> $\forall x \exists y \;p(x,y)$ is closed.
\item<6-> Universal closure: \pause[3] $\forall x \forall y \;p(x,y)$.
\item<8-> Existential closure: \pause[2] $\exists x \exists y \;p(x,y)$.
\end{itemize}
\pause
\begin{ex}{7.14}
$\forall x \; p(x) \land q(x)$
\end{ex}
\vspace*{-2ex}
\begin{itemize}
\item<11-> Bound: \pause[2] $x$ in $p(x)$
\item<13-> Free: \pause[2] $x$ in $q(x)$
\item<15-> Universal Closure: \pause[2] $\forall x(\forall x \; p(x) \land q(x))$
\item<17-> How to write this better?
\item<18-> $\forall y(\forall x \; p(x) \land q(y))$
\end{itemize}
\end{wideslide}
