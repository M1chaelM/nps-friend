\begin{slide}[bm=,toc=]{The Case for Functions in FOL}
FOL can express many properties of relations. 
\\~\\
\pause
{\bf Example:}
\begin{itemize}
\item $A = \forall x \forall y \forall z (p(x,y) \land p (y,z) \imp p(x,z))$ with
\item $\mathcal{I}_A = (\Z,\{<\},\{\})$
\item expresses transitivity for $<$ in the domain of integers.
\end{itemize}
\vspace{2ex}
\pause
{\bf Compare to}:
\[
  for\; all\;  x,y,z: (x < y) \imp (x + z < y + z)
  \]

\vspace{-1ex}
\pause
\begin{itemize}
\item The difference is the $+$ function.
\item How do we express this in first order logic?
\end{itemize}

\end{slide}

\begin{wideslide}[bm=,toc=]{Functions and Terms}
\begin{defn}{9.1}[Ben Ari]
Let $\mathcal{F}$ be a countable set of \emph{function symbols}, where
each symbol has an \emph{arity} denoted by a superscript. 
\\~\\
\pause
\emph{Terms} are defined recursively as follows:
\end{defn}
\vspace{-2ex}
\begin{itemize}
\item<3-> A variable, constant or $0$-ary function symbol is a term.
\item<4-> If $f^n$ is an $n$-ary function symbol $(n > 0)$ and $\{t_1,t_2,...,t_n\}$
are terms, then $f^n(t_1,t_2,...,t_n)$ is a term.
\end{itemize}
\pause[3]
An \emph{atomic formula} is 
\begin{itemize}
\item<6-> an $n$-ary predicate 
\item<7-> followed by a list of $n$ \emph{arguments} 
\item<8-> where each argument $t_i$ is a term: $p(t_1,t_2,...,t_n)$.
\end{itemize}
\end{wideslide}

\begin{wideslide}[bm=,toc=]{Notation}
Conventions we adopt from Ben Ari:
\begin{itemize}
\item<2-> The word `function' refers only to syntactical symbols (replaces word `symbol') 
\item<3-> Functions are denoted by $\{f,g,h\}$ (and subscripts)
\item<4-> Superscript denoting arity is usually omitted.
\item<5-> Constant symbols are now unnecessary, but kept for convenience.
\begin{itemize}
\item<6-> A constant is equivalent to a $0$-ary function.
\end{itemize}
\end{itemize}
~\\
\pause[6]
{\bf Examples of terms:}
\[
  a,\;\; x,\;\; f(a,x),\;\; f(g(x),y),\;\; g(f(a,g(b)))
  \]
\pause
{\bf Examples of atomic formulas:}
\[
  p(a,b),\;\; p(x,f(a,x)),\;\; q(f(a,a), f(g(x),g(x)))
\]
\end{wideslide}

\begin{wideslide}[bm=,toc=]{Interpretations for FOL with Functions}
We must extend the definition of an interpretation to include function symbols:
\pause
\begin{defn}{9.3}[Ben Ari]
Let $U$ be a set of formulas such that 
\begin{itemize}
\item<3-> $\{p_1,...,p_k\}$ are all the predicate symbols, 
\item<4-> $\{f_1^{n_1},...,f_l^{n_l}\}$ are all the functions symbols and
\item<5-> $\{a_1,...,a_m\}$ are all the constant symbols appearing in $U$. 
\end{itemize}
\pause[4]
An \emph{interpretation} $\mathcal{I}$ is a $4$-tuple:
\vspace{-2mm}
\[
  \mathcal{I} = (D, \{R_1,...,R_k\}, \{F_1^{n_1},...,F_l^{n_l}\},\{d_1,...,d_m\})
  \]~\\
\vspace{-8mm}
\pause
\textbf{\emph{consisting of}}
\begin{itemize}
\item<8-> a \emph{non-empty} domain $D$,
\item<9-> $n_i$-ary relations $R_i$ on $D$ to be assigned to the $n_i$-ary predicate symbols $p_i$, 
\item<10-> $n_j$-ary functions $F_j^{n_j}$ on $D$ to be assigned to the function symbol $f_j^{n_j}$, and 
\item<11-> assignments of elements $d_n \in D$ to the constant symbols $a_n$ for $1 \leq n \leq m$.
\end{itemize}
\end{defn}
\end{wideslide}

\begin{wideslide}[bm=,toc=]{Meaning of Atomic Formulas with Functions}
Let $\mathcal{D}_{\mathcal{I}}$ be a map from terms to domain elements such that:
\[
 \mathcal{D}_{\mathcal{I}} (f_i(t_1,...,t_n)) = F_i(\mathcal{D}_{\mathcal{I}}(t_1),...,\mathcal{D}_{\mathcal{I}}(t_n)) 
  \]
\pause
{\bf Given an atomic formula} 
\begin{itemize}
\item $A = p_k(t_1,...,t_n)$
\end{itemize}
\pause
{\bf The truth value is given by}
\begin{itemize}
\item $v_{\sigma_{\mathcal{I}}}(A) = T$ \emph{if and only if} $(\mathcal{D}_{\mathcal{I}}(t_1),...,\mathcal{D}_{\mathcal{I}}(t_n)) \in R_k$
\end{itemize}
\end{wideslide}
\begin{wideslide}[bm=,toc=]{Examples of Interpretations with Functions}
\begin{ex}{9.4}
Consider the formula
\[
  A = \forall x \forall y(p(x,y) \imp p(f(x,a),f(y,a)))
\]
We claim it is true under the interpretation:
\[
  (\Z,\{\leq\}, \{+\}, \{1\}) 
\]
\vspace{-4ex}
\pause
\begin{proof}
~\\For arbitrary $m,n \in \Z$ assigned to $x,y$, $f(x,a)$ and $f(y,a)$ evaluate to:
\vspace{-1ex}
\pause
\[ \mathcal{D}_{\mathcal{I}}(f(x,a)) = +(\mathcal{D}_{\mathcal{I}}(x), \mathcal{D}_{\mathcal{I}}(a)) = +(m,1) = m+1 \]
\pause
\vspace{-4ex}
\[ \mathcal{D}_{\mathcal{I}}(f(y,a)) = +(\mathcal{D}_{\mathcal{I}}(y), \mathcal{D}_{\mathcal{I}}(a)) = +(n,1) = n+1 \]
\pause
Substituting $\leq$ for $p$ yields $(m \leq n) \imp (m+1 \leq n + 1)$, which is
true for $m,n \in \Z$.
\end{proof}
\end{ex}
\end{wideslide}

\begin{wideslide}[bm=,toc=]{Examples of Interpretations with Functions (cont)}
\begin{ex}{9.4}[cont]
Consider the same formula
\[
  A = \forall x \forall y(p(x,y) \imp p(f(x,a),f(y,a)))
\]
Under the interpretation:
\[
  (\Sigma^*,\{\id{suffix}\}, \{\cdot\}, \{\id{tuv}\}) 
\]
\vspace{-4ex}
\end{ex}
\begin{itemize}
\item<2-> $\id{suffix}$ is the relation such that $(s_1,s_2) \in \id{suffix}$ iff $s_1$ is a suffix of $s_2$.
\item<3-> $\cdot$ is the function that concatenates its arguments
\item<4-> $\id{tuv}$ is a string
\item<5-> What is $v_{\mathcal{I}_A}$?
\item<6-> Is $A$ satisfiable?
\item<7-> Valid?
\begin{itemize}
\item<8-> No. Consider $(\Z,\{>\},\{\cdot\},\{-1\})$
\end{itemize}
\end{itemize}
\end{wideslide}



