\documentclass[style=sailor,size=12pt]{powerdot}
\usepackage{epic,array,ecltree,url,calrsfs}
\usepackage[nointegrals]{wasysym}
\usepackage{listings}
\usepackage{epsfig}
\usepackage{amsmath}
\usepackage{amsfonts}
\usepackage{amssymb}
\usepackage{amsxtra}
\usepackage{amsthm}
\usepackage{mlextra} % Must be below ams packages
\usepackage{mathrsfs}
\usepackage{color}
\usepackage{array}
\usepackage{graphicx}
\graphicspath{ {../art/} }
\usepackage{bm}
\usepackage{tikz}
\usepackage{multicol}
\usepackage{enumitem}

\pdsetup{method=normal}

\title{First Order Logic with Functions and Prenex Conjunctive Normal Form}
\author{Foundations of Computer Science}
\date{\today}

\begin{document}
\maketitle
\section[slide=false]{FOL with Functions}
\begin{slide}[bm=,toc=]{The Case for Functions in FOL}
FOL can express many properties of relations. 
\\~\\
{\bf Example:}
\begin{itemize}
\item $A = \forall x \forall y \forall z (p(x,y) \land p (y,z) \imp p(x,z))$ with
\item $\mathcal{I}_A = (\Z,\{<\},\{\})$
\item expresses transitivity for $<$ in the domain of integers.
\end{itemize}
\vspace{2ex}
{\bf Compare to}:
\[
  for\; all\;  x,y,z: (x < y) \imp (x + z < y + z)
  \]

\vspace{-1ex}
\begin{itemize}
\item The difference is the $+$ function.
\item How do we express this in first order logic?
\end{itemize}

\end{slide}

\begin{wideslide}[bm=,toc=]{Functions and Terms}
\begin{defn}{9.1}[Ben Ari]
Let $\mathcal{F}$ be a countable set of \emph{function symbols}, where
each symbol has an \emph{arity} denoted by a superscript. \emph{Terms} are
defined recursively as follows:
\end{defn}
\vspace{-2ex}
\begin{itemize}
\item A variable, constant or $0$-ary function symbol is a term.
\item If $f^n$ is an $n$-ary function symbol $(n > 0)$ and $\{t_1,t_2,...,t_n\}$
are terms, then $f^n(t_1,t_2,...,t_n)$ is a term.
\end{itemize}
An \emph{atomic formula} is an $n$-ary predicate followed by a list of $n$
\emph{arguments} where each argument $t_i$ is a term: $p(t_1,t_2,...,t_n)$.

\end{wideslide}

\begin{wideslide}[bm=,toc=]{Notation}
Conventions we adopt from Ben Ari:
\begin{itemize}
\item The word `function' refers only to syntactical symbols (replaces word `symbol') 
\item Functions are denoted by $\{f,g,h\}$ (and subscripts)
\item Superscript denoting arity is usually omitted.
\item Constant symbols are now unnecessary, but kept for convenience.
\begin{itemize}
\item A constant is equivalent to a $0$-ary function.
\end{itemize}
\end{itemize}
~\\
{\bf Examples of terms:}
\[
  a,\;\; x,\;\; f(a,x),\;\; f(g(x),y),\;\; g(f(a,g(b)))
  \]
{\bf Examples of atomic formulas:}
\[
  p(a,b),\;\; p(x,f(a,x)),\;\; q(f(a,a), f(g(x),g(x)))
\]
\end{wideslide}

\begin{wideslide}[bm=,toc=]{Interpretations for FOL with Functions}
We must extend the definition of an interpretation to include function symbols:
\begin{defn}{9.3}[Ben Ari]
Let $U$ be a set of formulas such that $\{p_1,...,p_k\}$ are all the predicate
symbols, $\{f_1^{n_1},...,f_l^{n_l}\}$ are all the functions symbols and
$\{a_1,...,a_m\}$ are all the constant symbols appearing in $U$. An
\emph{interpretation} $\mathcal{I}$ is a $4$-tuple:
\[
  \mathcal{I} = (D, \{R_1,...,R_k\}, \{F_1^{n_1},...,F_l^{n_l}\},\{d_1,...,d_m\})
  \]
{\bf \emph{consisting of}}
\end{defn}
\vspace{-3ex}
\begin{itemize}
\item<2-> a \emph{non-empty} domain $D$,
\item<3-> an assignment of an $n_i$-ary relation $R_i$ on $D$ to the $n_i$-ary predicate 
      symbols $p_i$ for $1 \leq i \leq k$, 
\item<4-> an assignment of an $n_j$-ary function $F_j^{n_j}$ on $D$ to the
      function symbol $f_j^{n_j}$ for $1 \leq j \leq l$, and 
\item<5-> an assignment of an element $d_n \in D$ to the constant symbol $a_n$ for $1 \leq n \leq m$.
\end{itemize}
\end{wideslide}

\begin{wideslide}[bm=,toc=]{Meaning of Atomic Formulas with Functions}
Let $\mathcal{D}_{\mathcal{I}}$ be a map from terms to domain elements such
that:
\[
 \mathcal{D}_{\mathcal{I}} (f_i(t_1,...,t_n)) = F_i(\mathcal{D}_{\mathcal{I}}(t_1),...,\mathcal{D}_{\mathcal{I}}(t_n)) 
  \]
{\bf Given an atomic formula} 
\begin{itemize}
\item $A = p_k(t_1,...,t_n)$
\end{itemize}
{\bf The truth value is given by}
\begin{itemize}
\item $v_{\sigma_{\mathcal{I}}}(A) = T$ \emph{if and only if} $(\mathcal{D}_{\mathcal{I}}(t_1),...,\mathcal{D}_{\mathcal{I}}(t_n)) \in R_k$
\end{itemize}
\end{wideslide}
\begin{wideslide}[bm=,toc=]{Examples of Interpretations with Functions}
\begin{ex}{9.4}
Consider the formula
\[
  A = \forall x \forall y(p(x,y) \imp p(f(x,a),f(y,a)))
\]
We claim it is true under the interpretation:
\[
  (\Z,\{\leq\}, \{+\}, \{1\}) 
\]
\vspace{-4ex}
\begin{proof}
~\\For arbitrary $m,n \in \Z$ assigned to $x,y$, $f(x,a)$ and $f(y,a)$ evaluate to:
\vspace{-1ex}
\[ \mathcal{D}_{\mathcal{I}}(f(x,a)) = +(\mathcal{D}_{\mathcal{I}}(x), \mathcal{D}_{\mathcal{I}}(a)) = +(m,1) = m+1 \]
\vspace{-4ex}
\[ \mathcal{D}_{\mathcal{I}}(f(y,a)) = +(\mathcal{D}_{\mathcal{I}}(y), \mathcal{D}_{\mathcal{I}}(a)) = +(n,1) = n+1 \]
Substituting $\leq$ for $p$ yields $m \leq n \imp (m+1) \leq n + 1$, which is
true for $m,n \in \Z$.
\end{proof}
\end{ex}
\end{wideslide}

\begin{wideslide}[bm=,toc=]{Examples of Interpretations with Functions (cont)}
\begin{ex}{9.4}[cont]
Consider the same formula
\[
  A = \forall x \forall y(p(x,y) \imp p(f(x,a),f(y,a)))
\]
Under the interpretation:
\[
  (\Sigma^*,\{\id{suffix}\}, \{\cdot\}, \{\id{tuv}\}) 
\]
\vspace{-4ex}
\end{ex}
\begin{itemize}
\item $\id{suffix}$ is the relation such that $(s_1,s_2) \in \id{suffix}$ iff
$s_1$ is a suffix of $s_2$.
\item $\cdot$ is the function that concatenates its arguments
\item $\id{tuv}$ is a string
\item What is $v_{\mathcal{I}_A}$?
\item Is $A$ satisfiable?
\item Valid?
\begin{itemize}
\item<2-> No. Consider $(\Z,\{>\},\{\cdot\},\{-1\})$
\end{itemize}
\end{itemize}
\end{wideslide}

\section[slide=false]{Prenex Conjunctive Normal Form}

\begin{wideslide}[bm=,toc=]{PCNF for First Order Logic}
{\bf Recall:}
\begin{itemize}
\item CNF in propositional logic: conjunction of disjunctions of literals.
\item Clausal form: represented as a set of literals.
\end{itemize}
We now generalize to first order logic:
\begin{defn}{9.9}
A formula is in \emph{prenex conjunctive normal form (PCNF)} iff it is of the
form:
\[
  Q_1x_1\cdots Q_n x_n M
  \]
  where the $Q_i$ are quantifiers and $M$ is a quantifier-free formula in CNF.
  The sequence $Q_1x_1 \cdots Q_n x_n$ is the \emph{prefix} and $M$ is the
  \emph{matrix}.
\end{defn}

\end{wideslide}


\begin{wideslide}[bm=,toc=]{Clausal Form for First Order Logic}
\begin{defn}{9.11}
Let $A$ be a \emph{closed} formula in PCNF whose prefix consists only of 
\emph{universal} quantifiers. The \emph{clausal form} of $A$ consists of
the matrix of $A$ written as a set of clauses.
\end{defn}
\begin{ex}{9.10}
The following formula is in PCNF:
\[
  \forall y \forall z([p(f(y)) \lor \ngg p (g(z)) \lor q(z)] \land [ \ngg q(z)
      \lor \ngg p (g(z)) \lor q(y)])
  \]
\end{ex}
\begin{ex}{9.12}
The formula in example 9.10 is closed and has only universal quantifiers, so it
can be written in clausal form as:
\[
   \{\{p(f(y)), \ngg p(g(z)), q(z)\}, \{ \ngg q(z), \ngg p (g(z)), q(y)\}\}
  \]
\end{ex}
\end{wideslide}
\begin{wideslide}[bm=,toc=]{Skolem's Theorem}
{\bf Recall:} every formula in PL has an equivalent in CNF.
\begin{itemize}
\item Not true for first-order logic.
\item But, for every formula in FOL there is an equisatisfiable formula
in clausal form.
\end{itemize}
\begin{thm}{9.13}[Skolem]
Let $A$ be a closed formula. Then there exists a formula $A'$ in clausal form
such that $A \approx A'$.
\end{thm}

\begin{itemize}
\item $A \approx A'$: $A$ is satisfiable iff $A'$ is satisfiable. 
\item We can find $A'$ using a process called \emph{Skolemization}. 
\item This process uses functions to eliminate existential quantifiers.
\end{itemize}
\end{wideslide}

\begin{wideslide}[bm=,toc=]{Intuition for Skolem's Algorithm}
Existential quantifiers indicate a function-like relationship.
\begin{itemize}
\item Consider $A = \forall x \exists y p(x,y)$
\item ``For all $x$, \emph{produce} a value $y$ associated with that $x$ such
that $p$ is true.''
\item Similar in sense to $y = f(x)$.
\item Replacing gives $A' = \forall x p(x,(f(x))$
\end{itemize}
\vspace{2ex}
Effects of replacement process:
\begin{itemize}
\item Eliminates existential quantifiers.
\item Introduction of function symbols narrows the choice of models.
\item Relations are \emph{many-many}.
\item Functions are relations that are \emph{many-one}.
\item The modification ``finds'' \emph{one} example necessary to satisfy the existence claim.
\item Many other possible examples are excluded.
\end{itemize}
\end{wideslide}

\begin{wideslide}[bm=,toc=]{Skolemization Example}
\begin{ex}{9.14}
~\\Consider the formulas:
\vspace{-2ex}
\[A = \forall x \exists y p(x,y) \text{ and } A' = \forall x p(x,(f(x))\]
\end{ex}
\vspace{-2ex}
Under the following interpretations:
\begin{itemize}
\item $\mathcal{I} = (\Z,\{>\},\{\})$
\item $\mathcal{I}' = (\Z,\{>\},\{F(x) = x + 1\},\{\})$
\item $\mathcal{I}'' = (\Z,\{>\},\{F(x) = x - 1\},\{\})$
\end{itemize}
Note that:
\begin{itemize}
\item $\mathcal{I} \models A$, $\mathcal{I}' \models A$ and $\mathcal{I}''
\models A$ (ignore the functions)
\item $\mathcal{I}'' \models A'$ but $\mathcal{I}' \not \models A'$. 
\item $A \not \equiv A'$ but $A \approx A'$ 
\end{itemize}
\end{wideslide}

\begin{wideslide}[bm=,toc=]{Skolem's Algorithm}
\begin{itemize}
\item Rename bound variables to remove name conflicts.
\item Eliminate Boolean operators except $\ngg, \land, \lor$.
\item Push negation operators inward to atomic formulas. Use
\begin{itemize}
\item $\ngg \forall x A(x) \equiv \exists x \ngg A (x)$ and
\item $\ngg \exists x A(x) \equiv \forall x \ngg A (x)$ 
\end{itemize}
\item Extract quantifiers from the matrix using equivalence laws:
\begin{itemize}
\item $A \;op\; Qx B(x) \equiv Qx(A \;op \; B(x))$ and
\item $QxA(x) \;op\; B \equiv Qx(A(x) \;op \; B)$ 
\end{itemize}
\item Use distributive laws to transform matrix to CNF.
\item Eliminate existential quantifiers by adding Skolem functions.
\begin{itemize}
\item For each existential quantifier, $\exists x$, create a new $n-ary$ function $f$ 
\item $n = $ number of universally quantified variables preceding $\exists x$ in
prenex normal form.
\end{itemize}\end{itemize}
\end{wideslide}


\end{document}

