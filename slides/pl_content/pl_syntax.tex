\begin{wideslide}[bm=,toc=]{Post System definition}
Recall the following Post system defining all well-formed formulae in
propositional logic:
\begin{displaymath}
\begin{array}{lll}
        \begin{array}[t]{l}
        \bid{T}\in P
        \end{array}
&
        \begin{array}[t]{l}
        \bid{F}\in P
        \end{array}
&
	\begin{array}[t]{l}
        x \in P \\
        \hline
        \ngg x \in P
        \end{array} \\[6ex]

	\begin{array}[t]{l}
	x \in P \;\;y \in P \\
	\hline
	x \land y \in P
	\end{array}
&
	\begin{array}[t]{l}
	x \in P \;\;y \in P \\
	\hline
	x \lor y \in P
	\end{array}
&
	\begin{array}[t]{l}
	x \in P \;\;y \in P \\
	\hline
	x \imp y \in P
	\end{array} \\[6ex]

	\begin{array}[t]{l}
        x \in \id{Var} \\
        \hline
        x \in P
        \end{array}
\end{array}
\end{displaymath}
%\item Consequents have larger terms than antecedents except in last rule.
\end{wideslide}

\begin{wideslide}[bm=,toc=]{Atoms and Operators}
\begin{defn}{2.1}[Ben Ari]
Formulas in propositional logic are composed of the following sets of symbols:
\begin{itemize}
\item $\mathcal{P} = \{p,q,r...\}$, the set of \emph{atomic propositions}.
\begin{itemize}
\item $\mathcal{P}$ is unbounded---there is no limit to the number of distinct
propositions.
\item \emph{Atoms} (propositions) will be denoted by lower case letters.
\end{itemize}
\item A finite set of \emph{Boolean operators}:
\begin{center}
\begin{tabular}{|c c||c c|}
\hline
\emph{negation}     & $\ngg$  & \emph{equivalence}  & $\eqv$  \\
\emph{disjunction}  & $\lor$  & \emph{exclusive or} & $\oplus$  \\
\emph{conjunction}  & $\land$ & \emph{nor}          & $\downarrow$  \\
\emph{implication}  & $\imp$   & \emph{nand}         & $\uparrow$  \\
\hline
\end{tabular}
\end{center}

\end{itemize}
\end{defn}
\end{wideslide}

\begin{wideslide}[bm=,toc=]{Formulas as Trees}
\begin{defn}{2.2}[Ben Ari]
A \emph{formula} in propositional logic is an extended binary tree defined recursively:
\begin{itemize}
\item A formula is a \emph{leaf} labeled by an atomic proposition.
\item A formula is a node labeled by $\neg$ with a single child that is a
formula.
\item A formula is a node labeled by one of the binary operators with two
children, both of which are formulas. 
\end{itemize}
\end{defn}
Each formula can be classified by its top-level operator:
\begin{defn}{2.10}[Ben Ari]
Let $A \in \mathcal{F}$. If $A$ is not an atom, the operator labeling
the root of the formula $A$ is the \emph{principal operator} of $A$.
\end{defn}

\end{wideslide}

\begin{wideslide}[bm=,toc=]{Two formulas represented as trees}
\begin{center}
\begin{tabular}{c@{\hspace{4em}}c}
Formula 1 & Formula 2 \\
\setlength{\GapWidth}{8mm}
\setlength{\GapDepth}{8mm}
\begin{bundle}{$\eqv$}
\chunk{
  \begin{bundle}{$\imp$}
  \chunk{$p$}
  \chunk{$q$}
  \end{bundle}
}
\chunk{
  \begin{bundle}{$\imp$}
  \chunk{
    \begin{bundle}{$\ngg$}
    \chunk{$p$}
    \end{bundle}
  }
  \chunk{
    \begin{bundle}{$\ngg$}
    \chunk{$q$}
    \end{bundle}
  }
  \end{bundle}
}
\end{bundle} &
\setlength{\GapWidth}{8mm}
\setlength{\GapDepth}{8mm}
\begin{bundle}{$\imp$}
\chunk{$p$}
\chunk{
  \begin{bundle}{$\eqv$}
  \chunk{$q$}
  \chunk{
    \begin{bundle}{$\ngg$}
    \chunk{
      \begin{bundle}{$\imp$}
      \chunk{$p$}
      \chunk{
        \begin{bundle}{$\ngg$}
        \chunk{$q$}
        \end{bundle}
      }
      \end{bundle}
    }
    \end{bundle}
  }
  \end{bundle}
}
\end{bundle}
\end{tabular}
\end{center}

\end{wideslide}

\begin{wideslide}[bm=,toc=,method=direct]{Formulas as Strings}
\begin{itemize}
\item A \emph{formula} can also be represented as a string.
\item The string form of the formula is obtained by \emph{inorder} traversal of
the tree:
\end{itemize}

\begin{verbatim}
Inorder(F)
  if F is a leaf
    write its label
    return
  let F1 and F2 be the left and right subtrees of F
  Inorder(F1)
  write the label of the root of F
  Inorder(F2)

\end{verbatim}

\end{wideslide}

\begin{wideslide}[bm=,toc=, method=file]{Producing strings from trees}
\begin{center}
\begin{tabular}{c@{\hspace{4em}}c}
Formula 1 & Formula 2 \\
\setlength{\GapWidth}{8mm}
\setlength{\GapDepth}{8mm}
\begin{bundle}{$\eqv$}
\chunk{
  \begin{bundle}{$\imp$}
  \chunk{$p$}
  \chunk{$q$}
  \end{bundle}
}
\chunk{
  \begin{bundle}{$\imp$}
  \chunk{
    \begin{bundle}{$\ngg$}
    \chunk{$p$}
    \end{bundle}
  }
  \chunk{
    \begin{bundle}{$\ngg$}
    \chunk{$q$}
    \end{bundle}
  }
  \end{bundle}
}
\end{bundle} &
\setlength{\GapWidth}{8mm}
\setlength{\GapDepth}{8mm}
\begin{bundle}{$\imp$}
\chunk{$p$}
\chunk{
  \begin{bundle}{$\eqv$}
  \chunk{$q$}
  \chunk{
    \begin{bundle}{$\ngg$}
    \chunk{
      \begin{bundle}{$\imp$}
      \chunk{$p$}
      \chunk{
        \begin{bundle}{$\ngg$}
        \chunk{$q$}
        \end{bundle}
      }
      \end{bundle}
    }
    \end{bundle}
  }
  \end{bundle}
}
\end{bundle}
\end{tabular}
\end{center}
\texttt{Inorder(Formula 1)}: \onslide{2-}{$p \imp q \eqv \ngg p \imp \ngg q$}

\texttt{Inorder(Formula 2)}: \onslide{3-}{$p \imp q \eqv \ngg p \imp \ngg q$}

\onslide{4-}{Our procedure produces an \emph{ambiguous} string representation!}

\end{wideslide}
\begin{wideslide}[bm=,toc=,method=file]{Resolving Ambiguity with Parentheses}

\begin{verbatim}
Inorder(F)
  if F is a leaf
    write its label
    return
  let F1 and F2 be the left and right subtrees of F
  write a left parenthesis '('
  Inorder(F1)
  write the label of the root of F
  Inorder(F2)
  write a right parenthesis ')'

\end{verbatim}

\texttt{Inorder(Formula 1)}: \onslide{2-}{$((p \imp q) \eqv ((\ngg p) \imp (\ngg q)))$}

\texttt{Inorder(Formula 2)}: \onslide{3-}{$(p \imp (q \eqv (\ngg (p \imp (\ngg q)))))$}

\end{wideslide}

\begin{wideslide}[bm=,toc=]{Other methods of resolving ambiguity}
\begin{itemize}
\item Precedence
\begin{enumerate}
\item {\bf $\ngg$} 
\item {\bf $\land,\uparrow$} 
\item {\bf $\lor,\downarrow$} 
\item {\bf $\imp$} 
\item {\bf $\eqv,\oplus$} 
\end{enumerate}
\item Polish Notation 
\begin{itemize}
\item Uses \emph{pre-order traversal}.
\item \texttt{Preorder(Formula 1)}: \onslide{2-}{$\eqv \imp p q \imp \ngg p \ngg q$}

\item \texttt{Preorder(Formula 2)}: \onslide{3-}{$\imp p \eqv q \ngg \imp p \ngg q$}

\end{itemize}
\end{itemize}

\end{wideslide}

\begin{wideslide}[bm=,toc=]{Alternative Notations}
\begin{center}
\begin{tabular}{|c|c|c|c|}
\hline
Operator          & Alternates                  & Java         & Python \\ \hline
\hline
$\ngg$            & $\sim$                      & \p{!}        & \p{not, \~} \\ \hline
$\wedge$          & $\&$                        & \p{\&, \&\&} & \p{and, \&} \\ \hline
$\vee$            &                             & \p{|, ||}    & \p{or,  |} \\ \hline
$\rightarrow$     & $\supset$, $\Rightarrow$    &              & \\ \hline
$\leftrightarrow$ & $\equiv$, $\Leftrightarrow$ &              & \\ \hline
$\neqv$           & $\not\equiv$                & \p{\p{\^{}}} & \p{\p{\^{}}} \\ \hline
$\uparrow$        & $\mid$                      &              & \\ \hline
\end{tabular}
\end{center}

\end{wideslide}
