\begin{wideslide}[bm=,toc=]{Overview}
Resolution is a refutation procedure to check whether a formula is
unsatisfiable.\\
~\\
{\bf \underline{Strategy:}}\\
\begin{enumerate}
\item Convert a formula, $A$, to \emph{clausal form} (producing a set of clauses $S$).
\item Apply the ``resolution rule'' to $S$ to produce an equisatisfiable set of
clauses.
\item Continue until:
\begin{enumerate}
\item The resolution rule cannot be further applied, or
\item an unsatisfiable clause is produced.
\end{enumerate}
\end{enumerate}
Since the resolution rule preserves satisfiability, the production of an
unsatisfiable clause indicates the original formula was unsatisfiable.
\end{wideslide}

\begin{wideslide}[bm=,toc=]{Clausal Form}
Clausal form is a notational variant of CNF.
\begin{itemize}
\item Convenient for describing resolution and DPLL.
\end{itemize}

\begin{defn}{4.5}[Ben Ari]
\end{defn}
\vspace*{-3ex}
\begin{itemize}
\item A \emph{clause} is a set of literals.  
\item A clause is considered to be an implicit disjunction of its literals.
\item A \emph{unit clause} is a clause consisting of exactly one literal.
\item The empty set of literals is the \emph{empty clause}, denoted by $\Box$.
\item A formula in \emph{clausal form} is a set of clauses.
\item A formula is considered to be an implicit conjunction of its clauses.
\item The formula that is the \emph{empty set of clauses} is denoted by
$\emptyset$.
\end{itemize}

The significant difference: now using sets instead of trees.
\begin{itemize}
\item This works because of the properties of CNF. 
\end{itemize}
\end{wideslide}

\begin{wideslide}[bm=,toc=]{Clausal Form Example}
\begin{ex}{4.7}
The CNF formula:
\[ (p \lor r)\land (\ngg q \lor \ngg p \lor q) \land (p \lor \ngg p \lor q) \] is
logically equivalent to its clausal form:
\[ \{\{p,r\},\{\ngg q, \ngg p, q\},\{p, \ngg p, q\} \} \]
\end{ex}
\end{wideslide}

\begin{wideslide}[bm=,toc=]{Clausal Form: Additional Notation}
\textbf{Conventions for clausal form:}
\begin{itemize}
\item Set delimiters are removed from each clause.
\item Negated literals are denoted by a bar over the literal.
\item Example: \[ \{\{p,r\},\{\ngg q, \ngg p, q\},\{p, \ngg p, q\} \} \] is
written as:
\[\{pr, \bar{q}\bar{p}q, p\bar{p}q \}  \]
\vspace{-5mm}
\item Note: product denotes \emph{disjunction}.
\end{itemize}
\vspace{3mm}
\textbf{Literals and their complements:}
\begin{itemize}
\item If $l$ is a literal, $l^c$ is its complement.
\item Example: 
\begin{itemize}
\item If $l = p$ then $l^c = \bar{p}$. 
\item If $l = \bar{p}$ then $l^c = p$.
\end{itemize}
\end{itemize}
\end{wideslide}


\begin{wideslide}[bm=,toc=]{Trivial Clauses}
\begin{defn}{4.8}
A clause is \emph{trivial} if it contains a pair of clashing literals.
\end{defn}
\pause
\textbf{Properties of trivial clauses:}
\begin{itemize}
\item<3-> Recall that a clause is a disjunction of literals.
\item<4-> Furthermore, $l \lor l^c \equiv \top$, and $A \lor \top \equiv \top$.
\item<5-> Therefore a clause containing a pair of clashing literals is always $true$.
\end{itemize} 
\pause[4]
\begin{lem}{4.9}
Let $S$ be a set of clauses and let $C \in S$ be a trivial clause. Then $S -
\{C\}$ is logically equivalent to $S$.
\end{lem}
\pause
\textbf{Rationale:}
\begin{itemize}
\item<8-> Recall that $S$ is a conjunction of clauses.
\item<9-> Recall that $A \land \top \equiv A$.
\item<10-> Therefore removing $C$ does not affect $v_{\mathcal{I}}(S)$ under any
interpretation. 
\end{itemize}
\end{wideslide}

\begin{wideslide}[bm=,toc=]{Empty Clauses and the Empty Set of Clauses}
\begin{lem}{4.10}[Part 1]\label{lem:resolution}
~\\$\Box$, the empty clause, is unsatisfiable. 
\end{lem}
\pause
\textbf{Rationale for Part 1:}
\begin{itemize}
\item Recall that a clause is satisfiable iff there is \emph{some} interpretation
under which \emph{at least one literal} in the clause is true.
\item The empty clause, $\Box$, contains no literals, so therefore no literals with a value of T.
\end{itemize}

\pause
\begin{lem}{4.10}[Part 2]\label{lem:resolution}
~\\$\emptyset$, the empty set of clauses, is valid.
\end{lem}
\pause
\textbf{Rationale for Part 2:}
\begin{itemize}
\item A set of clauses is valid iff \emph{every} clause is true in
\emph{every} interpretation.
\item There are no clauses in $\emptyset$, so this is vacuously true.
\end{itemize}

\end{wideslide}


\begin{wideslide}[bm=,toc=]{Resolution Rule}
Let $C_1, C_2$ be clauses such that $l \in C_1$, $l^c \in C_2$. The clauses
$C_1,C_2$ are said to be clashing clauses and to clash on the complementary
pair of literals $l,l^c$. $C$, the resolvent of $C_1$ and $C_2$, is the
clause:\\
\[ 
  Res(C_1,C_2) = (C_1 - \{l\}) \cup (C_2 - \{l^c\})
\]
\pause
\begin{ex}{4.15}[Ben Ari]
The pair of clauses $C_1 = ab\bar{c}$ and $C_2 = bc\bar{e}$ clash on the pair
of complementary literals $c, \bar{c}$. The resolvent is:

\[ 
  C = (ab\bar{c} - \{\bar{c}\}) \cup (bc\bar{e} - \{c\}) = ab\cup b\bar{e} = ab\bar{e}
  \]

\end{ex}
\pause
\begin{thm}{4.17}\label{thm:resolvent}
The resolvent $C$ is satisfiable if and only if the parent clauses $C_1$ and
$C_2$ are both satisfiable.
\end{thm}
\pause
\begin{itemize}
\item The intuition for Theorem \ref{thm:resolvent} hinges on the fact that one of 
$\{l,l^c\}$ must be \emph{false} and further that $A \lor \bot \equiv A$.
\end{itemize}
\end{wideslide}

\begin{wideslide}[bm=,toc=]{Clauses that Clash on More Than One Literal}
\begin{lem}{4.16}[Ben Ari]
If two clauses clash on more than one literal, their resolvent is a trivial
clause.
\end{lem}
\pause
{ \bf \underline{Intuition}}\\
Consider the following pair of clauses:
\[
  \{l_1, l_2\} \cup C1,  \{l_1^c, l_2^c\} \cup C2
  \]
\pause
Now perform resolution on $\{l_1, l_1^c\}$: 
\pause
\[
  (\{l_1, l_2\} \cup C1 - \{l_1\})\cup (\{l_1^c, l_2^c\} \cup C2 - \{l_1^c \})
\]
\pause
\[
  = (\{l_2\} \cup C1) \cup (\{l_2^c\} \cup C2)
\]
\pause
\[
  = \{l_2, l_2^c\} \cup C1 \cup C2
\]
\pause
Note that the resolvent is a trivial clause.
\end{wideslide}

\begin{wideslide}[bm=,toc=]{The Resolution Procedure}
{ \bf Input:} A set of clauses $S$.\\
{ \bf Output:} $S$ is satisfiable or unsatisfiable.\\
\begin{enumerate}
\item<2-> Let $S$ be a set of clauses and define $S_0 = S$.
\item<3-> Repeat the following steps to obtain $S_{i+1}$ from $S_i$ until
the procedure terminates.
\begin{enumerate}
\item<4-> \emph{Choose} a pair of clashing clauses $\{C_1,C_2\} \subseteq S_i$ that
has not been chosen before.
\item<5->Compute $C = Res(C_1,C_2)$ according to the resolution rule.
\item<6->If $C$ is not a trivial clause, let $S_{i + 1} = S_i \cup \{C\}$;
otherwise, $S_{i+1} = S_i$.
\end{enumerate}
\end{enumerate}
\pause[6]
Terminate the procedure if:
\begin{itemize}
\item Either $C = \Box$, or
\item All pairs of clashing clauses have been resolved. 
\end{itemize}

\end{wideslide}

\begin{wideslide}[bm=,toc=]{Resolution Example}
\begin{ex}{4.19}
Consider the set of clauses:
\[
  S  = S_0 = \{p,\bar{p}q,\bar{r},\bar{p}\bar{q}r\}
  \]
\end{ex}
\begin{enumerate}
\item<2-> $Res(\bar{r},\bar{p}\bar{q}r) = \bar{p}\bar{q}$, which is not trivial.
\item<3-> $S_1  = \{p,\bar{p}q,\bar{r},\bar{p}\bar{q}r, \bar{p}\bar{q}\}$
\item<4-> $Res(\bar{p}q,\bar{p}\bar{q}) = \bar{p}$, which is not trivial.
\item<5-> $S_2  = \{p,\bar{p}q,\bar{r},\bar{p}\bar{q}r, \bar{p}\bar{q}, \bar{p}\}$
\item<6-> $Res(p,\bar{p}) = \Box $, so we are finished.
\end{enumerate}
\pause[6]
{\bf Conclusion:} $S$ is unsatisfiable.\\
\pause ~\\
Note that the algorithm must \emph{choose} which clauses to resolve.
\end{wideslide}

\begin{wideslide}[bm=,toc=]{Figure 4.1: Resolution refutation as a tree}
\unitlength=1.0pt
\begin{center}
\begin{picture}(200,140)
\put(130,140){
  \put( -10,   0){\makebox(20,10){$\Box$}}
  \put( -70, -40){\makebox(20,10){$\bar{p}$}}
  \put(  50, -40){\makebox(20,10){$p$}}
  \put(-110, -90){\makebox(20,10){$\bar{p}\bar{q}$}}
  \put( -30, -90){\makebox(20,10){$\bar{p}q$}}
  \put(-130,-140){\makebox(20,10){$\bar{p}\bar{q}r$}}
  \put(- 90,-140){\makebox(20,10){$\bar{r}$}}
  \put(   0,   0){\line(-2,-1){58}}
  \put(   0,   0){\line( 2,-1){58}}
  \put( -60, -40){\line(-1,-1){38}}
  \put( -60, -40){\line( 1,-1){38}}
  \put(-100, -90){\line(-1,-2){18}}
  \put(-100, -90){\line( 1,-2){18}}
}
\end{picture}
\end{center}
\end{wideslide}
