\section[slide=false]{Semantics}
\begin{wideslide}[bm=,toc=]{Figure 13.1: State transition diagram}
\begin{defn}{13.4}[Ben Ari]
A \emph{state transition diagram} is a directed graph. The nodes are
\emph{states} and the edges are \emph{transitions}. Each state is
labeled with a set of propositional literals such that clashing
literals do not appear in any state.
\end{defn}
\unitlength=1.3pt
\begin{center}
\begin{picture}(140,105)
\put(0,60){
  \put(10,10){\state{\shortstack{$p$\\$\ngg q$}}{$s_{0}$}}
  \put(30,20){\vector(1,0){39}}
  \put(70,10){\state{\shortstack{$p$\\$q$}}{$s_{1}$}}
  \put(77, 9){\line(0,-1){4}}
  \put(65, 7){\oval(24,12)[b]}
  \put(65, 7){\oval(24,12)[tl]}
  \put(64,13){\vector(1,0){8}}
  \put(89,25){\vector(1,0){41}}
  \put(131,15){\vector(-1,0){41}}
  \put(130,10){\state{\shortstack{$\ngg p$\\$q$}}{$s_{2}$}}
  \put(127,35){\oval(26,20)[tr]}
  \put(127,45){\line(-1,0){94}}
  \put(33, 35){\oval(27,20)[tl]}
  \put(140,33){\vector(0,-1){0}}
}
\put(70,0){
\put(0,10){\state{\shortstack{$\ngg p$\\$\ngg q$}}{$s_{3}$}}
\put(10, 9){\line(0,-1){3}}
\put(22, 7){\oval(24,12)[b]}
\put(22, 7){\oval(24,12)[tr]}
\put(24,13){\vector(-1,0){6}}
\put(10,69){\vector(0,-1){38}}
\put(18,28){\vector(1,1){44}}
}
\end{picture}
\end{center}
\end{wideslide}

\begin{wideslide}[bm=,toc=]{Interpretations: PTL}
\begin{defn}{13.8}[Ben Ari]
An \emph{interpretation} $\mathcal{I}$ for a formula $A$ in PTL
is a pair $(\mathcal{S},\rho)$, where $\mathcal{S} = \{s_1,...,s_n\}$
is a set of states each of which is an assignment of truth values to the atomic
propositions in $A$, $s_i: \mathcal{P} \to \{T,F\}$, and $\rho$ is a binary
relation on the states, $\rho \subseteq S \times S$.
\end{defn}
Note that $(s_1,s_2) \in \rho$ can be written functionally as $s_2 \in \rho(p_1)$.
\vspace{2ex}
\begin{multicols}{2}
In graphical representations:
\begin{itemize}
\item states are usually labeled only with the atomic propositions that
are assigned $T$.
\item Atoms not shown are assigned $F$.
\end{itemize}
\unitlength=1pt
\begin{center}
\begin{picture}(160,105)
\put(0,60){
  \put(10,10){\state{$p$}{$s_{0}$}}
  \put(30,20){\vector(1,0){39}}
  \put(70,10){\state{\shortstack{$p$\\$q$}}{$s_{1}$}}
  \put(77, 9){\line(0,-1){4}}
  \put(65, 7){\oval(24,12)[b]}
  \put(65, 7){\oval(24,12)[tl]}
  \put(64,13){\vector(1,0){8}}
  \put(89,25){\vector(1,0){41}}
  \put(131,15){\vector(-1,0){41}}
  \put(130,10){\state{$q$}{$s_{2}$}}
  \put(127,35){\oval(26,20)[tr]}
  \put(127,45){\line(-1,0){94}}
  \put(33, 35){\oval(27,20)[tl]}
  \put(140,33){\vector(0,-1){0}}
}
\put(70,0){
\put(0,10){\state{}{$s_{3}$}}
\put(10, 9){\line(0,-1){3}}
\put(22, 7){\oval(24,12)[b]}
\put(22, 7){\oval(24,12)[tr]}
\put(24,13){\vector(-1,0){6}}
\put(10,69){\vector(0,-1){38}}
\put(18,28){\vector(1,1){44}}
}
\end{picture}
\end{center}
\end{multicols}
\end{wideslide}

\begin{wideslide}[bm=,toc=]{Graphical and Non-graphical Representations of $\mathcal{S}$ and $\rho$}
\begin{ex}{13.9}[Ben Ari]
\end{ex}
\begin{multicols}{2}
{\bf Graphical:}
\vspace{-4ex}
\unitlength=1.2pt
\begin{center}
\begin{picture}(160,120)
\put(0,60){
  \put(10,10){\state{$p$}{$s_{0}$}}
  \put(30,20){\vector(1,0){39}}
  \put(70,10){\state{\shortstack{$p$\\$q$}}{$s_{1}$}}
  \put(77, 9){\line(0,-1){4}}
  \put(65, 7){\oval(24,12)[b]}
  \put(65, 7){\oval(24,12)[tl]}
  \put(64,13){\vector(1,0){8}}
  \put(89,25){\vector(1,0){41}}
  \put(131,15){\vector(-1,0){41}}
  \put(130,10){\state{$q$}{$s_{2}$}}
  \put(127,35){\oval(26,20)[tr]}
  \put(127,45){\line(-1,0){94}}
  \put(33, 35){\oval(27,20)[tl]}
  \put(140,33){\vector(0,-1){0}}
}
\put(70,0){
\put(0,10){\state{}{$s_{3}$}}
\put(10, 9){\line(0,-1){3}}
\put(22, 7){\oval(24,12)[b]}
\put(22, 7){\oval(24,12)[tr]}
\put(24,13){\vector(-1,0){6}}
\put(10,69){\vector(0,-1){38}}
\put(18,28){\vector(1,1){44}}
}
\end{picture}
\end{center}
\columnbreak

{\bf Non-Graphical:}
\begin{align*}
s_0(p) = T&,\;\; s_0(q) = F\\
s_1(p) = T&,\;\; s_1(q) = T\\
s_2(p) = F&,\;\; s_2(q) = T\\
s_3(p) = F&,\;\; s_3(q) = F\\
\rho(s_0)&= \{s_1,s_2\} \\
\rho(s_1)&= \{s_1,s_2,s_3\}\\
\rho(s_2)&= \{s_1\}\\
\rho(s_3)&= \{s_2,s_3\}\\
\end{align*}

\end{multicols}
\end{wideslide}

\begin{wideslide}[bm=,toc=]{Models of Time}
\begin{itemize}
\item Placing restrictions on $\rho$, the transition function, produces
different variations of temporal logic.
\item These variations have practical significance.
\begin{itemize}
\item Contrast with PL or FOL.
\end{itemize}
\item We focus on three:
\begin{itemize}
\item Reflexivity
\item Transitivity
\item Linearity
\end{itemize}
\item The combination produces \emph{Linear Temporal Logic}.
\item Especially useful in computer science.
\end{itemize}
\end{wideslide}


\begin{wideslide}[bm=,toc=]{Reflexivity}
\begin{defn}{13.16}[Ben Ari]
An interpretation $\mathcal{I} = (\mathcal{S},\rho)$ is \emph{reflexive}
iff $\rho$ is a reflexive relation: for all $s \in \mathcal{S}$, $(s,s) \in
\rho$, or $s \in \rho(s)$ in functional notation.
\end{defn}
{\bf Impact of Restriction:}
\begin{itemize}
\item $\Diamond running$: ``eventually the programming is in the state
`running'.
\item If it is running in the current state, the intuitive meaning is true.
\begin{itemize}
\item That is, intuitively, runnning \emph{now} $\imp$ running \emph{eventually}.
\end{itemize}
\item The above holds if $\rho$ is reflexive.
\end{itemize}
\begin{thm}{13.17}
An interpretation with a reflexive transition relation is characterized by the
formula $\Box A \imp A$ (or, by duality, by the formula $A \imp \Diamond A$).
\end{thm}
\end{wideslide}

\begin{wideslide}[bm=,toc=]{Transitivity}
\begin{defn}{13.18}[Ben Ari]
An interpretation $\mathcal{I} = (\mathcal{S},\rho)$ is \emph{transitive}
iff $\rho$ is a transitive relation: for all $s_1,s_2,s_3 \in \mathcal{S}$, 
$s_2 \in \rho(s_1) \land s_3 \in \rho(s_2) \imp s_3 \in \rho(s_1)$.
\end{defn}
{\bf Impact of Restriction:}
\begin{itemize}
\item Requiring transitivity again produces an intuitive result.
\item Assume the following is proven:
\begin{itemize}
\item $s_2 \models running$ for $s_2 \in \rho(s_1) \imp s_1 \models \Diamond running$
\item $s_3 \models running$ for $s_3 \in \rho(s_2) \imp s_2 \models \Diamond running$
\end{itemize}
\item We expect that $s_3 \models running$ should prove $s_1 \models \Diamond running$ 
\end{itemize}
\begin{thm}{13.17}
An interpretation with a transitive transition relation is characterized by the
formula $\Box A \imp \Box \Box A$ (or by the formula 
$\Diamond \Diamond A \imp \Diamond A$).
\end{thm}
\end{wideslide}
\begin{wideslide}[bm=,toc=]{Properties of Reflexive, Transitive Interpretations}
{\bf Corollary 13.21}
In an interpretation that is both reflexive and transitive:
\[
\models \Box A \eqv \Box \Box A
\]
\[and\]
\[
\models \Diamond A \eqv \Diamond \Diamond A
\]

\end{wideslide}

\begin{wideslide}[bm=,toc=]{Truth in PTL}
Let $A$ be a formula in PTL, and $\rho^*$ be the reflexive, transitive closure
of $\rho$.\\
$\nu_{\mathcal{I},s}(A)$, the {\em truth value\/} of $A$ in $s$, is defined recursively as follows:
\begin{itemize}
\item If $A$ is $p\in\mathcal{P}$ then $\nu_{\mathcal{I},s}(A) = s(p)$.
\item If $A$ is $\neg A'$ then $\nu_{\mathcal{I},s}(A) = T$ iff $\nu_{\mathcal{I},s}(A') = F$.
\item If $A$ is $A'\vee A''$ then $\nu_{\mathcal{I},s}(A) = T$ iff $\nu_{\mathcal{I},s}(A') = T$ or
$\nu_{\mathcal{I},s}(A'') = T$.

and similarly for the other Boolean operators.

\item If $A$ is $\Box A'$ then $\nu_{\mathcal{I},s}(A) = T$ iff $\nu_{\mathcal{I},s}(A') = T$ for
all states $s'\in\rho^*(s)$.

\item If $A$ is $\Diamond A'$ then $\nu_{\mathcal{I},s}(A) = T$ iff $\nu_{\mathcal{I},s}(A') = T$ for
some state $s'\in\rho^*(s)$.

\end{itemize}
{\bf Notation:}\\
\begin{itemize}
\item We abbreviate $\nu_{\mathcal{I},s}(A)$ as $s\models_\mathcal{I} A$. 
\item When interpretation $\mathcal{I}$ is clear from the context, it is omitted
giving: 
\item $s\models A$ iff $\nu_s(A) = T$.
\end{itemize}
\end{wideslide}

\begin{wideslide}[bm=,toc=]{Example: Computing Truth Value}
\begin{ex}{13.11}[Modified]
Computing truth value for $\Box p \lor \Box q$ for each state
$s$. 
\end{ex}
\begin{itemize}
\item $s_0 \in \rho^*(s_0)$, but $s_0 \not \models p$ and $s_0 \not \models q$, so 
$s_0 \not \models p \lor q$. Thus, $s_0 \not \models \Box p \lor \Box q$.

\item $\rho^*(s_1) = \{s_1,s_2,s_3\}$. $s_2 \not \models p$ so $s_1 \not \models
\Box p$. Also, $s_3 \not \models q$ so $s_1 \not \models \Box q$. Therefore, $s_1 \not
\models \Box p \lor \Box q$.
\item $\rho^*(s_2) = \{s_2\}$. Since $s_2 \models q$, we have $s_2 \models \Box q$
and $s_2 \models \Box p \lor \Box q$.
\item $\rho^*(s_3) = \{s_3\}$. Since $s_3 \models p$,  $s_3 \models \Box p$ and
therefore $s_3 \models \Box p \lor \Box q$.
\end{itemize}
\begin{center}
\begin{picture}(160,105)
\put(0,60){
  \put(10,10){\state{}{$s_{0}$}}
  \put(30,20){\vector(1,0){39}}
  \put(70,10){\state{\shortstack{$p$\\$q$}}{$s_{1}$}}
  \put(77, 9){\line(0,-1){4}}
  \put(65, 7){\oval(24,12)[b]}
  \put(65, 7){\oval(24,12)[tl]}
  \put(64,13){\vector(1,0){8}}
  \put(90,20){\vector(1,0){40}}
% \put(131,15){\vector(-1,0){41}}
  \put(130,10){\state{$q$}{$s_{2}$}}
  \put(127,35){\oval(26,20)[tr]}
  \put(150,10){\oval(24,12)[b]}
  \put(150,10){\oval(24,12)[tr]}
  \put(155,16){\vector(-1,0){6}}
  \put(127,45){\line(-1,0){94}}
  \put(33, 35){\oval(27,20)[tl]}
  \put(140,33){\vector(0,-1){0}}
}
\put(70,0){
\put(0,10){\state{$p$}{$s_{3}$}}
\put(10, 9){\line(0,-1){3}}
\put(22, 7){\oval(24,12)[b]}
\put(22, 7){\oval(24,12)[tr]}
\put(24,13){\vector(-1,0){6}}
\put(10,69){\vector(0,-1){38}}
%\put(65,70){\vector(-1,-1){45}}
%\put(15,30){\vector(1,1){45}}
}
\end{picture}
\end{center}

\end{wideslide}
