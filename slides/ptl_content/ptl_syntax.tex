\begin{slide}[bm=,toc=]{Propositional Temporal Logic: Syntax}
\begin{defn}{13.2}[Ben Ari]
The syntax of \emph{propositional temporal logic (PTL)} is defined like
the syntax of propositional logic except for the addition of two unary
operators:
\end{defn}
\vspace{-2ex}
\begin{itemize}
\item $\Box$, read \emph{always},
\item $\Diamond$, read \emph{eventually}.
\end{itemize}
The two unary operators have the same precedence as negation.\\~\\
{\bf Informally:}
\begin{itemize}
\item $\Box$ is the universal operator meaning `for \emph{any} time in the
future'
\item $\Diamond$ is the existential operator meaning `for \emph{some} time in
the future.
\end{itemize}
\end{slide}

\begin{wideslide}[bm=,toc=]{The ``Next'' Operator}
{\bf Models of time can be:}
\begin{itemize}
\item continuous (with real-valued intervals) or
\item discrete (series of snapshots or steps).
\item Discrete models are natural for program execution.
\item Useful to express the \emph{next} instant in time.
\item We add another operator for this purpose:
\end{itemize}
\begin{defn}{13.23}[Ben Ari]
The unary operator $\Circle$ is called \emph{next}.
\end{defn}
{\bf Applications of Next:}
\begin{itemize}
\item Useful for deciding properties like satisfiability.
\item Rarely used to express properties of programs.
\item In concurrent programs, we often don't know which step is ``next.''
\end{itemize}
\end{wideslide}
