\begin{slide}[bm=,toc=]{Linear Temporal Logic}

\begin{defn}{13.22}
An interpretation $\mathcal{I} = (\mathcal{S},\rho)$ is \emph{linear}
iff $\rho$ is a function. That is, for all $s \in \mathcal{S}$, 
there is \emph{at most} one $s' \in \mathcal{S}$ such that $s' \in \rho(s)$.
\end{defn}

\begin{defn}{13.22b}[Volpano]
An interpretation $\mathcal{I} = (\mathcal{S},\rho)$ is \emph{linear}
and \emph{total} iff $\rho$ is a total function. That is, for all $s \in \mathcal{S}$, 
there is \emph{exactly} one $s' \in \mathcal{S}$ such that $s' \in \rho(s)$.
\end{defn}

The logic produced by adopting a total, linear transition function and following the truth
assignment given (over the reflexive transitive closure of its transition
    function) is called \emph{Linear Temporal Logic}.

\end{slide}

\begin{wideslide}[bm=,toc=]{Interpretations in Linear Temporal Logic}

The restrictions on $\rho$ imply that interpretation in LTL can be represented
as infinite paths:
\unitlength=1.2pt
\begin{center}
\begin{picture}(200,30)
\put(  0, 0){\state{$\ngg p$}{$s_{0}$}}
\put( 20,10){\vector(1,0){40}}
\put( 60, 0){\state{$\ngg p$}{$s_{1}$}}
\put( 80,10){\vector(1,0){40}}
\put(120, 0){\state{$p$}{$s_{2}$}}
\put(140,10){\vector(1,0){40}}
\put(180, 5){\makebox(20,10){\ldots}}
\end{picture}
\end{center}

Note that the \emph{next} operator is self-dual in a linear interpretation.
\begin{thm}{13.25} A linear interpretation whose relation $\rho$
is a function is characterized by the formula $\Circle A \eqv \ngg
\Circle \ngg A$.
\end{thm}

\end{wideslide}

\begin{wideslide}[bm=,toc=]{Examples of Valid Formulas in LTL}
The following are direct consequences of the restriction on
interpretations in LTL.
\begin{thm}{13.31}
\[
\begin{array}{ll}
\models \Box p \imp \Circle p, & \models \Circle p \imp \Diamond p,\\ 
~&~ \\
\models \Box p \imp \Diamond p, & \models \Circle p \eqv \ngg \Circle \ngg p. \\
\end{array}
\]
\end{thm}
\vspace{2ex}
Other valid formulas establish the following important properties:
\begin{itemize}
\item Inductive
\item Distributive
\item Commutative
\end{itemize}

\end{wideslide}

\begin{wideslide}[bm=,toc=]{Induction in LTL}
The following provides a method for proving properties of LTL formulas
inductively:\\
\begin{thm}{13.32}
\[
\begin{array}{ll}
\models \Box p \eqv p \land \Circle \Box p, & \models \Diamond p \eqv p \lor \Circle \Diamond p,\\ 
\end{array}
\]
\end{thm}
\vspace{2ex}
Induction in LTL is based upon the following valid formula:
\[
\begin{array}{ll}
\models \Box (p \imp \Circle p) \imp (p \imp \Box p)\\
\end{array}
\]
(Recall induction rules from the first section of the course.)

\end{wideslide}

\begin{wideslide}[bm=,toc=]{Distributivity}
$\Circle$ distributes over conjuction and disjunction:
\begin{align*}
&\models \Circle (p \lor q) \eqv (\Circle p \lor \Circle q),\\
&\models \Circle (p \land q) \eqv (\Circle p \land \Circle q).
\end{align*}

$\Box$ and $\Diamond$ behave analogously to FOL quantifiers:
\begin{align*}
&\models \Box (p \land q) \eqv (\Box p \land \Box q), & \models \Diamond (p \lor q) \eqv (\Diamond p \lor \Diamond q), \\
&\models (\Box p \lor \Box q) \imp \Box (p \lor q) & \models \Diamond (p \land q) \imp (\Diamond p \land \Diamond q).
\end{align*}
Similarly, over implication:
\begin{align*}
&\models \Box (p \imp q) \imp (\Box p \imp \Box q),\\
&\models (\Diamond p \imp \Diamond q) \imp \Diamond (p \imp q),\\
&\models \Circle (p \imp q) \eqv (\Circle p \imp \Circle q).
\end{align*}
%
\end{wideslide}
\begin{wideslide}[bm=,toc=]{Commutativity}
$\Circle$ commutes with $\Box$ and $\Diamond$: 
\begin{align*}
&\models \Box \Circle p \eqv \Circle \Box p,\\
&\models \Diamond \Circle p \eqv \Circle \Diamond p.
\end{align*}

$\Box$ and $\Diamond$ commute in one direction: 
\begin{align*}
&\models  \Diamond \Box p \imp \Box \Diamond  p
\end{align*}

Note that if $p$ is eventually always true and $q$ is always eventually
true, we have:
\begin{thm}{13.34}
\begin{align*}
\models (\Diamond \Box p \land \Box \Diamond q) \imp \Box \Diamond(p \land q)
\end{align*}
\end{thm}
\end{wideslide}
