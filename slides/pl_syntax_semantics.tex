\documentclass[style=sailor,size=12pt]{powerdot}
\usepackage{epic,array,ecltree,url,calrsfs}
\usepackage[nointegrals]{wasysym}
\usepackage{listings}
\usepackage{epsfig}
\usepackage{amsmath}
\usepackage{amsfonts}
\usepackage{amssymb}
\usepackage{amsxtra}
\usepackage{amsthm}
\usepackage{mlextra} % Must be below ams packages
\usepackage{mathrsfs}
\usepackage{color}
\usepackage{array}
\usepackage{graphicx}
\graphicspath{ {../art/} }
\usepackage{bm}
\usepackage{tikz}
\usepackage{multicol}
\usepackage{enumitem}

\pdsetup{method=normal}

\title{Propositional Logic: Syntax and Semantics}
\author{Foundations of Computer Science}
\date{\today}


\begin{document}
\maketitle
\section[slide=true]{Syntax}

\begin{wideslide}[bm=,toc=]{Post System definition}
Recall the following Post system defining all well-formed formulae in
propositional logic:
\begin{displaymath}
\begin{array}{lll}
        \begin{array}[t]{l}
        \bid{T}\in P
        \end{array}
&
        \begin{array}[t]{l}
        \bid{F}\in P
        \end{array}
&
	\begin{array}[t]{l}
        x \in P \\
        \hline
        \neg x \in P
        \end{array} \\[6ex]

	\begin{array}[t]{l}
	x \in P \;\;y \in P \\
	\hline
	x \wedge y \in P
	\end{array}
&
	\begin{array}[t]{l}
	x \in P \;\;y \in P \\
	\hline
	x \vee y \in P
	\end{array}
&
	\begin{array}[t]{l}
	x \in P \;\;y \in P \\
	\hline
	x \Rightarrow y \in P
	\end{array} \\[6ex]

	\begin{array}[t]{l}
        x \in \id{Var} \\
        \hline
        x \in P
        \end{array}
\end{array}
\end{displaymath}
%\item Consequents have larger terms than antecedents except in last rule.

\end{wideslide}



\begin{wideslide}[bm=,toc=]{Atoms and Operators}
\begin{defn}{2.1}[Ben Ari]
Formulas in propositional logic are composed of the following sets of symbols:
\begin{itemize}
\item $\mathcal{P} = \{p,q,r...\}$, the set of \emph{atomic propositions}.
\begin{itemize}
\item $\mathcal{P}$ is unbounded---there is no limit to the number of distinct
propositions.
\item \emph{Atoms} (propositions) will be denoted by lower case letters.
\end{itemize}
\item A finite set of \emph{Boolean operators}:
\begin{center}
\begin{tabular}{|c c||c c|}
\hline
\emph{negation}     & $\ngg$  & \emph{equivalence}  & $\eqv$  \\
\emph{disjunction}  & $\lor$  & \emph{exclusive or} & $\oplus$  \\
\emph{conjunction}  & $\land$ & \emph{nor}          & $\downarrow$  \\
\emph{implication}  & $\imp$   & \emph{nand}         & $\uparrow$  \\
\hline
\end{tabular}
\end{center}

\end{itemize}
\end{defn}
\end{wideslide}

\begin{wideslide}[bm=,toc=]{Formulas as Trees}
\begin{defn}{2.2}[Ben Ari]
A \emph{formula} in propositional logic is an extended binary tree defined recursively:
\begin{itemize}
\item A formula is a \emph{leaf} labeled by an atomic proposition.
\item A formula is a node labeled by $\neg$ with a single child that is a
formula.
\item A formula is a node labeled by one of the binary operators with two
children, both of which are formulas. 
\end{itemize}
\end{defn}
Each formula can be classified by its top-level operator:
\begin{defn}{2.10}[Ben Ari]
Let $A \in \mathcal{F}$. If $A$ is not an atom, the operator labeling
the root of the formula $A$ is the \emph{principal operator} of $A$.
\end{defn}

\end{wideslide}

\begin{wideslide}[bm=,toc=]{Two formulas represented as trees}
\begin{center}
\begin{tabular}{c@{\hspace{4em}}c}
Formula 1 & Formula 2 \\
\setlength{\GapWidth}{8mm}
\setlength{\GapDepth}{8mm}
\begin{bundle}{$\eqv$}
\chunk{
  \begin{bundle}{$\imp$}
  \chunk{$p$}
  \chunk{$q$}
  \end{bundle}
}
\chunk{
  \begin{bundle}{$\imp$}
  \chunk{
    \begin{bundle}{$\ngg$}
    \chunk{$p$}
    \end{bundle}
  }
  \chunk{
    \begin{bundle}{$\ngg$}
    \chunk{$q$}
    \end{bundle}
  }
  \end{bundle}
}
\end{bundle} &
\setlength{\GapWidth}{8mm}
\setlength{\GapDepth}{8mm}
\begin{bundle}{$\imp$}
\chunk{$p$}
\chunk{
  \begin{bundle}{$\eqv$}
  \chunk{$q$}
  \chunk{
    \begin{bundle}{$\ngg$}
    \chunk{
      \begin{bundle}{$\imp$}
      \chunk{$p$}
      \chunk{
        \begin{bundle}{$\ngg$}
        \chunk{$q$}
        \end{bundle}
      }
      \end{bundle}
    }
    \end{bundle}
  }
  \end{bundle}
}
\end{bundle}
\end{tabular}
\end{center}

\end{wideslide}

\begin{wideslide}[bm=,toc=,method=direct]{Formulas as Strings}
\begin{itemize}
\item A \emph{formula} can also be represented as a string.
\item The string form of the formula is obtained by \emph{inorder} traversal of
the tree:
\end{itemize}

\begin{verbatim}
Inorder(F)
  if F is a leaf
    write its label
    return
  let F1 and F2 be the left and right subtrees of F
  Inorder(F1)
  write the label of the root of F
  Inorder(F2)

\end{verbatim}

\end{wideslide}

\begin{wideslide}[bm=,toc=, method=file]{Producing strings from trees}
\begin{center}
\begin{tabular}{c@{\hspace{4em}}c}
Formula 1 & Formula 2 \\
\setlength{\GapWidth}{8mm}
\setlength{\GapDepth}{8mm}
\begin{bundle}{$\eqv$}
\chunk{
  \begin{bundle}{$\imp$}
  \chunk{$p$}
  \chunk{$q$}
  \end{bundle}
}
\chunk{
  \begin{bundle}{$\imp$}
  \chunk{
    \begin{bundle}{$\ngg$}
    \chunk{$p$}
    \end{bundle}
  }
  \chunk{
    \begin{bundle}{$\ngg$}
    \chunk{$q$}
    \end{bundle}
  }
  \end{bundle}
}
\end{bundle} &
\setlength{\GapWidth}{8mm}
\setlength{\GapDepth}{8mm}
\begin{bundle}{$\imp$}
\chunk{$p$}
\chunk{
  \begin{bundle}{$\eqv$}
  \chunk{$q$}
  \chunk{
    \begin{bundle}{$\ngg$}
    \chunk{
      \begin{bundle}{$\imp$}
      \chunk{$p$}
      \chunk{
        \begin{bundle}{$\ngg$}
        \chunk{$q$}
        \end{bundle}
      }
      \end{bundle}
    }
    \end{bundle}
  }
  \end{bundle}
}
\end{bundle}
\end{tabular}
\end{center}
\texttt{Inorder(Formula 1)}: \onslide{2-}{$p \imp q \eqv \ngg p \imp \ngg q$}

\texttt{Inorder(Formula 2)}: \onslide{3-}{$p \imp q \eqv \ngg p \imp \ngg q$}

\onslide{4-}{Our procedure produces an \emph{ambiguous} string representation!}

\end{wideslide}
\begin{wideslide}[bm=,toc=,method=file]{Resolving Ambiguity with Parentheses}

\begin{verbatim}
Inorder(F)
  if F is a leaf
    write its label
    return
  let F1 and F2 be the left and right subtrees of F
  write a left parenthesis '('
  Inorder(F1)
  write the label of the root of F
  Inorder(F2)
  write a right parenthesis ')'

\end{verbatim}

\texttt{Inorder(Formula 1)}: \onslide{2-}{$((p \imp q) \eqv ((\ngg p) \imp (\ngg q)))$}

\texttt{Inorder(Formula 2)}: \onslide{3-}{$(p \imp (q \eqv (\ngg (p \imp (\ngg q)))))$}

\end{wideslide}

\begin{wideslide}[bm=,toc=]{Other methods of resolving ambiguity}
\begin{itemize}
\item Precedence
\begin{enumerate}
\item {\bf $\ngg$} 
\item {\bf $\land,\uparrow$} 
\item {\bf $\lor,\downarrow$} 
\item {\bf $\imp$} 
\item {\bf $\eqv,\oplus$} 
\end{enumerate}
\item Polish Notation 
\begin{itemize}
\item Uses \emph{pre-order traversal}.
\item \texttt{Preorder(Formula 1)}: \onslide{2-}{$\eqv \imp p q \imp \ngg p \ngg q$}

\item \texttt{Preorder(Formula 2)}: \onslide{3-}{$\imp p \eqv q \ngg \imp p \ngg q$}

\end{itemize}
\end{itemize}

\end{wideslide}

\begin{wideslide}[bm=,toc=]{Alternative Notations}
\begin{center}
\begin{tabular}{|c|c|c|c|}
\hline
Operator          & Alternates                  & Java         & Python \\ \hline
\hline
$\ngg$            & $\sim$                      & \p{!}        & \p{not, \~} \\ \hline
$\wedge$          & $\&$                        & \p{\&, \&\&} & \p{and, \&} \\ \hline
$\vee$            &                             & \p{|, ||}    & \p{or,  |} \\ \hline
$\rightarrow$     & $\supset$, $\Rightarrow$    &              & \\ \hline
$\leftrightarrow$ & $\equiv$, $\Leftrightarrow$ &              & \\ \hline
$\neqv$           & $\not\equiv$                & \p{\p{\^{}}} & \p{\p{\^{}}} \\ \hline
$\uparrow$        & $\mid$                      &              & \\ \hline
\end{tabular}
\end{center}

\end{wideslide}


\section[slide=true]{Semantics}
\begin{slide}[bm=,toc=]{Interpretations and Truth}
\begin{defn}{2.15}[Ben Ari]
Let $A \in \mathcal{F}$ be a formula and let $\mathcal{P}_A$ be the
  set of atoms appearing in $A$. An \emph{interpretation} for $A$ is a 
total function $\mathcal{I}_A: \mathcal{P}_A \mapsto \{T,F\}$ that assigns
one of the \emph{truth values} $T$ or $F$ to \emph{every} atom in
$\mathcal{P}_A$.
\end{defn}
\begin{defn}{2.16}[Ben Ari]
Let $\mathcal{I}_A$ be an interpretation for $A \in \mathcal{F}$.
$v_{\mathcal{I}_A}(A)$, the \emph{truth value of A under} $\mathcal{I}_A$
is defined inductively on the structure of $A$ as shown in Fig. $2.3$.
\end{defn}
\end{slide}

\begin{wideslide}[bm=,toc=]{Figure 2.3: Truth values of formulas}
\noindent
\begin{minipage}{0.42\textwidth}
\begin{tabular}{|c|c|c|c|}
\hline
$A$ & $v(A_{1})$ & $v(A_{2})$ & $v(A)$ \\ \hline \hline
$\ngg A_{1}$ & $T$ & & $F$  \\
$\ngg A_{1}$ & $F$ & & $T$ \\ \hline
$A_{1} \vee A_{2}$ & $F$ & $F$ & $F$ \\
$A_{1} \vee A_{2}$ & \multicolumn{2}{c|}{otherwise}  & $T$ \\ \hline
$A_{1} \wedge A_{2}$ & $T$ & $T$ & $T$ \\
$A_{1} \wedge A_{2}$ & \multicolumn{2}{c|}{otherwise}  & $F$ \\ \hline
$A_{1} \imp A_{2}$ & $T$ & $F$ & $F$ \\
$A_{1} \imp  A_{2}$ &
   \multicolumn{2}{c|}{otherwise}  & $T$ \\ \hline
\end{tabular}
\end{minipage}
\hspace{0.08\textwidth}
\begin{minipage}{0.42\textwidth}
\begin{tabular}{|c|c|c|c|}
\hline
$A$ & $v(A_{1})$ & $v(A_{2})$ & $v(A)$ \\ \hline \hline
$A_{1} \uparrow A_{2}$ & $T$ & $T$ & $F$ \\
$A_{1} \uparrow  A_{2}$ &
   \multicolumn{2}{c|}{otherwise}  & $T$ \\ \hline
$A_{1} \downarrow A_{2}$ & $F$ & $F$ & $T$ \\
$A_{1} \downarrow  A_{2}$ &
   \multicolumn{2}{c|}{otherwise}  & $F$ \\ \hline
$A_{1} \eqv A_{2}$ & \multicolumn{2}{c|}{$v(A_{1})=v(A_{2})$} & $T$ \\
$A_{1} \eqv A_{2}$ & \multicolumn{2}{c|}{$v(A_{1})\neq v(A_{2})$} &
  $F$ \\ \hline
$A_{1} \neqv A_{2}$ & \multicolumn{2}{c|}{$v(A_{1})\neq v(A_{2})$} & $T$ \\
$A_{1} \neqv A_{2}$ & \multicolumn{2}{c|}{$v(A_{1})=v(A_{2})$} &
  $F$ \\ \hline
\end{tabular}
\end{minipage}
\end{wideslide}
\begin{wideslide}[bm=,toc=]{Evaluating the Truth Value of a Formula}
\begin{ex}{2.17}[Ben Ari]
Let $A = (p \imp q) \eqv (\ngg q \imp \ngg p)$ and let $\mathcal{I}_A$ be
the interpretation:
\begin{center}
\begin{tabular}{ c c }
$\mathcal{I}_A(p) = F$, & $\mathcal{I}_A(q) = T$.
\end{tabular}
\end{center}
The truth value of $A$ can be evaluated inductively using Fig. $2.3$ (see previous
    slide):
\begin{center}
\begin{tabular}{ l l }
\onslide{2-}{$v_\mathcal{I}(p)$}                                     & \onslide{3-}{$= \mathcal{I}_A(p) = F$}\\
\onslide{4-}{$v_\mathcal{I}(q)$}                                     & \onslide{5-}{$= \mathcal{I}_A(q) = T$} \\
\onslide{6-}{$v_\mathcal{I}(p \imp q)$}                              & \onslide{7-}{$= T$} \\
\onslide{8-}{$v_\mathcal{I}(\ngg q)$}                                & \onslide{9-}{$= F$} \\
\onslide{10-}{$v_\mathcal{I}(\ngg p)$}                               & \onslide{11-}{$= T$} \\
\onslide{12-}{$v_\mathcal{I}(\ngg q \imp \ngg p)$}                   & \onslide{13-}{$= T$} \\
\onslide{14-}{$v_\mathcal{I}((p \imp q) \eqv (\ngg q \imp \ngg p))$} & \onslide{15-}{$= T$} \\
\end{tabular}
\end{center}\end{ex}
\end{wideslide}

\begin{wideslide}[bm=,toc=]{Interpretation for a Set of Formulas}
\begin{defn}{2.24}[Ben Ari]
Let $S = \{A_1,...\}$ be a set of formulas and let $\mathcal{P}_S = \bigcup_i
\mathcal{P}_{A_i}$, that is, $\mathcal{P}_S$ is the set of all the atoms that
appear in the formulas of $S$. An \emph{interpretation} for $S$ is a function
$\mathcal{I}_S:\mathcal{P}_S \mapsto \{T,F\}$.
\end{defn}
For any $A_i \in S$, $v_{\mathcal{I}_S}(A_i)$, the \emph{truth value of} $A_i$
\emph{under} $\mathcal{I}_S$ is defined as in Definition $2.16$.

\begin{itemize}
\item Note the truth value of atoms must be consistent across all formulas in $S$.
\item This is guaranteed by defining $\mathcal{P}_S$ as the union of sets of
atoms. 
\end{itemize}
\end{wideslide}

\begin{wideslide}[bm=,toc=]{Logical Equivalence}
\begin{defn}{2.24}[Ben Ari]
Let $A_1,A_2 \in \mathcal{F}$. If $v_{\mathcal{I}}(A_1) = v_{\mathcal{I}}(A_2)$
  for all interpretations $\mathcal{I}$, then $A_1$ is \emph{logically
    equivalent} to $A_2$, denoted $A_1 \equiv A_2$. 
\end{defn}
Note that $\eqv$ and $\equiv$ have distinct meanings.
\begin{itemize}
\item The symbol $\eqv$
\begin{itemize}
\item Is a Boolean operator in propositional logic.
\item Appears in formulas of the logic.
\end{itemize}

\item The symbol $\equiv$
\begin{itemize}
\item Is not a Boolean operator.
\item Describes a property of pairs of formulas in propositional logic.
\item Is part of the \emph{metalanguage} we use to reason about propositional
logic.
\end{itemize}
\end{itemize}

\begin{thm}{2.29}[Ben Ari]
  $A_1 \equiv A_2$ \emph{if and only if} $A_1 \eqv A_2$ \emph{is true in every
    interpretation.}
\end{thm}
\end{wideslide}

\begin{wideslide}[bm=,toc=]{Constant Atomic Propositions}
\begin{itemize}
\item We extend the syntax of Boolean Formulas to include two constant formulas: 
       \emph{true} and \emph{false}.
\item \emph{true}:
\begin{itemize}
\item Also written: $\top$
\item Semantics: $\mathcal{I}(true) = T$
\item Equivalent to: $p \lor \ngg p$
\end{itemize}
\item \emph{false}:
\begin{itemize}
\item Also written: $\bot$
\item Semantics: $\mathcal{I}(false) = F$
\item Equivalent to: $p \land \ngg p$
\end{itemize}
\item These are distinct from the truth values $T$ and $F$ that we use to define
an interpretation: they are constant under all interpretations.
\end{itemize}
\end{wideslide}
\begin{wideslide}[bm=,toc=]{Logically equivalent formulas (1)}
\vspace*{-2ex}
Absorption of Constants:
\vspace*{-2ex}
\begin{displaymath}
\begin{array}{lcl@{\hspace{4em}}lcl}
A \vee  \mathit{true} &\logeqv& \mathit{true} &
  A \wedge \mathit{true} &\logeqv& A \\
A \vee \mathit{false} &\logeqv& A &
  A \wedge \mathit{false} &\logeqv&\mathit{false}\\
A \imp \mathit{true} &\logeqv& \mathit{true} &
  \mathit{true} \imp A &\logeqv& A \\
A\imp \mathit{false} &\logeqv& \ngg A &
  \mathit{false} \imp A &\logeqv& \mathit{true}\\
A\eqv \mathit{true} &\logeqv& A &
A \neqv  \mathit{true}&\logeqv&\ngg A\\
A\eqv \mathit{false} &\logeqv& \ngg A &
A \neqv  \mathit{false}&\logeqv& A\\
\end{array}
\end{displaymath}
\vspace*{-3ex}
Identical Operands:
\begin{displaymath}
\begin{array}{lcl@{\hspace{4em}}lcl}
A &\logeqv& \ngg \ngg A & \\
A &\logeqv& A \wedge A & A &\logeqv& A \vee A \\
A \vee \ngg A &\logeqv& true &
  A \wedge \ngg A &\logeqv& \mathit{false} \\
A \imp A &\logeqv&  \mathit{true}& \\
A \eqv A &\logeqv&  \mathit{true} &
A \neqv A   &\logeqv& \mathit{false} \\
\ngg A &\logeqv& A \uparrow A &
  \ngg A &\logeqv& A \downarrow A\\
\end{array}
\end{displaymath}
\end{wideslide}

\begin{wideslide}[bm=,toc=]{Logically equivalent formulas (2)}
Commutativity:
\begin{displaymath}
\begin{array}{lcl@{\hspace{5em}}lcl}
A\vee B &\logeqv& B\vee A & A\wedge B&\logeqv& B\wedge A\\
A\eqv B &\logeqv& B\eqv A &
   A\neqv B&\logeqv& B\neqv A\\
A\uparrow B &\logeqv& B\uparrow A &
   A\downarrow B&\logeqv& B\downarrow A\\
\end{array}
\end{displaymath}

Associativity:
\begin{displaymath}
\begin{array}{lcl@{\hspace{2em}}lcl}
A \vee (B\vee C) &\logeqv& (A\vee B) \vee C &
   A \wedge (B\wedge C) &\logeqv& (A\wedge B) \wedge C \\
A \eqv (B\eqv C) &\logeqv& (A\eqv B) \eqv C &
   A \neqv(B\neqv C) &\logeqv&
   (A\neqv B)\neqv C\\
\end{array}
\end{displaymath}

Distributivity:
\begin{displaymath}
\begin{array}{lcl}
A\vee (B\wedge C) &\logeqv& (A\vee B)\wedge (A\vee C) \\
   A\wedge (B\vee C) &\logeqv& (A\wedge B) \vee (A\wedge C)\\
\end{array}
\end{displaymath}
\end{wideslide}

\begin{wideslide}[bm=,toc=]{Logically equivalent formulas (3)}
Equivalences for eliminating operators:
\begin{displaymath}
\begin{array}{lcl}
A \eqv B &\logeqv& (A\imp B) \wedge (B\imp A) \\
A \neqv B &\logeqv& \ngg\,(A\imp B) \vee \ngg\,(B\imp A)\\
A \imp B &\logeqv& \ngg A\vee B\\
   A \imp B &\logeqv& \ngg\,(A\wedge \ngg B) \\
A \vee B &\logeqv& \ngg\,(\ngg A \wedge \ngg B) \\
   A \wedge B &\logeqv& \ngg\,(\ngg A\vee \ngg B)\\
A \vee B &\logeqv& \ngg A \imp B  \\
   A \wedge B &\logeqv& \ngg\,(A\imp \ngg B)\\
\end{array}
\end{displaymath}
\end{wideslide}


\begin{wideslide}[bm=,toc=]{Satisfiability and Validity}
\begin{defn}{2.38}[Ben Ari]
Let $A \in \mathcal{F}$.
\begin{itemize}
\item $A$ is \emph{satisfiable} iff $v_{\mathcal{I}}(A) = T$ for \emph{some}
      interpretation $\mathcal{I}$.\\
      An intepretation satisfying $A$ is a \emph{model} for $A$.

\item $A$ is valid, denoted $\models A$, iff $v_{\mathcal{I}}(A)=T$ for
      \emph{all} interpretations $\mathcal{I}$.\\
      A valid propositional formula is also called a \emph{tautology}.

\item $A$ is \emph{unsatisfiable} iff it is... not satisfiable. That is, if
      $v_{\mathcal{I}}(A) = F$ for \emph{all} interpretations $\mathcal{I}$.

\item $A$ is \emph{falsifiable}, denoted $\not\models A$, iff it is not valid,
      that is, if $v_{\mathcal{I}}(A) = F$ for \emph{some} interpretation
      $\mathcal{I}$.
\end{itemize}
\end{defn}
\begin{thm}{2.39}[Ben Ari]
Let $A \in \mathcal{F}$. $A$ is valid if and only if $\ngg A$ is unsatisfiable.
$A$ is satisfiable if and only if $\ngg A$ is falsifiable.
\end{thm}
\end{wideslide}


\begin{wideslide}[bm=,toc=]{Example 2.21}
\begin{center}
\begin{tabular}{|c|c||c|}
\hline
$p$ & $q$ & $p \imp q$ \\ \hline \hline
$T$ & $T$ & $T$  \\ \hline
$T$ & $F$ & $F$  \\ \hline
$F$ & $T$ & $T$  \\ \hline
$F$ & $F$ & $T$  \\ \hline
\end{tabular}
\end{center}
\begin{itemize}
\item Is $p \imp q$ satisfiable?
\item Is $p \imp q$ unsatisfiable?
\item Is $p \imp q$ valid?
\item Is $p \imp q$ falsifiable?
\end{itemize}
\end{wideslide}

\begin{wideslide}[bm=,toc=]{Example 2.22}
\begin{center}
\begin{tabular}{|c|c||c|c|c|c|c|}
\hline
$p$ & $q$ & $p \imp q$ & $\ngg p$ & $\ngg q$ 
& $\ngg q \imp \ngg p$ & $(p\imp q)\eqv  (\ngg q\imp \ngg p)$\\ \hline \hline
$T$ & $T$ & $T$  & $F$ & $F$  & $T$ & $T$  \\ \hline
$T$ & $F$ & $F$ & $F$ & $T$  & $F$ & $T$   \\ \hline
$F$ & $T$ & $T$ & $T$ & $F$  & $T$ & $T$   \\ \hline
$F$ & $F$ & $T$ & $T$ & $T$  & $T$ & $T$   \\ \hline
\end{tabular}
\end{center}
\begin{itemize}
\item Is $(p\imp q)\eqv  (\ngg q\imp \ngg p)$ satisfiable?
\item Is $(p\imp q)\eqv  (\ngg q\imp \ngg p)$ unsatisfiable?
\item Is $(p\imp q)\eqv  (\ngg q\imp \ngg p)$ valid?
\item Is $(p\imp q)\eqv  (\ngg q\imp \ngg p)$ falsifiable?
\end{itemize}
\end{wideslide}

\begin{wideslide}[bm=,toc=]{Example 2.41}
\begin{displaymath}
\begin{array}{|c|c||c|c|c|c|}
\hline
p & q & p \vee q & \ngg p & \ngg q &
(p \vee q) \wedge \ngg p \wedge \ngg q \\ \hline \hline
T & T & T & F & F & F \\ \hline
T & F & T & F & T & F \\ \hline
F & T & T & T & F & F \\ \hline
F & F & F & T & T & F \\ \hline
\end{array}
\end{displaymath}
\begin{itemize}
\item Is $(p\imp q)\eqv  (\ngg q\imp \ngg p)$ satisfiable?
\item Is $(p\imp q)\eqv  (\ngg q\imp \ngg p)$ unsatisfiable?
\item Is $(p\imp q)\eqv  (\ngg q\imp \ngg p)$ valid?
\item Is $(p\imp q)\eqv  (\ngg q\imp \ngg p)$ falsifiable?
\end{itemize}
\end{wideslide}


\begin{wideslide}[bm=,toc=]{Valid, satisfiable, falsifiable and unsatisfiable formulas}
\begin{center}
\unitlength=1.5pt
\begin{picture}(200,100)
\put(0,20){
\put(100,40){\oval(200,80)}
\put(60,30){\oval(80,30)}
\put(35,15){\makebox(50,30){\shortstack[l]%
{True in\\all interpretations.}}}
\put(35,50){\makebox(50,30){\shortstack[l]%
{True in some interpretations;\\false in others.}}}
\put(120,0){\line(0,1){80}}
\put(120,35){\makebox(80,30){\shortstack[l]%
{False in\\all interpretations.}}}
}
\put(101,0){\makebox(40,20)[b]{Falsifiable}}
\put(110,10){\vector(0,1){18}}
\put(130,10){\vector(0,1){18}}
\put(151,0){\makebox(40,20)[b]{Unsatisfiable}}
\put(170,10){\vector(0,1){18}}
\put(50,0){\makebox(50,20)[b]{Satisfiable}}
\put(65,10){\vector(0,1){30}}
\put(85,10){\vector(0,1){18}}
\put(15,0){\makebox(50,20)[b]{Valid}}
\put(40,10){\vector(0,1){30}}
\end{picture}
\end{center}
\end{wideslide}

\begin{wideslide}[bm=,toc=]{Satisfiability of a Set of Formulas}
\begin{defn}{2.42}[Ben Ari]
A set of formulas $U = \{A_1,....\}$ is (\emph{simultaneously})
\emph{satisfiable} iff there exists an interpretation $\mathcal{I}$
such that $v_{\mathcal{I}}(A_i) = T$ for all $i$. The satisfying interpretation
is a \emph{model} of $U$. $U$ is \emph{unsatisfiable} iff for every
interpretation $\mathcal{I}$, there exists an $i$ such that $v_{\mathcal{I}}(A_i) = F$.
\end{defn}
\begin{ex}{2.43}[Ben Ari]
\end{ex}
\vspace*{-3ex}
\begin{itemize}
\item $U_1 = \{p, \ngg p \lor q, q \land r \}$
\item $U_2 = \{p, \ngg p \lor q, \ngg p \}$
\end{itemize}
\end{wideslide}

\begin{wideslide}[bm=,toc=]{Satisfiability of Sets of Formulas:\\ Some Elementary
Theorems}
\begin{thm}{2.44}
{ \em If} $U$ {\em is satisfiable, then so is} $U - \{A_i\}$ {\em for all} $i$. 
\end{thm}
\begin{thm}{2.45}
{ \em If} $U$ {\em is satisfiable and B is valid, then} $U \cup \{B\}$ {\em is
  satisfiable}. 
\end{thm}
\begin{thm}{2.46}
{ \em If} $U$ {\em is unsatisfiable, then for any formula B,} $U \cup \{B\}$ {\em is
  unsatisfiable}. 
\end{thm}
\begin{thm}{2.47}
{ \em If} $U$ {\em is unsatisfiable and for some } $i$, $A_i$ { \em is valid,
  then} $U - \{A_i\}$ {\em is unsatisfiable}. 
\end{thm}
\end{wideslide}

\begin{wideslide}[bm=,toc=]{Logical Consequence}
\begin{defn}{2.48}[Ben Ari]
Let $U$ be a set of formulas and $A$ a formula. $A$ is a \emph{logical
  consequence} of $U$, denoted $U \models A$, iff every model of $U$ is a
  model of $A$.
\end{defn}
The effect of $U$ is to reduce the number of interpretations that must satisfy
$A$.
\begin{ex}{2.49}[Ben Ari]
Let $A = (p \lor r) \land (\ngg q \lor \ngg r)$ and $U = \{p,\ngg q\}$.
Then $U \models A$, but $\not\models A$.
\end{ex}
Note that $\implies$ is an operator in the object language, whereas $\models$
is a symbol for a concept in the metalanguage. However:
\begin{thm}{2.50}
$U \models A$ \emph{if and only if } $\bigwedge\limits_{i} A_i \implies A$. 
\end{thm}
\end{wideslide}

\begin{wideslide}[bm=,toc=]{Logical Consequence}
\begin{ex}{2.52}[Ben Ari]
~\\
$\{p,\ngg q\} \models (p \lor r) \land (\ngg q \lor \ngg r)$, so it follows that\\
$\models (p \land \ngg q) \implies (p \lor r) \land (\ngg q \lor \ngg r)$
(by Theorem 2.50).
\end{ex}
\begin{thm}{2.53}
If $U \models A$ \emph{then} $U \cup \{B\} \models A$ \emph{for any formula} $B$. 
\end{thm}
\begin{thm}{2.54}
If $U \models A$ \emph{ and B is valid then} $U - \{B\} \models A$. 
\end{thm}

\end{wideslide}

\begin{wideslide}[bm=,toc=]{Theories}
\begin{defn}{2.55}[Ben Ari]
Let $\mathcal{T}$ be a set of formulas. $\mathcal{T}$ \emph{is closed under
logical consequence} iff for all formulas $A$, if $\mathcal{T} \models A$ then
$A \in \mathcal{T}$. A set of formulas that is closed under logical consequence
is a \emph{theory}. The elements of $\mathcal{T}$ are \emph{theorems}.
\end{defn}
\begin{defn}{2.56}[Ben Ari]
Let $\mathcal{T}$ be a theory. $\mathcal{T}$ is said to be \emph{axiomatizable} 
iff there exists a set of formulas $U$ such that  $\mathcal{T} = \{A | U \models
A\}$. The set of formulas $U$ are the \emph{axioms} of $\mathcal{T}$. If $U$ is
finite, $\mathcal{T}$ is said to be \emph{finitely axiomatizable}.
\end{defn}
Examples:
\begin{itemize}
\item Euclidean Geometry---consequences of $5$ formulas.
\item Peano Arithmetic---axiomatizable, but not finitely axiomatizable.
\end{itemize}


\end{wideslide}


\end{document}

