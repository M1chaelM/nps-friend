\documentclass[style=sailor,size=12pt]{powerdot}
\usepackage{epic,array,ecltree,url,calrsfs}
\usepackage[nointegrals]{wasysym}
\usepackage{listings}
\usepackage{epsfig}
\usepackage{amsmath}
\usepackage{amsfonts}
\usepackage{amssymb}
\usepackage{amsxtra}
\usepackage{amsthm}
\usepackage{mlextra} % Must be below ams packages
\usepackage{mathrsfs}
\usepackage{color}
\usepackage{array}
\usepackage{graphicx}
\graphicspath{ {../art/} }
\usepackage{bm}
\usepackage{tikz}
\usepackage{multicol}
\usepackage{enumitem}

\def\labelitemi{\textbullet}
\def\labelitemii{$\circ$}

\pdsetup{method=normal}

\title{First Order Logic: Motivation and Applications}
\author{Foundations of Computer Science}
\date{\today}

\begin{document}
\maketitle

\begin{wideslide}[bm=,toc=]{Limitations of Propositional Logic}
Previously, we used propositional logic to simplify the conditional statement:
\begin{program}
if (a < b || (a >= b \&\& c == d)) ...
\end{program}
\pause 
to the equivalent statement
\vspace{-1em}
\begin{program}
if (a < b || c == d) ...
\end{program}
\pause
However, there are other equivalences that propositional logic cannot justify.
\end{wideslide}

\begin{wideslide}[bm=,toc=]{First Order Logic Equivalence Example}
Consider the expression:
\begin{program}
if ((a < b) and (a < c) and (b < c))...
\end{program}
\pause
which can be simplified to
\vspace{-1em}
\begin{program}
if ((a < b) and (b < c))...
\end{program}
\vspace{-1em}
\pause
In propositional logic, this equivalence would be written as:\\~\\
$p \land q \land r \equiv p \land r$\\~\\
\pause
which is not true for the interpretation $v_{\mathcal{I}}(p) = T$,
   $v_{\mathcal{I}}(q) = F$, $v_{\mathcal{I}}(r) = T$
\end{wideslide}

\begin{wideslide}[bm=,toc=]{First Order Logic and Predicates}
First-order logic allows propositions to contain ``\emph{arguments}.''
\begin{itemize}
\item<2-> Called \emph{predicates}
\item<3->Interpretations are no longer restricted to assignments of $T$ or $F$.
\item<4->Predicates can express set-theoretic relations.
\end{itemize}
\pause[4]
\begin{ex}{}[Ullman]
Define the relation $\id{lt}$ such that $\id{lt}(x,y)$ iff $x < y$ and 
let $p = \id{lt}(a,b)$, $q = \id{lt}(a,c)$, and $r = \id{lt}(b,c)$.
\\~\\
\pause
We can now rewrite the equivalent statements as:\\
$(\id{lt}(a,b) \land \id{lt}(a,c) \land \id{lt}(b,c)) \equiv (\id{lt}(a,b) \land \id{lt}(b,c))$
\\~\\
\pause
First order logic allows us to express that $\id{lt}$ is transitive, and further that the above
equivalence holds for \emph{all} predicates that express transitive relations.
\end{ex}
\end{wideslide}

\begin{wideslide}[bm=,toc=]{FOL Application to Access Control Policies}
First order logic can also be used to express access control policies.
\\~\\
\pause
\textbf{Motivation:} 
\begin{itemize}
\item Informal policy statements often invite ambiguity.\footnote{ \footnotesize
J. Halpern and V. Weissman, 
Using First-Order Logic to Reason about Policies, {\em ACM Trans on Information and System Security\/},
(11)4, July 2008.}
\item Difficult to ensure consistency automatically. \pause
\begin{itemize}
\item (How difficult? How do we measure the difficulty?) 
\end{itemize}
\end{itemize}
\pause
~\\
\textbf{Example:} 
\begin{itemize}
\item ``Only librarians may edit the online catalog.''
\begin{itemize}
\item<5-> Forbids anyone who is not a librarian from editing the catalog.
\item<6-> But can a librarian edit the catalog?
\end{itemize}
\end{itemize}
\end{wideslide}

\begin{wideslide}[bm=,toc=]{Policies in first-order logic}
\textbf{Approaches:}
\begin{itemize}
\item<2-> Structured policy languages
\begin{itemize}
\item<3-> Extensible rights Markup Language, Open Digital Rights Language
\item<4-> Permit more formal statements but their semantics are usually in English.
\end{itemize}
\item<2-> First-order logic 
\begin{itemize}
\item<5-> Has a precise semantics.
\item<6-> Encourages precise policy specification. 
\end{itemize}
\end{itemize}
\pause[6]
\textbf{Consider:}
\vspace{-2mm}
\begin{displaymath}
\begin{array}{l}
\forall x(\neg\bid{Librarian}(x))\imp \neg\bid{Permitted}(x,\rid{edit catalog}) \\[1ex]
\forall x(\bid{Librarian}(x))\imp \bid{Permitted}(x,\rid{edit catalog})
\end{array}
\end{displaymath}
\vspace{3mm}
\pause
First formula alone may convey the meaning of the English statement.
\end{wideslide}

\begin{wideslide}[bm=,toc=]{Halpern and Weissman's Access Control Proposal}
\textbf{General form of a policy:}
\begin{displaymath}
\begin{array}{l}
\forall x_1\ldots x_m(f \imp (\neg)\bid{Permitted}(t,t'))
\end{array}
\end{displaymath}
\vspace{-15mm}
\begin{itemize}
\item[] \phantom{empty bullet}
\begin{itemize}
\item $f$ is a first-order formula
\item $t$ denotes a subject and 
\item $t'$ an action performed by that subject.
\end{itemize}
\end{itemize}
\vspace{3mm}
\pause
\textbf{Example:}\\
\vspace{2mm}
``An article may be downloaded if a fee has been paid within the past 6 weeks.''
\vspace{-2mm}
\begin{displaymath}
\begin{array}{l}
\forall i\, \forall t\, \forall a\, ((\bid{PaidFee}(i,t)\wedge (\bid{now} - 6 < t < \bid{now})\wedge \\
\bid{Customer}(i,\bid{now})\wedge\bid{Article}(a) \Rightarrow \bid{Permitted}(i,\rid{download}(a)))
\end{array}
\end{displaymath}
$\bid{now}$ is a constant denoting current time (provided by a global clock).
\end{wideslide}

\begin{wideslide}[bm=,toc=]{Halpern and Weissman's Access Control Proposal}
\textbf{Expressing Permission:}
\begin{itemize}
\item<2-> Let $E$ be an {\em environment\/} of statements (e.g.\ $\bid{Librarian}(\id{Alice})$).
\item<3-> Let $p_1,\ldots ,p_n$ be policies governing subjects.
\item<4-> Subject $t$ can perform action $t'$ iff the following is {\em valid\/}:
\pause[4]
\begin{displaymath}
E\wedge p_1 \wedge \cdots \wedge p_n \imp \bid{Permitted}(t,t')
\end{displaymath}
\vspace{-5mm}
\item<6->  that is, $E\wedge p_1 \wedge \cdots \wedge p_n \wedge \neg\bid{Permitted}(t,t')$ 
is {\em unsatisfiable\/}.
\end{itemize}
\end{wideslide}

\begin{wideslide}[bm=,toc=]{Halpern and Weissman's Access Control Proposal}
\textbf{Checking Consistency:}
\begin{itemize}
\item Policies $p_1,\dots p_n$ and environment $E$ are {\em consistent\/} iff
the following is {\em satisfiable\/}:
\begin{displaymath}
E\wedge p_1 \wedge \cdots \wedge p_n 
\end{displaymath}
\end{itemize}
\pause
\textbf{Difficulty of Checking Consistency:}
\begin{itemize}
\item<3-> As we shall see, neither validity or satisfiability is decidable in first-order logic!
\item<4-> Halpern and Weissman give restrictions on $E$ and $p_1,\ldots ,p_n$ under which
these problems are:
\begin{itemize}
\item<5->decidable,
\item<5->tractable
\item<5->sufficiently expressive to remain useful. 
\end{itemize}
\end{itemize}
\end{wideslide}
\end{document}

