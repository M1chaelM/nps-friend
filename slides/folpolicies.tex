
\documentclass[style=sailor,size=12pt]{powerdot}
\usepackage{epic,array,ecltree,url,calrsfs}
\usepackage[nointegrals]{wasysym}
\usepackage{mlextra}
\usepackage{listings}
\graphicspath{ {../art/} }

\newcommand{\id}[1]{\mbox{\it #1\/}}
\newcommand{\rid}[1]{\mbox{\rm #1}}
\newcommand{\sid}[1]{\mbox{\sf #1}}
\newcommand{\bid}[1]{\mbox{\bf #1}}
\newcommand{\tinysz}[1]{\mbox{\tiny $#1$}}

\usepackage{epsfig}
\usepackage{amsmath}
\usepackage{amsfonts}
\usepackage{amssymb}
\usepackage{amsxtra}
\usepackage{amsthm}
\usepackage{mathrsfs}
\usepackage{color}
\usepackage{array}
\usepackage{graphicx}
\graphicspath{ {../art/} }
\usepackage{bm}
\usepackage{tikz}
\usepackage{multicol}
\usepackage{enumitem}

\renewcommand\qedsymbol{$\blacksquare$}

%Here I define some theorem styles and shortcut commands for symbols I use often
\theoremstyle{definition}

\newtheorem{innerdefn}{Definition}
\newenvironment{defn}[1]
  {\renewcommand\theinnerdefn{#1}\innerdefn}
  {\endinnerdefn}

\newtheorem{innerexample}{Example}
\newenvironment{ex}[1]
  {\renewcommand\theinnerexample{#1}\innerexample}
  {\endinnerexample}

\newtheorem{innerthm}{Theorem}
\newenvironment{thm}[1]
  {\renewcommand\theinnerthm{#1}\innerthm}
  {\endinnerthm}

\newtheorem{innerlem}{Lemma}
\newenvironment{lem}[1]
  {\renewcommand\theinnerlem{#1}\innerlem}
  {\endinnerlem}




%\newtheorem*{defn}{Definition}
%\newtheorem{thm}{Theorem}
%\newtheorem{cor}{Corollary}
%\newtheorem*{rmk}{Remark}
%\newtheorem{lem}{Lemma}
%\newtheorem*{joke}{Joke}
%\newtheorem{ex}{Example}
%\newtheorem*{soln}{Solution}
%\newtheorem{prop}{Proposition}


\newcommand{\lra}{\longrightarrow}
\newcommand{\ra}{\rightarrow}
\newcommand{\surj}{\twoheadrightarrow}
\newcommand{\graph}{\mathrm{graph}}
\newcommand{\bb}[1]{\mathbb{#1}}
\newcommand{\Ell}{\mathscr{L}}
\newcommand{\Z}{\bb{Z}}
\newcommand{\Q}{\bb{Q}}
\newcommand{\R}{\bb{R}}
\newcommand{\C}{\bb{C}}
\newcommand{\N}{\bb{N}}
\newcommand{\M}{\mathbf{M}}
\newcommand{\m}{\mathbf{m}}
\newcommand{\MM}{\mathscr{M}}
\newcommand{\HH}{\mathscr{H}}
\newcommand{\Om}{\Omega}
\newcommand{\Ho}{\in\HH(\Om)}
\newcommand{\bd}{\partial}
\newcommand{\del}{\partial}
\newcommand{\bardel}{\overline\partial}
\newcommand{\textdf}[1]{\textbf{\textsf{#1}}\index{#1}}
\newcommand{\img}{\mathrm{img}}
\newcommand{\ip}[2]{\left\langle{#1},{#2}\right\rangle}
\newcommand{\inter}[1]{\mathrm{int}{#1}}
\newcommand{\exter}[1]{\mathrm{ext}{#1}}
\newcommand{\cl}[1]{\mathrm{cl}{#1}}
\newcommand{\ds}{\displaystyle}
\newcommand{\vol}{\mathrm{vol}}
\newcommand{\cnt}{\mathrm{ct}}
\newcommand{\osc}{\mathrm{osc}}
\newcommand{\LL}{\mathbf{L}}
\newcommand{\UU}{\mathbf{U}}
\newcommand{\support}{\mathrm{support}}
\newcommand{\AND}{\;\wedge\;}
\newcommand{\OR}{\;\vee\;}
\newcommand{\Oset}{\varnothing}
\newcommand{\st}{\ni}
\newcommand{\wh}{\widehat}
\newcommand{\mli}[1]{\mathit{#1}}
\newcommand{\ndiv}{\hspace{-3pt}\not|\hspace{2pt}}


\pdsetup{method=normal}

\begin{document}

\begin{wideslide}[bm=,toc=]{Policies in first-order logic}
\begin{itemize}
\item English lacks precision, often inviting ambiguity:\footnote{
J. Halpern and V. Weissman, 
Using First-Order Logic to Reason about Policies, {\em ACM Trans on Information and System Security\/},
(11)4, July 2008.}
\begin{quote}
``only librarians may edit the online catalog''
\end{quote}
\item Forbids anyone who is not a librarian from editing the catalog.
\item But can a librarian edit the catalog?
\item Structured policy languages (Extensible rights Markup Language and Open Digital Rights Language)
permit more formal statements but their semantics are usually in English.
\item First-order logic has a precise semantics and encourages precise policy specification.
\end{itemize}
\end{wideslide}

\begin{wideslide}[bm=,toc=]{Policies in first-order logic}
\begin{displaymath}
\begin{array}{l}
\forall x(\neg\bid{Librarian}(x)\Rightarrow \neg\bid{Permitted}(x,\rid{edit catalog}) \\[1ex]
\forall x(\bid{Librarian}(x)\Rightarrow \bid{Permitted}(x,\rid{edit catalog})
\end{array}
\end{displaymath}
First formula alone may convey meaning of English statement.
\end{wideslide}

\begin{wideslide}[bm=,toc=]{Policies in first-order logic}
\begin{itemize}
\item An article may be downloaded if a fee has been paid within the past 6 weeks:
\begin{displaymath}
\begin{array}{l}
\forall i\, \forall t\, \forall a\, ((\bid{PaidFee}(i,t)\wedge (\bid{now} - 6 < t < \bid{now})\wedge \\
\bid{Customer}(i,\bid{now})\wedge\bid{Article}(a) \Rightarrow \bid{Permitted}(i,\rid{download}(a)))
\end{array}
\end{displaymath}
$\bid{now}$ is a constant denoting current time (provided by a global clock).
\end{itemize}
\end{wideslide}

\begin{wideslide}[bm=,toc=]{Policies in first-order logic}
\begin{itemize}
\item General form of a policy:
\begin{displaymath}
\begin{array}{l}
\forall x_1\ldots x_m(f \Rightarrow (\neg)\bid{Permitted}(t,t'))
\end{array}
\end{displaymath}
$f$ is a first-order formula, $t$ denotes a subject and $t'$ an action performed by that subject.
\item Let $E$ be an {\em environment\/} of statements (e.g.\ $\bid{Librarian}(\id{Alice})$).
\item Let $p_1,\ldots ,p_n$ be policies governing subjects.
\item Subject $t$ can perform action $t'$ iff 
\begin{displaymath}
E\wedge p_1 \wedge \cdots \wedge p_n \Rightarrow \bid{Permitted}(t,t')
\end{displaymath}
is {\em valid\/}; i.e.\ $E\wedge p_1 \wedge \cdots \wedge p_n \wedge \neg\bid{Permitted}(t,t')$ 
is {\em unsatisfiable\/}.
\end{itemize}
\end{wideslide}

\begin{wideslide}[bm=,toc=]{Policies in first-order logic}
\begin{itemize}
\item Policies $p_1,\dots p_n$ and environment $E$ are {\em consistent\/} iff
\begin{displaymath}
E\wedge p_1 \wedge \cdots \wedge p_n 
\end{displaymath}
is {\em satisfiable\/}.
\item As we shall see, neither validity ot satisfiability is decidable in first-order logic!
\item Halpern and Weissman give restrictions on $E$ and $p_1,\ldots ,p_n$ under which
these problems are decidable without compromising the utility of first-order logic for policies.
\end{itemize}
\end{wideslide}

\begin{wideslide}[bm=,toc=]{Policies in first-order logic}
\begin{itemize}
\item Recall the policy:
\begin{displaymath}
\begin{array}{l}
\forall i\, \forall t\, \forall a\, ((\bid{PaidFee}(i,t)\wedge (\bid{now} - 6 < t < \bid{now})\wedge \\
\bid{Customer}(i,\bid{now})\wedge\bid{Article}(a) \Rightarrow \bid{Permitted}(i,\rid{download}(a)))
\end{array}
\end{displaymath}
%$\bid{now}$ is a constant denoting current time 
\item {\bf Customer} is parameterized on time.
\item Its truth varies across time (dynamic truth).
\item Contrast with {\bf Article} whose truth is static.
\item A more natural way to express this policy uses {\em temporal logic\/}.
\end{itemize}
\end{wideslide}

\end{document}
\begin{center}
\begin{picture}(255,50)
%\begin{picture}(255,25)
\put(  -15, -20){\makebox(2,12){\small time:}}
\put(  0, 0){\state{\tinysz{n=0}}{$s_{0}$}}
\put(  10, -20){\makebox(2,12){\small $t_0$}}
\put( 20,10){\vector(1,0){40}}
\put( 60, 0){\state{\tinysz{n=1}}{$s_{1}$}}
\put(  70, -20){\makebox(2,12){\small $t_1$}}
\put( 80,10){\vector(1,0){40}}
\put(120, 0){\state{\tinysz{n=2}}{$s_{2}$}}
\put( 130, -20){\makebox(2,12){\small $t_2$}}
\put(140,10){\vector(1,0){40}}
\put(180, 0){\state{\tinysz{n=3}}{$s_{3}$}}
\put( 198, -20){\makebox(2,12){\small $t_3\ldots$}}
%\put(200,10){\vector(1,0){40}}
%\put(245,5){\makebox(10,10){$\cdots$}}
	\put(190,0){\line(0,-1){3}}
	\put(202,-2){\oval(24,12)[b]}
	\put(202,-2){\oval(24,12)[tr]}
	\put(203,4){\vector(-1,0){5}}
	%\put(70,0){\line(0,-1){3}}
	%\put(82,-2){\oval(24,12)[b]}
	%\put(82,-2){\oval(24,12)[tr]}
	%\put(83,4){\vector(-1,0){5}}
\end{picture}
\end{center}
\end{wideslide}

\begin{wideslide}[bm=,toc=]{Why Temporal Logic?}
\begin{displaymath}
\begin{array}{|l|l|}
\hline
\bid{First-order logic} & \bid{Temporal logic analog} \\\hline
\forall i\geq 0.\, n=_{t_i}2\;\rid{is False} & s_0\not\models\Box n = 2\\\hline
\exists i\geq 0.\, n=_{t_i}2\;\rid{is True}  & s_0\models\Diamond n = 2\\\hline
\forall i.\exists j.\, n=_{t_j}3 \wedge j\geq i\;\rid{is True} & s_0\models\Box\Diamond n = 3\\\hline
\exists i.\exists j.\, n =_{t_i}2 \wedge n=_{t_j}3 \wedge i > j\;\rid{is False} & 
s_0\not\models\Diamond (n = 3\wedge\Diamond n = 2)\\\hline
\exists i.\exists j.\, n =_{t_i}2 \wedge n=_{t_j}3 \wedge i < j\;\rid{is True} & 
s_0\models\Diamond (n = 2\wedge\Diamond n = 3)\\\hline
\exists i.\exists j.\, n =_{t_i}2 \wedge n=_{t_j}3\;\rid{is True} & 
\rid{no temporal analog as it} \\
 & \rid{must order}\;t_i\;\rid{and}\;t_j\\\hline
\end{array}
\end{displaymath}
\end{wideslide}

\begin{wideslide}[bm=,toc=]{Why Temporal Logic?}
\begin{enumerate}
\item FOL doesn't care about temporal ordering.  Can write
\begin{displaymath}
\exists i.\exists j.\, n =_{t_i}2 \wedge n=_{t_j}3
\end{displaymath}
which is true for the C program.
\item Says nothing about which predicate became true first.
\item Hence it has no analog in temporal logic.
\item Temporal logic cares about temporal ordering of events.
\item Awkward predicates in FOL as they're parameterized on time.
\item But time is not referenced anywhere in the C program.
\item Predicates not parametrized on time in temporal logic.
\end{enumerate}
\end{wideslide}
\end{document}
