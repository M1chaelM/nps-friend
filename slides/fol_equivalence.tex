\begin{wideslide}[bm=,toc=]{Definition of Logical Equivalence}
\begin{defn}{7.28}[Ben Ari]
\end{defn}
\vspace{-2ex}
\begin{itemize}
\item<2-> Let $U = \{A_1,A_2\}$ be a pair of closed formulas. 
\item<3-> $A_1$ is \emph{logically equivalent} to $A_2$ iff \pause[3] $v_{{\mathcal{I}_U}}(A_1) =
v_{{\mathcal{I}_U}}(A_2)$ for \emph{all} $\mathcal{I}_U$.
\begin{itemize}
\item<5-> Notation: $A_1 \equiv A_2$.
\end{itemize}
\item<6-> Let $A$ be a closed formula and $U$ a set of closed formulas.
\item<7-> $A$ is a \emph{logical consequence} of $U$ iff \pause[4] for all interpretations
$\mathcal{I}_{U \cup \{A\}}$, $v_{\mathcal{I}_{U \cup \{A\}}}(A_i) = T$ for
all $A_i \in U$ implies $v_{\mathcal{I}_{U \cup \{A\}}}(A) = T$. 
\begin{itemize}
\item<9-> Notation: $U \models A$.
\end{itemize}
\end{itemize}
\vspace{-5mm}
\pause[2]
\begin{thm}{7.29}
Let $A,B$ be closed formulas and $U = \{A_1,...,A_n\}$ be a set of
closed formulas. Then:
\end{thm}
\vspace{-5ex}
\begin{tabbing}
closed formulas. Then X \= \kill
\> $A \equiv B$ iff $\models A \eqv B$\\
\> $U \models A$ iff $\models (A_1 \land \cdots \land A_n) \imp A$
\end{tabbing}

\end{wideslide}

\begin{wideslide}[bm=,toc=]{Duality}
The two quantifiers are duals:
\pause
\begin{tabbing}
The two quantifiers are dua \= \kill
\> $\models \forall x A(x) \eqv \ngg \exists x \ngg A(x)$.\\
\> ~\\ 
\pause
\> $\models \exists x A(x) \eqv \ngg \forall x \ngg A(x)$.
\end{tabbing}
\pause
\begin{itemize}
\item It is only necessary to define one of the two.
\item If $\forall$ is defined, $\exists$ can be considered an abbreviation of
      $\ngg \forall \ngg$.
\end{itemize}

\end{wideslide}

\begin{wideslide}[bm=,toc=]{Commutativity}
Quantifiers of the same type commute:
\pause
\begin{tabbing}
Quantifiers of the same \= \kill
\> $\models \forall x \forall y A(x,y) \eqv \forall y \forall x A(x,y)$.\\
\pause
\> ~\\ 
\> $\models \exists x \exists y A(x,y) \eqv \exists y \exists x A(x,y)$.\\
\end{tabbing}

\pause
but $\forall$ and $\exists$ commute only in one direction:
\pause
\begin{tabbing}
Quantifiers of the same \= \kill
\> $\models \exists x \forall y A(x,y) \imp \forall y \exists x A(x,y)$.\\
\end{tabbing}

\end{wideslide}

\begin{wideslide}[bm=,toc=]{Distributivity}
\begin{tabbing}
Universal \= \kill
Universal quantifiers distribute over conjunction:  \> \\[2mm]
\pause
\>$\models \forall x(A(x) \land B(x)) \eqv \forall x A(x) \land \forall x B(x)$.\\
~\\
\pause
Existential quantifiers distribute over disjunction:\\[2mm]
\pause
\>$\models \exists x(A(x) \lor B(x)) \eqv \exists x A(x) \lor \exists x B(x)$.\\
~\\
\pause
Distributing universal over disjunction goes only one direction:\\[2mm]
\pause
\>$\models \forall x A(x) \lor \forall x B(x) \imp \forall x(A(x) \lor B(x))$.\\
~\\
\pause
Distributing existential over conjunction goes the other direction:\\[2mm]
\pause
\>$\models \exists x(A(x) \land B(x)) \imp \exists x A(x) \land \exists x B(x)$.\\
\end{tabbing}

\end{wideslide}

\begin{wideslide}[bm=,toc=]{Quantification Without the Free Variable in Its Scope}

Quantifiers over disjunction or conjunction are always distributive when
one subformula does not contain the quantified variable:

\vspace*{-5mm}
\begin{displaymath}
\def\arraystretch{1.5}
\begin{array}{ll}
 \models \exists xA(x) \vee B \eqv \exists x(A(x) \vee  B), &
 \models \forall xA(x) \vee B \eqv \forall x(A(x) \vee  B), \\
 \models B \vee \exists xA(x) \eqv \exists x(B \vee A(x)), &
 \models B \vee \forall xA(x) \eqv \forall x(B \vee A(x)), \\
 \models \exists xA(x) \wedge B \eqv \exists x(A(x) \wedge B), &
 \models \forall xA(x) \wedge B \eqv \forall x(A(x) \wedge B), \\
 \models B \wedge \exists xA(x) \eqv \exists x(B \wedge A(x)), &
 \models B \wedge \forall xA(x) \eqv \forall x(B \wedge A(x)).
\end{array}
\end{displaymath}
\pause
For implication, the following equivalences hold (note they are not symmetric):
\begin{displaymath}
\begin{array}{l}
 \models \forall x(A\imp B(x)) \eqv (A \imp \forall xB(x)) \\ 
 \models \forall x(A(x)\imp B) \eqv (\exists xA(x) \imp B) \\
\end{array}
\end{displaymath}

\end{wideslide}
\begin{wideslide}[bm=,toc=]{Quantification Over Implication}
The following formulas hold for distributing quantifiers over implication:
\vspace*{-2mm}
\begin{displaymath}
\def\arraystretch{1.5}
\begin{array}{l}
\models \forall x(A(x)\imp B(x))\imp (\forall xA(x)\imp\forall xB(x)) \\
\models \forall x(A(x)\imp B(x))\imp (\exists xA(x)\imp\exists xB(x)) \\
\models \forall x(A(x)\imp B(x))\imp (\forall xA(x)\imp\exists xB(x)) \\
\models \exists x(A(x)\imp B(x))\eqv (\forall xA(x)\imp\exists xB(x))  \\
\models (\exists xA(x)\imp\forall xB(x))\imp \forall x(A(x)\imp B(x))  \\
\end{array}
\end{displaymath}
\pause
Example to verify intuition:
\begin{itemize}
\item Let $A(x)$ mean ``a person, $x$, is rich.''
\item Let $B(x)$ mean ``a person, $x$, is happy.''
\item Check the formulas under this interpretation.
\end{itemize}
\end{wideslide}

\begin{wideslide}[bm=,toc=]{Quantification Over Equivalence}
Distribution of universal quantification works in one direction:
\begin{displaymath}
\begin{array}{l}
 \models \forall x(A(x)\eqv B(x))\imp (\forall xA(x) \eqv \forall xB(x)) \\
\end{array}
\end{displaymath}
\pause
The following formula holds for existential quantification:
\begin{displaymath}
\begin{array}{l}
 \models \forall x(A(x)\eqv B(x))\imp (\exists xA(x) \eqv \exists xB(x)) \\
\end{array}
\end{displaymath}

\end{wideslide}

\begin{wideslide}[bm=,toc=]{Example Derivation for Distributing Quantifiers Over
Implications / Biconditionals}
To derive an equivalent formula, translate $\imp$ or $\eqv$ into disjuctions and
conjunctions, then apply relevance substitutions.
\begin{displaymath}
\def\arraystretch{1.5}
\begin{array}{ll}
\exists x (A(x) \imp B(x))  & \equiv \exists x(\ngg A(x) \lor B(x)) \\
\pause &\equiv \exists x\ngg A(x) \lor \exists x B(x) \\
\pause  &\equiv \ngg \exists x\ngg A(x) \imp \exists x B(x) \\
\pause &\equiv \forall x A(x) \imp \exists x B(x) \\

\end{array}
\end{displaymath}

\end{wideslide}


