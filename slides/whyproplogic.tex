\documentclass[style=sailor,size=12pt]{powerdot}
\usepackage{epic,array,ecltree,url,calrsfs}
\usepackage[nointegrals]{wasysym}
\usepackage{listings}
\usepackage{epsfig}
\usepackage{amsmath}
\usepackage{amsfonts}
\usepackage{amssymb}
\usepackage{amsxtra}
\usepackage{amsthm}
\usepackage{mlextra} % Must be below ams packages
\usepackage{mathrsfs}
\usepackage{color}
\usepackage{array}
\usepackage{graphicx}
\graphicspath{ {../art/} }
\usepackage{bm}
\usepackage{tikz}
\usepackage{multicol}
\usepackage{enumitem}

\pdsetup{method=normal}

\title{Some Common Applications of Propositional Logic}
\author{Foundations of Computer Science}
\date{\today}


\begin{document}
\maketitle

\begin{wideslide}[bm=,toc=]{Propositional logic}
Propositional logic provides a way to
\begin{itemize}
\item express assertions precisely
\item examine consistency among assertions
%\item express behaviors of some systems
\item justify simplifications
\item capture circuits
\end{itemize}
\end{wideslide}

\begin{wideslide}[bm=,toc=]{Expressing assertions}
\begin{itemize}
\item ``The diagnostic message is stored in the buffer or retransmitted.''
\item ``The diagnostic message is not stored in the buffer.''
\item ``If the diagnostic message is stored in the buffer then it's retransmitted.''
\item Let $p$ denote ``The diagnostic message is stored in the buffer.''
\item Let $q$ denote ``The diagnostic message is retransmitted.''
\item $p$ and $q$ are called {\em propositional variables\/}.
\item The 3 facts in order become $p\lor q$, $\ngg p$ and $p \imp q$.
\item Can be {\em satisfied\/} iff $p$ is false and $q$ is true.
\end{itemize}
\begin{center}
\begin{tabular}{c|c||c|c|c}
$p$ & $q$ & $p \lor q$ & $\ngg p$ & $p \imp q$ \\
\hline
$1$ & $1$ & $1$ & $0$ & $1$ \\
$1$ & $0$ & $1$ & $0$ & $0$ \\
$0$ & $1$ & $1$ & $1$ & $1$ \\
$0$ & $0$ & $0$ & $1$ & $1$ 
\end{tabular}
\end{center}


\end{wideslide}

\begin{wideslide}[bm=,toc=]{Examine consistency}
\begin{itemize}
\item Add to 3 facts: ``The diagnostic message is not retransmitted.''
\item The 4 facts then become $p\lor q$, $\ngg p$, $p\imp q$ and $\ngg q$.
\item Can no longer be satisfied.
\item Assertions are {\em inconsistent\/}.
\end{itemize}
\begin{center}
\begin{tabular}{c|c||c|c|c|c}
$p$ & $q$ & $p \lor q$ & $\ngg p$ & $p \imp q$ & $\ngg q$ \\
\hline
$1$ & $1$ & $1$ & $0$ & $1$ & $0$\\
$1$ & $0$ & $1$ & $0$ & $0$ & $1$\\
$0$ & $1$ & $1$ & $1$ & $1$ & $0$\\
$0$ & $0$ & $0$ & $1$ & $1$ & $1$
\end{tabular}
\end{center}
\end{wideslide}

\begin{wideslide}[bm=,toc=]{Justify simplifications}
\begin{itemize}
\item Consider the expression
\vspace{-1em}
\begin{program}
if (a < b || (a >= b \&\& c == d)) ...
\end{program}
It can be simplified to
\vspace{-1em}
\begin{program}
if (a < b || c == d) ...
\end{program}
\item Let $p$ and $q$ stand for propositions $\tid{(a < b)}$ and $\tid{(c == d)}$.
\item The simplification is justfied by $(p\vee ((\neg p) \wedge q)) \equiv (p\vee q)$.
\item ``$\equiv$'' means ``logically equivalent to''.\footnote{
A. Aho and J. Ullman, Foundations of Computer Science, W.H. Freeman, 1994.}
\end{itemize}
\end{wideslide}

\begin{wideslide}[bm=,toc=]{Example 1.2: A half-adder}
\begin{center}
\begin{picture}(280,120)
\put(-10,  0){\makebox(20,20)[l]{Bit2}}
\put(-10,100){\makebox(20,20)[l]{Bit1}}
\put(260, 20){\makebox(25,20)[l]{Sum}}
\put(260, 90){\makebox(25,20)[l]{Carry}}
\put(20, 10){\line(1,0){60}}
\put(20,110){\line(1,0){50}}
\put(30, 10){\line(1,2){40}}
\put(30,110){\line(1,-2){40}}
\put(70, 30){\line(1,0){10}}
\put(70,80){ %AND gate
  \put(20,20){\oval(40,40)[r]}
  \put(0,0){\line(0,1){40}}
  \put(0,0){\line(1,0){20}}
  \put(0,40){\line(1,0){20}}
}
\put(70,0){ %OR gate
  \put(20,20){\oval(40,40)[r]}
  \put(0,20){\oval(20,40)[r]}
  \put(0,0){\line(1,0){20}}
  \put(0,40){\line(1,0){20}}
}
\put(110,100){\line(1,0){140}}
\put(110, 20){\line(1,0){60}}
\put(210, 30){\line(1,0){40}}
\put(170,10){ %AND gate
  \put(20,20){\oval(40,40)[r]}
  \put(0,0){\line(0,1){40}}
  \put(0,0){\line(1,0){20}}
  \put(0,40){\line(1,0){20}}
}
\put(130,80){  % NOT gate
  \put( 0, 0){\line(1,0){20}}
  \put( 0, 0){\line(1,-2){10}}
  \put(20, 0){\line(-1,-2){10}}
  \put(10,-23){\circle{6}}
}
\put(140,80){\line(0,1){20}}
\put(140,54){\line(0,-1){14}}
\put(140,40){\line(1,0){30}}
\end{picture}
\end{center}

Carry: $\rid{Bit1}\wedge\rid{Bit2}$

Sum: $(\rid{Bit1}\vee\rid{Bit2})\wedge \neg(\rid{Bit1}\wedge\rid{Bit2})$
\end{wideslide}

%\begin{wideslide}[bm=,toc=]{Propositional logic}
%\begin{itemize}
%\item Begin discussion of propositional logic.
%\end{itemize}
%\end{wideslide}

\end{document}
