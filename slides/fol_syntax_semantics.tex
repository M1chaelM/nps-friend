\documentclass[style=sailor,size=12pt]{powerdot}
\usepackage{epic,array,ecltree,url,calrsfs}
\usepackage[nointegrals]{wasysym}
\usepackage{listings}
\usepackage{epsfig}
\usepackage{amsmath}
\usepackage{amsfonts}
\usepackage{amssymb}
\usepackage{amsxtra}
\usepackage{amsthm}
\usepackage{mlextra} % Must be below ams packages
\usepackage{mathrsfs}
\usepackage{color}
\usepackage{array}
\usepackage{graphicx}
\graphicspath{ {../art/} }
\usepackage{bm}
\usepackage{tikz}
\usepackage{multicol}
\usepackage{enumitem}

\pdsetup{method=normal}

\title{First Order Logic: Syntax and Semantics}
\author{Foundations of Computer Science}
\date{\today}

\begin{document}
\maketitle
\section[slide=false]{Syntax}
\begin{slide}[bm=,toc=]{First Order Logic: Syntax}
\begin{defn}{7.6}[Ben Ari]
Let $\mathcal{P}$,$\mathcal{A}$ and $\mathcal{V}$ be countable sets of
\emph{predicate symbols}, \emph{constant symbols} and \emph{variables}.
Each predicate symbol $p^n \in \mathcal{P}$ is associated with an \emph{arity},
the number $n \geq 1$ of \emph{arguments} that it takes. $p^n$ is called an
$n$-ary predicate. For $n = 1,2$, the terms \emph{unary} and \emph{binary},
respectively, are also used.
\end{defn}
{\bf Notation}
\begin{itemize}
\item By convention, we associate the following lower-case letters with
each set: $\mathcal{P} = \{p,q,r\}$, $\mathcal{A} = \{a,b,c\}$, $\mathcal{V} =
\{x,y,z\}$.
\item Superscript denoting the arity of the predicate is omitted when clear (can
    be inferred from the number of arguments).

\end{itemize}
\begin{defn}{7.7}[Ben Ari]
~\\
$\forall$ is the \emph{universal quantifier} and is read \emph{for all}.\\
$\exists$ is the \emph{existential quantifier} and is read \emph{there exists}.
\end{defn}

\end{slide}

\begin{wideslide}[bm=,toc=]{Formulas in FOL}
\begin{defn}{7.8}[Ben Ari]
An \emph{atomic formula} is an $n$-ary predicate followed by a list of $n$
arguments in parentheses: $p(t_1,t_2,...,t_n)$, where each argument $t_i$ is
either a variable or a constant. A \emph{formula} in first-order logic is
a tree defined recursively as follows:
\end{defn}
\vspace{-3ex}
\begin{itemize}
\item A formula is a leaf labeled by an atomic formula.
\item A formula is a node labeled by $\ngg$ with a single child that is a
formula.
\item A formula is a node labeled by $\forall x$ or $\exists x$ (for some
 variable $x$) with a single child that is a formula.
\item A formula is a node labeled by a binary Boolean operator with two
children both of which are formulas.
\end{itemize}
{\bf Universal vs Existential}:\\
A formula of the form $\forall x A$ is a \emph{universal formula}.\\
A formula of the form $\exists x A$ is an \emph{existential formula}.\\
\end{wideslide}

\begin{wideslide}[bm=,toc=]{Figure 7.1: Tree for
$\forall x(\ngg\exists yp(x,y)\vee\ngg\exists yp(y,x))$}
\begin{multicols}{2}
\begin{itemize}
\item Formula tree structure follows PL.
\item Similar concept of induction on structure.
\item In the string representation, $\exists$ and $\forall$ have the same precedence as $\ngg$.
\end{itemize}
\begin{center}
\setlength{\GapWidth}{8mm}
\setlength{\GapDepth}{8mm}
\begin{bundle}{$\forall x$\rule[-1mm]{0mm}{1mm}}
\chunk{
  \begin{bundle}{$\vee$\rule[-1mm]{0mm}{1mm}}
  \chunk{
    \begin{bundle}{$\ngg$\rule[-1mm]{0mm}{1mm}}
    \chunk{
      \begin{bundle}{$\exists y$\rule[-1mm]{0mm}{1mm}}
      \chunk{$p(x,y)$}
      \end{bundle}
    }
    \end{bundle}
  }
  \chunk{
    \begin{bundle}{$\ngg$\rule[-1mm]{0mm}{1mm}}
    \chunk{
      \begin{bundle}{$\exists y$\rule[-1mm]{0mm}{1mm}}
      \chunk{$p(y,x)$}
      \end{bundle}
    }
    \end{bundle}
  }
  \end{bundle}
}
\end{bundle}
\end{center}
\end{multicols}
\end{wideslide}

\begin{wideslide}[bm=,toc=]{Scope of Variables}
\begin{defn}{7.11}
A universal or existential formula $\forall x A$ or $\exists x A$
is a \emph{quantified formula}. $x$ is the \emph{quantified variable}
and its \emph{scope} is the formula $A$. It is not required that $x$
actually appear in the scope of its quantification.
\end{defn}
\begin{itemize}
\item Conceptually similar to scope of variables in programming languages.
\item Technically, local declaration hides global declarations.
\item Best practice is to avoid name conflicts.
\end{itemize}
\end{wideslide}

\begin{wideslide}[bm=,toc=]{Figure 7.2: Global and local variables}
\begin{multicols}{2}
\begin{itemize}
\item $x$ is declared globally and locally within $p$.
\item Local declaration takes precedence in $p$.
\item This code is error prone and confusing.
\item Using another variable name improves readability.
\end{itemize}
\vspace*{-2ex}
\begin{program}
class MyClass \{\\
\>int x;\\
\>void p() \{\\
\>\>int x;\\
\>\>x = 1;\\
\>\>// Print the value of x\\
\>\}\\
\>void q() \{\\
\>\>// Print the value of x\\
\>\}\\
\>... void main(...) \{\\
\>x = 5;\\
\>p;\\
\>q;\\
\}
\end{program}
\end{multicols}
\end{wideslide}

\begin{wideslide}[bm=,toc=]{Free and Bound Variables}
\begin{defn}{7.12}
Let $A$ be a formula. An occurrence of a variable $x$ in $A$ is
a \emph{free variable of A} iff $x$ is not within the scope of
a quantified variable $x$. A variable which is not free is \emph{bound}.
\end{defn}
\vspace*{-2ex}
\begin{itemize}
\item A formula with no free variable is \emph{closed}.
\item If $\{x_1,...,x_n\}$ are all free variables of $A$...
  \begin{itemize}
      \item $\forall x_1 \cdots \forall x_n A$ is the \emph{universal closure}.
      \item $\exists x_1 \cdots \exists x_n A$ is the \emph{existential closure}.
  \end{itemize}
\end{itemize}
\vspace*{-2ex}
\begin{ex}{7.13}
\end{ex}
\vspace*{-2ex}
\begin{itemize}
\item $p(x,y)$ has two free variables ($x$ and $y$).
\item $\exists y\; p(x,y)$ has one free variable($x$).
\item $\forall x \exists y \;p(x,y)$ is closed.
\item Universal closure: $\forall x \forall y \;p(x,y)$.
\item Existential closure: $\exists x \exists y \;p(x,y)$.
\end{itemize}
\end{wideslide}
\begin{wideslide}[bm=,toc=]{Scope Example}
\begin{ex}{7.14}
$\forall x \; p(x) \land q(x)$
\end{ex}
\vspace*{-2ex}
\begin{itemize}
\item<2-> Bound: \pause[2] $x$ in $p(x)$
\item<4-> Free: \pause[2] $x$ in $q(x)$
\item<6-> Universal Closure: \pause[2] $\forall x(\forall x \; p(x) \land q(x))$
\item<8-> How to write this better?
\item<9-> $\forall y(\forall x \; p(x) \land q(y))$
\end{itemize}
\end{wideslide}
\section[slide=false]{Semantics}
\begin{slide}[bm=,toc=]{Interpretations: FOL vs PL}
{\bf Interpretation in PL:} 
\begin{itemize}
\item Map from atomic \emph{propositions} to truth values.
\end{itemize}
{\bf Interpretation in FOL:} 
\begin{itemize}
\item Map from atomic \emph{formulas} to truth values.
\end{itemize}
\vspace{1ex}
{\bf \emph{Mapping atomic formulas is more complex}}
\begin{itemize}
\item Atomic formulas contain variables and constants.
\item First these must be assigned elements from some domain.
\item Then predicates are interpreted as relations on that domain. 
\end{itemize}
\end{slide}

\begin{wideslide}[bm=,toc=]{Interpretations in FOL}
\begin{defn}{7.16}
Let $A$ be a formula where $\{p_1,...p_m\}$ are all the predicates
appearing in $A$ and $\{a_1,...,a_k\}$ are all the constants appearing in $A$.
An \emph{interpretation} $\mathcal{I}_A$ for $A$ is a triple:
\[(D,\{R_1,...R_m\}, \{d_1,...d_k\},)\]
where $D$ is a \emph{non-empty} set called the \emph{domain}, $R_i$ is an
$n_i$-ary relation on $D$ that is assigned to the $n_i$-ary predicate
$p_i$ and $d_i \in D$ is assigned to the constant $a_i$.
\end{defn}
\end{wideslide}
\begin{wideslide}[bm=,toc=]{Examples of Interpretations in FOL}
\begin{ex}{7.17}
Four interpretations for the formula $\forall x \; p(a,x)$:
\end{ex}
\vspace*{-2ex}
\begin{enumerate}
\item<2-> $\mathcal{I}_1 = (\N_0,\{\leq\},\{0\})$
\begin{itemize}
\item<3-> $\N_0$ is assigned to $D$.
\item<3-> The \emph{less-than or equal} relation is assigned to $p$.
\item<3-> $0$ is assigned to $a$.
\end{itemize}
\item<4-> $\mathcal{I}_2 = (\N_0,\{\leq\},\{1\})$
\begin{itemize}
\item<5-> Same as above but $1$ is assigned to $a$. 
\end{itemize}
\item<6-> $\mathcal{I}_3 = (\Z,\{\leq\},\{0\})$
\begin{itemize}
\item<7-> Same as first example but $\Z$ is assigned to $D$. 
\end{itemize}
\item<8-> $\mathcal{I}_4 = (\mathcal{S},\{substr\},\{ \epsilon \})$
\begin{itemize}
\item The domain, $\mathcal{S}$, is a set of strings; $\epsilon$ is the empty string.
\item The $substr$ relation is a binary relation such that $(s_1,s_2) \in substr$
iff $s_1$ is a substring of $s_2$.
\end{itemize}

\end{enumerate}
\end{wideslide}


\begin{wideslide}[bm=,toc=]{Assignment}
\begin{defn}{7.18}[Ben Ari]
Let $\mathcal{I}_A$ be an interpretation for a formula $A$. An
\emph{assignment} $\sigma_{\mathcal{I}_A}: \mathcal{V} \mapsto D$
is a function which maps every variable $v \in \mathcal{V}$ to an
element $d \in D$, the domain of $\mathcal{I}_A$.\\~\\
$\sigma_{\mathcal{I}_A}[x_i \leftarrow d_i]$ is an assignment that is
the same as $\sigma_{\mathcal{I}_A}$ except that $x_i$ is mapped to
$d_i$.
\end{defn}
\begin{itemize}
\item That is, $\sigma_{\mathcal{I}_A}[x_i \leftarrow d_i]$ can be
understood as a \emph{modification} of the assignment 
$\sigma_{\mathcal{I}_A}$ in which the value of $x_i$ is assigned
to $d_i$ instead.
\end{itemize}
\end{wideslide}


\begin{wideslide}[bm=,toc=]{Truth in First Order Logic}
\begin{defn}{7.19}[Ben Ari]
Let $A$ be a formula, $\mathcal{I}_A$ an interpretation and
$\sigma_{\mathcal{I}_A}$ an assignment. $v_{\sigma_{\mathcal{I}_A}}(A)$,
the \emph{truth value} of $A$ \emph{under} $\mathcal{I}_A$ \emph{and}
$\sigma_{\mathcal{I}_A}$, is defined by induction on the structure of
$A$ (for simplicity we write $v_{\sigma}$ for $v_{\sigma_{\mathcal{I}_A}}$):
\end{defn}
\vspace{-2ex}
\begin{itemize}
\item Let $A = p_k(c_1,...,c_n)$ be an atomic formula where each $c_i$ is
either a variable $x_i$ or a constant $a_i$. $v_{\sigma}(A) = T$ iff $(d_1,...,d_n)
\in R_k$ where $R_k$ is the relation assigned by $\mathcal{I}_A$ to $p_k$, and
$d_i$ is the domain element assigned to $c_i$, either by $\mathcal{I}_A$ if
$c_i$ is a constant or by $\sigma_{\mathcal{I}_A}$ if $c_i$ is a variable.
\item $v_{\sigma}(\ngg A) = T$ iff $v_{\sigma}(A) = F$.
\item $v_{\sigma}(A_1 \lor A_2) = T$ iff $v_{\sigma}(A_1) = T$ or
$v_{\sigma}(A_2) = T$, and similarly for the other Boolean operators.
\item $v_{\sigma}(\forall x A_1) = T$ iff $\sigma_{[x \leftarrow d]}(A_1) = T$
for \emph{all} $d \in D$.
\item $v_{\sigma}(\exists x A_1) = T$ iff $\sigma_{[x \leftarrow d]}(A_1) = T$
for \emph{some} $d \in D$.
\end{itemize}
\end{wideslide}

\begin{wideslide}[bm=,toc=]{Truth Values of Interpretations in Example 7.17}
\begin{ex}{7.21}
Recall the four interpretations given for the formula $\forall x \; p(a,x)$:
\end{ex}
\vspace*{-2ex}
\begin{enumerate}
\item<2-> $\mathcal{I}_1 = (\N_0,\{\leq\},\{0\})$
\begin{itemize}
\item<3-> $T$: all natural numbers are greater than or equal to $0$.
\end{itemize}
\item<4-> $\mathcal{I}_2 = (\N_0,\{\leq\},\{1\})$
\begin{itemize}
\item<5-> $F$: $0 \in \N_0$ but $1 \not \leq 0$. 
\end{itemize}
\item<6-> $\mathcal{I}_3 = (\Z,\{\leq\},\{0\})$
\begin{itemize}
\item<7-> $F$: Not all integers are greater than or equal to $0$.
\end{itemize}
\item<8-> $\mathcal{I}_4 = (\mathcal{S},\{substr\},\{ \epsilon \})$
\begin{itemize}
\item<9-> $T$: $\epsilon$ is a substring of all strings, therefore also all
strings in $\mathcal{S}$. 
\end{itemize}
\end{enumerate}
\end{wideslide}

\begin{wideslide}[bm=,toc=]{Truth for Some / All Assignments}
\begin{thm}{7.22}[Ben Ari]
Let $A' = A(x_1,...,x_n)$ be a (non-closed) formula with free variables
$x_1,...,x_n$, and let $\mathcal{I}$ be an interpretation. Then:
\end{thm}
\vspace*{-2ex}
\begin{itemize}
\item $v_{\sigma_{\mathcal{I}_A}}(A') = T$ for \emph{some} assignment $\sigma_{\mathcal{I}_A}$ iff $v_{\mathcal{I}}(\exists x_1 ... \exists x_n A') = T$.
\item $v_{\sigma_{\mathcal{I}_A}}(A') = T$ for \emph{all} assignments
$\sigma_{\mathcal{I}_A}$ iff $v_{\mathcal{I}}(\forall x_1 ... \forall x_n A') = T$.
\end{itemize}
Informally:
\begin{itemize}
\item The value of $A'$ under interpretation $\mathcal{I}$ and assignment
$\sigma$ is $T$ for some assignment iff the existential closure of $A'$ is true 
under $\mathcal{I}$. 
\item Similarly, the value of $A'$ under interpretation $\mathcal{I}$ and assignment
$\sigma$ is $T$ for all assignments iff the universal closure of $A'$ is true 
under $\mathcal{I}$. 
\end{itemize}
\end{wideslide}

\section[slide=false]{Validity and Satisfiability}

\begin{wideslide}[bm=,toc=]{Validity and Satisfiability}
\begin{defn}{7.23}[Ben Ari]
Let $A$ be a closed formula of first-order logic.
\end{defn}
\vspace{-2ex}
\begin{itemize}
\item $A$ is \emph{true} in $\mathcal{I}$ or $\mathcal{I}$ is a \emph{model} for
$A$ iff $v_{\mathcal{I}}(A) = T$. Notation: $\mathcal{I} \models A$.
\item $A$ is \emph{valid} if for \emph{all} interpretations $\mathcal{I}$,
$\mathcal{I} \models A$. Notation: $\models A$ 
\item $A$ is \emph{satisfiable} if for \emph{some} interpretation $\mathcal{I}$,
$\mathcal{I} \models A$.
\item $A$ is \emph{unsatisfiable} if it is not satisfiable. 
\item $A$ is \emph{falsifiable} if it is not valid. 
\end{itemize}

\end{wideslide}

\begin{wideslide}[bm=,toc=]{Satisfiability and Validity in $\forall x \; p(a,x)$}
Let $A$ be the formula $\forall x \; p(a,x)$ from examples $7.17$ and $7.21$:
\begin{enumerate}
\item<2-> $\mathcal{I}_1 = (\N_0,\{\leq\},\{0\})$
\begin{itemize}
\item<3-> $T$: $A$ is satisfiable. 
\end{itemize}
\item<4-> $\mathcal{I}_2 = (\N_0,\{\leq\},\{1\})$
\begin{itemize}
\item<5-> $F$: $A$ is not valid. 
\end{itemize}
\item<6-> $\mathcal{I}_3 = (\Z,\{\leq\},\{0\})$
\begin{itemize}
\item<7-> $F$: Also shows $A$ is not valid. 
\end{itemize}
\item<8-> $\mathcal{I}_4 = (\mathcal{S},\{substr\},\{ \epsilon \})$
\begin{itemize}
\item<9-> $T$: Also shows $A$ is satisfiable. 
\end{itemize}
\end{enumerate}
\end{wideslide}


\begin{wideslide}[bm=,toc=]{Satisfiability and Validity Examples}
\begin{defn}{7.25}{Ben Ari}
\end{defn}
\begin{enumerate}
\item<2-> $\forall x \forall y (p(x,y) \implies p(y,x))$
\begin{itemize}
\item<3-> Satisfiable? \pause[3] Yes, in interpretations where $p$ is a
symmetric relation.
\item<3-> Valid? \pause No. Falisified in interpretations where $p$ is
non-symmetric. 
\end{itemize}
\item<6-> $\forall x \exists y p(x,y)$
\begin{itemize}
\item<7-> Satisfiable? \pause[3] Satisfiable in interpretations where 
$p$ is a total function. For example, $(x,y) \in R$ iff $y = x + 1$
for $x,y, \in \Z$.
\item<7-> Valid? \pause No. Falsified if the domain is changed to $\Z^-$.  
\end{itemize}
\item<10-> $\exists x \exists y(p(x) \land \ngg p (y))$
\begin{itemize}
\item<11-> Satisfiable? \pause[3] Yes. Assume for example that $D$ is $\N$ and
$p$ is the unary predicate corresponding to the property of being an
odd number.
\item<11-> Valid? \pause No. Consider a domain with only one element.
\end{itemize}
\end{enumerate}
\end{wideslide}

\begin{wideslide}[bm=,toc=]{Satisfiability and Validity Examples (continued)}
\begin{defn}{7.25}{Ben Ari}
(continued from previous slide)
\end{defn}
\begin{enumerate}
\setcounter{enumi}{3}
\item<2-> $\forall x p(a,x)$
\begin{itemize}
\item<3-> Expresses the existence of an element with special properties.
\item<3-> Satisfiable? \pause[3] Yes, when $D$ is $\N$ and $p$ is $\leq$.
\item<3-> Valid? \pause No. Falisified if we change $D$ to $\Z$ 
\end{itemize}
\item<6-> $\forall x (p(x) \land q(x)) \eqv (\forall x p(x) \land \forall x
    q(x))$
\begin{itemize}
\item<7-> Satisfiable? \pause[3] Yes. Ex: $D$ is $\N$, $p$ is the property of
not being negative and $q$ is the property of being divisible by $1$. 
\item<7-> Valid? \pause Yes. Proof uses theorem 7.22. 
\end{itemize}
\item<10-> $\forall x (p(x) \imp q(x)) \imp (\forall x p(x) \imp \forall x q(x))$
\begin{itemize}
\item<11-> Valid? \pause Yes. 
\item<11-> What about the converse: $(\forall x p(x) \imp \forall x q(x)) \imp \forall x (p(x) \imp q(x))$?
\end{itemize}
\end{enumerate}
\end{wideslide}

\begin{wideslide}[bm=,toc=]{Extending to Sets of Formulas}
As in PL, we can extend concepts of interpretation, satisfiability and other
properties to sets of formulas.
\begin{defn}{7.26}[Ben Ari]
Let $U = \{A_1,...\}$ be a set of formulas where $\{p_1,...,p_m\}$ are all
the predicates appearing in all $A_i \in U$ and $\{a_1,...,a_k\}$ are all the
constants appearing in all $A_i \in U$. An \emph{interpretation} $\mathcal{I}_U$
for $U$ is a triple:
\[(D, \{R_1,...,R_m\},\{d_1,...,d_k\})\]
where $D$ is a \emph{non-empty} set called the \emph{domain}, $R_i$ is an
$n_i$-ary relation on $D$ that is assigned to the $n_i$-ary predicate $p_i$
and $d_i \in D$ is an element of $D$ that is assigned to the constant $a_i$.
\end{defn}

\end{wideslide}

\begin{wideslide}[bm=,toc=]{Consistency}
\begin{defn}{7.27}[Ben Ari]
A set of closed formulas $U = \{A_1,...\}$ is \emph{(simultaneously)
  satisfiable} iff there exists an interpretation $\mathcal{I}_U$ such that
  $v_{\mathcal{I}_U}(A_i) = T$ for all $i$. The satisfying interpretation
  is a \emph{model} of $U$. $U$ is \emph{valid} iff for every interpretation
  $\mathcal{I}_U$, $v_{\mathcal{I}_U}(A_i) = T$ for all $i$.
\end{defn}

\end{wideslide}

\section[slide=false]{Logical Equivalence}

\begin{wideslide}[bm=,toc=]{Definition of Logical Equivalence}
\begin{defn}{7.28}[Ben Ari]
\end{defn}
\vspace{-2ex}
\begin{itemize}
\item Let $U = \{A_1,A_2\}$ be a pair of closed formulas. $A_1$ is
\emph{logically equivalent} to $A_2$ iff $v_{{\mathcal{I}_U}}(A_1) =
v_{{\mathcal{I}_U}}(A_2)$ for all interpretations $\mathcal{I}_U$.
Notation: $A_1 \equiv A_2$.
\item Let $A$ be a closed formula and $U$ a set of closed formulas.
$A$ is a \emph{logical consequence} of $U$ iff for all interpretations
$\mathcal{I}_{U \cup \{A\}}$, $v_{\mathcal{I}_{U \cup \{A\}}}(A_i) = T$ for
all $A_i \in U$ implies $v_{\mathcal{I}_{U \cup \{A\}}}(A) = T$. Notation: $U \models A$.
\end{itemize}
\begin{thm}{7.29}
Let $A,B$ be closed formulas and $U = \{A_1,...,A_n\}$ be a set of
closed formulas. Then:
\end{thm}
\vspace{-5ex}
\begin{tabbing}
closed formulas. Then X \= \kill
\> $A \equiv B$ iff $\models A \eqv B$\\
\> $U \models A$ iff $\models (A_1 \land \cdots \land A_n) \imp B$
\end{tabbing}

\end{wideslide}

\begin{wideslide}[bm=,toc=]{Duality}
The two quantifiers are duals:
\begin{tabbing}
The two quantifiers are dua \= \kill
\> $\models \forall x A(x) \eqv \ngg \exists x \ngg A(x)$.\\
\> $\models \exists x A(x) \eqv \ngg \forall x \ngg A(x)$.
\end{tabbing}
\begin{itemize}
\item It is only necessary to define one of the two.
\item If $\forall$ is defined, $\exists$ can be considered an abbreviation of
      $\ngg \forall \ngg$.
\end{itemize}

\end{wideslide}

\begin{wideslide}[bm=,toc=]{Commutativity}
Quantifiers of the same type commute:
\begin{tabbing}
Quantifiers of the same \= \kill
\> $\models \forall x \forall y A(x,y) \eqv \forall y \forall x A(x,y)$.\\
\> ~\\ 
\> $\models \exists x \exists y A(x,y) \eqv \exists y \exists x A(x,y)$.\\
\end{tabbing}

but $\forall$ and $\exists$ commute only in one direction:
\begin{tabbing}
Quantifiers of the same \= \kill
\> $\models \exists x \forall y A(x,y) \imp \forall y \exists x A(x,y)$.\\
\end{tabbing}

\end{wideslide}

\begin{wideslide}[bm=,toc=]{Distributivity}
\begin{tabbing}
Universal \= \kill
Universal quantifiers distribute over conjunction:  \> \\
\>$\models \forall x(A(x) \land B(x)) \eqv \forall x A(x) \land \forall x B(x)$.\\
~\\
Existential quantifiers distribute over disjunction:\\
\>$\models \exists x(A(x) \lor B(x)) \eqv \exists x A(x) \lor \exists x B(x)$.\\
~\\
Distributing universal over disjunction goes only one direction:\\
\>$\models \forall x A(x) \lor \forall x B(x) \imp \forall x(A(x) \lor B(x))$.\\
~\\
Distributing existential over conjunction goes the other direction:\\
\>$\models \exists x(A(x) \land B(x)) \imp \exists x A(x) \land \exists x B(x)$.\\
\end{tabbing}

\end{wideslide}

\begin{wideslide}[bm=,toc=]{Quantification Without the Free Variable in Its Scope}

Quantifiers over disjunction or conjunction are always distributive when
one subformula does not contain the quantified variable:

\vspace*{-5mm}
\begin{displaymath}
\def\arraystretch{1.5}
\begin{array}{ll}
 \models \exists xA(x) \vee B \eqv \exists x(A(x) \vee  B), &
 \models \forall xA(x) \vee B \eqv \forall x(A(x) \vee  B), \\
 \models B \vee \exists xA(x) \eqv \exists x(B \vee A(x)), &
 \models B \vee \forall xA(x) \eqv \forall x(B \vee A(x)), \\
 \models \exists xA(x) \wedge B \eqv \exists x(A(x) \wedge B), &
 \models \forall xA(x) \wedge B \eqv \forall x(A(x) \wedge B), \\
 \models B \wedge \exists xA(x) \eqv \exists x(B \wedge A(x)), &
 \models B \wedge \forall xA(x) \eqv \forall x(B \wedge A(x)).
\end{array}
\end{displaymath}
For implication, the following equivalences hold (note they are not symmetric):
\begin{displaymath}
\begin{array}{l}
 \models \forall x(A\imp B(x)) \eqv (A \imp \forall xB(x)) \\ 
 \models \forall x(A(x)\imp B) \eqv (\exists xA(x) \imp B) \\
\end{array}
\end{displaymath}

\end{wideslide}
\begin{wideslide}[bm=,toc=]{Quantification Over Implication}
The following formulas hold for distributing quantifiers over implication:
\vspace*{-2mm}
\begin{displaymath}
\def\arraystretch{1.5}
\begin{array}{l}
\models \forall x(A(x)\imp B(x))\imp (\forall xA(x)\imp\forall xB(x)) \\
\models \forall x(A(x)\imp B(x))\imp (\exists xA(x)\imp\exists xB(x)) \\
\models \forall x(A(x)\imp B(x))\imp (\forall xA(x)\imp\exists xB(x)) \\
\models \exists x(A(x)\imp B(x))\eqv (\forall xA(x)\imp\exists xB(x))  \\
\models (\exists xA(x)\imp\forall xB(x))\imp \forall x(A(x)\imp B(x))  \\
\end{array}
\end{displaymath}
Example to verify intuition:
\begin{itemize}
\item Let $A(x)$ mean ``a person, $x$, is rich.''
\item Let $B(x)$ mean ``a person, $x$, is happy.''
\item Check the formulas under this interpretation.
\end{itemize}
\end{wideslide}

\begin{wideslide}[bm=,toc=]{Quantification Over Equivalence}
Distribution of universal quantification works in one direction:
\begin{displaymath}
\begin{array}{l}
 \models \forall x(A(x)\eqv B(x))\imp (\forall xA(x) \eqv \forall xB(x)) \\
\end{array}
\end{displaymath}
The following formula holds for existential quantification:
\begin{displaymath}
\begin{array}{l}
 \models \forall x(A(x)\eqv B(x))\imp (\exists xA(x) \eqv \exists xB(x)) \\
\end{array}
\end{displaymath}

\end{wideslide}

\begin{wideslide}[bm=,toc=]{Example Derivation for Distributing Quantifiers Over
Implications / Biconditionals}
To derive an equivalent formula, translate $\imp$ or $\eqv$ into disjuctions and
conjunctions, then apply relevance substitutions.
\begin{displaymath}
\def\arraystretch{1.5}
\begin{array}{ll}
\exists x (A(x) \imp B(x))  & \equiv \exists x(\ngg A(x) \lor B(x)) \\
\pause &\equiv \exists x\ngg A(x) \lor \exists x B(x) \\
\pause  &\equiv \ngg \exists x\ngg A(x) \imp \exists x B(x) \\
\pause &\equiv \forall x A(x) \imp \exists x B(x) \\

\end{array}
\end{displaymath}

\end{wideslide}

%\begin{wideslide}[bm=,toc=]{Logically equivalent formulas (2)}
%\begin{displaymath}
%\begin{array}{l}
% \forall x(A(x)\vee B(x)) \imp (\forall xA(x) \vee \exists xB(x)) \\
%
%\end{array}
%\end{displaymath}
%\end{wideslide}
%
\end{document}

