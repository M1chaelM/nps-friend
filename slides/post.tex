
\documentclass[style=sailor,size=12pt]{powerdot}
\usepackage{epic,array,ecltree,url}
\usepackage[nointegrals]{wasysym}
\usepackage{mathtools}
\usepackage{graphicx}

\newcommand{\id}[1]{\mbox{\it #1\/}}
\newcommand{\rid}[1]{\mbox{\rm #1}}
\newcommand{\sid}[1]{\mbox{\sf #1}}
\newcommand{\bid}[1]{\mbox{\bf #1}}
\newcommand{\tinysz}[1]{\mbox{\tiny $#1$}}

\newcommand{\lra}{\longrightarrow}
\newcommand{\ra}{\rightarrow}
\newcommand{\surj}{\twoheadrightarrow}
\newcommand{\graph}{\mathrm{graph}}
\newcommand{\bb}[1]{\mathbb{#1}}
\newcommand{\Ell}{\mathscr{L}}
\newcommand{\Z}{\bb{Z}}
\newcommand{\Q}{\bb{Q}}
\newcommand{\R}{\bb{R}}
\newcommand{\C}{\bb{C}}
\newcommand{\N}{\bb{N}}
\newcommand{\M}{\mathbf{M}}
\newcommand{\m}{\mathbf{m}}
\newcommand{\MM}{\mathscr{M}}
\newcommand{\HH}{\mathscr{H}}
\newcommand{\Om}{\Omega}
\newcommand{\Ho}{\in\HH(\Om)}
\newcommand{\bd}{\partial}
\newcommand{\del}{\partial}
\newcommand{\bardel}{\overline\partial}
\newcommand{\textdf}[1]{\textbf{\textsf{#1}}\index{#1}}
\newcommand{\img}{\mathrm{img}}
\newcommand{\ip}[2]{\left\langle{#1},{#2}\right\rangle}
\newcommand{\inter}[1]{\mathrm{int}{#1}}
\newcommand{\exter}[1]{\mathrm{ext}{#1}}
\newcommand{\cl}[1]{\mathrm{cl}{#1}}
\newcommand{\ds}{\displaystyle}
\newcommand{\vol}{\mathrm{vol}}
\newcommand{\cnt}{\mathrm{ct}}
\newcommand{\osc}{\mathrm{osc}}
\newcommand{\LL}{\mathbf{L}}
\newcommand{\UU}{\mathbf{U}}
\newcommand{\support}{\mathrm{support}}
\newcommand{\AND}{\;\wedge\;}
\newcommand{\OR}{\;\vee\;}
\newcommand{\Oset}{\varnothing}
\newcommand{\st}{\ni}
\newcommand{\wh}{\widehat}
\newcommand{\mli}[1]{\mathit{#1}}
\newcommand{\ndiv}{\hspace{-3pt}\not|\hspace{2pt}}

\pdsetup{method=normal,
list={labelsep=1em,leftmargin=1cm,itemsep=0pt,topsep=5pt,parsep=0pt}
}
% import from truth and PTL.
\title{A Brief Introduction to Post Systems}
\author{Foundations of Computer Science}
\date{\today}


\begin{document}
\maketitle

\begin{wideslide}[bm=,toc=]{Models of computation}
\begin{itemize}
\item Different computational models have been developed independently:
%\item Machine independent: recursive function theory.
%\item Machine dependent:
\begin{enumerate}
\item Turing machines, Alan Turing (covered in CS3101)
\item Lambda calculus, Alonzo Church
\item Post systems, Emil Post
\item and others.
\end{enumerate}
\item Post systems are of interest to us.
\item They can be used to define something (data structure, program).
\item Definitions may be recursive 

\hspace{2em}$A \equiv \ldots A\ldots$

or mutually recursive

\hspace{2em}$A \equiv \ldots B\ldots$

\hspace{2em}$B \equiv \ldots A\ldots$

\item Post systems admit formal proofs---is a definition correct?
\end{itemize}
\end{wideslide}

\begin{wideslide}[bm=,toc=]{Recursive definitions}
\begin{itemize}
\item A recursive definition can be represented in a {\em Post system\/}.
\item Named after mathematician Emil Leon Post.
\item A Post system comprises a finite set of {\em variables\/}, {\em signs\/} and {\em productions\/} 
(inference rules).
\item An inference rule has the form
\begin{displaymath}
\begin{array}{c}
t_1\;t_2\;\cdots\;t_n\\
\hline
t
\end{array}
\end{displaymath}
where $t,t_1,\ldots ,t_n$ ($n\geq 0$) are terms (strings of signs and variables).
\item $t_1,\ldots , t_n$ are the {\em antecedents\/} and $t$ the {\em consequent\/} or conclusion.
\item When $n=0$, an inference rule is called an {\em axiom\/}.
\end{itemize}
\end{wideslide}

\begin{wideslide}[bm=,toc=]{Recursive definitions}
\begin{itemize}
\item A Post system of two productions:
\vspace{-1em}
\begin{tabbing}
{\bf R}XX \=  \kill
{\bf B} \>
        \(\begin{array}[t]{l}
        3\in S
        \end{array}\) \\[2ex]
{\bf R} \>
        \(\begin{array}[t]{l}
        x \in S \;\;\;y \in S \\
        \hline
        x + y \in S
        \end{array}\)
\end{tabbing}
\item The set of signs is $\{3,\in,+\}$ and variables $\{x,y\}$.
\item Claim: system defines the set of all positive multiples of 3.
\item Must prove $n\in S$ iff $n$ is a positive multiple of 3.
\item ``$n\in S$ only if $n$ is a positive multiple of 3'' (system {\em soundness\/}).
\item ``$n\in S$ if $n$ is a positive multiple of 3'' (system {\em completeness\/}).
\item Example derivation showing $9 \in S$:\\
\vspace{2mm}
\begin{tabular}{lllll}
  $\bid{B}$  & $\id{3}\in \id{S}$           & $\id{3}\in \id{S}$ & $\bid{B}$\\ 
  \cline{2-3}
  $\bid{R}$  & $\id{3} + \id{3} \in \id{S}$ &                    & $\id{3}\in \id{S}$ & $\bid{B}$ \\ 
  \cline{2-4} 
  $\bid{R}$  & $\id{3} + \id{3} + \id{3} \in \id{S}$ &           &                    &  \\ 
\end{tabular}
\end{itemize}
\end{wideslide}
\begin{wideslide}[bm=,toc=]{Recursive Definitions}
\begin{itemize}
\item A Post system with multiple axioms.
\begin{tabbing}
{\bf R1}XX \=  \kill
{\bf B1} \>
        \(\begin{array}[t]{l}
        4\in S
        \end{array}\) \\[2ex]
{\bf B2} \>
        \(\begin{array}[t]{l}
        5\in S
        \end{array}\) \\[2ex]
        
{\bf R} \>
        \(\begin{array}[t]{l}
        x \in S \;\;\;y \in S \\
        \hline
        x + y \in S
        \end{array}\)
\end{tabbing}
\item Compare to postage stamp problem in Rosen p. 337.
\item Claim: This system defines the set of all positive integers greater than or
equal to 12.
\item Well-suited to strong induction.
\end{itemize}
\end{wideslide}



\begin{wideslide}[bm=,toc=]{Recursive definitions}
\begin{itemize}
\item A Post system recursively defining the set {\em P\/} of all well-formed formulae in propositional logic:
\begin{displaymath}
\begin{array}{lll}
        \begin{array}[t]{l}
        \bid{T}\in P
        \end{array}
&
        \begin{array}[t]{l}
        \bid{F}\in P
        \end{array}
&
	\begin{array}[t]{l}
        x \in P \\
        \hline
        \neg x \in P
        \end{array} \\[6ex]

	\begin{array}[t]{l}
	x \in P \;\;y \in P \\
	\hline
	x \wedge y \in P
	\end{array}
&
	\begin{array}[t]{l}
	x \in P \;\;y \in P \\
	\hline
	x \vee y \in P
	\end{array}
&
	\begin{array}[t]{l}
	x \in P \;\;y \in P \\
	\hline
	x \Rightarrow y \in P
	\end{array} \\[6ex]

	\begin{array}[t]{l}
        x \in \id{Var} \\
        \hline
        x \in P
        \end{array}
\end{array}
\end{displaymath}
%\item Consequents have larger terms than antecedents except in last rule.
\end{itemize}
\end{wideslide}
\begin{wideslide}[bm=,toc=]{Rooted Tree Review}
  Recall the definition of rooted trees given in Rosen (p. 351).
  
  The set of \emph{rooted trees}, where a rooted tree consists of a set of
  vertices containing a distinguished vertex called the \emph{root}, and
  edges connecting these vertices, can be defined recursively by these steps:
  
  \begin{tabbing}
  {\bf RECURSIVE}XX \=  \kill
  {\bf BASIS} \>
           A single vertex $r$ is a rooted tree.\\[2ex]
  {\bf RECURSIVE} \>
          Suppose that $T_1,T_2,...,T_n$ are disjoint rooted trees with roots\\
  {\bf } \>
          $r_1,r_2,...r_n$, respectively. Then the graph formed by starting with\\
  {\bf } \>
          a root $r$, which is not in any of the rooted trees $T_1,T_2,...T_n$,\\
  {\bf } \>
          and adding an edge from $r$ to each of the vertices $r_1,r_2,...r_n$, \\
  {\bf } \>
          is also a rooted tree.\\[2ex] \\
\end{tabbing}

\end{wideslide}

\begin{wideslide}[bm=,toc=]{Rooted Tree Review}
  \begin{itemize}
    \item Let RT be the set of rooted trees (i.e. $t \in$ RT iff $x$ is a rooted tree).
    \item A list of rooted trees, $l$, is a finite ordered list of elements of RT ($t_1,t_2...t_n$).
    \item Let RTL be the set of all possible lists of a rooted trees.
    \item $nil$ denotes an empty list. 
    \item We define two helper functions: $node$ and $cons$.
  \end{itemize}
  The $node(l)$ function:
  \begin{itemize}
      \item Takes a list, $l$ as an argument and creates a new tree node that is 
            connected to every element in the list $l$.
      \item If the list is empty, the new node will be connected to nothing.
  \end{itemize}



\end{wideslide}

\begin{wideslide}[bm=,toc=]{The cons function}
    \begin{itemize}
      \item It \emph{constructs} a two-cell memory object with references to its
      arguments. For example, if we write $ pair = cons(x,y)$, $pair$ will store
      references to the elements $x$ and $y$.
      \item Repeated calls can be used to create a list. For example, $list =
      cons(z,pair)$ creates a list containing elements $z,x,y$, in that order.

    \item If $t$ is a tree and $l$ is a list of trees, $cons(t,l)$ will produce a 
          new list of trees that begins with $t$.
    \end{itemize}
\end{wideslide}

\begin{wideslide}[bm=,toc=]{Recursive definitions}
\begin{itemize}
\item A mutually-recursive definition of {\em rooted trees\/} {\em RT\/}:
\vspace{-1em}
\begin{tabbing}
{\bf R1}XX \=  \kill
{\bf B} \>
        \(\begin{array}[t]{l}
        \id{nil}\in\id{RTL}
        \end{array}\) \\[2ex]
{\bf R1} \>
        \(\begin{array}[t]{l}
        x\in\id{RT}\;\;\;y\in\id{RTL} \\
        \hline
        \id{cons}(x,y)\in\id{RTL}
        \end{array}\) \\[2ex]
{\bf R2} \>
        \(\begin{array}[t]{l}
        x\in\id{RTL} \\
        \hline
        \id{node}(x)\in\id{RT}
        \end{array}\)
\end{tabbing}
\item Compare with sloppy definition in Rosen 7th Ed., pg.\ 351.
\item It ignores mutual recursion and hence mutual induction.
\item It is not induction friendly.
\end{itemize}
\end{wideslide}

\begin{wideslide}[bm=,toc=]{Sample RT derivation}
\begin{center}
\begin{tabular}{llllll}
             &                                 & $\bid{B}$             & $\id{nil}\in\id{RTL}$           &                       & \\ \cline{4-4}
  $\bid{B}$  & $\id{nil}\in\id{RTL}$           &$\bid{R2}$             & $\id{node}(\id{nil})\in\id{RT}$ & $\id{nil}\in\id{RTL}$ & $\bid{B}$ \\ 
  \cline{2-2} \cline{4-5}
  $\bid{R2}$ & $\id{node}(\id{nil})\in\id{RT}$ & $\bid{R1}$            & \multicolumn{2}{l}{
  $\id{cons}(\id{node}(\id{nil}),\id{nil})\in\id{RTL}$} & \\
  \cline{2-5}
  $\bid{R1}$ & \multicolumn{4}{l}{$\id{cons}(\id{node}(\id{nil}),\id{cons}(\id{node}(\id{nil}),\id{nil})) \in \id{RTL}$} \\
  \cline{2-5}
  $\bid{R2}$ & \multicolumn{4}{l}{$\id{node}(\id{cons}(\id{node}(\id{nil}),\id{cons}(\id{node}(\id{nil}),\id{nil})))\in\id{RT}$}
\end{tabular}
\end{center}
\begin{figure}[h]
\centering
\includegraphics[width=3.5in, height=1.5in,keepaspectratio=true]{art/3_node_tree.eps}
\caption{3 node rooted tree derived above}
\label{2sp}
\end{figure}
\end{wideslide}




\begin{wideslide}[bm=,toc=]{Beyond recursive definitions}
\begin{itemize}
\item Post systems are as powerful as any programming language.
\item Productions can define any computation.
\item In practice they are used as {\em deductive systems\/}.
\item Gentzen and Hilbert systems are systems for deducing validity in propositional logic.
\item For instance, here's a rule of inference for {\em modus ponens\/}:
\vspace{-1em}
\begin{tabbing}
{\bf MP}XX \=  \kill
{\bf MP} \>
        \(\begin{array}[t]{l}
        x\Rightarrow y \in\id{Thm}\;\;\;x\in\id{Thm} \\
        \hline
        y\in\id{Thm}
        \end{array}\)
\end{tabbing}
\item Notice term ``$x\Rightarrow y$'' is larger than ``$y$'' in the consequent.
\item So structural induction is not possible here.
\end{itemize}
\end{wideslide}

\end{document}
