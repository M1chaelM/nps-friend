\begin{slide}[bm=,toc=]{Satisfiability and Validity}

\begin{defn}{13.12}[Ben Ari]
Let $A$ be a formula in Propositional Temporal Logic.
\end{defn}

\vspace{-2ex}

\begin{itemize}
 \item A is \emph{satisfiable} iff there is an interpretation $\mathcal{I} = (\mathcal{S}, \rho)$ for $A$ and a state $s \in \mathcal{S}$ such that $s \models_{\mathcal{I}} A$.

 \item A is \emph{valid} iff for all interpretations $\mathcal{I} = (\mathcal{S},\rho)$ for $A$ and for all states $s \in \mathcal{S}$, $s \models_{\mathcal{I}} A$. 
\begin{itemize}
\item {\bf Notation}: $\models A$. 
\end{itemize}
\end{itemize}

\end{slide}

\begin{wideslide}[bm=,toc=]{Example: Satisfiability and Validity}
\begin{ex}{13.11}[Modified]
The previous example shows $\Box p \lor \Box q$ is satisfiable, but
not valid.
\end{ex}
\begin{itemize}
\item  Satisfiable: $s_2,s_3 \models \Box p \lor \Box q$.
\item  Falsifiable: $s_0,s_1 \not \models \Box p \lor \Box q$.
\end{itemize}
\begin{center}
\begin{picture}(160,105)
\put(0,60){
  \put(10,10){\state{}{$s_{0}$}}
  \put(30,20){\vector(1,0){39}}
  \put(70,10){\state{\shortstack{$p$\\$q$}}{$s_{1}$}}
  \put(77, 9){\line(0,-1){4}}
  \put(65, 7){\oval(24,12)[b]}
  \put(65, 7){\oval(24,12)[tl]}
  \put(64,13){\vector(1,0){8}}
  \put(90,20){\vector(1,0){40}}
% \put(131,15){\vector(-1,0){41}}
  \put(130,10){\state{$q$}{$s_{2}$}}
  \put(127,35){\oval(26,20)[tr]}
  \put(150,10){\oval(24,12)[b]}
  \put(150,10){\oval(24,12)[tr]}
  \put(155,16){\vector(-1,0){6}}
  \put(127,45){\line(-1,0){94}}
  \put(33, 35){\oval(27,20)[tl]}
  \put(140,33){\vector(0,-1){0}}
}
\put(70,0){
\put(0,10){\state{$p$}{$s_{3}$}}
\put(10, 9){\line(0,-1){3}}
\put(22, 7){\oval(24,12)[b]}
\put(22, 7){\oval(24,12)[tr]}
\put(24,13){\vector(-1,0){6}}
\put(10,69){\vector(0,-1){38}}
%\put(65,70){\vector(-1,-1){45}}
%\put(15,30){\vector(1,1){45}}
}
\end{picture}
\end{center}
\end{wideslide}

\begin{wideslide}[bm=,toc=]{Satisfiability and Validity: PL vs PTL}
{\bf Definition:} \\
A \emph{substitution instance} of a propositional logic formula is
obtained by substituting PTL formulas uniformly for propositional letters.\\~\\
{\bf Example:}\\
$\Box p \land \Box q \imp \Box p$ is a substitution instance of $A \land B \imp
A$.\\~\\
{\bf Facts:}\\
\begin{itemize}
\item Any valid formula of propositional logic is a valid formula of PTL.
\begin{itemize}
\item Ex: $(A \imp B) \imp A$ is a valid formula in both PL and PTL.
\end{itemize}
\item Any \emph{substitution instance} of a valid PL formula is valid.
\begin{itemize}
\item Ex: $(\Box p \imp \Box q) \imp \Box p$ is a substitution instance of a
valid PL formula, and therefore is also valid.
\end{itemize}
\end{itemize}
\end{wideslide}

\begin{wideslide}[bm=,toc=]{Other Valid Formulas in PTL}
Not all valid formulas are substitution instances of propositional validities.
\begin{thm}{13.14}[Duality]
$\models \Box p \eqv \ngg \Diamond \ngg p$
\end{thm}

\begin{thm}{13.15}
$\models \Box (p \imp q) \imp (\Box p \imp \Box q)$
\end{thm}



\end{wideslide}


