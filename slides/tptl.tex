\begin{wideslide}[bm=,toc=]{Timed Propositional Temporal Logic}
\begin{itemize}
\item Recall the download policy:
\begin{displaymath}
\begin{array}{l}
\forall i\, \forall t\, \forall a\, ((\bid{PaidFee}(i,t)\wedge (\bid{now} - 6 < t < \bid{now})\wedge \\
\bid{Customer}(i,\bid{now})\wedge\bid{Article}(a) \Rightarrow \bid{Permitted}(i,\rid{download}(a)))
\end{array}
\end{displaymath}
%$\bid{now}$ is a constant denoting current time
\item {\bf Customer} is parameterized on time.
\item Its truth varies across time (dynamic truth).
\item Contrast with {\bf Article} whose truth is static.
\item A more natural way to express this policy uses {\em timed propositional temporal logic\/} (TPTL).
\end{itemize}
\end{wideslide}

\begin{wideslide}[bm=,toc=]{Timed Propositional Temporal Logic}
\begin{itemize}
\item Temporal logic is missing a {\bf now} constant.
\item TPTL introduces it via the ``freeze'' quantifier ``$x.$'' which freezes $x$ to the
time of the local temporal context.\footnote{
R. Alur and T. Henzinger, A Really Temporal Logic, {\em J. ACM\/},
(41)1, Jan 1994.}
\item We can rewrite the download policy in TPTL.  Suppose
\begin{enumerate}
\item $\bid{PaidFee}[i]$ is an integer-valued variable storing
the discrete time at which customer $i$ paid the fee (one for each customer).
\item $\bid{Customer}(i)$ is a predicate that is true iff $i$ is a customer.
\item $\bid{Article}(a)$ is a predicate that is true iff $a$ is an article.
\item $\bid{Permitted}(i,a)$ is a predicate that is true iff 
customer $i$ is permitted to download article $a$.
\end{enumerate}
\end{itemize}
\end{wideslide}

\begin{wideslide}[bm=,toc=]{Timed Propositional Temporal Logic}
\begin{enumerate}
\item Imagine a database updated over time to reflect current customers and when they paid their fees.
\item Represented as a {\em timed\/} linear interpretation of database states.
\item So the download policy in TPTL becomes
\begin{displaymath}
\begin{array}{l}
\forall i\, \forall a\, \\
\>\> \Box\; t.\;t - 6 < \bid{PaidFee}[i] < t\; \wedge \\
\>\>\>\>\bid{Customer}(i) \wedge \bid{Article}(a) \Rightarrow
\bid{Permitted}(i,a)
\end{array}
\end{displaymath}
which can be regarded as a correctness criterion for the database.
\end{enumerate}
\end{wideslide}
