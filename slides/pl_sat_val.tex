\documentclass[style=sailor,size=12pt]{powerdot}
\usepackage{epic,array,ecltree,url,calrsfs}
\usepackage[nointegrals]{wasysym}
\usepackage{listings}
\usepackage{epsfig}
\usepackage{amsmath}
\usepackage{amsfonts}
\usepackage{amssymb}
\usepackage{amsxtra}
\usepackage{amsthm}
\usepackage{mlextra} % Must be below ams packages
\usepackage{mathrsfs}
\usepackage{color}
\usepackage{array}
\usepackage{graphicx}
\graphicspath{ {../art/} }
\usepackage{bm}
\usepackage{tikz}
\usepackage{multicol}
\usepackage{enumitem}

\pdsetup{method=normal}

\title{Satisfiability, Validity and Consequence \\ in Propositional Logic}
\author{Foundations of Computer Science}
\date{\today}


\begin{document}
\maketitle

\begin{wideslide}[bm=,toc=]{Satisfiability and Validity}
\begin{defn}{2.38}[Ben Ari]
~\\Let $A \in \mathcal{F}$.
\begin{itemize}
\item<2-> $A$ is \emph{satisfiable} iff $v_{\mathcal{I}}(A) = T$ for \emph{some}
      interpretation $\mathcal{I}$.\\

\begin{itemize}
\item<3-> An intepretation satisfying $A$ is called a \emph{model} for $A$.
\end{itemize}

\item<4-> $A$ is valid, denoted $\models A$, iff $v_{\mathcal{I}}(A)=T$ for
      \emph{all} interpretations $\mathcal{I}$.\\
\begin{itemize}
\item<5-> A valid propositional formula is also called a \emph{tautology}.
\end{itemize}

\item<6-> $A$ is \emph{unsatisfiable} iff it is... \pause[6] not satisfiable. 
\begin{itemize}
\item<8-> That is, if
      $v_{\mathcal{I}}(A) = F$ for \emph{all} interpretations $\mathcal{I}$.
\end{itemize}

\item<9-> $A$ is \emph{falsifiable}, denoted $\not\models A$, iff it is not valid.
\begin{itemize}
\item<10->  That is, if $v_{\mathcal{I}}(A) = F$ for \emph{some} interpretation
      $\mathcal{I}$.
\end{itemize}
\end{itemize}
\end{defn}
\end{wideslide}


\begin{wideslide}[bm=,toc=]{Example 2.21}
\begin{center}
\begin{tabular}{|c|c||c|}
\hline
$p$ & $q$ & $p \imp q$ \\ \hline \hline
$T$ & $T$ & $T$  \\ \hline
$T$ & $F$ & $F$  \\ \hline
$F$ & $T$ & $T$  \\ \hline
$F$ & $F$ & $T$  \\ \hline
\end{tabular}
\end{center}
\begin{itemize}
\item Is $p \imp q$ satisfiable?
\item Is $p \imp q$ unsatisfiable?
\item Is $p \imp q$ valid?
\item Is $p \imp q$ falsifiable?
\end{itemize}
\end{wideslide}

\begin{wideslide}[bm=,toc=]{Example 2.22}
\begin{center}
\begin{tabular}{|c|c||c|c|c|c|c|}
\hline
$p$ & $q$ & $p \imp q$ & $\ngg p$ & $\ngg q$ 
& $\ngg q \imp \ngg p$ & $(p\imp q)\eqv  (\ngg q\imp \ngg p)$\\ \hline \hline
$T$ & $T$ & $T$  & $F$ & $F$  & $T$ & $T$  \\ \hline
$T$ & $F$ & $F$ & $F$ & $T$  & $F$ & $T$   \\ \hline
$F$ & $T$ & $T$ & $T$ & $F$  & $T$ & $T$   \\ \hline
$F$ & $F$ & $T$ & $T$ & $T$  & $T$ & $T$   \\ \hline
\end{tabular}
\end{center}
\begin{itemize}
\item Is $(p\imp q)\eqv  (\ngg q\imp \ngg p)$ satisfiable?
\item Is $(p\imp q)\eqv  (\ngg q\imp \ngg p)$ unsatisfiable?
\item Is $(p\imp q)\eqv  (\ngg q\imp \ngg p)$ valid?
\item Is $(p\imp q)\eqv  (\ngg q\imp \ngg p)$ falsifiable?
\end{itemize}
\end{wideslide}

\begin{wideslide}[bm=,toc=]{Example 2.41}
\begin{displaymath}
\begin{array}{|c|c||c|c|c|c|}
\hline
p & q & p \vee q & \ngg p & \ngg q &
(p \vee q) \wedge \ngg p \wedge \ngg q \\ \hline \hline
T & T & T & F & F & F \\ \hline
T & F & T & F & T & F \\ \hline
F & T & T & T & F & F \\ \hline
F & F & F & T & T & F \\ \hline
\end{array}
\end{displaymath}
\begin{itemize}
\item Is $(p\imp q)\eqv  (\ngg q\imp \ngg p)$ satisfiable?
\item Is $(p\imp q)\eqv  (\ngg q\imp \ngg p)$ unsatisfiable?
\item Is $(p\imp q)\eqv  (\ngg q\imp \ngg p)$ valid?
\item Is $(p\imp q)\eqv  (\ngg q\imp \ngg p)$ falsifiable?
\end{itemize}
\end{wideslide}

\begin{wideslide}[bm=,toc=]{Negations of Valid and Satisfiable Formulas}
\begin{thm}{2.39}[Ben Ari]
~\\Let $A \in \mathcal{F}$. 
\begin{itemize}
\item $A$ is valid if and only if $\ngg A$ is unsatisfiable.
\item $A$ is satisfiable if and only if $\ngg A$ is falsifiable.
\end{itemize}
\end{thm}
\emph{Intuition:}
\begin{itemize}
\item If $v_{\mathcal{I}}(A) = T$ for all interpretations, its
negation must be false for all interpretations.
\item If $v_{\mathcal{I}}(A) = T$ for at least one interpretation, its
negation must be false for at least one interpretation.
\end{itemize}
See text for a formal proof.
\end{wideslide}




\begin{wideslide}[bm=,toc=]{Valid, satisfiable, falsifiable and unsatisfiable formulas}
\begin{center}
\unitlength=1.5pt
\begin{picture}(200,100)
\put(0,20){
\put(100,40){\oval(200,80)}
\put(60,30){\oval(80,30)}
\put(35,15){\makebox(50,30){\shortstack[l]%
{True in\\all interpretations.}}}
\put(35,50){\makebox(50,30){\shortstack[l]%
{True in some interpretations;\\false in others.}}}
\put(120,0){\line(0,1){80}}
\put(120,35){\makebox(80,30){\shortstack[l]%
{False in\\all interpretations.}}}
}
\put(101,0){\makebox(40,20)[b]{Falsifiable}}
\put(110,10){\vector(0,1){18}}
\put(130,10){\vector(0,1){18}}
\put(151,0){\makebox(40,20)[b]{Unsatisfiable}}
\put(170,10){\vector(0,1){18}}
\put(50,0){\makebox(50,20)[b]{Satisfiable}}
\put(65,10){\vector(0,1){30}}
\put(85,10){\vector(0,1){18}}
\put(15,0){\makebox(50,20)[b]{Valid}}
\put(40,10){\vector(0,1){30}}
\end{picture}
\end{center}
\end{wideslide}

\begin{wideslide}[bm=,toc=]{Satisfiability of a Set of Formulas}
\begin{defn}{2.42}[Ben Ari]
A set of formulas $U = \{A_1,....\}$ is (\emph{simultaneously})
\emph{satisfiable} iff there exists an interpretation $\mathcal{I}$
such that $v_{\mathcal{I}}(A_i) = T$ for all $i$. The satisfying interpretation
is a \emph{model} of $U$. $U$ is \emph{unsatisfiable} iff for every
interpretation $\mathcal{I}$, there exists an $i$ such that $v_{\mathcal{I}}(A_i) = F$.
\end{defn}
\begin{ex}{2.43}[Ben Ari]
\end{ex}
\vspace*{-3ex}
\begin{itemize}
\item $U_1 = \{p, \ngg p \lor q, q \land r \}$
\item $U_2 = \{p, \ngg p \lor q, \ngg p \}$
\end{itemize}
\end{wideslide}

\begin{wideslide}[bm=,toc=]{Satisfiability of Sets of Formulas:\\ Some Elementary
Theorems}
\begin{thm}{2.44}
{ \em If} $U$ {\em is satisfiable, then so is} $U - \{A_i\}$ {\em for all} $i$. 
\end{thm}
\begin{thm}{2.45}
{ \em If} $U$ {\em is satisfiable and B is valid, then} $U \cup \{B\}$ {\em is
  satisfiable}. 
\end{thm}
\begin{thm}{2.46}
{ \em If} $U$ {\em is unsatisfiable, then for any formula B,} $U \cup \{B\}$ {\em is
  unsatisfiable}. 
\end{thm}
\begin{thm}{2.47}
{ \em If} $U$ {\em is unsatisfiable and for some } $i$, $A_i$ { \em is valid,
  then} $U - \{A_i\}$ {\em is unsatisfiable}. 
\end{thm}
\end{wideslide}

\begin{wideslide}[bm=,toc=]{Logical Consequence}
\begin{defn}{2.48}[Ben Ari]
Let $U$ be a set of formulas and $A$ a formula. $A$ is a \emph{logical
  consequence} of $U$, denoted $U \models A$, iff every model of $U$ is a
  model of $A$.
\end{defn}
The effect of $U$ is to reduce the number of interpretations that must satisfy
$A$.
\begin{ex}{2.49}[Ben Ari]
Let $A = (p \lor r) \land (\ngg q \lor \ngg r)$ and $U = \{p,\ngg q\}$.
Then $U \models A$, but $\not\models A$.
\end{ex}
Note that $\imp$ is an operator in the object language, whereas $\models$
is a symbol for a concept in the metalanguage. However:
\begin{thm}{2.50}
$U \models A$ \emph{if and only if } $\bigwedge\limits_{i} A_i \imp A$. 
\end{thm}
\end{wideslide}

\begin{wideslide}[bm=,toc=]{Logical Consequence}
\begin{ex}{2.52}[Ben Ari]
~\\Since we know:
\begin{itemize}
\item $\{p,\ngg q\} \models (p \lor r) \land (\ngg q \lor \ngg r)$
\end{itemize}
it follows by Theorem 2.50 that:
\begin{itemize}
\item $\models (p \land \ngg q) \imp (p \lor r) \land (\ngg q \lor \ngg r)$
\end{itemize}
\end{ex}
\begin{thm}{2.53}
If $U \models A$ \emph{then} $U \cup \{B\} \models A$ \emph{for any formula} $B$. 
\end{thm}
\begin{thm}{2.54}
If $U \models A$ \emph{ and B is valid then} $U - \{B\} \models A$. 
\end{thm}

\end{wideslide}

\begin{wideslide}[bm=,toc=]{Theories and Axioms}
\begin{defn}{2.55}[Ben Ari]
Let $\mathcal{T}$ be a set of formulas. $\mathcal{T}$ \emph{is closed under
logical consequence} iff for all formulas $A$, if $\mathcal{T} \models A$ then
$A \in \mathcal{T}$. A set of formulas that is closed under logical consequence
is a \emph{theory}. The elements of $\mathcal{T}$ are \emph{theorems}.
\end{defn}
\begin{defn}{2.56}[Ben Ari]
Let $\mathcal{T}$ be a theory. $\mathcal{T}$ is said to be \emph{axiomatizable} 
iff there exists a set of formulas $U$ such that  $\mathcal{T} = \{A | U \models
A\}$. The set of formulas $U$ are the \emph{axioms} of $\mathcal{T}$. If $U$ is
finite, $\mathcal{T}$ is said to be \emph{finitely axiomatizable}.
\end{defn}
Examples:
\begin{itemize}
\item Euclidean Geometry---consequences of $5$ formulas.
\item Peano Arithmetic---axiomatizable, but not finitely axiomatizable.
\end{itemize}


\end{wideslide}


\end{document}

