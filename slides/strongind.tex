
\documentclass[style=sailor,size=12pt]{powerdot}
\usepackage{epic,array,ecltree,wasysym,url}
\usepackage{mlextra}

\newcommand{\id}[1]{\mbox{\it #1\/}}
\newcommand{\rid}[1]{\mbox{\rm #1}}
\newcommand{\sid}[1]{\mbox{\sf #1}}
\newcommand{\bid}[1]{\mbox{\bf #1}}
\newcommand{\tinysz}[1]{\mbox{\tiny $#1$}}

\pdsetup{method=normal}

\begin{document}

\begin{wideslide}[bm=,toc=]{Strong induction}
\begin{itemize}
\item A Post system supposedly for positive multiples of 3:
\begin{tabbing}
{\bf R2}XX \=  \kill
{\bf B} \>
        \(\begin{array}[t]{l}
        3\in S
        \end{array}\) \\[2ex]
{\bf R} \>
        \(\begin{array}[t]{l}
        x \in S \;\;\;y \in S \\
        \hline
        x + y \in S
        \end{array}\)
\end{tabbing}
\item We shall prove the soundness and completeness of this system with respect to
the set of all positive multiples of 3.
\end{itemize}
\end{wideslide}

\begin{wideslide}[bm=,toc=]{Strong induction}
{\bf Theorem}. $n\in S$ iff $n$ is a positive multiple of 3.
\vspace{1em}

{\bf Proof}.  As the statement is an ``iff'', it has two parts, the ``only-if'' part (soundness)
and the ``if'' part (completeness).

\vspace{1em} 
(only-if): $n\in S$ only if $n$ is a positive multiple of 3.
This is proved by strong induction on the height of the derivation of $n\in S$.

\vspace{1em}
{\em Basis\/}: derivation height is zero.  Then the derivation must end with an
application of rule {\bf B}, which implies $n=3$.
And 3 is a positive multiple of 3 since $3\cdot 1=3$.

\vspace{1em}
{\em Inductive hypothesis\/}:  Suppose for all $k$ where $0\leq k\leq n$ and $n\geq 0$,
$n$ is a positive multiple of 3 if $n\in S$ has derivation height $k$.
\end{wideslide}

\begin{wideslide}[bm=,toc=]{Strong induction}
{\em Inductive step}.
We must show $m$ is a positive multiple of 3 if $m\in S$ has derivation height $n+1$ and $n\geq 0$.

\vspace{1em}
Suppose $m\in S$ has derivation height $n+1$ and $n\geq 0$.
Then since $n\geq 0$, the derivation must end with an application of rule {\bf R}.

\vspace{1em}
Thus $m$ can be expressed as the sum of $i$ and $j$ where $i\in S$ and $j\in S$.
By induction, $i$ and $j$ are positive multiples of 3, meaning there are positive integers
$a$ and $b$ such that $i=a3$ and $j=b3$.
Hence 
\[m = i + j = a3 + b3 = (a+b)3
\]
thus establishing $m$ is a positive multiple of 3.
\end{wideslide}

\begin{wideslide}[bm=,toc=]{Weak induction}
(if): $n\in S$ if $n$ is a positive multiple of 3.
Suppose $n$ is a positive multiple of 3.
Then it can be expressed as $3k$ for some positive integer $k$.
We prove by weak induction on $k$ that $3k\in S$.

\vspace{1em}
{\em Basis\/}: $k=1$.  By axiom {\bf B}, $3\in S$ so $3k\in S$
since $3k = 3$.

\vspace{1em}
{\em Inductive hypothesis\/}:  Suppose $3k\in S$ where $k\geq 0$.

\vspace{1em}
{\em Inductive step}.
We must show $3(k+1)\in S$ where $k\geq 0$.
We have $3(k+1) = 3k + 3$.
By induction, $3k\in S$ and by axiom {\bf B}, $3\in S$.
Therefore by rule {\bf R}, $3k + 3\in S$, or $3(k+1)\in S$.
\end{wideslide}

\end{document}

Suppose $t\in\id{RT}$ has derivation height $n+1$ where $n\geq 0$.
Then the derivation must end with {\bf R2}. 
Therefore $t$ has the form $\id{node}(x)$
where $x\in\id{RTL}$.
The derivation of $x\in\id{RTL}$ has height $n$.
%So by induction and $S_1$, $\id{\#edges}(x)=\id{\#nodes}(x)$.
Then
\begin{displaymath}
\begin{array}{lll}
\id{\#nodes}(\id{node}(x)) & = & 1 + \id{\#nodes}(x) \\
	& = & 1+\id{\#edges}(x) + |x|\;\;\;\rid{by induction and}\;S_1 \\
	& = & 1+\id{\#edges}(\id{node}(x))
\end{array}
\end{displaymath}
Ergo, $\id{\#edges}(\id{node}(x))=\id{\#nodes}(\id{node}(x))-1$.
\end{wideslide}

\begin{wideslide}[bm=,toc=]{Mutual induction}
Now we show $S_1$.
Suppose $l\in\id{RTL}$ has derivation height $n+1$ where $n\geq 0$.
Then the derivation must end with {\bf R1}. 
Therefore $l$ has the form $\id{cons}(x,y)$
where $x\in\id{RT}$ and $y\in\id{RTL}$.
Either the derivation of $x\in\id{RT}$ or $y\in\id{RTL}$ has height $n$ while the other 
has height {\em at most\/} $n$.
%By induction and $S_2$, $\id{\#edges}(x)=\id{\#nodes}(x)-1$.
%Also by induction and $S_1$, $\id{\#edges}(y)=\id{\#nodes}(y)$.
Then
\begin{displaymath}
\begin{array}{lll}
\id{\#edges}(\id{cons}(x,y)) & = & \id{\#edges}(x) + \id{\#edges}(y) \\
	& = & \id{\#nodes}(x) - 1 + \id{\#edges}(y) \\
 & & \>\>\>\>\>\>\rid{by induction and}\;S_2 \\
	& = & \id{\#nodes}(x) - 1 + \id{\#nodes}(y) - |y| \\
 & & \>\>\>\>\>\>\rid{by induction and}\;S_1 \\
	& = & \id{\#nodes}(\id{cons}(x,y)) - 1 - |y| \\
	& = & \id{\#nodes}(\id{cons}(x,y)) - (1 + |y|) \\
	& = & \id{\#nodes}(\id{cons}(x,y)) - |\id{cons}(x,y)|
\end{array}
\end{displaymath}
%{\em quod erat demonstrandum (Q.E.D.)\/}
\end{wideslide}
\end{document}

