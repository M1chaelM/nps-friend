\documentclass[style=sailor,size=12pt,mode=present]{powerdot}
\usepackage{epic,array,ecltree,url,calrsfs}
\usepackage[nointegrals]{wasysym}
\usepackage{mlextra}
\usepackage{listings}
\graphicspath{ {../art/} }

\newcommand{\id}[1]{\mbox{\it #1\/}}
\newcommand{\rid}[1]{\mbox{\rm #1}}
\newcommand{\sid}[1]{\mbox{\sf #1}}
\newcommand{\bid}[1]{\mbox{\bf #1}}
\newcommand{\tinysz}[1]{\mbox{\tiny $#1$}}

\usepackage{epsfig}
\usepackage{amsmath}
\usepackage{amsfonts}
\usepackage{amssymb}
\usepackage{amsxtra}
\usepackage{amsthm}
\usepackage{mathrsfs}
\usepackage{color}
\usepackage{array}
\usepackage{graphicx}
\graphicspath{ {../art/} }
\usepackage{bm}
\usepackage{tikz}
\usepackage{multicol}
\usepackage{enumitem}

\renewcommand\qedsymbol{$\blacksquare$}

%Here I define some theorem styles and shortcut commands for symbols I use often
\theoremstyle{definition}

\newtheorem{innerdefn}{Definition}
\newenvironment{defn}[1]
  {\renewcommand\theinnerdefn{#1}\innerdefn}
  {\endinnerdefn}

\newtheorem{innerexample}{Example}
\newenvironment{ex}[1]
  {\renewcommand\theinnerexample{#1}\innerexample}
  {\endinnerexample}

\newtheorem{innerthm}{Theorem}
\newenvironment{thm}[1]
  {\renewcommand\theinnerthm{#1}\innerthm}
  {\endinnerthm}

\newtheorem{innerlem}{Lemma}
\newenvironment{lem}[1]
  {\renewcommand\theinnerlem{#1}\innerlem}
  {\endinnerlem}

\newcommand{\lra}{\longrightarrow}
\newcommand{\ra}{\rightarrow}
\newcommand{\surj}{\twoheadrightarrow}
\newcommand{\graph}{\mathrm{graph}}
\newcommand{\bb}[1]{\mathbb{#1}}
\newcommand{\Ell}{\mathscr{L}}
\newcommand{\Z}{\bb{Z}}
\newcommand{\Q}{\bb{Q}}
\newcommand{\R}{\bb{R}}
\newcommand{\C}{\bb{C}}
\newcommand{\N}{\bb{N}}
\newcommand{\M}{\mathbf{M}}
\newcommand{\m}{\mathbf{m}}
\newcommand{\MM}{\mathscr{M}}
\newcommand{\HH}{\mathscr{H}}
\newcommand{\Om}{\Omega}
\newcommand{\Ho}{\in\HH(\Om)}
\newcommand{\bd}{\partial}
\newcommand{\del}{\partial}
\newcommand{\bardel}{\overline\partial}
\newcommand{\textdf}[1]{\textbf{\textsf{#1}}\index{#1}}
\newcommand{\img}{\mathrm{img}}
\newcommand{\ip}[2]{\left\langle{#1},{#2}\right\rangle}
\newcommand{\inter}[1]{\mathrm{int}{#1}}
\newcommand{\exter}[1]{\mathrm{ext}{#1}}
\newcommand{\cl}[1]{\mathrm{cl}{#1}}
\newcommand{\ds}{\displaystyle}
\newcommand{\vol}{\mathrm{vol}}
\newcommand{\cnt}{\mathrm{ct}}
\newcommand{\osc}{\mathrm{osc}}
\newcommand{\LL}{\mathbf{L}}
\newcommand{\UU}{\mathbf{U}}
\newcommand{\support}{\mathrm{support}}
\newcommand{\AND}{\;\wedge\;}
\newcommand{\OR}{\;\vee\;}
\newcommand{\Oset}{\varnothing}
\newcommand{\st}{\ni}
\newcommand{\wh}{\widehat}
\newcommand{\mli}[1]{\mathit{#1}}
\newcommand{\ndiv}{\hspace{-3pt}\not|\hspace{2pt}}


\pdsetup{method=normal}

\title{Propositional Temporal Logic: Syntax and Semantics}
\author{Foundations of Computer Science}
\date{\today}

\begin{document}
\maketitle
\section[slide=false]{Syntax}
\begin{slide}[bm=,toc=]{Propositional Temporal Logic: Syntax}
\begin{defn}{13.2}[Ben Ari]
The syntax of \emph{propositional temporal logic (PTL)} is defined like
the syntax of propositional logic except for the addition of two unary
operators:
\end{defn}
\vspace{-2ex}
\begin{itemize}
\item $\Box$, read \emph{always},
\item $\Diamond$, read \emph{eventually}.
\end{itemize}
The two unary operators have the same precedence as negation.\\~\\
{\bf Informally:}
\begin{itemize}
\item $\Box$ is the universal operator meaning `for \emph{any} time in the
future'
\item $\Diamond$ is the existential operator meaning `for \emph{some} time in
the future.
\end{itemize}
\end{slide}

\begin{wideslide}[bm=,toc=]{The ``Next'' Operator}
{\bf Models of time can be:}
\begin{itemize}
\item continuous (with real-valued intervals) or
\item discrete (series of snapshots or steps).
\item Discrete models are natural for program execution.
\item Useful to express the \emph{next} instant in time.
\item We add another operator for this purpose:
\end{itemize}
\begin{defn}{13.23}[Ben Ari]
The unary operator $\Circle$ is called \emph{next}.
\end{defn}
{\bf Applications of Next:}
\begin{itemize}
\item Useful for deciding properties like satisfiability.
\item Rarely used to express properties of programs.
\item In concurrent programs, we often don't know which step is ``next.''
\end{itemize}
\end{wideslide}


\section[slide=false]{Semantics}
\begin{wideslide}[bm=,toc=]{Figure 13.1: State transition diagram}
\begin{defn}{13.4}[Ben Ari]
A \emph{state transition diagram} is a directed graph. The nodes are
\emph{states} and the edges are \emph{transitions}. Each state is
labeled with a set of propositional literals such that clashing
literals do not appear in any state.
\end{defn}
\unitlength=1.3pt
\begin{center}
\begin{picture}(140,105)
\put(0,60){
  \put(10,10){\state{\shortstack{$p$\\$\ngg q$}}{$s_{0}$}}
  \put(30,20){\vector(1,0){39}}
  \put(70,10){\state{\shortstack{$p$\\$q$}}{$s_{1}$}}
  \put(77, 9){\line(0,-1){4}}
  \put(65, 7){\oval(24,12)[b]}
  \put(65, 7){\oval(24,12)[tl]}
  \put(64,13){\vector(1,0){8}}
  \put(89,25){\vector(1,0){41}}
  \put(131,15){\vector(-1,0){41}}
  \put(130,10){\state{\shortstack{$\ngg p$\\$q$}}{$s_{2}$}}
  \put(127,35){\oval(26,20)[tr]}
  \put(127,45){\line(-1,0){94}}
  \put(33, 35){\oval(27,20)[tl]}
  \put(140,33){\vector(0,-1){0}}
}
\put(70,0){
\put(0,10){\state{\shortstack{$\ngg p$\\$\ngg q$}}{$s_{3}$}}
\put(10, 9){\line(0,-1){3}}
\put(22, 7){\oval(24,12)[b]}
\put(22, 7){\oval(24,12)[tr]}
\put(24,13){\vector(-1,0){6}}
\put(10,69){\vector(0,-1){38}}
\put(18,28){\vector(1,1){44}}
}
\end{picture}
\end{center}
\end{wideslide}

\begin{wideslide}[bm=,toc=]{Interpretations: PTL}
\begin{defn}{13.8}[Ben Ari]
An \emph{interpretation} $\mathcal{I}$ for a formula $A$ in PTL
is a pair $(\mathcal{S},\rho)$, where $\mathcal{S} = \{s_1,...,s_n\}$
is a set of states each of which is an assignment of truth values to the atomic
propositions in $A$, $s_i: \mathcal{P} \to \{T,F\}$, and $\rho$ is a binary
relation on the states, $\rho \subseteq S \times S$.
\end{defn}
Note that $(s_1,s_2) \in \rho$ can be written functionally as $s_2 \in \rho(p_1)$.
\vspace{2ex}
\begin{multicols}{2}
In graphical representations:
\begin{itemize}
\item states are usually labeled only with the atomic propositions that
are assigned $T$.
\item Atoms not shown are assigned $F$.
\end{itemize}
\unitlength=1pt
\begin{center}
\begin{picture}(160,105)
\put(0,60){
  \put(10,10){\state{$p$}{$s_{0}$}}
  \put(30,20){\vector(1,0){39}}
  \put(70,10){\state{\shortstack{$p$\\$q$}}{$s_{1}$}}
  \put(77, 9){\line(0,-1){4}}
  \put(65, 7){\oval(24,12)[b]}
  \put(65, 7){\oval(24,12)[tl]}
  \put(64,13){\vector(1,0){8}}
  \put(89,25){\vector(1,0){41}}
  \put(131,15){\vector(-1,0){41}}
  \put(130,10){\state{$q$}{$s_{2}$}}
  \put(127,35){\oval(26,20)[tr]}
  \put(127,45){\line(-1,0){94}}
  \put(33, 35){\oval(27,20)[tl]}
  \put(140,33){\vector(0,-1){0}}
}
\put(70,0){
\put(0,10){\state{}{$s_{3}$}}
\put(10, 9){\line(0,-1){3}}
\put(22, 7){\oval(24,12)[b]}
\put(22, 7){\oval(24,12)[tr]}
\put(24,13){\vector(-1,0){6}}
\put(10,69){\vector(0,-1){38}}
\put(18,28){\vector(1,1){44}}
}
\end{picture}
\end{center}
\end{multicols}
\end{wideslide}

\begin{wideslide}[bm=,toc=]{Graphical and Non-graphical Representations of $\mathcal{S}$ and $\rho$}
\begin{ex}{13.9}[Ben Ari]
\end{ex}
\begin{multicols}{2}
{\bf Graphical:}
\vspace{-4ex}
\unitlength=1.2pt
\begin{center}
\begin{picture}(160,120)
\put(0,60){
  \put(10,10){\state{$p$}{$s_{0}$}}
  \put(30,20){\vector(1,0){39}}
  \put(70,10){\state{\shortstack{$p$\\$q$}}{$s_{1}$}}
  \put(77, 9){\line(0,-1){4}}
  \put(65, 7){\oval(24,12)[b]}
  \put(65, 7){\oval(24,12)[tl]}
  \put(64,13){\vector(1,0){8}}
  \put(89,25){\vector(1,0){41}}
  \put(131,15){\vector(-1,0){41}}
  \put(130,10){\state{$q$}{$s_{2}$}}
  \put(127,35){\oval(26,20)[tr]}
  \put(127,45){\line(-1,0){94}}
  \put(33, 35){\oval(27,20)[tl]}
  \put(140,33){\vector(0,-1){0}}
}
\put(70,0){
\put(0,10){\state{}{$s_{3}$}}
\put(10, 9){\line(0,-1){3}}
\put(22, 7){\oval(24,12)[b]}
\put(22, 7){\oval(24,12)[tr]}
\put(24,13){\vector(-1,0){6}}
\put(10,69){\vector(0,-1){38}}
\put(18,28){\vector(1,1){44}}
}
\end{picture}
\end{center}
\columnbreak

{\bf Non-Graphical:}
\begin{align*}
s_0(p) = T&,\;\; s_0(q) = F\\
s_1(p) = T&,\;\; s_1(q) = T\\
s_2(p) = F&,\;\; s_2(q) = T\\
s_3(p) = F&,\;\; s_3(q) = F\\
\rho(s_0)&= \{s_1,s_2\} \\
\rho(s_1)&= \{s_1,s_2,s_3\}\\
\rho(s_2)&= \{s_1\}\\
\rho(s_3)&= \{s_2,s_3\}\\
\end{align*}

\end{multicols}
\end{wideslide}

\begin{wideslide}[bm=,toc=]{Models of Time}
\begin{itemize}
\item Placing restrictions on $\rho$, the transition function, produces
different variations of temporal logic.
\item These variations have practical significance.
\begin{itemize}
\item Contrast with PL or FOL.
\end{itemize}
\item We focus on three:
\begin{itemize}
\item Reflexivity
\item Transitivity
\item Linearity
\end{itemize}
\item The combination produces \emph{Linear Temporal Logic}.
\item Especially useful in computer science.
\end{itemize}
\end{wideslide}


\begin{wideslide}[bm=,toc=]{Reflexivity}
\begin{defn}{13.16}[Ben Ari]
An interpretation $\mathcal{I} = (\mathcal{S},\rho)$ is \emph{reflexive}
iff $\rho$ is a reflexive relation: for all $s \in \mathcal{S}$, $(s,s) \in
\rho$, or $s \in \rho(s)$ in functional notation.
\end{defn}
{\bf Impact of Restriction:}
\begin{itemize}
\item $\Diamond running$: ``eventually the programming is in the state
`running'.
\item If it is running in the current state, the intuitive meaning is true.
\begin{itemize}
\item That is, intuitively, runnning \emph{now} $\imp$ running \emph{eventually}.
\end{itemize}
\item The above holds if $\rho$ is reflexive.
\end{itemize}
\begin{thm}{13.17}
An interpretation with a reflexive transition relation is characterized by the
formula $\Box A \imp A$ (or, by duality, by the formula $A \imp \Diamond A$).
\end{thm}
\end{wideslide}

\begin{wideslide}[bm=,toc=]{Transitivity}
\begin{defn}{13.18}[Ben Ari]
An interpretation $\mathcal{I} = (\mathcal{S},\rho)$ is \emph{transitive}
iff $\rho$ is a transitive relation: for all $s_1,s_2,s_3 \in \mathcal{S}$, 
$s_2 \in \rho(s_1) \land s_3 \in \rho(s_2) \imp s_3 \in \rho(s_1)$.
\end{defn}
{\bf Impact of Restriction:}
\begin{itemize}
\item Requiring transitivity again produces an intuitive result.
\item Assume the following is proven:
\begin{itemize}
\item $s_2 \models running$ for $s_2 \in \rho(s_1) \imp s_1 \models \Diamond running$
\item $s_3 \models running$ for $s_3 \in \rho(s_2) \imp s_2 \models \Diamond running$
\end{itemize}
\item We expect that $s_3 \models running$ should prove $s_1 \models \Diamond running$ 
\end{itemize}
\begin{thm}{13.17}
An interpretation with a transitive transition relation is characterized by the
formula $\Box A \imp \Box \Box A$ (or by the formula 
$\Diamond \Diamond A \imp \Diamond A$).
\end{thm}
\end{wideslide}
\begin{wideslide}[bm=,toc=]{Properties of Reflexive, Transitive Interpretations}
{\bf Corollary 13.21}
In an interpretation that is both reflexive and transitive:
\[
\models \Box A \eqv \Box \Box A
\]
\[and\]
\[
\models \Diamond A \eqv \Diamond \Diamond A
\]

\end{wideslide}

\begin{wideslide}[bm=,toc=]{Truth in PTL}
Let $A$ be a formula in PTL, and $\rho^*$ be the reflexive, transitive closure
of $\rho$.\\
$\nu_{\mathcal{I},s}(A)$, the {\em truth value\/} of $A$ in $s$, is defined recursively as follows:
\begin{itemize}
\item If $A$ is $p\in\mathcal{P}$ then $\nu_{\mathcal{I},s}(A) = s(p)$.
\item If $A$ is $\neg A'$ then $\nu_{\mathcal{I},s}(A) = T$ iff $\nu_{\mathcal{I},s}(A') = F$.
\item If $A$ is $A'\vee A''$ then $\nu_{\mathcal{I},s}(A) = T$ iff $\nu_{\mathcal{I},s}(A') = T$ or
$\nu_{\mathcal{I},s}(A'') = T$.

and similarly for the other Boolean operators.

\item If $A$ is $\Box A'$ then $\nu_{\mathcal{I},s}(A) = T$ iff $\nu_{\mathcal{I},s}(A') = T$ for
all states $s'\in\rho^*(s)$.

\item If $A$ is $\Diamond A'$ then $\nu_{\mathcal{I},s}(A) = T$ iff $\nu_{\mathcal{I},s}(A') = T$ for
some state $s'\in\rho^*(s)$.

\end{itemize}
{\bf Notation:}\\
\begin{itemize}
\item We abbreviate $\nu_{\mathcal{I},s}(A)$ as $s\models_\mathcal{I} A$. 
\item When interpretation $\mathcal{I}$ is clear from the context, it is omitted
giving: 
\item $s\models A$ iff $\nu_s(A) = T$.
\end{itemize}
\end{wideslide}

\begin{wideslide}[bm=,toc=]{Example: Computing Truth Value}
\begin{ex}{13.11}[Modified]
Computing truth value for $\Box p \lor \Box q$ for each state
$s$. 
\end{ex}
\begin{itemize}
\item $s_0 \in \rho^*(s_0)$, but $s_0 \not \models p$ and $s_0 \not \models q$, so 
$s_0 \not \models p \lor q$. Thus, $s_0 \not \models \Box p \lor \Box q$.

\item $\rho^*(s_1) = \{s_1,s_2,s_3\}$. $s_2 \not \models p$ so $s_1 \not \models
\Box p$. Also, $s_3 \not \models q$ so $s_1 \not \models \Box q$. Therefore, $s_1 \not
\models \Box p \lor \Box q$.
\item $\rho^*(s_2) = \{s_2\}$. Since $s_2 \models q$, we have $s_2 \models \Box q$
and $s_2 \models \Box p \lor \Box q$.
\item $\rho^*(s_3) = \{s_3\}$. Since $s_3 \models p$,  $s_3 \models \Box p$ and
therefore $s_3 \models \Box p \lor \Box q$.
\end{itemize}
\begin{center}
\begin{picture}(160,105)
\put(0,60){
  \put(10,10){\state{}{$s_{0}$}}
  \put(30,20){\vector(1,0){39}}
  \put(70,10){\state{\shortstack{$p$\\$q$}}{$s_{1}$}}
  \put(77, 9){\line(0,-1){4}}
  \put(65, 7){\oval(24,12)[b]}
  \put(65, 7){\oval(24,12)[tl]}
  \put(64,13){\vector(1,0){8}}
  \put(90,20){\vector(1,0){40}}
% \put(131,15){\vector(-1,0){41}}
  \put(130,10){\state{$q$}{$s_{2}$}}
  \put(127,35){\oval(26,20)[tr]}
  \put(150,10){\oval(24,12)[b]}
  \put(150,10){\oval(24,12)[tr]}
  \put(155,16){\vector(-1,0){6}}
  \put(127,45){\line(-1,0){94}}
  \put(33, 35){\oval(27,20)[tl]}
  \put(140,33){\vector(0,-1){0}}
}
\put(70,0){
\put(0,10){\state{$p$}{$s_{3}$}}
\put(10, 9){\line(0,-1){3}}
\put(22, 7){\oval(24,12)[b]}
\put(22, 7){\oval(24,12)[tr]}
\put(24,13){\vector(-1,0){6}}
\put(10,69){\vector(0,-1){38}}
%\put(65,70){\vector(-1,-1){45}}
%\put(15,30){\vector(1,1){45}}
}
\end{picture}
\end{center}

\end{wideslide}

\section[slide=false]{Satisfiability and Validity}
\begin{slide}[bm=,toc=]{Satisfiability and Validity}

\begin{defn}{13.12}[Ben Ari]
Let $A$ be a formula in Propositional Temporal Logic.
\end{defn}

\vspace{-2ex}

\begin{itemize}
 \item A is \emph{satisfiable} iff there is an interpretation $\mathcal{I} = (\mathcal{S}, \rho)$ for $A$ and a state $s \in \mathcal{S}$ such that $s \models_{\mathcal{I}} A$.

 \item A is \emph{valid} iff for all interpretations $\mathcal{I} = (\mathcal{S},\rho)$ for $A$ and for all states $s \in \mathcal{S}$, $s \models_{\mathcal{I}} A$. 
\begin{itemize}
\item {\bf Notation}: $\models A$. 
\end{itemize}
\end{itemize}

\end{slide}

\begin{wideslide}[bm=,toc=]{Example: Satisfiability and Validity}
\begin{ex}{13.11}[Modified]
The previous example shows $\Box p \lor \Box q$ is satisfiable, but
not valid.
\end{ex}
\begin{itemize}
\item  Satisfiable: $s_2,s_3 \models \Box p \lor \Box q$.
\item  Falsifiable: $s_0,s_1 \not \models \Box p \lor \Box q$.
\end{itemize}
\begin{center}
\begin{picture}(160,105)
\put(0,60){
  \put(10,10){\state{}{$s_{0}$}}
  \put(30,20){\vector(1,0){39}}
  \put(70,10){\state{\shortstack{$p$\\$q$}}{$s_{1}$}}
  \put(77, 9){\line(0,-1){4}}
  \put(65, 7){\oval(24,12)[b]}
  \put(65, 7){\oval(24,12)[tl]}
  \put(64,13){\vector(1,0){8}}
  \put(90,20){\vector(1,0){40}}
% \put(131,15){\vector(-1,0){41}}
  \put(130,10){\state{$q$}{$s_{2}$}}
  \put(127,35){\oval(26,20)[tr]}
  \put(150,10){\oval(24,12)[b]}
  \put(150,10){\oval(24,12)[tr]}
  \put(155,16){\vector(-1,0){6}}
  \put(127,45){\line(-1,0){94}}
  \put(33, 35){\oval(27,20)[tl]}
  \put(140,33){\vector(0,-1){0}}
}
\put(70,0){
\put(0,10){\state{$p$}{$s_{3}$}}
\put(10, 9){\line(0,-1){3}}
\put(22, 7){\oval(24,12)[b]}
\put(22, 7){\oval(24,12)[tr]}
\put(24,13){\vector(-1,0){6}}
\put(10,69){\vector(0,-1){38}}
%\put(65,70){\vector(-1,-1){45}}
%\put(15,30){\vector(1,1){45}}
}
\end{picture}
\end{center}
\end{wideslide}

\begin{wideslide}[bm=,toc=]{Satisfiability and Validity: PL vs PTL}
{\bf Definition:} \\
A \emph{substitution instance} of a propositional logic formula is
obtained by substituting PTL formulas uniformly for propositional letters.\\~\\
{\bf Example:}\\
$\Box p \land \Box q \imp \Box p$ is a substitution instance of $A \land B \imp
A$.\\~\\
{\bf Facts:}\\
\begin{itemize}
\item Any valid formula of propositional logic is a valid formula of PTL.
\begin{itemize}
\item Ex: $(A \imp B) \imp A$ is a valid formula in both PL and PTL.
\end{itemize}
\item Any \emph{substitution instance} of a valid PL formula is valid.
\begin{itemize}
\item Ex: $(\Box p \imp \Box q) \imp \Box p$ is a substitution instance of a
valid PL formula, and therefore is also valid.
\end{itemize}
\end{itemize}
\end{wideslide}

\begin{wideslide}[bm=,toc=]{Other Valid Formulas in PTL}
Not all valid formulas are substitution instances of propositional validities.
\begin{thm}{13.14}[Duality]
$\models \Box p \eqv \ngg \Diamond \ngg p$
\end{thm}

\begin{thm}{13.15}
$\models \Box (p \imp q) \imp (\Box p \imp \Box q)$
\end{thm}



\end{wideslide}

\section[slide=false]{Linear Temporal Logic}
\begin{slide}[bm=,toc=]{Linear Temporal Logic}

\begin{defn}{13.22}
An interpretation $\mathcal{I} = (\mathcal{S},\rho)$ is \emph{linear}
iff $\rho$ is a function. That is, for all $s \in \mathcal{S}$, 
there is \emph{at most} one $s' \in \mathcal{S}$ such that $s' \in \rho(s)$.
\end{defn}

\begin{defn}{13.22b}[Volpano]
An interpretation $\mathcal{I} = (\mathcal{S},\rho)$ is \emph{linear}
and \emph{total} iff $\rho$ is a total function. That is, for all $s \in \mathcal{S}$, 
there is \emph{exactly} one $s' \in \mathcal{S}$ such that $s' \in \rho(s)$.
\end{defn}

The logic produced by adopting a total, linear transition function and following the truth
assignment given (over the reflexive transitive closure of its transition
    function) is called \emph{Linear Temporal Logic}.

\end{slide}

\begin{wideslide}[bm=,toc=]{Interpretations in Linear Temporal Logic}

The restrictions on $\rho$ imply that interpretation in LTL can be represented
as infinite paths:
\unitlength=1.2pt
\begin{center}
\begin{picture}(200,30)
\put(  0, 0){\state{$\ngg p$}{$s_{0}$}}
\put( 20,10){\vector(1,0){40}}
\put( 60, 0){\state{$\ngg p$}{$s_{1}$}}
\put( 80,10){\vector(1,0){40}}
\put(120, 0){\state{$p$}{$s_{2}$}}
\put(140,10){\vector(1,0){40}}
\put(180, 5){\makebox(20,10){\ldots}}
\end{picture}
\end{center}

Note that the \emph{next} operator is self-dual in a linear interpretation.
\begin{thm}{13.25} A linear interpretation whose relation $\rho$
is a function is characterized by the formula $\Circle A \eqv \ngg
\Circle \ngg A$.
\end{thm}

\end{wideslide}

\begin{wideslide}[bm=,toc=]{Examples of Valid Formulas in LTL}
The following are direct consequences of the restriction on
interpretations in LTL.
\begin{thm}{13.31}
\[
\begin{array}{ll}
\models \Box p \imp \Circle p, & \models \Circle p \imp \Diamond p,\\ 
~&~ \\
\models \Box p \imp \Diamond p, & \models \Circle p \eqv \ngg \Circle \ngg p. \\
\end{array}
\]
\end{thm}
\vspace{2ex}
Other valid formulas establish the following important properties:
\begin{itemize}
\item Inductive
\item Distributive
\item Commutative
\end{itemize}

\end{wideslide}

\begin{wideslide}[bm=,toc=]{Induction in LTL}
The following provides a method for proving properties of LTL formulas
inductively:\\
\begin{thm}{13.32}
\[
\begin{array}{ll}
\models \Box p \eqv p \land \Circle \Box p, & \models \Diamond p \eqv p \lor \Circle \Diamond p,\\ 
\end{array}
\]
\end{thm}
\vspace{2ex}
Induction in LTL is based upon the following valid formula:
\[
\begin{array}{ll}
\models \Box (p \imp \Circle p) \imp (p \imp \Box p)\\
\end{array}
\]
(Recall induction rules from the first section of the course.)

\end{wideslide}

\begin{wideslide}[bm=,toc=]{Distributivity}
$\Circle$ distributes over conjuction and disjunction:
\begin{align*}
&\models \Circle (p \lor q) \eqv (\Circle p \lor \Circle q),\\
&\models \Circle (p \land q) \eqv (\Circle p \land \Circle q).
\end{align*}

$\Box$ and $\Diamond$ behave analogously to FOL quantifiers:
\begin{align*}
&\models \Box (p \land q) \eqv (\Box p \land \Box q), & \models \Diamond (p \lor q) \eqv (\Diamond p \lor \Diamond q), \\
&\models (\Box p \lor \Box q) \imp \Box (p \lor q) & \models \Diamond (p \land q) \imp (\Diamond p \land \Diamond q).
\end{align*}
Similarly, over implication:
\begin{align*}
&\models \Box (p \imp q) \imp (\Box p \imp \Box q),\\
&\models (\Diamond p \imp \Diamond q) \imp \Diamond (p \imp q),\\
&\models \Circle (p \imp q) \eqv (\Circle p \imp \Circle q).
\end{align*}
%
\end{wideslide}
\begin{wideslide}[bm=,toc=]{Commutativity}
$\Circle$ commutes with $\Box$ and $\Diamond$: 
\begin{align*}
&\models \Box \Circle p \eqv \Circle \Box p,\\
&\models \Diamond \Circle p \eqv \Circle \Diamond p.
\end{align*}

$\Box$ and $\Diamond$ commute in one direction: 
\begin{align*}
&\models  \Diamond \Box p \imp \Box \Diamond  p
\end{align*}

Note that if $p$ is eventually always true and $q$ is always eventually
true, we have:
\begin{thm}{13.34}
\begin{align*}
\models (\Diamond \Box p \land \Box \Diamond q) \imp \Box \Diamond(p \land q)
\end{align*}
\end{thm}


\end{wideslide}



\end{document}

