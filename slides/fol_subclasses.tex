\begin{slide}[bm=,toc=]{Decidable Subclasses}
\begin{thm}{12.5}
There are decision procedures for the validity of pure PCNF formulas whose
prefixes are of one of the following forms (where $m,n \geq 0$):
\[ \forall x_1\cdots \forall x_n \exists y_1 \cdots \exists y_m, \]
\[ \forall x_1\cdots \forall x_n \exists y \forall z_1 \cdots \forall z_m, \]
\[ \forall x_1\cdots \forall x_n \exists y_1 \exists y_2 \forall z_1 \cdots \forall z_m. \]
\pause
These classes are conveniently abbreviated: 

\[    \forall^*\exists^*, \forall^*\exists \forall^*,   \forall^*\exists \exists \forall^*\]
\end{thm}
\end{slide}

\begin{wideslide}[bm=,toc=]{Undecidable Subclasses}
\begin{thm}{12.6}
There are \emph{no} decision procedures for the validity of pure PCNF formulas whose
prefixes have one of the following forms:
\[ \exists z \forall x_1\cdots \forall x_n \exists y_1 \cdots \exists y_m, \]
\[ \forall x_1\cdots \forall x_n \exists y_1 \exists y_2 \exists y_2 \forall z_1 \cdots \forall z_m. \]
\pause
For the first prefix, the result holds even if $n = m = 1$:
\pause
\[ \exists z \forall x_1 \exists y_1, \]
\pause
and for the second prefix, the result holds even if $n = 0, m = 1$:
\pause
\[ \exists y_1 \exists y_2 \exists y_2 \forall z_1. \]
\pause
Even if the matrix is restricted to contain only binary predicate symbols, there
is still no decision procedure.
\end{thm}
\end{wideslide}

\begin{wideslide}[bm=,toc=]{Application to Policy Statements}
\begin{ex}{3.2}[Halpern and Weissman]
Consider the policy `anyone who is accompanied by a librarian
may enter the stacks'.\\~\\
\pause
In first-order logic:
\begin{align*}
& \forall x_1 (\exists x_2 (\bid{Librarian}(x_2) \land \bid{Accompanies}(x_2,x_1)) \imp \\
& \bid{Permitted}(x_1, enter(stacks)))
\end{align*}
\end{ex}
\begin{itemize} 
\item<3-> Note that \emph{enter} is a function.
\item<4-> Validity for existential formulas with functions is undecidable. 
\item<5-> Need a more restricted sublanguage.
\item<6-> Decidable is not enough---must also be \emph{tractable}.
\end{itemize}
\end{wideslide}

\begin{wideslide}[bm=,toc=]{A Tractable Sublanguage for Access
  Control}
Halpern and Weissman propose and analyze the complexity and expressiveness
of several fragments of FOL. One example:
\pause
\begin{thm}{4.2}[Halpern and Weissman]
Let $\Phi$ be a vocabulary that contains {\bf Permitted} (and possibly
other predicate, constant, and function symbols). Let $\mathcal{L}_6$
consist of all closed formulas in $\mathcal{L}^{fo}(\Phi)$ of the
form $E \land P \imp \bid{Permitted}(t,t')$, where $P$ is a conjunction
of standard policies and both $t$ and $t'$ and closed terms of the
appropriate sort, such that
\end{thm}
\vspace{-2ex}
\begin{enumerate}
\renewcommand{\labelenumi}{\alph{enumi})}
\item<3-> E is a basic environment with $m$ constants,
\item<4-> no policy in P has an inequality in its antecendent, and
\item<5-> there are no bipolars in P relative to the equality statements
in E.
\end{enumerate}
\pause[4]
\emph{If each literal in each policy has at most one variable
  that does not appear in {\bf Permitted}, then we can determine
    the validity of the formula in time $O((|E| + m|P|)log|E|)$.}

    {\tiny (Refer to paper for detailed explanation of these constraints.)}
\end{wideslide}



