\begin{wideslide}[bm=,toc=]{PCNF for First Order Logic}
{\bf Recall:}
\begin{itemize}
\item CNF in propositional logic: conjunction of disjunctions of literals.
\item Clausal form: represented as a set of literals.
\end{itemize}
\pause
We now generalize to first order logic:
\begin{defn}{9.9}
A formula is in \emph{prenex conjunctive normal form (PCNF)} iff it is of the
form:
\pause
\[
  Q_1x_1\cdots Q_n x_n M
  \]
\pause
  where 
  \begin{itemize}
  \item<5-> the $Q_i$ are quantifiers and 
  \item<6-> $M$ is a quantifier-free formula in CNF.
  \item<7-> The sequence $Q_1x_1 \cdots Q_n x_n$ is the \emph{prefix} and 
  \item<8-> $M$ is the \emph{matrix}.
  \end{itemize}
\end{defn}

\end{wideslide}


\begin{wideslide}[bm=,toc=]{Clausal Form for First Order Logic}
\begin{defn}{9.11}
Let $A$ be a formula in first order logic such that $A$
\begin{itemize}
\item<2-> is \emph{closed},
\item<3-> is in PCNF 
\item<4-> has a prefix that consists only of \emph{universal} quantifiers. 
\end{itemize}
\pause[4]
The \emph{clausal form} of $A$ consists of the matrix of $A$ written as a set of clauses.
\end{defn}
\pause
\begin{ex}{9.10}
The following formula is in PCNF:
\[
  \forall y \forall z([p(f(y)) \lor \ngg p (g(z)) \lor q(z)] \land [ \ngg q(z)
      \lor \ngg p (g(z)) \lor q(y)])
  \]
\end{ex}
\pause
\begin{ex}{9.12}
The formula in example 9.10 is closed and has only universal quantifiers, so it
can be written in clausal form as:
\[
   \{\{p(f(y)), \ngg p(g(z)), q(z)\}, \{ \ngg q(z), \ngg p (g(z)), q(y)\}\}
  \]
\end{ex}
\end{wideslide}
\begin{wideslide}[bm=,toc=]{Skolem's Theorem}
{\bf Recall:} every formula in PL has an equivalent in CNF.
\begin{itemize}
\item<2-> Not true for first-order logic.
\item<3-> But, for every formula in FOL there is an equisatisfiable formula
in clausal form.
\end{itemize}
\pause[3]
\begin{thm}{9.13}[Skolem]
Let $A$ be a closed formula. Then there exists a formula $A'$ in clausal form
such that $A \approx A'$.
\end{thm}
\pause
\textbf{Additional Notes:}
\begin{itemize}
\item<6-> $A \approx A'$: $A$ is satisfiable iff $A'$ is satisfiable. 
\item<7-> We can find $A'$ using a process called \emph{Skolemization}. 
\item<8-> This process uses functions to eliminate existential quantifiers.
\end{itemize}
\end{wideslide}

\begin{wideslide}[bm=,toc=]{Intuition for Skolem's Algorithm}
Existential quantifiers indicate a function-like relationship.
\begin{itemize}
\item<2-> Consider $A = \forall x \exists y p(x,y)$
\item<3-> ``For all $x$, \emph{produce} a value $y$ associated with that $x$ such
     that $p$ is true.''
\item<4-> Similar in sense to $y = f(x)$.
\item<5-> Replacing gives $A' = \forall x p(x,(f(x))$
\end{itemize}
\vspace{2ex}
\pause[5]
Effects of replacement process:
\begin{itemize}
\item<7-> Eliminates existential quantifiers.
\item<8-> Introduction of function symbols narrows the choice of models.
\item<9-> Relations are \emph{many-many}.
\item<10-> Functions are relations that are \emph{many-one}.
\item<11-> The modification ``finds'' \emph{one} example necessary to satisfy the existence claim.
\item<12-> Many other possible examples are excluded.
\end{itemize}
\end{wideslide}

\begin{wideslide}[bm=,toc=]{Skolemization Example}
\begin{ex}{9.14}
~\\Consider the formulas:
\vspace{-2ex}
\[A = \forall x \exists y p(x,y) \text{ and } A' = \forall x p(x,f(x))\]
\end{ex}
\vspace{-2ex}
\pause
Under the following interpretations:
\begin{itemize}
\item<3-> $\mathcal{I} = (\Z,\{>\},\{\})$
\item<4-> $\mathcal{I}' = (\Z,\{>\},\{F(x) = x + 1\},\{\})$
\item<5-> $\mathcal{I}'' = (\Z,\{>\},\{F(x) = x - 1\},\{\})$
\end{itemize}
\pause[4]
Note that:
\begin{itemize}
\item $\mathcal{I} \models A$, $\mathcal{I}' \models A$ and $\mathcal{I}''
\models A$ (ignore the functions)
\item $\mathcal{I}'' \models A'$ but $\mathcal{I}' \not \models A'$. 
\item $A \not \equiv A'$ but $A \approx A'$ 
\end{itemize}
\end{wideslide}

\begin{wideslide}[bm=,toc=]{Skolem's Algorithm}
\begin{itemize}
\item<2-> Rename bound variables to remove name conflicts.
\item<3-> Eliminate Boolean operators except $\ngg, \land, \lor$.
\item<4-> Push negation operators inward to atomic formulas. Use
\begin{itemize}
\item $\ngg \forall x A(x) \equiv \exists x \ngg A (x)$ and
\item $\ngg \exists x A(x) \equiv \forall x \ngg A (x)$ 
\end{itemize}
\item<5-> Extract quantifiers from the matrix using equivalence laws:
\begin{itemize}
\item $A \;op\; Qx B(x) \equiv Qx(A \;op \; B(x))$ and
\item $QxA(x) \;op\; B \equiv Qx(A(x) \;op \; B)$ 
\end{itemize}
\item<6-> Use distributive laws to transform matrix to CNF.
\item<7-> Eliminate existential quantifiers by adding Skolem functions.
\begin{itemize}
\item<8-> For each existential quantifier, $\exists x$, create a new $n-ary$ function $f$ 
\item<9-> $n = $ number of universally quantified variables preceding $\exists x$ in
prenex normal form.
\end{itemize}\end{itemize}
\end{wideslide}


