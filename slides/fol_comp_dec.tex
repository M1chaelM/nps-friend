\documentclass[style=sailor,size=12pt]{powerdot}
\usepackage{epic,array,ecltree,url,calrsfs}
\usepackage[nointegrals]{wasysym}
\usepackage{listings}
\usepackage{epsfig}
\usepackage{amsmath}
\usepackage{amsfonts}
\usepackage{amssymb}
\usepackage{amsxtra}
\usepackage{amsthm}
\usepackage{mlextra} % Must be below ams packages
\usepackage{mathrsfs}
\usepackage{color}
\usepackage{array}
\usepackage{graphicx}
\graphicspath{ {../art/} }
\usepackage{bm}
\usepackage{tikz}
\usepackage{multicol}
\usepackage{enumitem}

\pdsetup{method=normal}
\title{First Order Logic: Computational Complexity and Decidable Subclasses}
\author{Foundations of Computer Science}
\date{\today}

\begin{document}
\maketitle
\section[slide=false]{Computational Complexity of FOL}
\begin{slide}[bm=,toc=]{Validity in FOL is Undecidable}
There is no decision procedure for validity in first order logic.
\begin{itemize}
\item First proved by Alonzo Church in 1936.
\item Proof uses a reduction \emph{from} the halting problem.
\item That is $\id{HALT} \leq \id{FOL-VAL}$
\begin{itemize}
\item The halting problem is proven to be undecidable.
\item Therefore, $\id{FOL-VAL}$ is undecidable.
\end{itemize}
\end{itemize}
How the reduction works:
\begin{itemize}
\item Begins with an arbitrary Turing machine $T$.
\item Constructs a formula $S_T$ in first-order logic, such that
\begin{itemize}
\item $S_T$ is valid iff $T$ halts on a blank tape.
\end{itemize}
\item This shows that a decision procedure for $\id{FOL-VAL}$ would also solve
the halting problem.
\end{itemize}
\end{slide}

\begin{wideslide}[bm=,toc=]{Church's Theorem}
\begin{thm}{12.3}[Church]

\emph{Validity in first-order logic is undecidable.}
\end{thm}
Holds even under the following restrictions:
\begin{enumerate}
\item The formulas contain only binary predicate symbols, one constant and one
unary function symbol.
\begin{itemize}
\item Ex: $\forall x (p(x,f(x)) \imp a$.
\end{itemize}
\item The formulas are logic programs.
\begin{itemize}
\item Restricted set of clauses structured to perform computations.
\item See Ben Ari chapter 11.
\end{itemize}
\item The formulas are pure (see below).
\end{enumerate}
\vspace{-2ex}
\begin{defn}{12.4}[Ben Ari]
A formula of first-order logic is \emph{pure} if it contains no function symbols
(including constants which are 0-ary functions symbols).
\end{defn}
{\bf Note:} Validity is undecidable \emph{with} or \emph{without} function symbols.
\end{wideslide}

\begin{wideslide}[bm=,toc=]{Additional Implications of Church's Theorem}
{\bf Recall:}\\
A formula is valid if and only if its complement (negation) is unsatisfiable.
\begin{itemize}
\item There is a simple reduction from a formula to its complement (negation).
\item Therefore, $\id{FOL-VAL} \leq \id{FOL-UNSAT}$ 
\item $\id{FOL-VAL}$ is undecidable.
\item Therefore, $\id{FOL-UNSAT}$ is undecidable.
\end{itemize}
\end{wideslide}

\begin{wideslide}[bm=,toc=]{Semi-decision Procedures for FOL-UNSAT}
\begin{itemize}
\item Resolution for FOL is semi-decision procedure (J.A Robinson) for FOL-UNSAT.
\item Another technique, constructing a \emph{semantic tableau} is also
semi-decision procedure.
\item Thus, $\id{FOL-UNSAT}$ is semi-decidable.
\item Since $\id{FOL-VAL} \leq \id{FOL-UNSAT}$ 
\item $\id{FOL-VAL}$ is semi-decidable. 
\end{itemize}
{\bf Recall:} 
\begin{itemize}
\item $\id{FOL-SAT}$ = $\id{FOL}$ - $\id{FOL-UNSAT}$
\item Properties of semi-decision procedures:
\begin{itemize}
\item Return ``yes'' if an element is in the target set.
\item Return ``no'' or runs forever if not.
\end{itemize}
\item Therefore, $\id{FOL-SAT}$ is \emph{neither} decidable \emph{nor}
semi-decidable.
\end{itemize}

\end{wideslide}

\begin{wideslide}[bm=,toc=]{Satisfiability vs Validity in PL and FOL}
In propositional logic:
\begin{itemize}
\item $\id{SAT}$ is NP-complete $\imp \id{UNSAT}$ is CoNP-complete. 
\item $\id{UNSAT} \leq_p \id{VAL}$, so $\id{VAL}$ is coNP-Hard.
\item $\id{INVAL}$ is in NP $\imp \id{VAL}$ is in CoNP.
\item $\id{VAL}$ is in CoNP and $\id{VAL}$ is CoNP-Hard $\imp \id{VAL}$ is Co-NP
complete.
\item Therefore, in PL, $\id{VAL}$ is harder than $\id{SAT}$ unless $NP = CoNP$.
\end{itemize}
In first-order logic:
\begin{itemize}
\item $\id{VAL} \leq_p \id{UNSAT}$
\begin{itemize}
\item $\id{VAL}$ is undecidable $\imp \id{UNSAT}$ is undecidable. 
\item $\id{UNSAT}$ is semi-decidable $\imp \id{VAL}$ is semi-decidable. 
\end{itemize}
\item $\id{UNSAT}$ is semi-decidable $\imp \id{SAT}$ \emph{not} semi-decidable. 
\item Therefore, in FOL, $\id{SAT}$ is harder than $\id{VAL}$.
\end{itemize}
\end{wideslide}


\section[slide=false]{Subclasses of First-Order Logic}
\begin{slide}[bm=,toc=]{Decidable Subclasses}
\begin{thm}{12.5}
There are decision procedures for the validity of pure PCNF formulas whose
prefixes are of one of the following forms (where $m,n \geq 0$):
\[ \forall x_1\cdots \forall x_n \exists y_1 \cdots \exists y_m, \]
\[ \forall x_1\cdots \forall x_n \exists y \forall z_1 \cdots \forall z_m, \]
\[ \forall x_1\cdots \forall x_n \exists y_1 \exists y_2 \forall z_1 \cdots \forall z_m. \]
These classes are conveniently abbreviated: 

\[    \forall^*\exists^*, \forall^*\exists \forall^*,   \forall^*\exists \exists \forall^*\]
\end{thm}
\end{slide}

\begin{wideslide}[bm=,toc=]{Undecidable Subclasses}
\begin{thm}{12.6}
There are \emph{no} decision procedures for the validity of pure PCNF formulas whose
prefixes have one of the following forms:
\[ \exists z \forall x_1\cdots \forall x_n \exists y_1 \cdots \exists y_m, \]
\[ \forall x_1\cdots \forall x_n \exists y_1 \exists y_2 \exists y_2 \forall z_1 \cdots \forall z_m. \]
For the first prefix, the result holds even if $n = m = 1$:
\[ \exists z \forall x_1 \exists y_1, \]
and for the second prefix, the result holds even if $n = 0, m = 1$:
\[ \exists y_1 \exists y_2 \exists y_2 \forall z_1. \]
Even if the matrix is restricted to contain only binary predicate symbols, there
is still no decision procedure.
\end{thm}
\end{wideslide}

\begin{wideslide}[bm=,toc=]{Application to Policy Statements}
\begin{ex}{3.2}[Halpern and Weissman]
Consider the policy `anyone who is accompanied by a librarian
may enter the stacks'.\\~\\
In first-order logic:
\begin{align*}
& \forall x_1 (\exists x_2 (\bid{Librarian}(x_2) \land \bid{Accompanies}(x_2,x_1)) \imp \\
& \bid{Permitted}(x_1, enter(stacks)))
\end{align*}
\end{ex}
\begin{itemize} 
\item Note that \emph{enter} is a function.
\item Validity for existential formulas with functions is undecidable. 
\item Need a more restricted sublanguage.
\item Decidable is not enough---must also be \emph{tractable}.
\end{itemize}
\end{wideslide}

\begin{wideslide}[bm=,toc=]{A Tractable Sublanguage for Access
  Control}
Halpern and Weissman propose and analyze the complexity and expressiveness
of several fragments of FOL. One example:
\begin{thm}{4.2}[Halpern and Weissman]
Let $\Phi$ be a vocabulary that contains {\bf Permitted} (and possibly
other predicate, constant, and function symbols). Let $\mathcal{L}_6$
consist of all closed formulas in $\mathcal{L}^{fo}(\Phi)$ of the
form $E \land P \imp \bid{Permitted}(t,t')$, where $P$ is a conjunction
of standard policies and both $t$ and $t'$ and closed terms of the
appropriate sort, such that
\end{thm}
\vspace{-2ex}
\begin{enumerate}
\renewcommand{\labelenumi}{\alph{enumi})}
\item E is a basic environment with $m$ constants,
\item no policy in P has an inequality in its antecendent, and
\item there are no bipolars in P relative to the equality statements
in E.
\end{enumerate}
\emph{If each literal in each policy has at most one variable
  that does not appear in {\bf Permitted}, then we can determine
    the validity of the formula in time $O((|E| + m|P|)log|E|)$.}

    {\tiny (Refer to paper for detailed explanation of these constraints.)}
\end{wideslide}


\end{document}

