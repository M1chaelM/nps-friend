
\documentclass[style=sailor,size=12pt]{powerdot}
\usepackage{epic,array,ecltree,wasysym,url}
\usepackage{mlextra}
\graphicspath{ {../art/} }

\newcommand{\id}[1]{\mbox{\it #1\/}}
\newcommand{\rid}[1]{\mbox{\rm #1}}
\newcommand{\sid}[1]{\mbox{\sf #1}}
\newcommand{\bid}[1]{\mbox{\bf #1}}
\newcommand{\tinysz}[1]{\mbox{\tiny $#1$}}

\pdsetup{method=normal}

\begin{document}

\begin{wideslide}[bm=,toc=]{Recursive definitions}
\begin{itemize}
\item A mutually-recursive definition of {\em rooted trees\/} {\em RT\/}:
\begin{tabbing}
{\bf R2}XX \=  \kill
{\bf B} \>
        \(\begin{array}[t]{l}
        \id{nil}\in\id{RTL}
        \end{array}\) \\[2ex]
{\bf R1} \>
        \(\begin{array}[t]{l}
        t\in\id{RT}\;\;\;l\in\id{RTL} \\
        \hline
        \id{cons}(t,l)\in\id{RTL}
        \end{array}\) \\[2ex]
{\bf R2} \>
        \(\begin{array}[t]{l}
        l\in\id{RTL} \\
        \hline
        \id{node}(l)\in\id{RT}
        \end{array}\)
\end{tabbing}
\item Compare with definition in Rosen 7th Ed.
\item Rosen ignores mutual recursion and hence mutual induction.
\item His definition is not induction friendly.
\end{itemize}
\end{wideslide}

\begin{wideslide}[bm=,toc=]{Mutual induction}
{\bf Theorem}. If $t\in\id{RT}$ then $\id{\#edges}(t)$ = $\id{\#nodes}(t) - 1$.
\vspace{1em}

{\bf Proof}.  By {\em strong\/} mutual induction on derivation height. 
We prove the following two statements:
\vspace{1em}

$S_1$: $l\in \id{RTL}\Rightarrow \id{\#edges}(l) = \id{\#nodes}(l) - |l|$.

$S_2$: $t\in \id{RT}\Rightarrow \id{\#edges}(t) = \id{\#nodes}(t) - 1$.

\vspace{2em}
Note: $|l|$ stands for the length of list $l$, $\id{\#edges}(l)$ the number of edges in $l$
and $\id{\#nodes}(l)$ the number of nodes in $l$.
\end{wideslide}

\begin{wideslide}[bm=,toc=]{Mutual induction}

\vspace{1em}
$S_1(n)$: if the derivation of $l\in \id{RTL}$ has height $n$ then
$\id{\#edges}(l) = \id{\#nodes}(l) - |l|$.
$S_2(n)$: if the derivation of $t\in \id{RT}$ has height $n$ then
$\id{\#edges}(t) = \id{\#nodes}(t) - 1$.

\vspace{1em}



\begin{displaymath}
\begin{array}[t]{l}
S_1(0)\wedge S_2(0) \\
\forall n\geq 0.\, S_1(0)\wedge S_2(0) \wedge \cdots \wedge
S_1(n)\wedge S_2(n)\Rightarrow S_1(n+1)\wedge S_2(n+1)\\
\hline
\forall n\geq 0.\, S(n)
\end{array}
\end{displaymath}

\vspace{1em}
Basis: Show $S_1(0)$ and $S_2(0)$.

\vspace{1em}
Inductive hypothesis: Suppose $S_1(0), S_2(0), S_1(1), S_2(1), \ldots ,S_1(n),S_2(n)$ {\bf where} $n\geq 0$.

\vspace{1em}
Inductive step: Show $S_1(n+1)$ and $S_2(n+1)$.

\end{wideslide}
\begin{wideslide}[bm=,toc=]{Mutual induction}
{\em Basis\/}: derivation height is zero.

\vspace{1em}
$S_1(0)$: If derivation height is zero then $l\in\id{RTL}$ implies $l=\id{nil}$.
Then $\id{\#edges}(\id{nil})=\id{\#nodes}(\id{nil}) - |\id{nil}| = 0$.

\vspace{1em}
$S_2(0)$: Vacuously true since there's no derivation of height zero for $t\in\id{RT}$.

\vspace{2em}
{\em Inductive hypothesis (restated)\/} :  Suppose for all $k$ where $0\leq k\leq n$ and $n\geq 0$,
$S_1(k)$ holds if $l\in\id{RTL}$ has derivation height $k$ and
$S_2(k)$ holds if $t\in\id{RT}$ has derivation height $k$.
\end{wideslide}

\begin{wideslide}[bm=,toc=]{Mutual induction}
{\em Inductive step}.
We must show $S_1(n+1)$ and $S_2(n+1)$ for $n\geq 0$.
We begin with $S_2(n+1)$.

\vspace{1em}
Suppose $t\in\id{RT}$ has derivation height $n+1$ where $n\geq 0$.
Then the derivation must end with {\bf R2}. 
Therefore $t$ has the form $\id{node}(l)$
where $l\in\id{RTL}$.
The derivation of $l\in\id{RTL}$ has height $n$.
%So by induction and $S_1$, $\id{\#edges}(x)=\id{\#nodes}(x)$.
Then
\begin{displaymath}
\begin{array}{lll}
\id{\#nodes}(\id{node}(l)) & = & 1 + \id{\#nodes}(l) \\
	& = & 1+\id{\#edges}(l) + |l|\;\;\;\rid{by induction and}\;S_1 \\
	& = & 1+\id{\#edges}(\id{node}(l))
\end{array}
\end{displaymath}
Ergo, $\id{\#edges}(\id{node}(l))=\id{\#nodes}(\id{node}(l))-1$.
\end{wideslide}

\begin{wideslide}[bm=,toc=]{Mutual induction}
Now we show $S_1(n+1)$.
Suppose $l'\in\id{RTL}$ has derivation height $n+1$ where $n\geq 0$.
Then the derivation must end with {\bf R1}. 
Therefore $l'$ has the form $\id{cons}(t,l)$
where $t\in\id{RT}$ and $l\in\id{RTL}$.
Either the derivation of $t\in\id{RT}$ or $l\in\id{RTL}$ has height $n$ while the other 
has height {\em at most\/} $n$.
%By induction and $S_2$, $\id{\#edges}(x)=\id{\#nodes}(x)-1$.
%Also by induction and $S_1$, $\id{\#edges}(y)=\id{\#nodes}(y)$.
Then
\begin{displaymath}
\begin{array}{lll}
\id{\#edges}(\id{cons}(t,l)) & = & \id{\#edges}(t) + \id{\#edges}(l) \\
	& = & \id{\#nodes}(t) - 1 + \id{\#edges}(l) \\
 & & \>\>\>\>\>\>\rid{by induction and}\;S_2 \\
	& = & \id{\#nodes}(t) - 1 + \id{\#nodes}(l) - |l| \\
 & & \>\>\>\>\>\>\rid{by induction and}\;S_1 \\
	& = & \id{\#nodes}(\id{cons}(t,l)) - 1 - |l| \\
	& = & \id{\#nodes}(\id{cons}(t,l)) - (1 + |l|) \\
	& = & \id{\#nodes}(\id{cons}(t,l)) - |\id{cons}(t,l)|\\
	& = & \id{\#nodes}(l') - |l'|
\end{array}
\end{displaymath}
%{\em quod erat demonstrandum (Q.E.D.)\/}
\end{wideslide}
\end{document}

